% \iffalse meta-comment
%<*internal>
\iffalse
%</internal>
%<*readme>
----------------------------------------------------------------
phd-utils --- some useful uitlities
E-mail: yannislaz@gmail.com
Released under the LaTeX Project Public License v1.3c or later
See http://www.latex-project.org/lppl.txt
----------------------------------------------------------------
This file provides package phd-i18n for handling i18n.
%</readme>
%<*readmemd>
###The `phd` LaTeX2e package

The `phd` latex package and the class with the same name provide
convenient methods to create new styles for books, reports
and articles. It also loads the most commonly used packages 
and resolves conflicts.

This work consists of the file  `phd.dtx`,
and the derived files   `phd.ins`,  `phd.pdf`, and `phd.sty`.

The `phd` bundle is a bunch of 15 packages, that manage all
aspects of document production.

They consist of:

1.0  The [phd-pkgmanager](https://github.com/yannisl/phd/blob/master/docs/phd-pkgmanager.md). This
     package manages all aspects of loading numerous packages and avoiding conflicts. It currently
     loads over 100 packages, directly or indirectly.
     
2.0  The [phd-fontmanager](https://github.com/yannisl/phd/blob/master/docs/phd-fontmanager.md). The
     `phd-fontmanager` sets up the fonts to be used for the document and provides an interface to
     all areas where font data is needed.
     
3.0  The [phd-handlers](https://github.com/yannisl/phd/blob/master/docs/phd-handlers.md). As we use
     extensively automatic key generation via code, this package provides numerous handlers.
     
4.0  The [phd-lowerlevelheadings](https://github.com/yannisl/phd/blob/master/docs/phd-lowerlevelheadings.md)     

5.0  The [phd-toc](https://github.com/yannisl/phd/blob/master/docs/phd-toc.md) package.
     Manages all aspects of Table of Contents via a key value interface. 

6.0  The [phd-counters](https://github.com/yannisl/phd/blob/master/docs/phd-counters.md) 

7.0  The [phd-colorpalette](https://github.com/yannisl/phd/blob/master/docs/phd-colorpalette.md). This
     package introduces the concept of a `color palette` to `LaTeX` coding. It groups all color
     data into `palettes`. By setting a single set of keys, the document can be updated with new
     color information.

8.0  The [phd-scriptsmanager](https://github.com/yannisl/phd/blob/master/docs/phd-scriptsmanager.md).
     Handles the settings and provides a key face interface to over scripts for use in typesetting
     texts from ancient to modern.

9.0  The [phd-documentation](https://github.com/yannisl/phd/blob/master/docs/phd-documentation.md).
     Provides numerous commands and keys for typesetting code. It also provides indexing shortcuts,
     including math symbols etc.
     
10.0 The [phd-epigraphs](https://github.com/yannisl/phd/blob/master/docs/phd-epigraphs.md).
     This package manages the typesetting of epigraphs.
     
11.0 The [phd-frontmatter](https://github.com/yannisl/phd/blob/master/docs/phd-epigraphs.md)
     package for the typesetting of frontmatter such as coverpages, copyright pages and title
     pages.
     
12.0 The [phd-lorems](https://github.com/yannisl/phd/blob/master/docs/phd-lorems.md)    
     package. Provides some additional commands for filler text.

13.0 The [phd-quote](https://github.com/yannisl/phd/blob/master/docs/phd-quote.md)    
     package. Provides some additional commands for filler text.
     
14.0 The [phd-lists](https://github.com/yannisl/phd/blob/master/docs/phd-lists.md)    
     package. Provides some additional commands and a key value interface for lists.  
     
15.0 The [phd-logos](https://github.com/yannisl/phd/blob/master/docs/phd-logos.md)    
     package. Supplementary commands for logos.         
      
###Installation

run
        `phd-lua phd-i18n.dtx` on windows
        
If you have any difficulties with the package come and join us at
http://tex.stackexchange.com and post a new question or
add a comment at http://tex.stackexchange.com/a/45023/963.
or send me a message at  yannislaz at gmail.com

### Documentation

The package was written using the `doc` and `docscript` packages,
so that it is self documented in a literary programming style. 
The .pdf is a fat document, providing over fifty book styles (the
equivalent of classes) plus there is a lot of write-up on the inner
workings of TeX and LaTeX2e. However, you don't need to know much
to use it.

      \usepackage{phd-i18n}
      %%%%%%%%%%%%%%%%%%%%%%%%%%%%%%%%%%%%%%%%%%%
%%%%%%  STYLE 13
%%%%%%%%%%%%%%%%%%%%%%%%%%%%%%%%%%%%%%%%%%%

\cxset{style13/.style={
 name={Chapter},
 numbering=arabic,
 number font-size=\HUGE,
 number font-family=\sffamily,
 number font-weight=\bfseries,
 number color=\color{gray!50},
 number before=\par\vspace*{5pt}\hfill\hfill,
 number dot=,
 number after={\hspace*{7pt}\par},
 number position=rightname,
 chapter font-family=\sffamily,
 chapter font-weight=\normalfont,
 chapter font-size=\LARGE,
 chapter before={\thickrule\vspace*{20pt}\par\hfill\hfill},
 chapter after={\vskip0pt\par},
 chapter color={black!50},
 title beforeskip={\vspace*{10pt}},
 title afterskip={\vspace*{50pt}\par},
 title before={\hfill\hfill\raggedleft},
 title after={},
 title font-family=\sffamily,
 title font-color=\color{thered},
 title font-weight=\bfseries,
 title font-size=\huge,
 section indent=-1em,
 section align=\raggedright,
 section numbering=arabic,
 section indent=0pt,
 section beforeskip=0pt,
 section afterskip=\baselineskip,
 subsection align=\raggedright,
 subsection font-family=\sffamily,
 subsection font-weight=\bfseries,
 subsection font-size=\large,
 subsection font-shape=\itshape,
 subparagraph number after=\space,
}
}

\def\setstyle#1{\cxset{style#1}%
 \renewsection\renewsubsection\renewsubsubsection%
 \renewparagraph\renewsubparagraph}

\setstyle{13}


\chapter{Introduction to Chapter\\ Style Thirteen}

\section{A Brief History of Biomedical\\ Fluid Mechanics}
\lorem
\medskip
\begin{figure}[ht]
\centering
\includegraphics[width=0.45\textwidth]{./chapters/chapter14}
\includegraphics[width=0.45\textwidth]{./chapters/chapter14a}
\end{figure}
\lorem


All choices, are made via an extended key-value interface. 
Although not a compliment, it resembles CSS and the keys are a bit verbose but
attributes are easy to change and have a consistent and easy to remember interface.

To set or add a key we only use one command:

      \cxset{chapter name font-size: Huge,
             chapter number font-size: HUGE} 

### Future Development

This is still an experimental version, but I will retain the
interface in future releases. There is a large amount of
work still to be carried out to improve the template styles
provided, to test it more thoroughly and to add a number of
improvements in the special designs. At present I estimate
that I have completed about 70% of the work that needs
to be done.

__The package as it stands is not production stable.__ 


%</readmemd>
%
%<*TODO>
1. On final round add pkg options. This was left as last in order not to solve problems by adding
    options. Too many options are not a good User Interface.
2.  Finish symbol management, both text and math. Math already 60% incorporated.
3.  Better integration of indexing commands.   
4.  Revisit layout manager for Chapters. Broke again in tests.
5.  Docs. Add all references.
6.  Incorporate phd class for more flexibility.
7. Improve package manager.
8. Group script loading for better font management.
9. General font management to relook it again.
10. Add all style sections (about 100 already prepared). Once they
     are all working issue beta version.
%</TODO>
%<*internal>
\fi
\def\nameofplainTeX{plain}
\ifx\fmtname\nameofplainTeX\else
  \expandafter\begingroup
\fi
%</internal>
%<*install>
\input docstrip.tex
\keepsilent
\askforoverwritefalse
\preamble
----------------------------------------------------------------
phd-i18n --- A package to beautify documents.
E-mail: yannislaz@gmail.com
Released under the LaTeX Project Public License v1.3c or later
See http://www.latex-project.org/lppl.txt
----------------------------------------------------------------
\endpreamble
%\BaseDirectory{C:/users/admin/my documents/github/phd}
%\usedir{MWE}
\generate{\file{\jobname.sty}{\from{\jobname.dtx}{package}}}
%</install>
%<install>\endbatchfile
%<*internal>
%\usedir{tex/latex/phd}
\generate{
  \file{\jobname.ins}{\from{\jobname.dtx}{install}}
}
\nopreamble\nopostamble

\generate{
	\file{README.txt}{\from{\jobname.dtx}{readme}}
  }

\generate{
  \file{README.md}{\from{\jobname.dtx}{readmemd}}
}
\generate{
  \file{TODO.tex}{\from{\jobname.dtx}{TODO}}
}

\ifx\fmtname\nameofplainTeX
  \expandafter\endbatchfile
\else
  \expandafter\endgroup
\fi
 
\immediate\write18{makeindex -s gglo.ist -g phd.gls phd.glo}  %needs checking from trivfloat
\immediate\write18{makeindex -s gind.ist -g phd.ind phd.idx} %needs checking from Joseph’s trivfloat
%</internal>
%<*driver>
\NeedsTeXFormat{LaTeX2e}[2017/04/15]%
\RequirePackage[2017/04/15]{latexrelease}
\documentclass[book,twoside,10pt,a4paper]{phddoc}
\usepackage{phdfilecontents}
\def\partname{Part}
\let\HUGE\huge
\usepackage[bottom=2cm]{geometry}
\savegeometry{std}
\usepackage[microtype=on]{phd}
\usepackage{phd-pkgmanager}
\usepackage{phd-scriptsmanager}
\usepackage{phd-fontmanager}
\usepackage{phd-runningheads}
\usepackage{phd-lowersections}
\usepackage{phd-documentation}
\usepackage{phd-toc}
\usepackage{phd-lists}
\usepackage{phd-quote}
\usepackage{phd-i18n}
\sethyperref
\usepackage{makeidx}
\makeindex
\cxset{part format=stewart,
       chapter afterindent=on,
       subsection afterindent=off,
       part afterindent=on,
       section format=hang,
       chapter format=block,
       chapter opening=left,
       chapter label background-color=white,
       chapter number background-color=white,
       chapter title font-family = upshape}
%\newfontfamily\symbola{Symbola.ttf} 
\addbibresource{phd1.bib}
\usepackage{numprint}
\setmonofont[Scale=.88,FakeStretch=.95]{Apl385} %was Noto Mono
\definecolor{messages}{rgb}{.66,.13,.27}
\makeatletter
\def\@begintheorem#1#2{%
  \list{}{}%
  \global\advance\@listdepth\m@ne
  \item[{\sffamily\bfseries\color{messages}\hspace*{1.3em}%
        \MakeUppercase{#1}}]}%
\makeatother
\newtheorem{warning}{Warning}
\newtheorem{note}{Note}
\newtheorem{examples}{Examples}
\newtheorem{troubleshooting}{Troubleshooting}
\let\aegean\panunicode
\begin{document}
\cxset{locale finnish}
\DEBUGOFF
\let\luacmd\docAuxCommand
\frontmatter
\tableofcontents
%\listoffigures
%\listoftables
\mainmatter
\pagestyle{headings-sugar-hearts}
\parindent1em
%\index{Katakana}\index{Hiragana}
\index{Bopomofo}\index{Hangul}\index{Yi}
\index{East Asian Scripts>Katakana}
\index{East Asian Scripts>Hiragana}
\index{East Asian Scripts>Hangul}
\index{East Asian Scripts>Bopomofo}
\index{East Asian Scripts>Yi}
\index{scripts>cjk}
\pagestyle{headings}
\index{Yi fonts>Microsoft Yi Baiti}
\chapter{East Asian Scripts}
\epigraph{

For writing is the foundation of the classics and the arts, the beginning of
royal government. It is the means by which people of the past reach posterity,
by which people of the future know the past. 

{\cjk 蓋文字者,經藝之本,王政之始。前人所以垂後,後人所以識古。}
}{ Xu Shen  in the ``Postface'' of the \emph{Shuowen}}

\bigskip

\noindent This chapter presents the most common scripts currently in use in East Asia. This includes Chinese, Japanese and Korean. It also discusses several scripts for minority languages spoken in southern China. The scripts discussed are as follows:


\begin{center}
\begin{tabular}{lll}
\nameref{s:han} &Hiragana &Hangul\\
\nameref{s:bopomofo} &Katakana &\nameref{s:yi}\\
\end{tabular}
\end{center}
\bigskip

\parindent1em

Settings for |cjk| languages and scripts follow:

\begin{docKey}[phd]{cjk font}{\meta{font name}}{default none, initial code2000.ttf}
This key when set produces all necessary command to set the font for cjk typesetting.
\end{docKey}

\parindent1em
\section{Han CJK Unified Ideographs}
\label{s:han}
\index{CJK}
The Chinese, Japanese and Korean (CJK) scripts share a common background. In the process called Han unification the common (shared) characters were identified, and named "CJK Unified Ideographs". Unicode defines a total of 74,617 CJK Unified Ideographs.[1]\footnote{\protect\url{http://shahon.org/wp-content/uploads/2010/02/Galambos-2006-Orthography-of-early-Chinese-writing.pdf}}

The terms ideographs or ideograms may be misleading, since the Chinese script is not strictly a picture writing system.
Historically, Vietnam used Chinese ideographs too, so sometimes the abbreviation "CJKV" is used. This system was replaced by the Latin-based Vietnamese alphabet in the 1920s.


\unicodetable{cjk}{"4E00,"4E10,"4E20,"4E30,"4E40,"4}




\section{Bopomofo}
\label{s:bopomofo}
Bopomofo is the colloquial name of the \textit{zhuyin fuhao} or \textit{zhuyin} system of phonetic notation for the transcription of spoken Chinese, particularly the Mandarin dialect. Consisting of 37 characters and four tone marks, it transcribes all possible sounds in Mandarin. 

Bopomofo was introduced in China by the Republican Government, in the 1910s and used alongside the Wade-Giles system, which used a modified Latin alphabet. The Wade system was replaced by \textit{Hanyu Pinyin} in 1958 by the Government of the People's Republic of China,[1] at the International Organization for Standardization (ISO) in 1982 (ISO 7098:1982). Bopomofo remains widely used as an educational tool and electronic input method in Taiwan. On Windows the font Microsoft JhengHei can be used. 

Windows fonts that can be used \texttt{Microsoft JhengHei} and \texttt{SimSun}.

U+3100–U+312F
\newfontfamily\bopomofo{Microsoft JhengHei}

\begin{scriptexample}[]{Bopomofo}
{\centering\bopomofo 

伯帛勃脖舶博渤霸壩灞

}

\hfill \texttt{Typeset with \cmd{\bopomofo} and Microsoft JhengHei font }
\end{scriptexample}

\begin{scriptexample}[]{Bopomofo}

{\centering\bopomofo

伯帛勃脖舶博渤霸壩灞

}
\hfill \texttt{Typeset with \cmd{\bopomofo} and JhengHei font }
\end{scriptexample}


The Bopomofo Extended block, running from \unicodenumber{U+31A0-U31BF}, contains less universally recognized Bopomofo characters used to write various non-Mandarin Chinese languages. A few additional tone marks are unified with characters in the Spacing Modifier Letters block. 












\section{Yi}
\label{s:yi}

The Yi script (Yi: {\yi ꆈꌠꁱꂷ} nuosu bburma [nɔ̄sū bū̠mā]; Chinese: {\cjk 彝文}; pinyin: Yí wén) is an umbrella term for two scripts used to write the Yi language; Classical Yi, an ideogram script, the later Yi Syllabary. The script is also historically known in Chinese as Cuan Wen (Chinese: {\cjk 爨文}; pinyin: Cuàn wén) or Wei Shu (simplified Chinese: {\cjk韪书}; traditional Chinese: {\cjk 違書}; pinyin: Wéi shū) and various other names ({\cjk夷字、倮語、倮倮文、毕摩文}), among them "tadpole writing" ({\cjk蝌蚪文}).[1]

This is to be distinguished from romanized Yi ({\yi 彝文罗马拼音} Yiwen Luoma pinyin) which was a system (or systems) invented by missionaries and intermittently used afterwards by some government institutions.[2][3] There was also a Yi abugida or alphasyllabary devised by Sam Pollard, the Pollard script for the Miao language, which he adapted into "Nasu" as well.[4][5] Present day traditional Yi writing can be sub-divided into five main varieties (Huáng Jiànmíng 1993); Nuosu (the prestige form of the Yi language centred on the Liangshan area), Nasu (including the Wusa), Nisu (Southern Yi), Sani (撒尼) and Azhe (阿哲).[6][7]

The Unicode block for Modern Yi is Yi syllables (U+A000 to U+A48C), and comprises 1,164 syllables (syllables with a diacritic mark are encoded separately, and are not decomposable into syllable plus combining diacritical mark) and one syllable iteration mark (U+A015, incorrectly named YI SYLLABLE WU). In addition, a set of 55 radicals for use in dictionary classification are encoded at U+A490 to U+A4C6 (Yi Radicals).[11] Yi syllables and Yi radicals were added as new blocks to Unicode Standard Version 3.0.[12]

Classical Yi - which is an ideographic script like the Chinese characters - has not yet been encoded in Unicode, but a proposal to encode 88,613 Classical Yi characters was made in 2007.[13]

\bgroup
\yi \char"A000: Yi Syllable It\\

\yi \char"A001: Yi Syllable Ix\\

\yi \char"A002: Yi Syllable I\\
\egroup

\begin{scriptexample}[]{Yi}
\unicodetable{yi}{"A000,"A010,"A020,"A030,"A040,"A050,"A060,"A070,"A080,"A090,"A0A0,"A0B0,"A0C0}
\end{scriptexample}



%\newfontfamily\korean{NotoSerifCJKkr}

\def\textko#1{\bgroup\korean #1\egroup}

\chapter{Korean}

\section{Origins}

Korea has a fairly homogeneous population so the question as where did the language came from has perplexed linguists. 
This origin question is of ultimate interest to linguists, but it has also captured the imagination of the
Korean lay public, who have tended to conflate the question with broader ones about their own ethnic origin. Linguistic nomenclature has added to the confusion. When specialists speak to the public about \enquote{family trees} and
\enquote{related languages,} the non-specialist naturally thinks that the Korean language
has relatives and a biological family like those people do. And when
a people as homogeneous as Koreans are told that their language belongs to a
family that includes Mongolian and Manchu, they envision their ancestors
arriving in the cul-de-sac of the Korean peninsula as horse-riding warriors.
It becomes a personal kind of romance.\footcite{ki-moon2011}

Nevertheless, the answer to the question of where Korean came from is
still incomplete. In order for a genetic hypothesis to be truly convincing, the
proposed rules of correspondence must lead to additional, often unsuspected
discoveries about the relationship. Concrete facts must emerge about the
history of each language being compared in order to put the hypothesis
beyond challenges to its validity, and that has so far not happened in the case
of Korean. As a result, we cannot yet say with complete certainty what the
origin of Korean was.\footcite{ki-moon2011}



\section{Hangul}
Hangul or Hangeul (English pronunciation: /ˈhɑːnɡuːl/ HAHN-gool;[1] from Korean {\korean 한글} [ha(ː)n.ɡɯl]) is the South Korean term for the Korean alphabet (called Chosŏn'gŭl ({\korean 조선글}) in North Korea), which has been used to write the Korean language since its creation in the 15th century by King Sejong the Great.[2][3]

It is the official writing system of North Korea and South Korea. It is a co-official writing system in the Yanbian Korean Autonomous Prefecture and Changbai Korean Autonomous County in Jilin Province, China. It is also used to write the Cia-Cia language spoken near the town of Bau-Bau, Indonesia.

Korean is known as an alphabetic language in the literature ( Perfetti, 2003;
 Wang et al ., 2003 ; Simpson and Kang, 2004;
 Perfetti and Liu, 2005 ). However, the Korean language has a distinctive feature of a syllabic writing
system. Korean is alphabetic in that each letter maps onto a phoneme and each grapheme ties with
a vowel to form a syllabic unit. Each grapheme in Korean (a consonant or a vowel) has its individual
sound, but a consonant has to glue together with a vowel for the consonant to be vocalized. A consonant
string in the initial, middle, and ending positions is unlikely to occur in Korean. Talylor and Olson (1995), as well as H.K. Pay claims that Korean is an alphabetic syllabary or a syllabic alphabet. \footcite{perfetti2003}


The alphabet consists of 14 consonants and 10 vowels. Its letters are grouped into syllabic blocks, vertically and horizontally. For example, the Korean word for \enquote{honeybee} is written \textko{꿀벌}, not \textko{ㄲㅜㄹㅂㅓㄹ}.[4] As it combines the features of alphabetic and syllabic writing systems, it has been described as an \enquote{alphabetic syllabary} by some linguists.[5][6] As in traditional Chinese writing, Korean texts were traditionally written top to bottom, right to left, and are occasionally still written this way for stylistic purposes. Today, it is typically written from left to right with spaces between words and western-style punctuation.[7]

Some linguists consider it the most logical writing system in the world, partly because the shapes of its consonants mimic the shapes of the speaker's mouth when pronouncing each consonant.[5][7][8]

\section{The transcription of Sino-Korean}

If we take Sejong at his word, the new symbols were devised explicitly to
represent the sounds of Korean. In the preface to the Hunmin cho˘ngu˘m he
wrote:

\begin{quotation}
The sounds of our country’s language are different from those of the Middle Kingdom
and are not smoothly adaptable to those of Chinese characters. Therefore, among the
simple people, there are many who have something they wish to put into words but are
never able to express their feelings. I am distressed by this, and have newly designed
twenty-eight letters. I desire only that everyone practice them at their leisure and make
them convenient for daily use.
\end{quotation}

But from the very beginning, the new letters were used to transcribe
the readings of Chinese characters as well as to write native Korean words,
and both are found together in the texts of the period. As we have said,
these character readings do not represent natural Korean but rather the
prescriptive pronunciations spelled out in detail in the Tongguk chongun
of 1447.

\subsection{Vowels}

Vowel letters are based on three elements:

\begin{enumerate}
\item A horizontal line representing the flat Earth, the essence of yin.

\item A point for the Sun in the heavens, the essence of yang. (This becomes a short stroke when written with a brush.)

\item A vertical line for the upright Human, the neutral mediator between the Heaven and Earth.
\end{enumerate}

Short strokes (dots in the earliest documents) were added to these three basic elements to derive the vowel letter. These are divided into three categories a) simple vowels b) compound vowels and c) iotized vowels. 

\subsubsection{Simple vowels}
\paragraph{Horizontal letters:} these are mid-high back vowels.

bright {\korean ㅗ} [o]

dark {\korean ㅜ} [u]

neutral {\korean ㅡ} eu (ŭ)

\paragraph{Vertical letters:} these were once low vowels.

bright \textko{ㅏ} a

dark {\korean ㅓ} eo (ŏ)

neutral \textko{\char12643 } i

\subsubsection{Compound vowels}

The Korean alphabet never had a w, except in Sino-Korean vocabulary. Since an o or u before an a or eo became a [w] sound, and [w] occurred nowhere else, [w] could always be analyzed as a phonemic o or u, and no letter for [w] was needed. However, vowel harmony is observed: \enquote{dark} \textko{ㅜ} u with \enquote{dark} \textko{ㅓ} eo for \textko{ㅝ} wo; \enquote{bright} \textko{ㅗ} o with \enquote{bright} \textko{ㅏ} a for \textko{ㅘ} wa:

{\korean
ㅘ wa = ㅗ o + ㅏ a

ㅝ wo = ㅜ u + ㅓ eo

ㅙ wae = ㅗ o + ㅐ ae

ㅞ we = ㅜ u + ㅔ e
}


\subsubsection{Iotized vowels}



The table below shows the 21 vowels used in the modern Korean Alphabet in South Korean alphabetic order with Revised Romanization equivalents for each letter. Linguists disagree on the number of phonemes versus diphthongs among vowels in the Korean alphabet.[40]

\bigskip

\begingroup
\setlength{\tabcolsep}{2pt}
\korean
\begin{tabular}{p{2cm}lllllllllllllllllllll}
Letters       &ㅏ &ㅐ	&ㅑ	&ㅒ	&ㅓ	&ㅔ	&ㅕ	&ㅖ	&ㅗ	&ㅘ	&ㅙ	&ㅚ	&ㅛ	&ㅜ	&ㅝ	&ㅞ	&ㅟ	&ㅠ	&ㅡ	&ㅢ	&ㅣ\\
Revised Romanization	&a	&ae	&ya	&yae	&eo	&e	 &yeo	&ye	&o	&wa	&wae	&oe	&yo	&u	&wo	&we	&wi	&yu	&eu	&ui	&i\\
\end{tabular}
\endgroup

\section{The Korean Letters: Jamo}

Every Korean syllable consists of a \textit{lead consonant}, a medial \textit{vowel} and a \textit{tail} consonant. To write syllables with an initial vowel  a special sign for a mute lead consonant must be used. In open syllables (syllables ending in a vowel), the tail consonant is omitted. Isolated vowels can be considered regular syllables with a mute initial and a missing tail.

There are 19 different lead consonants, including the mute consonant. The following table gives the consonants in their canonical order, and their Unicode values. Consonant number 12 is the mute consonant.

\begingroup
\arrayrulecolor{thetablevrulecolor}%
\begin{longtable}{ll >{\korean}l >{\korean}l >{\ttfamily}l}
\toprule
Number	&Lead	&Jamo	& &Character\\ 
        &     &     & &reference\\
\midrule                     
1	&G	&ᄀ	&ㄱ	&U+1100\\
2	&GG	&ᄁ	&ㄲ	&U+1101\\
3	&N	   &ᄂ	&ㄴ	&U+1102\\
4	&D	 &ᄃ	&ㄷ	&U+1103\\
5	&DD	&ᄄ	&ㄸ	&U+1104\\
6	&R	&ᄅ	&ㄹ	&U+1105\\
7	&M	&ᄆ	&ㅁ	&U+1106\\
8	&B	&ᄇ	&ㅂ	&U+1107\\
9	&BB	&ᄈ	&ㅃ	&U+1108\\
10	&S	&ᄉ	&ㅅ	&U+1109\\
11	&SS	&ᄊ	&ㅆ	&U+110A\\
12	&ᄋ	&ㅇ	&ㅇ&U+110B\\
13	&J	  &ᄌ	&ㅈ	&U+110C\\
14	&JJ	&ᄍ	&ㅉ	&U+110D\\
15	&C	   &ᄎ	&ㅊ	&U+110E\\
16	&K	   &ᄏ	&ㅋ	&U+110F\\
17	&T	   &ᄐ	&ㅌ	&U+1110\\
18	&P	   &ᄑ	&ㅍ	&U+1111\\
19	&H	   &ᄒ	&ㅎ	&U+1112\\
\bottomrule
\end{longtable}
\endgroup

\begingroup
\arrayrulecolor{thetablevrulecolor}%
\begin{longtable}{ll >{\korean}l >{\korean}l >{\ttfamily}l}
\toprule
Number	&Lead	&Jamo	& &Character\\ 
        &     &     & &reference\\
\midrule 
1	&G	 &ᆨ	&ㄱ	& U+11A8\\
2	&GG	&ᆩ	&ㄲ	& U+11A9\\
3	&GS	&ᆪ	&ㄳ	& U+11AA\\
4	&N	   &ᆫ	&ㄴ	& U+11AB\\
5	&NJ	&ᆬ	&ㄵ	& U+11AC\\
6	&NH	&ᆭ	&ㄶ	& U+11AD\\
7	&D	&ᆮ	&ㄷ	& U+11AE\\
8	&L	&ᆯ	&ㄹ	& U+11AF\\
9	&LG	&ᆰ	&ㄺ	& U+11B0\\
10	&LM	&ᆱ	&ㄻ	& U+11B1\\
11	&LB	&ᆲ	&ㄼ	& U+11B2\\
12	&LS	&ᆳ	&ㄽ	& U+11B3\\
13	&LT	&ᆴ	&ㄾ	& U+11B4\\
14	&LP	&ᆵ	&ㄿ	& U+11B5\\
15	&LH	&ᆶ	&ㅀ	& U+11B6\\
16	&M	&ᆷ	&ㅁ	& U+11B7\\
17	&B	&ᆸ	&ㅂ	& U+11B8\\
18	&BS	&ᆹ	&ㅄ	& U+11B9\\
19	&S	&ᆺ	&ㅅ	& U+11BA\\
20	&SS	&ᆻ	&ㅆ	& U+11BB\\
21	&NG	&ᆼ	&ㅇ	& U+11BC\\
22	&J	&ᆽ	&ㅈ	& U+11BD\\
23	&C	&ᆾ	&ㅊ	& U+11BE\\
24	&K	&ᆿ	&ㅋ	& U+11BF\\
25	&T	&ᇀ	&ㅌ	& U+11C0\\
26	&P	&ᇁ	&ㅍ	& U+11C1\\
27	&H	&ᇂ	&ㅎ	& U+11C2\\
\bottomrule
\end{longtable}
\endgroup

A useful utility for composing Hangul can be found at gernot-katzers website.
\footnote{\url{http://gernot-katzers-spice-pages.com/var/korean_hangul_unicode.html}} The utility helped me personally to understand how the syllabels are formed from the lead, medial and final letters to syllables and how the syllables are shaped by the text shaping engines. The Tables above and much of the text here are based on these webpages. 

\section{Typography}


\begin{figure}[htbp]
\includegraphics[width=\textwidth]{hangul-direction}
\caption[Korean horizontal and vertical writing]{Horizontal writing and vertical writing (arrow indicates the text direction), from \protect\url{https://www.w3.org/TR/klreq}}
\end{figure}

\paragraph{Line adjustment}

Text can be line adjusted (justified) both in the vertical or horizontal direction. 


\section{LaTeX}

Using one of the newer TeX engines such as LuaLaTeX enables one to typeset Korean almost effortlessly. 
The character \char12643 can give problems sometimes, if you copy paste as it can be confused with the \textbar. There are tow issues for capturing text. If you are on windows you can use a windows virtual keyboard, which tends to work well. The keyboard also has a predictive feature that pops up a menu to choose the correct syllable or word (very similar to pinyn keyboards for Chinese).

\begin{figure}[htbp]
\centering

\includegraphics[width=0.5\textwidth]{hangul-keyboard}

\caption{Virtual Korean keyboard for windows 10. The keyboard can be rendered long or squarish. I personally find the squarish shape more convenient, as it takes less screen space.}

\end{figure}

An alternative way is to use a web based interface such as \href{https://r12a.github.io/pickers/}{r12a}. Google also offers virtual keyboards that can be downloaded.


There are two sets of commands that are usefull utilities for unicode. The first set if you know the name of the 
unicode character it can be used t print the character. These commands are generated by the phd-i18n utility scripts, based
on the latest UCD database (version 11.0) as of this writing.

\begin{texexample}{Printing Korean Characters}{ex:kochar}
\ExplSyntaxOn
\cs_set:cpn {yu}
  {
    HANGUL~JUNGSEONG~YU\par 
    \space
    \large
    \begingroup
      \korean
      \centering
      \char"1172 
    \endgroup
  }
\use:c {yu}
\ExplSyntaxOff
\end{texexample}

The second set of commands is used for transcription of texts, based on the revised romanization scheme. 

















%\chapter{Phags-pa}
\label{s:phagspa}
\newfontfamily\phagspa{code2000.ttf}
\arial 
The 'Phags-pa script, (Mongolian: дөрвөлжин үсэг "Square script") was an alphabet designed by the Tibetan monk and vice-king Drogön Chögyal Phags-pa for the Mongol Yuan emperor Kublai Khan as a unified script for the literary languages of the Yuan. 


It was first promulgated in 1269, although there is an inscription to testify its use before that time. ThePP alphabet is considered to have been designed for all the languages of the Mongol empire, but it appears it was almost used exclusively for Mongolian and Chinese. 

Widespread use was limited to about a hundred years during the Yuan Dynasty, and it fell out of use with the advent of the Ming dynasty. The documentation of its use provides clues about the changes in the varieties of Chinese, the Tibetic languages, Mongolian and other neighboring languages during the Yuan era.

After the fall of the Yuan dynasty in 1368, PP fell out of use, although it may have survived on seals and in some copybooks (although some hold that these descend froma Tibetan seal script rather than from PP).  

PP was based on the Tibetan alphabet and was used for writing both Chinese and Mongolian.

The script as a whole system comes down to us in two different traditions. On the one hand we have the letters and arrangements as they were recorded in the \textit{Shu shih hui yao} and the Fas shu k'ao, two 14th century works on calligraphy. Here the letters are presented in a more Buddist and Tibetan tradition.


\begin{figure}[htbp]
\includegraphics[width=1\linewidth]{./images/phags-pa.jpg}

credit \protect\url{http://turfan.bbaw.de/dta/monght/images/monght009_seite2.jpg}
\end{figure}


\begin{scriptexample}[]{Phags-pa}
\bgroup
\unicodetable{phagspa}{"A840,"A850,"A860,"A870}

\arial
\hfill Typeset with \texttt{code2000.ttf} and \cmd{\phagspa}

\egroup
\end{scriptexample}
\medskip

Phags-pa is a historical script related to Tibetan that was created as the national script of
the Mongol empire. Even though Phags-pa was used mostly in Eastern and Central Asia for
writing text in the Mongolian and Chinese languages, it is discussed in this chapter because
of its close historical connection to the Tibetan script. The script has very limited modern use. It bears similarity to Tibetan and has no case distinctions. It is written vertically in columns running for left to right, like Mongolian. Units are often composed of several syllables and sometimes are separated by whitespace.


\printunicodeblock{./languages/phags-pa.txt}{\phagspa}

\cxset{script/.code={}}
\cxset{script=phags-pa}

\begin{docKey}[phd]{script}{ = \meta{phags-pa}} {}
The key |script| will activate the commands available for typesetting the phags-pa script.
\end{docKey}















%\chapter{Unicode}

Unicode is an encoding of \textit{characters}, and it is the first encoding that took the trouble to define what a
\textit{character} is. The distinction between a character and a \textit{glyph} has come to be of interest to philosophers with the Japanese philosopher Shigeki Moro to say that Unicode’s approach is Aristotelian essentialist. 

In this book we adopt the practical definition given by Spyropoulos in his book Unicode \& Encodings. 

\begin{itemize}

\item A glyph is the image of a symbol used in a writing system (in an alphabet, a syllabary, a set of ideographs, etc.) or in a notational system (like music, mathematics, cartography etc.)

\item A \textit{character} is the simple description, primarily linguistic or logical, of an equivalence class of glyphs.
\end{itemize}

\section{Unicode's Principles.}

Unicode subscribes to ten principles.

\medskip
\begin{tabular}{ll}
Universal repertoire &Logical order\\
Efficiency &Unification\\
Characters, not glyphs &Dynamic composition\\
Semantics &Stability\\
Plain Text &Convertibility\\
\end{tabular}
\medskip


The character sets of many existing international, national and corporate standards are incorporated within the Unicode Standard. For example, its first 256 characters are taken from the widely used Latin-1 character set.

Duplicate encoding of characters is avoided by unifying characters within scripts across languages; characters that are equivalent in form are given a single code. Chinese/Japanese/Korean (CJK) consolidation is achieved by assigning a single code for each ideograph that is common to more than one of these languages. This is instead of providing a separate code for the ideograph each time it appears in a different language. (These three languages share many thousands of identical characters because their ideograph sets evolved from the same source.)

The Unicode Standard specifies an algorithm for the presentation of text with bidirectional behavior, for example, Arabic and English. Characters are stored in logical order. The Unicode Standard includes characters to specify changes in direction when scripts of different directionality are mixed. For all scripts Unicode text is in logical order within the memory representation, corresponding to the order in which text is typed on the keyboard.


\section{Unicode Character Database}

Unicode provides all the raw data in its database in the form of specially formatted text files.\footnote{\url{http://www.unicode.org/reports/tr44/tr44-20.html\#Unicode_10.0.0}}

\newfontfamily\panuni{aegean}

\section{Character Data}

Each character is defined by a unique codepoint. Unicode does not care about how it looks, but how it is described, so each character is defined by a unique codepoint.  In addition every character has a number of properties associated with it that defines how a character is to be used by various processes. Below we illustrate some of these properties by parsing a single unicode point, using a custom Lua script.

\bgroup
\parindent=0pt
\begin{multicols}{2}
\small
\panuni
\luadirect{
   dofile("./i18n/parseunicode.lua")
}
\end{multicols}
\egroup


Among the properties that each character has are:

\begin{enumerate}
\item The character’s code-point value and name.

\item The character’s general category. All of the characters in Unicode are grouped into 30 categories,
17 of which are considered normative. The category tells you things like whether the character is
a letter, numeral, symbol, whitespace character, control code, etc.

\item The character’s decomposition, along with whether it’s a canonical or compatibility
decomposition, and for compatibility composites, a tag that attempts to indicate what data is lost
when you convert to the decomposed form.

\item The character’s case mapping. If the character is a cased letter, the database includes the mapping
from the character to its counterpart in the opposite case.

\item For characters that are considered numerals, the database includes the character’s numeric value.
(That is, the numeric value the character represents, not the character’s code point value.)

\item The character’s directionality. (e.g., whether it’s left-to-right, right-to-left, or takes on the
directionality of the surrounding text). The Unicode Bidirectional Layout Algorithm uses this
property to determine how to arrange characters of different directionalities on a single line of
text.

\item The character’s mirroring property. This says whether the character take on a mirror-image glyph
shape when surrounded by right-to-left text.

\item The character’s combining class. This is used to derive the canonical representation of a character
with more than one combining mark attached to it (it’s used to derive the canonical ordering of
combining characters that don’t interact with each other).

\item The character’s line-break properties. This is used by text rendering processes to help figure out
where line divisions should go.

\item  Many more… 
\end{enumerate}


Most of the information for each character is obtained from UnicodeData.txt. This file contains most of the  
Unicode Character Database. As the database has grown, and as supplementary information has been
added to the database, various pieces of it have been split out into separate files, but the most
important parts of the standard are still in UnicodeData.txt. 

\section{Code Point Ranges}

A range of code points is specified by the form |"X..Y"|.
Each code point in a range has the associated property value specified on a data file. For example (from \docFile{Blocks.txt}),


\begin{dispListing}
0000..007F; Basic Latin
0080..00FF; Latin-1 Supplement
\end{dispListing}

Block ranges are different from Scripts which are defined in the \docFile{Scripts.txt} file. A block range is defined by a range of hexadecimal codepoints. 

All block ranges start with a value where |(cp MOD 16) = 0|,
  and end with a value where |(cp MOD 16) = 15|. In other words,
  the last hexadecimal digit of the start of range is |...0|
  and the last hexadecimal digit of the end of range is |...F.|
  This constraint on block ranges guarantees that allocations
  are done in terms of whole columns, and that code chart display
  never involves splitting columns in the charts.

  All code points not explicitly listed for Block
  have the value |No_Block|.

The advantage for providing the database in specially formatted text files, is that they can be parsed into any computer language easily. In our case I have parsed the files both using Lua, as well as modified variants using \latexe. 
The only frustration is to make sure the files are somewhere where |texlua| can find them. 

The list of all the blocks can be obtained and typeset using the |phdlua| modules and is shown next.

{\parindent0pt
\leavevmode\luadirect{dofile("./i18n/parseunicodeblocks.lua")}
}

We can use the same module to determine the current total unicode points and blocks defined by UNICODE.




%
\chapter{GROUPING AND SCOPING RULES}
\index{Grouping}
\label{ch:grouping}

Like most computer languages \tex\ has a scoping mechanism that is able to confine most changes to a particular locality. This chapter explains what sort of actions can be local, and how groups are formed.
\medskip

\begin{docCommand}{bgroup}{}
Implicit beginning of group character.
\end{docCommand}

\begin{docCommand}{egroup}{}
 Implicit end of group character.
 \end{docCommand}

\begin{docCommand}{begingroup}{}
 Open a group that must be closed with |\endgroup|.
\end{docCommand}

\begin{docCommand}{endgroup}{} 
Close a group that was opened with |\begingroup|.
\end{docCommand}

\begin{docCommand}{aftergroup}{} 
Save the next token for insertion after the current group ends.
\end{docCommand}

\begin{docCommand}{global}{}
 Make assignments, macro definitions, and arithmetic global.
\end{docCommand} 

\begin{docCommand}{globaldefs}{}
 Parameter for overriding |\global| prefixes. IniTEX default: 0.
\end{docCommand}



The grouping mechanism can be thought of a bit like scope in other programming languages, with the
exception that in \tex the mechanism is much more Pascal-like. Most assignments made inside a group are local to that group
unless explicitly indicated otherwise, and outside the group old values are restored (pretty much like in Pascal). 

The most common way to group a portion of your program is to use braces. If we type the following  example:

\begin{texexample}{}{}
\def\i{42} 

{
  \def\i{43}
  \def\b{2}
}

The value of the \textbackslash i is now \i

\def\x{a}
\let\y\x
\bgroup
  \def\x{b}
  Within group \x\par
\egroup
  Outside group \x
\end{texexample}
We get   \texttt{The value of the \textbackslash i is now 42}. Due to the way \tex scoping rules work, the old program state
will be restored \textit{completely} after returning from the local group. Neither the change to |\i| nor the definition of |\b| will survive. This is also true for register changes or other assignments.



\section{Local and global assignments}

An assignment or macro definition is usually made global by prefixing it with \cs{global}, but nonzero
values of the integer parameter |globaldefs| override |doccmd{global}|
is positive every assignment is implicitly prefixed with \docAuxCommand{global}, and if |\globaldefs| is negative,
|\global| is ignored. Ordinarily this parameter is zero. It has very
limited use and even in the \latex\ kernel we can only find 3-4 uses when defining math fonts.\footnote{In file \texttt{ltfssbas.dtx}.}


Some assignment are always global: the \marg{global} assignments are:

\begin{description}
\item[font assignment] assignments involving \cs{fontdimen}, \cs{hyphenchar}, and \cs{skewchar}.

\item[hyphenation] assignment \cs{hyphenation} and \cs{patterns} commands.

\item[hbox size assignment] altering box dimensions with \cs{ht}, \cs{dp}, and \cs{wd} 

\item[interaction mode assignment] run modes for a \tex job.

\item[intimate assignment] assignments to a special integer or special dimen
\end{description}

\section{Braces}

The most common way to group is to use braces. They are used for two purposes:

\begin{enumerate}
\item to indicate the start and end of a group. For example |{\small here is some text}|.

\item to indicate that a string of tokens should be treated as one unit. For example in |\def\abc{...}| the braces are used
to delimit the argument.
\end{enumerate}

It is important to note that the characters `\{', `\}' are not hardwired in \tex. Any tokens with catcodes 1 and 2 can be used.
The plain format starts [343] by defining:

\begin{teX}
\catcode`\{ =1
\catcode `} = 2
\end{teX}

Tokens with catcodes 1 and 2 are called \emph{explicit braces}. An \emph{implicit} brace is a control sequence whose replacement text is an explicit brace. Thus the two |plain| control sequences 
|\bgroup| and |\egroup| are implicit braces. 

There is also a low-level \tex operator pair for creating groups. It works
just as the braces. A group is started with \cs{begingroup} and ended with
\cs{endgroup}. These operators may be freely mixed with braces but pairs
should be properly matched. So |{ \begingroup \endgroup }| is allowed
but |{ \begingroup } \endgroup| is not.

\begin{teX}
\let\bgroup={
let\egroup=}
\end{teX}

They can be used where unbalanced braces are needed.

Salomon gives an example to typeset a number of paragraphs with a negative indentation\footnote{This style can sometimes be found in old books.}:

\begin{teX}
\def\negIndent{\brgoup\parindent=-20pt}
\def\endIndent{\par\egroup}

\negIndent
  \small\lipsum[1]
\endIndent
\end{teX}

This will typeset:

\def\beginindent{\bgroup\parindent=-20pt}
\def\endindent{\par\egroup}

\beginindent
  \small\lipsum[1-3]
\endindent

\section{Forming Groups Using \textbackslash begingroup and \textbackslash endgroup} 

The other two primitives \docAuxCommand{begingroup} and \docAuxCommand{endgroup} can also be used to define a group. However a group that starts with a |\begingroup| must end with an |\endgroup|. This provides a mechanism for error checking, which \tex's parsing routines can easily catch.

Note that |\begingroup| and |\endgroup| can only be used to define a group, not to delimit a string. You can say:

\begin{teX}
\begingroup
  \it abc
\endgroup
\end{teX}

but the following will get \tex to complain about missing braces

\begin{teX}
\hbox\begingroup\it abc\endgroup
\end{teX}

It should be pointed out that |\begingroup| and |\endgroup| do not really
add any new grouping functionality that could not be provided by curly braces
or |\bgroup| and |\egroup|. On the other hand, these two instructions are very
useful in nested groups of complicated structures, where one wants to make sure
that a certain "begin group instruction" is matched by a certain "end group
instruction." For this pair of grouping instructions, and this pair only, use |\begingroup|
and |\endgroup|. In case a |\begingroup| is not matched by a |\endgroup|,
an error is generated by \tex.\footcite{bechto1993} 

The case when not to use |begingroup| is clear. However, if one should use it for cases where
|\bgroup| is possible, is a subject with different opinions.\footnote{See \url{https://tex.stackexchange.com/questions/1930/when-should-one-use-begingroup-instead-of-bgroup/1932\#1932}.} Unless you are using |mathmode| or have deeply nested structures, |bgroup| is fine to use. In all
other cases it is preferable to use |\begingroup|.

\section*{Examples}
From the TexBook Exercise 7.4

Suppose that the commands
\begin{texexample}{}{}
{\catcode`\<=1 \catcode`\>=2
 \bfseries test
>
 test
\end{texexample}

appear near the beginning of a group that begins with |{| these specifications instruct
TEX to treat |<| and |>| as group delimiters. According to \tex's rules of locality, the
characters |<| and |>| will revert to their previous categories when the group ends. But
should the group end with |}| or with |>| ?

It ends with either |>| or |}| or any character of category 2; then the effects of all
\cs{catcode} definitions within the group are wiped out, except those that were global.
\tex  doesn't have any built-in knowledge about how to pair up particular kinds of
grouping characters. New category codes take effect as soon as a |\catcode| assignment
has been digested. For example,

\begin{teX}
{\catcode`\>=2 >
\end{teX}

is a complete group. But without the space after |2|  it would not be complete, since TEX
would have read the |>|  and converted it to a token before knowing what category code
was being specified; \tex always reads the token following a constant before evaluating
that constant.

\topline

\textbf{Example}: \textsc{Adjusting the spacing of a font} An interesting example that illustrates some of the concepts that were discussed so far is to try and change the \textit{inter word spacing} of text using the \cs{fontdimen2} parameter. The interesting aspect of this example is that
we want to change the spacing, but since the font changes are global, we want to revert back to the original font at the end of the group. Although there are many other ways of achieving this we will use the \cs{aftergroup}.

\begin{teX}
\font \roman=cmr10
\font\specroman=cmr10
%% Next, the special registers
\newdimen\savedvalue
\savedvalue=\fontdimen2\roman
\newdimen\specialvalue
\specialvalue=13.0pt
%% Finally, definitions.
\def \rm{%
  \fontdimen2\roman=\savedvalue }
\def\specrm{%
  \aftergroup\restoredimen
  \fontdimen2\specroman=\specialvalue
  \specroman  }
\def\restoredimen{%
\fontdimen2\roman=\savedvalue }
\end{teX}
{
%% First, fonts.
\font \roman=cmr10
\font\specroman=cmr10
%% Next, the special registers
\newdimen\savedvalue
\savedvalue=\fontdimen2\roman
\newdimen\specialvalue
\specialvalue=13.0pt
%% Finally, definitions.
\def \rm{%
  \fontdimen2\roman=\savedvalue }
\def\specrm{%
  \aftergroup\restoredimen
  \fontdimen2\specroman=\specialvalue
  \specroman  }
\def\restoredimen{%
\fontdimen2\roman=\savedvalue }


{\bf Spaced Out Text} 
\medskip
{\specrm \lorem} dimension2 the interword   value \the\fontdimen2\font


{\bf  Back to Normal}
\medskip

\rm
\lorem

}

\section{\textbackslash aftergroup}

The \cs{aftergroup} control sequence saves a token for insertion after the current group. Several
tokens can be set aside by this command, and they are inserted in the left-to-right order in which
they were stated.

\begin{texexample}{}{}
\def\x#1;{#1}
\def\y{15}
{\globaldefs1
\bgroup
   \def\y{0}
   \aftergroup\x\aftergroup\y\aftergroup;
   \aftergroup}
\egroup
\y


\globaldefs0

\def\z{1}
{\def\z{0}
\z
}

\z

\end{texexample}

\begin{texexample}{}{}
{ \def\z{1}
  {\def\z{0}\globaldefs1
     \z
    {
	\z
    }
   \z
  }
 \z
}
\end{texexample}
\section{afterassignment}

An interesting primitive is \docAuxCommand{afterassignment}. The primitive saves the token immediately following it without
expansion. Nothing happens until after the next assignment; immediately after the next assignment the saved token is expanded.

\begin{texexample}{Aftergroup}{ex:aftergroup}
\def\yy{%
  \afterassignment\yyb
  \let\yyDiscard = 
}

\def\yyb{%
 ``%
 \bgroup
 \itshape
 \aftergroup\yyc
}
\def\yyc{%
  ''%
}

\yy{This is a test}  
\end{texexample}

The above example is not a very common or idiomatic way of writing macros. So what is |\afterassignment| good for? Its main use is to write macros with \enquote{arguments} similar to the way \tex assigns registers. Afterassignment allow you to define macros which avoid curly braces to enclose arguments.

The most common use of |\afterassignment| is in a macro whose parameter is glue or dimen. Consider the definition of a macro such as:
\begin{quote}
 |\def\myglue#1{\leftskip=#1 \rightskip=#1}|
\end{quote}

Such a macro can be called as |\myglue{3pt plus5pt minus3pt}|, but if we want to keep the same conventions as \tex we might prefer to have the ability to call it as |\myglue 3pt plus5pt minus3pt|. To achieve this we can do:

\begin{texexample}{Afterassignment}{ex:afterassignment}
\bgroup
\font\larger=cmr10 scaled\magstep1
\larger
\newskip\tempskip
\def\myglue{\afterassignment\myglueaux \tempskip}
\def\myglueaux{\leftskip=\tempskip \rightskip=\tempskip}
\myglue=30pt plus1pt minus1pt
\lorem\par
\egroup
\lorem
\end{texexample}



\section{Scoping Rules for boxes}

The scoping rules for boxes work similarly to those for other command sequences, since they are just macros defined by \latex or |plain|. In the example below, we define a box |\mybox| and we save a sentence both in global scope as well as local scope.

\begin{teX}
\documentclass{article}
\begin{document}
  \newsavebox{\mybox}
  \savebox{\mybox}{Outside scope}
  \usebox\mybox
  \begin{minipage}{5cm}
    \sbox{\mybox}{from first minipage}(*@ \label{global} @*)
    \usebox\mybox
  \end{minipage}
  \usebox{\mybox}
\end{document}
\end{teX}


This will typeset:
\medskip

\newsavebox{\myboxi}
\savebox{\myboxi}{\tt > Outside scope}

\noindent\usebox\myboxi

\noindent\begin{minipage}{5cm}
\sbox{\myboxi}{\tt > from first minipage}
\noindent\usebox\myboxi
\end{minipage}

\noindent\usebox{\myboxi}


\medskip 
Changing line [\ref{global}] to |\global\sbox| will make the definition of |\mybox| within the minipage environment global and would change the output to:
\medskip


To save memory space, box registers become empty by using them: \tex assumes
that after you have inserted a box by calling |\boxnn| in some mode, you do not need the contents of that register any more and empties it. In case you do need the contents of a box register more
than once, you can |\copy| it. Calling |\copynn| is equivalent to |\boxnn| in all respects except that the register is not cleared.


There are 256 box registers, numbered 0–255. Either a box register is empty (‘void’), or it contains
a horizontal or vertical box. This section discusses specifically box registers; the sizes of boxes,
and the way material is arranged inside them, is treated below.




\newbox\MyBox

\setbox\MyBox=\hbox{\hfil Test\hfill}

\unhbox\MyBox


\noindent\unhbox\MyBox

\noindent{\hfill Test \hfill}



\framebox{\parbox{\linewidth}{\color{theblue}
\textbf{\textcolor{purple}{\textsf{CAUTION}}}
\begin{enumerate}
\itemsep-5pt
\item \latex will not empty a box as it uses the \cs{copy} command in the definition of the \cs{newsavebox}.
\item It is better to use \LaTeX\ commands rather than \tex primitives, when defining boxes, as \latex tests for duplication of names - which is very important if a user uses a lot of different packages.
\item Give always preferences to local definitions rather than global. Globals always create maintenance problems in programming.
\end{enumerate}
}}


\section{Implicit Grouping}

There are  instances where grouping is \textit{implicit}. What this means is that \text starts and ends a group automatically and without any action by the user. There are two major cases where this happens:

\begin{enumerate}
\item The text inside a box such as |\hbox|, |\vbox|, |\vtop|, |\vcenter| etc. is automatically treated by \tex as a group.  For example |\hbox{\bf My Heading}|, will print  \hbox{\bf My Heading}  and it will not continue with the bold font once outside the group. All these commands have curly brackets and these curly brackets form implicit groups.
\item In five cases \tex forms implicit groups. In some of these cases not even curly braces are involved.
\end{enumerate}

\begin{enumerate}
\item The text inside math mode is treated as a group. This is true both for inline math as well as display math.
\item Matching |\left| and |\right| primitives treat the formula in between them as a group.
\item Fractions are treated as a group.
\item The execution of an ouput routine is implicitly enclosed in a group.
\item Columns in |\halign| based tables are local.
\end{enumerate} 

\subsection{\texttt{afterssignment and grouping}}

\begin{macro}{\afterassignment}
The primitive |\afterasignment| does not follow grouping in that it does not save the definition of a token when |\afterassignment| is executed. Consider the following example:
\end{macro}

Define the two macros |\xx| and |\yy|.

\begin{texexample}{afterassignment}{}
\def\xx{\string\xx\ executed\par }

\def\yy{\string\yy\ executed\par }

\afterassignment\xx
\end{texexample}

We start a group, where we have two definitions of |\xx| and |\yy|

\begin{texexample}{afterassignment}{}
\def\yy{42}
{
  \def\xx{\string\xx executed inside a group\par}

  \def\yy{\string\yy executed inside a group\par}

The second afterassignment is execute

  \afterassignment\yy

The group is ended

}
\end{texexample}

Note \cs{afterassignment} saves the token following \cs{afterassignment} without expanding it. Nothing happens until after the next assignment; immediately after the next assignment the saved token is expanded. This is a bit of a tricky part and you can go over it to make sure you understand it well.
\footnote{\url{http://tug.org/TUGboat/tb32-2/tb101grunewald.pdf}}
\footnote{\url{http://tex.stackexchange.com/questions/65462/plain-tex-theory-afterassignment}}


\begin{texexample}{Combining bgroup and begingroup}{}
\begingroup
\newbox\savedparbox

\def\saveparbox{\par\begingroup
  \def\par{\egroup\endgroup}
  \global\setbox\savedparbox\vbox\bgroup}

Ordinary paragraph.
\saveparbox
This paragraph will be saved in \string\box\string\savedparbox.
If you wish, you can unpack the box and do all kinds of processing on it.
In this demo, I won't do any processing.
Look in the log file to examine the box contents.

Another ordinary paragraph.
\endgroup
\end{texexample}


































%\chapter{A more flexible and robust method of defining functions with LaTeX3 and xparse}
\label{ch:xparse}

\section{Introduction}

The \latex2e |\newcommand| macro is the most popular user command for creating macros. The command
provides a number of checks and also has the ability to define macros with an optional argument.
For more complex macros, users have to revert to using |\def| or use packages which extend |\newcommand|,
such as \pkg{twoopt}.\footcite{twoopt}

The \LaTeX3 Team developed the package \pkg{xparse} to provide document level 
authors with some powerful commands that extend those such as \cs{newcommand}
of \latexe. The code is been stable and the interface is not expected to change. 
Although targetted at document level, the commands offered can be used effectively to produce code used in packages.\footnote{\protect\url{http://tex.stackexchange.com/questions/98152/always-use-newdocumentcommand-instead-of-newcommand}} The functions offered by the package enable commands with star, or optional arguments to be produced easily.\tcbdocmarginnote{Revised\\ July 2018} A good introduction to the package was published in TUGboat by Joseph Wright.\footcite{wright2010} 


\section{Creating document commands}

\begin{docCommand}{DeclareDocumentCommand}{\marg{function}\marg{argument specification}\marg{code}}
This family of commands are used to create a document-level \emph{function}. The argument
specification for the function is given by \textit{arg spec}, and expanding to be replaced by the
\textit{code}. Unlike \latex's definition commands, all xparse commands take two arguments.
The first one is the \textit{argument specifier}, and the second is the \textit{code.}
\end{docCommand}

\begin{texexample}{DeclareDocumentCommand}{l3:1}
\DeclareDocumentCommand \foo { m o m } { 
    arg 1 = #1, arg 2 = #2,  arg 3 = #3 }
\foo{A}[B]{C}  

\foo{A}{B}    
\end{texexample}

In the example above |{m o m}| is the argument specifier. It tells the function  to expect, two mandatory arguments and one optional denoted by the letter \textbf{o}. There are many more specifiers. For example \textbf{O} takes an parameter as a default value.\index{argument specifier}

\begin{texexample}{DeclareDocumentCommand}{l3:1}
\DeclareDocumentCommand \foo { m O{\ldots} m } { 
    arg 1 = #1, arg 2 = #2,  arg 3 = #3 }
\foo{A}[B]{C}  

\foo{A}{B}    
\end{texexample}

The argument markers can be entered in any order. In the following example we will also add an optional argument in a curly bracket. Although this is frowned upon in certain contexts it is useful. Consider the case of a chapter title that also has a subtitle. \docAuxCommand*{Chapter}, it maybe more natural and useful to have input of the form, as shown in Example~\ref{l3:g}. 

\begin{texexample}{DeclareDocumentCommand}{l3:g}
\DeclareDocumentCommand \MyChapter { o m g } { 
\centering #2\par #3\par }
    
\MyChapter{THIS IS THE MAIN TITLE}{This is a subtitle}  

\MyChapter[]{THIS IS THE MAIN TITLE}{This is a subtitle}        
\end{texexample}

\begin{texexample}{DeclareDocumentCommand}{l4:g}
\DeclareDocumentCommand \MyChapter {s o m g } { 
\IfBooleanTF {#1} {\gdef\fonta{\bfseries\selectfont}}{\gdef\fonta{}}
\IfNoValueTF {#2} {No option\par}{#2}

\centering {\fonta #3}\par #4\par 
  }   
\MyChapter{ THIS IS THE MAIN TITLE}  

\MyChapter*[short title]{THIS IS THE MAIN TITLE}{This is a subtitle}        
\end{texexample}

As you can see, it is fairly easy to produce starred and unstarred versions of commands as well as as any form of optional arguments. Let us now see some of the other command definition functions, before we continue with other specifiers.

\begin{docCommand}{NewDocumentCommand}{\marg{function}\marg{argument specification}\marg{code}}
will issue an error if \meta{function} has already been defined
\end{docCommand}

\begin{docCommand}{RenewDocumentCommand}{\marg{function}\marg{argument specification}\marg{code}}
For changing a definition,
issuing an error message if the macro does
not already exist.
\end{docCommand}


As the \cmd{\DeclareDocumentCommand} always updates a definition, it is used for the examples in this chapter to avoid any errors.

What sets the above commands apart from \latexe \cmd{\newcommand} is the argument specification.



\begin{texexample}{DeclareDocumentCommand}{l3:1}
\DeclareDocumentCommand \teststar {s o m } { 
\IfBooleanTF {#1}
  { \typesetnormalchapter {#2} {#3} }
  { \typesetstarchapter {#3} }
}  
\newcommand\typesetnormalchapter[2][]{
  normal chapter
}
\newcommand\typesetstarchapter[1]{
  #1
}
\teststar{Test}

\teststar*{test}
\end{texexample}    

    
The argument specification \textbf{m o m} in the example enables the function to accept three arguments, two mandatory and one optional. 


\section{Argument specifications}

The basic idea of an argument specification is that each argument is listed as a single letter. 
As the argument specification is a mandatory argument, a function with no arguments still needs an arg spec. The number of letters in the argument specification tells you how many arguments a function takes, while the letters themeselves determine the type of argument. Unlike |\newcommand| or |\def| a function with no arguments still needs to be specified with an empty
argument specification.

\begin{texexample}{Empty arg spec}{}
\DeclareDocumentCommand\atest{}{some text}
\atest
\end{texexample}

There is a wide range of argument specifier letters. Mandatory arguments are created using the letter \textbf{m}.

\begin{marglist}
\item [m] Mandatory. This is a standard mandatory argument, which can either be a single token alone or multiple tokens surrounded by curly braces. Regardless of the input, the argument will
be passed to the internal code surrounded by a brace pair. This is the \pkgname{xparse} type
specifier for a normal \tex argument.
\end{marglist}

\begin{texexample}{Mandatory Values, verbatim}{}
\DeclareDocumentCommand\testverbatim{ v }{
    \ttfamily#1
}
\testverbatim+ \this is a test +

\testverbatim * &^%$#\test *

\testverbatim{\ttfamily \bfseries\normalfont test}
\end{texexample}

The \textbf{l} specifier reads its argument, until it encounters a left brace. It is equivalent to \tex \# argument. Can be used basically for |\hbox| type comands.

\begin{texexample}{Mandatory arguments l-specifier}{}
\DeclareDocumentCommand\myhbox{ l }{
   \hbox to \dimexpr(#1)\relax
}
\fbox{\myhbox 12pt+1em+13ex  {test}}
\end{texexample}

One difference between |xparse| functions and |\newcommand| is that the functions defined are not |\long|. In |xparse| the argument specidfier can b used to allow paragraph tokens or not. This is done by preceding the arg spec letter by |+|:

\begin{texexample}{Long macros with xparse}{ex:xparse1}
\DeclareDocumentCommand \mylorem {m +m }{%
  #1
  #2
}

\mylorem{\hrule}{\lorem\par}
\end{texexample}

If you expect longer texts, it is always a good idea to make the macros |long|. 

\begin{marglist}
\item [o] Optional argument in  []. Returns |-NoValue-| if not present.
\item [O] As for \textbf{o}, but returns \meta{default}, if no value is given. Should be given as |O{default}|.

\item [s] Starred version
\item [v] Verbatim. Reads an argument “verbatim”, between the following character and its next occurrence,
in a way similar to the argument of the LATEX2" command \cmd{\verb}. Thus
a v-type argument is read between two matching tokens, which cannot be any of
\%, \#, \{, \}, \^ or  . The verbatim argument can also be enclosed between braces,
\{ and \}. A command with a verbatim argument will not work when it appears
within an argument of another function.
\item [l] An argument which reads everything up to the first open group token: in standard
\latex this is a left brace.
\item [u] Reads an argument “until \meta{tokens} are encountered, where the desired \meta{tokens}
are given as an argument to the specifier: |u|\meta{tokens}.
\item [d] An optional argument that is delimited. 
\item [D] As for d, but returns \meta{default} if no value is given: D\meta{token1} \meta{token2}\marg{default}.
Internally, the o, d and O types are short-cuts to an appropriated-constructed D
type argument.
\item [t]  An optional \meta{token}, which will result in a value \cs{BooleanTrue} if \meta{token} is 
            present and \cs{BooleanFalse} otherwise. Given as \meta{token}.
\item [g] An optional argument given inside a pair of \tex group tokens (in standard \latex,
              \{ . . . \}, which returns |-NoValue-| if not present.
\item [G] As  for \textbf{g} but returns \meta{default} if no value is given: |G|\marg{default}.
\end{marglist}

\begin{texexample}{Default Values}{}
\DeclareDocumentCommand\testcolor{ O{red} m }{
    \textcolor{#1}{#2}
}
\testcolor{This is typeset in red}
\testcolor[blue]{This is typeset in blue.}
\end{texexample}

\section{Testing special values}

The optional arguments of a function defined using |xparse| use dedicated variables to return
information about the naure of the argument received.

\begin{docCommand}{IfNoValueTF}{\marg{argument}\marg{true code}\marg{false code}}
The function tests if the argument has a value and executes the true of false code, by means
of a |-NoValue-| marker. 
\begin{texexample}{special values}{}
\DeclareDocumentCommand\doccmd{O{red} m}
    {
        \IfNoValueTF{#1}
            {\doccmdnocolor{#1}}
            {\doccmdcolor{\textcolor{#1}{#2}}}
     }
\newcommand\doccmdnocolor[1]{#1}
\newcommand\doccmdcolor[2]{#1 #2}     
This is \doccmd[blue]{text}  and this is \doccmd{text}.   
\end{texexample}
\end{docCommand}

\begin{texexample}{special values}{}
\DeclareDocumentEnvironment{allbold}{o}
    {
        \bfseries 
        \IfNoValueTF{#1}
            {\color{red}}
            {\color{#1}}
    }
    {                 }
\begin{allbold}[magenta]
\lorem
\end{allbold}
\end{texexample}

\begin{texexample}{variants}{ex:variants}
\ExplSyntaxOn
\cs_set:Npn \foo_something:Nn #1#2 {
   \csname\expandafter#1\endcsname{blue}{a a a} 
   { #2}
  }
\cs_generate_variant:Nn \foo_something:Nn { c }
%\meaning\foo_something:cn
\ExplSyntaxOff
\lorem
\end{texexample}

\section{Robustness}

|xparse| craetes functions which are naturally \enquote{robust}. This means that they can be used in section names
and so on without needing to be protected using
|\protect|. This makes using functions created using
xparse much more reliable than using those created
using |\newcommand|, particularly when there are optional
arguments.
xparse is also designed so that optional arguments
can themselves contain optional material. For
example, if you try:

\begin{teXXX}
\newcommand*\foo[2][]{%
% Code
}
\foo[\baz[arg1]{arg2}]{arg3}
\end{teXXX}

you will find that |\foo| will pick up ‘|\baz[arg1|’ as
\#1 and ‘arg2’ as \#2: not what is intended. However,
the same code with xparse

\begin{teXXX}
\DeclareDocumentCommand \foo { o m } {%
% Code
}
\foo[\baz[arg1]{arg2}]{arg3}
\end{teXXX}
will parse ‘|\baz[arg1]{}|’ as \#1 and ‘arg’ as \#2, as
anticipated.\footcite{wright2010}







%
\@specialtrue
\cxset{steward,
  numbering=arabic,
  custom=stewart,
  offsety=0cm,
  image=hine03,
  texti={When Lamport designed the original \LaTeX\ sectioning commands, limitations of computer power forced him to restrict the abstraction of complicated chapter layouts. With current tools available improvements are much easier to program.},
%
  textii={In this chapter we discuss a method that allows the production of fancy sectionr headings and formatting, based on a set of key values. Central  to this process is the separation of content from presentation.
We also discuss the basic formatting tools that are available and how one can modify them to mould new book designs.
 }
}



\raggedbottom

\chapter{Lower Level Headings}
\@specialfalse

\section{Introduction}

Good book design dictates that sectioning styles follow that of the general book design and theme. An academic publication for example might have chapters and section numbered in arabic numerals, whereas a high school textbook might have sections marked in colored boxes.

Similarly to the chapter key value interface, the package offers a key value interface to adjust sectioning command parameters.



\cxset{section beforeskip={10pt},
      section indent=0pt}
\cxset{section afterskip={10pt}}
\renewsection

\section{Section styling}

In a similar fashion to the chapter commands the following keys are provided.

\subsection{Fonts and numerals}

Font and numeral keys are shown below.
\medskip

  \keyval{section font-size}{\marg{cmd}}{Font size command such as \cs{large.}}
  \keyval{section font-weight}{\marg{cmd}}{Font weight command such as \cs{bfseries.}}
  \keyval{section font-family}{\marg{cmd}}{Font family command such as \cs{sffamily.}}
  \keyval{section font-shape}{\marg{cmd}}{Font shape command such as \cs{itshape}}
  \keyval{section color}{\marg{color}}{Color of section.}
  \keyval{section numbering}{\marg{arabic|roman|Roman|alph|Alph|words|WORDS}}{Section number style.}
  \begin{marglist}
  \item [arabic] Typesers the section number in arabic numerals.
  \item [roman] Typesets the section number in lowercase roman numerals.
  \item [Roman] Typesets the section number in uppercase roman numerals.
  \item [alph] Typesets the section number in lowercase alphabetic numbering.
  \item [Alph] Typesets the section number in uppercase alphabetic numerals.
  \item [words] Typesets the numbers in words (lowercase).
  \item [WORDS] Typesets the number in words (uppercase).
  \end{marglist}

\subsection{Skip and indentation commands}

The keys for indentaion and above and below skips are shown below.
\medskip

\keyval{section beforeskip}{}{}
\keyval{section afterskip}{}{}
\keyval{section indent}{\marg{dim}}{Indentation from margin as per standard LaTeX class definitions.}
\keyval{section spaceout}{}{}
\begin{marglist}
 \item[soul]
 \item[none]
\end{marglist}

\subsection{align}

\keyval{section align}{\marg{cmd}}{One of the alignment commands centering, ragged right, raggedleft}

\subsection{Hooks}

Hooks for adding material are shown in the following sketch.
\medskip

\fbox{aboveskip}

\fbox{indent} \fbox{number}\fbox{hook}\fbox{title}

\fbox{belowskip}

%\lipsum

\section{Example usage}

\cxset{
 chapter toc=false,
 name=CHAPTER,
 numbering=arabic,
 number font-size=\huge,
 number font-family=\sffamily,
 number font-weight=\bfseries,
 number before=,
 number dot=,
 number after=\hspace{1em},
 number position=rightname,
 chapter opening=anywhere,
 chapter font-family=\sffamily,
 chapter font-weight=\bfseries,
 chapter font-size=\huge,
 chapter before={\vspace*{0.1\textheight}\hfill},
 chapter after={\hfill\hfill\vskip0pt\thinrule\par},
 chapter color={black!90},
 number color=\color{black!90},
 title beforeskip={\vspace*{30pt}},
 title afterskip={\vspace*{30pt}\par},
 title before={\hfill},
 title after={\hfill\hfill},
 title font-family=\sffamily,
 title font-color=\color{black!90},
 title font-weight=\bfseries,
 title font-size=\huge,
%%%%%%%%%% Sections
 section font-size=\LARGE,
 section font-weight=\normalfont,
 section font-family=\sffamily,
 section align=\centering,
 section numbering=arabic,
 section indent=0em,
 section align=\centering,
 section beforeskip=20pt,
 section afterskip=10pt,
 section spaceout=soul,
 section font-shape=\itshape,
}
\cxset{book/.style={
 section numbering=arabic,
 section font-size=\Large,
 section font-weight=\bfseries,
 section font-family=\rmfamily,
 section font-shape=\normalfont,
 section align=\raggedright,
 %section numbering custom=\color{gray}{Section} (\thechapter-\@arabic\c@section),
 subsection font-size=\large
 section indent=0em,
 section beforeskip=-3.5ex \@plus -1ex\@minus -0.2ex,
 section afterskip=2.3ex\@plus.2ex,
 subsection beforeskip=-3.5ex \@plus -1ex\@minus -0.2ex,
 subsection afterskip= 1.5ex \@plus .2ex,
}}


\begin{example}{Adjusting section parameters}{}
\cxset{ section font-size=\LARGE,
 section font-weight=\normalfont,
 section font-family=\sffamily,
 section align=\centering,
 section numbering=(roman),
 section indent=0em,
 section align=\centering,
 section beforeskip=20pt,
 section afterskip=10pt,}
\chapter{A First Look at the Sectioning Keys}
\section{First section}
\lorem
\end{example}

One notable thing to keep in mind is that the numbering of the chapter is independent of that for the section, so if you need to have strange combinations rather define a section numbering custom.\index{section formatting!vertical space}

\cxset{section numbering=arabic}
\subsection{Adjusting vertical spaces}

Perhaps the most important issues we need to consider is the adjusting of vertical spaces; example~\ref{ex:latex}, that follows illustrates settings from the Octavo class and compare them with those of standard the \LaTeXe\ book class. The Octavo class through settings that are based on baselineskip fractions and multiples endeavours to achieve a grid layout. The class also tones down the `loudness' of some of the headings compared to those of the book class.


\cxset{octavo/.style={
 section font-size=\large,
 section font-weight=\normalfont,
 section font-family=\rmfamily,
 section font-shape=\scshape,
 section indent=0em,
 section align=\centering,
 section beforeskip=-1.666\baselineskip\@minus -2\p@,
 section afterskip=0.835\baselineskip \@minus 2\p@,
 subsection numbering=none,
 subsection font-family=\rmfamily,
 subsection font-size=\normalfont,
 subsection font-shape=\scshape,
 subsection font-weight=\normalfont,
 subsection indent=1em,
 subsection align=\raggedright,
 subsection beforeskip=-0.666\baselineskip\@minus -2\p@,
 subsection afterskip=0.333\baselineskip \@minus 2\p@
 }}




\cxset{book/.style={
 section numbering=arabic,
 section font-size=\Large,
 section font-weight=\bfseries,
 section font-family=\rmfamily,
 section font-shape=\normalfont,
 section align=\raggedright,
 %section numbering custom=\color{gray}{Section} (\thechapter-\@arabic\c@section),
 subsection font-size=\large,
 section indent=0em,
 section beforeskip=-3.5ex \@plus -1ex\@minus -0.2ex,
 section afterskip=2.3ex\@plus.2ex,
 subsection font-size=\large,
 subsection font-weight=\bfseries,
 subsection numbering=arabic,
 subsection indent=0pt,
 subsection beforeskip=-3.5ex \@plus -1ex\@minus -0.2ex,
 subsection afterskip= 1.5ex \@plus .2ex,
}}

\cxset{octavo headings/.style={%
 section numbering=none,section font-size=\large,section font-weight=\normalfont,
 section font-family=\rmfamily, section font-shape=\scshape,
 section indent=0em, section align=\centering, section beforeskip=-1.666\baselineskip\@minus -2\p@,
 section afterskip=0.835\baselineskip \@minus 2\p@, subsection numbering=none,
 subsection font-family=\rmfamily, subsection font-size=\normalfont, subsection font-shape=\scshape,
 subsection font-weight=\normalfont, subsection indent=1em, subsection align=\raggedright,
 subsection beforeskip=-0.666\baselineskip\@minus -2\p@,
 subsection afterskip=0.333\baselineskip \@minus 2\p@,
 subsubsection numbering=none,
 subsubsection font-family=\rmfamily,
 subsubsection font-size=\normalfont,
 subsubsection font-shape=\itshape,
 subsubsection font-weight=\normalfont,
 subsubsection indent=1em,
 subsubsection align=\raggedright,
 subsubsection beforeskip=-0.666\baselineskip\@minus -2\p@,
 subsubsection afterskip=0.333\baselineskip \@minus 2\p@,
 paragraph numbering=none,
 paragraph font-family=\rmfamily,
 paragraph font-size=\normalfont,
 paragraph font-shape=\normalfont,
 paragraph font-weight=\normalfont,
 paragraph indent=-1em,
 paragraph align=\raggedright,
 paragraph beforeskip=\z@,
 paragraph afterskip=0\p@,
% subparagraph numbering=none,
% subparagraph font-family=\rmfamily,
% subparagraph font-size=\normalfont,
% subparagraph font-shape=\normalfont,
% subparagraph font-weight=\normalfont,
% subparagraph indent=0em,
% subparagraph align=\raggedright,
% subparagraph beforeskip=\z@,
% subparagraph afterskip=0\p@,
}}
\cxset{octavo headings}
\renewsection\renewsubsection\renewsubsubsection\renewparagraph

\begin{example}{Octavo class headings, settings}{}
\cxset{octavo headings/.style={%
 section numbering=none,section font-size=\large,section font-weight=\normalfont,
 section font-family=\rmfamily, section font-shape=\scshape,
 section indent=0em, section align=\centering, section beforeskip=-1.666\baselineskip\@minus -2\p@,
 section afterskip=0.835\baselineskip \@minus 2\p@, subsection numbering=none,
 subsection font-family=\rmfamily, subsection font-size=\normalfont, subsection font-shape=\scshape,
 subsection font-weight=\normalfont, subsection indent=1em, subsection align=\raggedright,
 subsection beforeskip=-0.666\baselineskip\@minus -2\p@,
 subsection afterskip=0.333\baselineskip \@minus 2\p@,
 subsubsection numbering=none,
 subsubsection font-family=\rmfamily,
 subsubsection font-size=\normalfont,
 subsubsection font-shape=\itshape,
 subsubsection font-weight=\normalfont,
 subsubsection indent=1em,
 subsubsection align=\raggedright,
 subsubsection beforeskip=-0.666\baselineskip\@minus -2\p@,
 subsubsection afterskip=0.333\baselineskip \@minus 2\p@,
 paragraph numbering=none,
 paragraph font-family=\rmfamily,
 paragraph font-size=\normalfont,
 paragraph font-shape=\normalfont,
 paragraph font-weight=\normalfont,
 paragraph indent=-1em,
 paragraph align=\raggedright,
 paragraph beforeskip=\z@,
 paragraph afterskip=0\p@,}}

\cxset{octavo headings}
\renewsection\renewsubsection\renewsubsubsection\renewparagraph
\section{Octavo Class Heading}
\lorem
\subsection{Octavo subsection}
This is some text short text\par
\subsubsection{Octavo sub-subsection}
\lorem
\paragraph{paragraph heading} This is some short text.
\end{example}

\begin{example}{}{}
\cxset{octavo}
\section{Octavo Class Heading}
\lorem
\subsection{Octavo subsection}
\lorem
\subsubsection{Octavo sub-subsection}
\lorem
\paragraph{paragraph heading} This is some short text.
\lorem
\paragraph{paragraph heading} This is some short text.
\lorem
\end{example}



\begin{example}{\LaTeXe\ book class headings settings}{ex:latex}
\cxset{book/.style={
 section numbering=arabic,
 section font-size=\Large,
 section font-weight=\bfseries,
 section font-family=\rmfamily,
 section font-shape=\normalfont,
 section align=\raggedright,
 %section numbering custom=\color{gray}{Section} (\thechapter-\@arabic\c@section),
 subsection font-size=\large,
 section indent=0em,
 section beforeskip=-3.5ex \@plus -1ex\@minus -0.2ex,
 section afterskip=2.3ex\@plus.2ex,
 subsection font-size=\large,
 subsection font-shape=\normalfont,
 subsection font-weight=\bfseries,
 subsection numbering=arabic,
 subsection indent=0pt,
 subsection beforeskip=-3.5ex \@plus -1ex\@minus -0.2ex,
 subsection afterskip= 1.5ex \@plus .2ex,
}}
\cxset{book}
\renewsubsection
\section{LaTeX Book  Class Heading}
\lorem
\subsection{A subsection}
\lorem
\end{example}

\section{Grid example}

One problem sometimes is that the sectioning commands create problems with grid layouts. Example~\ref{ex:grid} shows example settings.

\begin{example}{Section styles from the grid package}{ex:grid}
\cxset{grid/.style={
 section numbering=arabic,
 section font-size=\normalsize,
 section font-weight=\bfseries\mathversion{bold},
 section font-family=\rmfamily,
 section font-shape=\normalfont\bfseries\mathversion{bold},
 section beforeskip=-.999\baselineskip,
 section afterskip=0.001\baselineskip,
 section align=\raggedright,
 %section numbering custom=\color{gray}{Section} (\thechapter-\@arabic\c@section),
 subsection font-size=\normalsize,
 section indent=0em,
% section beforeskip=-3.5ex \@plus -1ex\@minus -0.2ex,
 %section afterskip=2.3ex\@plus.2ex,
 subsection font-shape=,
 subsection font-weight=\bfseries\mathversion{bold},
 subsection numbering=arabic,
 subsection indent=0pt,
 subsection beforeskip=\baselineskip,
 subsection afterskip= -.35\baselineskip,
% subsub section
 subsubsection font-shape=\itshape,
 subsubsection font-weight=\bfseries\mathversion{bold},
 subsubsection numbering=numeric,
 subsubsection indent=0pt,
 subsubsection beforeskip=\baselineskip,
 subsubsection afterskip= -.35\baselineskip,
}}
\cxset{grid}
\renewsubsection
\begin{multicols}{2}
\section{Grid  Class Heading}
\lorem
\subsection{Grid  subsection.}
\lorem
\subsubsection{A subsection grid.}
\lorem
\subsubsection{Another subsection grid.}
\lorem
\end{multicols}
\end{example}



The key \option{\bfseries section numbering custom}=\marg{code} is quite powerfull and can be used to define any type of section number style. Just remember that the numbering so far depends on two counters, the c@chapter and c@section. What the section numbering does, it redefines the macro \cs{thesection} to the new definition provided as argument for the key.

Although the temptation to define a lot of key combinations one would rather define new styles as a more user friendly approach.

\cxset{section numbering=arabic, section align=\raggedright, section font-shape=\upshape, section font-family=\rmfamily}
\section{Handling Other Section Levels}

Other sectioning commands such as \cs{subsubsection}, \cs{paragraph} and \cs{subparagraph} have equivalent keys. Examples can be found in the chapters that follow for specific styles.

\section{Technical discussion}

The standard LaTeX classes, book report and article have sections showing dot leaders, whereas in the article class the sections are shown without the dotted lines, as the l@section macro is redefined for articles.

\index{macros!\textbackslash @seccntformat}

\subsection{Indexing of Lower Section Headings}
\LaTeXe\ offers two pathways in redefining section commands, the first one is @startsection and the second is \cs{@seccntformat} \index{sectioning macros}. It also uses the macro \cs{secdef} to create the starred and unstarred versions of the sectioning commands.

\begin{tcolorbox}{}
\begin{lstlisting}
% \begin{macro}{\l@section}
%    In the article document class the entry in the table of contents
%    for sections looks much like the chapter entries for the report
%    and book document classes.
%
%    First we make sure that if a pagebreak should occur, it occurs
%    \emph{before} this entry. Also a little whitespace is added and a
%    group begun to keep changes local.
% \changes{v1.0h}{1993/12/18}{Replaced -\cs{@secpenalty} by
%    \cs{@secpenalty}.  ASAJ.}
% \changes{v1.2i}{1994/04/28}{Don't print a toc line when the tocdepth
%    counter is less than 1.}
% \changes{v1.4a}{1998/10/12}{we should use \cs{@tocrmarg}; see PR/2881.}
%    \begin{macrocode}
%<*article>
\newcommand*\l@section[2]{%
  \ifnum \c@tocdepth >\z@
    \addpenalty\@secpenalty
    \addvspace{1.0em \@plus\p@}%
%    \end{macrocode}
%
%    The macro |\numberline| requires that the width of the box that
%    holds the part number is stored in \LaTeX's scratch register
%    |\@tempdima|. Therefore we put it there. We begin a group, and
%    change some of the paragraph parameters (see also the remark at
%    \cs{l@part} regarding \cs{rightskip}).
%    \begin{macrocode}
    \setlength\@tempdima{1.5em}%
    \begingroup
      \parindent \z@ \rightskip \@pnumwidth
      \parfillskip -\@pnumwidth
%    \end{macrocode}
%    Then we leave vertical mode and switch to a bold font.
%    \begin{macrocode}
      \leavevmode \bfseries
%    \end{macrocode}
%    Because we do not use |\numberline| here, we have do some fine
%    tuning `by hand', before we can set the entry. We discourage but
%    not disallow a pagebreak immediately after a section entry.
%    \begin{macrocode}
      \advance\leftskip\@tempdima
      \hskip -\leftskip
      #1\nobreak\hfil \nobreak\hb@xt@\@pnumwidth{\hss #2}\par
    \endgroup
  \fi}
%</article>
\end{lstlisting}
\end{tcolorbox}

As you can see the dot leaders are not present in the above definition. Although we can get rid of dot leaders in other section by redefining them, it is not as easy to add them back.

As our aim is to be able to have all the classes used a common denominator we can define a command as follows (using book as a base)

\begin{tcolorbox}{}
\begin{lstlisting}
\def\articlesection{
\newcommand*\l@section[2]{%
  \ifnum \c@tocdepth >\z@
    \addpenalty\@secpenalty
    \addvspace{1.0em \@plus\p@}%
    \setlength\@tempdima{1.5em}%
    \begingroup
      \parindent \z@ \rightskip \@pnumwidth
      \parfillskip -\@pnumwidth
      \leavevmode \bfseries
      \advance\leftskip\@tempdima
      \hskip -\leftskip
      #1\nobreak\hfil \nobreak\hb@xt@\@pnumwidth{\hss #2}\par
    \endgroup
  \fi}
}
\end{lstlisting}
\end{tcolorbox}

%\articlesection

The \cs{@starredsection} macro is one of those locomotive type of commands. It takes 7 required arguments and 2 optional ones and hidden within it are two booleans. The full set looks like this:

\cs{@startsection} \marg{name} \marg{level} \marg{indent} \marg{beforeskip} \marg{afterskip} \marg{style}[*]
  [\marg{altheading}]\marg{heading}.

\begin{marglist}
\item[name] The name of the level command.
\item [level] A number denoting the depth of the section, chapter=1, section=2, etc. A section number will be printed only if \marg{level} is equal or smaller than the value of \textit{secnumdepth}
\item[indent] The indentation of the heading from the left margin.
\item[beforeskip]  The absolute value of this argument is the skip to leave above the heading. If it is negative, then the paragraph indent of the text following the heading is suppressed.
\item [afterskip] If positive, it is the skip to leave below the heading, else it is the skip to the right of a run-in heading.
\item [style] Sets the style of the heading.
\item[\textup{[*]}] When this is missing the heading is numbered and the corresponding counter is incremented.
\item[\textup{[\textit{altheading}]}] Gives an alternative heading to use in the table of contents and in the running heads. This should be present when the * form is used.
\item[heading] The heading of the new section.
\end{marglist}

\begin{example}{Example formatting run-in section}{}
\makeatletter
\bgroup
\renewcommand\section{%
    \@startsection{section}%
    {1}%
    {0em}%
    {-0.8em}%
    {-0.5em}%
    {\large\normalfont\scshape}}
\makeatother
\section[]{test}
\lorem
\egroup
\end{example}

Note we run the example in a group so that we will not influence the formatting of this document.

As mentioned earlier there is an additional way to introduce formatting for sections and this is using the command \cs{@seccntformat}, which is responsible for typesetting the counter part of a section title. The default definition of the command typesets the \cs{the} representation of the section counter.

\begin{example}{}{}
\bgroup
\renewcommand\section{%
    \@startsection{section}%
    {1}%
    {0em}%
    {-0.8em}%
    {-0.5em}%
    {\large\normalfont\scshape}}
\renewcommand\@seccntformat[1]{\fbox
{\csname the#1\endcsname}\hspace{0.5em}}
\makeatother
\section[]{test}\label{sec:ok}
\lorem

See section \ref{sec:ok}.
\egroup
\end{example}

The definition of \cs{@seccntformat} applies to all headings
defined with the \cs{@startsection} command (which is described in the next
section). Therefore, if you wish to use different definitions of \cs{@seccntformat}
for different headings, you must put the appropriate code into every heading
definition.

\begin{tcolorbox}
\begin{lstlisting}
\def\@seccntformat##1{\csname the##1\endcsname{}}
\end{lstlisting}
\end{tcolorbox}

\section{Custom headings}

It is also possible to define section headings without resorting to any of the above. To do this.

\begin{tcolorbox}
\begin{lstlisting}
\newcommand\part{\secdef\cmda\cmdb}
\end{lstlisting}
\end{tcolorbox}

the part and chapter and sometimes appendix are defined this way, but nothing stops us from doing the same for other sections. A generic section command can be defined as follows:

\begin{example}{}{}
\bgroup
\renewcommand\section[2] [?]{% % Complex form:
\refstepcounter{section}% % step counter/ set label
\addcontentsline{toc}{appendix}% % generate toe entry
{\protect\numberline{section-\thesection}#1}%
{\raggedright\large\bfseries section %\appendixname\ % typeset the title
\thesection\par \centering#2\par}% % and number
\sectionmark{#1}% % add to running header
\@afterheading % prepare indentation handling
%\addvspace{\baselineskip}
}
\section{Test}
\lorem
\egroup
\end{example}

Many other strategies can also be implemented that are perhaps easier to grasp.

\begin{example}{}{}
\bgroup
\def\strut{\vrule height12pt depth1pt width0pt}
\renewcommand\section[2] []{% % Complex form:
\refstepcounter{section}% % step counter/ set label
\addcontentsline{toc}{section}% % generate toc entry
{\protect\numberline{\thesection} }%
{\raggedright\large\bfseries\scshape %
\parbox[b]{\dimexpr(\linewidth-0.5\columnsep)}{\colorbox{brown!80}%
{{\vbox{\strut\raise2pt\hbox{#2}}}}}}\vskip0pt% % and number
\sectionmark{#1}% % add to running header
\@afterheading % prepare indentation handling
\vspace{\dimexpr\baselineskip+6pt}%must have a parameter
}
\chapter{Fossil Insects}
\begin{multicols*}{2}\raggedcolumns
\section[Insect Fossilization]{\raggedright \thinspace Insect Fossilization}
\lipsum[1]
\end{multicols*}
\egroup
\end{example}
% To answer http://tex.stackexchange.com/questions/52998/change-title-to-small-caps-but-not-in-toc

Of course some work is needed to center the text properly in the middle of the colour box. For all practical purposes it is lining up as per the sample.

In Chapter we discussed a forward, but this may not apply if there are no chapters or we need to treat these as sections, the example \ref{ex:forwardsection} shows such a method.

\begin{example}{Defining a Foreward Section}{ex:forwardsection}

\newcommand\prematter@sp[1]{% % Complex form:
%\refstepcounter{section}% % step counter/ set label
\addcontentsline{toc}{section}% % generate toe entry
{\protect\numberline{}\textsc{#1}}%
\sectionmark{#1}% % add to running header
{\LARGE\centering\normalfont\sffamily\colorbox{brown!80}{ \textsc{#1}}\par}%
\@afterheading % prepare indentation handling
\addvspace{\baselineskip}
\@afterindentfalse
}

\newenvironment{prematter}[1]{%
   \prematter@sp{#1}}
{}
\begin{multicols}{2}
\label{theok}
\begin{prematter}{Foreward}
\lipsum[1]
\end{prematter}\ref{theok}
\end{multicols}
\end{example}

\section{underlining}

I am aware that some people have no choice but have some sections underlined as dictated by archaic regulations in some establishments for thesis submission. If nobody is forcing you to underline it is best to avoid it. We use Donald Arsenau's ulem package to achieve underlining.

\newtcolorbox{scriptexample}[2][shavian]{colback=graphicbackground,
boxrule=0pt,toprule=0pt,colframe=white}


\chapter{Those Other Languages}
\minitoc
\parindent1em

\pagestyle{myheadings}

Probably there are more users of \latexe whose mother tongue is not English than those who speak the language. \tex out of the box does not offer facilities for using non-latin based scripts easily; presents numerous problems. The biggest problem---which has been solved to a large extent---was the entering of text without having to mark all the special
characters such as umlauts (\"o) with commands. The second issue and which has been addressed by packages such as Babel, is redefining the strings such as "Chapter" to another language. In software this is called internationalization and a governing standard is |i18n|. None of the current packages take such an approach and none of them as yet offer a satisfactory solution for |LuaLaTeX|. 

Another issue with writing systems and scripts is that of appropriate fonts. Most writing systems that have ever existed are now extinct. Only minute vestiges of one of the most ancient - Egyptian hieroglyphs - live on, unrecognized, in the Latin alphabet in which English, among hundreds of other languages, is conveyed today. The latin \textit{m}, for example, ultimately derives from the Egyptian's cononantal n-sign, depicting waves.

Many of the scripts have other peculiarities, some languages such as Hanunó'o is written vertically from bottom to top. Others from top to bottom and many others from right to left. 

\section{TeX's support for different languages}

TeX's primitives such as \cmd{\language}=\meta{number} can be used to store hyphenation patterns and exceptions for up to 256 different languages. This primitive is then used by TeX to apply an appropriate set of hyphenation rules for each paragraph or part of a paragraph in a document\footnote{\url{http://www.tug.org/utilities/plain/cseq.html language-rp}}. When TeX begins a ne paragraph it sets the \emph{current language} to \cmd{\language}. Just before it adds each new character to the paragraph in unrestricted horizontal mode, it compares the current language to \cmd{\language}. If they are different, TeX : a) changes the current language to \cmd{\language}; b) inserts a whatsit\index{whatsit>language} containing the new language and the values of |\lefthyphenmin| and |\righthyphenmin|; and c) inserts the character. The |\setlanguage| command should be used to change languages in restricted horizontal mode (i.e., inside an |\hbox|). If \meta{number} is less than 0 or greater than 255, 0 is used [455]. Plain TeX has a |\newlanguage| command which may be used to allocate numbers for languages [347]. Changes made to |\language| are local to the group containing the change 

\section{LaTeX}

As far as hyphenation patterns are concerned \latexe follows very closely to the methods employed by \tex and Plain Tex. In the source2e the File |lthyphen.dtx| describes the approach to loading the default file |hyphen.ltx| . If a file hyphen.cfg is found \latexe will load the appropriate hyphenaion patterns. Traditionally language management was achieved via Johan 
Braams package Babel which we describe in the next section.


\section{The Babel Package} 

Babel \citet{babel} was the first package to systematically offer foreign language
support for \tex. It has been updated for use with |XeTeX| and |LuaTeX| and provides an environment
in which documents can be typeset in a language
other than US English, or in more than one language
or script. However, no attempt has been done to
take full advantage of the features provided by the
latter, which would require a completely new core
(as for example polyglossia or as part of \latex3).

The package has a number of predefined language files with the extension |ldf|. 


\Describe\selectlanguage{\marg{language}}{}
When a user wants to switch from one language to another he can
do so using the macro |\selectlanguage|. This macro takes the
language, defined previously by a language definition file, as
its argument. It calls several macros that should be defined in
the language definition files to activate the special definitions
for the language chosen. For ``historical reasons'', a macro name is
converted to a language name without the leading |\|; in other words,
the two following declarations are equivalent:
\begin{verbatim}
\selectlanguage{german}
\selectlanguage{\german}
\end{verbatim}

\Describe\foreignlanguage{\marg{language}\marg{text}}
The command |\foreignlanguage| takes two arguments; the second
argument is a phrase to be typeset according to the rules of the
language named in its first argument. This command (1) only
switches the extra definitions and the hyphenation rules for the
language, \emph{not} the names and dates, (2) does not send
information about the language to auxiliary files (i.e., the
surrounding language is still in force), and (3) it works even if
the language has not been set as package option (but in such a
case it only sets the hyphenation patterns and a warning is shown).

\Describe{otherlanguage*}%
{\marg{language}{otherlanguage*}}

Same as |\foreignlanguage| but as environment. Spaces after the
environment are \textit{not} ignored.



\section{The Polyglossia package}

The \pkgname{polyglossia} package has a lot of potential and has solved many issues
but its integration with large parts of the traditional |pdfLaTeX| world
is still under development and will probably take a while before one could
declare it easy to use and bug free. For example anything with the |bidi| package has issues with loading orders for a number of packages and least of which is with
the Ams packages. So if you are going to mix a number of languages in a \XeTeX\ document
you need to take extra care.

 Polyglossia is a package for facilitating multilingual typesetting with
 \XeLaTeX\ and (at an early stage) \LuaLaTeX.  Basically, it
 can be used as a replacement of \pkg{babel} for performing the following
 tasks automatically:
 
 \begin{enumerate}
 \item Loading the appropriate hyphenation patterns.
 \item Setting the script and language tags of the current font (if possible and
       available), via the package \pkg{fontspec}.
 \item Switching to a font assigned by the user to a particular script or language.
 \item Adjusting some typographical conventions according to the current language
       (such as afterindent, frenchindent, spaces before or after punctuation marks,
       etc.).
 \item Redefining all document strings (like chapter, “figure”, “bibliography”).
 \item Adapting the formatting of dates (for non-Gregorian calendars via external
       packages bundled with polyglossia: currently the Hebrew, Islamic and Farsi
       calendars are supported).
 \item For languages that have their own numbering system, modifying the formatting
       of numbers appropriately (this also includes redefining the alphabetic sequence
       for non-Latin alphabets).\footnote{ %
         For the Arabic script this is now done by the bundled package \pkg{arabicnumbers}.}
 \item Ensuring proper directionality if the document contains languages
       that are written from right to left (via the package \pkg{bidi},
       available separately).
 \end{enumerate}
 
 Several features of \pkg{babel} that do not make sense in the \XeTeX\ world (like font
 encodings, shorthands, etc.) are not supported.
 Generally speaking, \pkg{polyglossia} aims to remain as compatible as possible
 with the fundamental features of \pkg{babel} while being cleaner, light-weight,
 and modern. The package \pkg{antomega} has been very beneficial in our attempt to
 reach this objective.


\section{Loading language definition files}

The recommended way of \pkg{polyglossia} to load language definition files
is given in the manual as:
 
\Describe{\setdefaultlanguage}{\oarg{options}\marg{lang}}
 (or equivalently \cmd\setmainlanguage).
 Secondary languages can be loaded with

\Describe{\setotherlanguage}{\oarg{options}\marg{lang}}
 These commands have the advantage of being explicit and of allowing you to set
 language-specific options.\footnote{ %
 More on language-specific options below.}
 It is also possible to load a series of secondary languages at once using

\Describe\setotherlanguages{\marg{lang1,lang2,lang3,\ldots}}

 Language-specific options can be set or changed at any time by means of
\Describe\setkeys{\marg{lang}\marg{opt1=value1,opt2=value2,\ldots}}

\subsection{Bidirectional languages}





\begin{comment}
\begin{Arabic}
ّ هو إذ الغاية؛ شريف الفوائد، جم المذهب، عزيز فنّ التاريخ فنّ أنّ اعلم
والملوك سيرهم، في والأنبياء أخلاقهم، في الأمم من الماضين أحوال على يوقفنا
ّ أحوال في يرومه لمن ذلك في الإقتداء فائدة تتم حتّى وسياستهم؛ دولهم في
والدنيا. الدين
\end{Arabic}
\end{comment}

The Greek language is represented both in modern Greek as well as its ancient variants.

\begin{verbatim}
\begin{greek}
\textbf{Η ελληνική γλώσσα} είναι μία από τις ινδοευρωπαϊκές γλώσσες, για την
οποία έχουμε γραπτά κείμενα από τον 15ο αιώνα π.Χ. μέχρι σήμερα. Αποτελεί το
μοναδικό μέλος ενός κλάδου της ινδοευρωπαϊκής οικογένειας γλωσσών. Ανήκει
επίσης στον βαλκανικό γλωσσικό δεσμό.\\	
(\today) 
\end{greek}
\end{verbatim}

\topline

\textbf{Η ελληνική γλώσσα} είναι μία από τις ινδοευρωπαϊκές γλώσσες, για την
οποία έχουμε γραπτά κείμενα από τον 15ο αιώνα π.Χ. μέχρι σήμερα. Αποτελεί το
μοναδικό μέλος ενός κλάδου της ινδοευρωπαϊκής οικογένειας γλωσσών. Ανήκει
επίσης στον βαλκανικό γλωσσικό δεσμό.\\	
(\today) 

\bottomline

\begin{verbatim}
\begin{russian}
\textbf{Русский язык} — один из восточнославянских языков, один из 
крупнейших языков мира, в том числе самый распространённый из славянских
языков и самый распространённый язык Европы, как географически, так и по
числу носителей языка как родного (хотя значительная, и географически бо́
льшая, часть русского языкового ареала находится в Азии).	\\
(\today)
\end{russian}
\end{verbatim}



\textbf{Русский язык} — один из восточнославянских языков, один из крупнейших языков мира, в том числе самый распространённый из славянских языков и самый распространённый язык Европы, как географически, так и по числу носителей языка как родного (хотя значительная, и географически бо́льшая, часть русского языкового ареала находится в Азии).	\\
(\today)


\section{The Translator package}

The \pkgname{translator} package was developed by \person{Till Tantau} \citep{translator}. It provides a flexible
mechanism for translating individual words into different languages.
For example, it can be used to translate a word like ``figure'' into,
say, the German word ``Abbildung''. Such a translation mechanism is
useful when the author of some package would like to localize the
package such that texts are correctly translated into the language
preferred by the user. The translator package is \emph{not} intended
to be used to automatically translate more than a few words. 

You may wonder whether the translator package is really necessary
since there is the (very nice) |babel| package available for
\LaTeX. This package already provides translations for words like
``figure''. Unfortunately, the architecture of the babel package was
designed in such a way that there is no way of adding translations of
new words to the (very short) list of translations directly build into
babel.

The translator package was specifically designed to allow an easy
extension of the vocabulary. It is both possible to add new words that
should be translated and translations of these words.

\subsection{Using the Translator Package}

  The \pkg{Translator} needs to be used with Babel and I am not too sure yet 
  if it is ready  to be used with Polyglossia.

Once the package has loaded a language or a set of languages the optional argument to the
\cmd{\translate} can be used to translate a string. 

\begin{texexample}{Translating strings}{ex:translator}
  \translate[to=german]{rightpagename}
  \translate[to=dutch]{rightpagename}
\end{texexample}

Before you can provide the translations you need to provide your own dictionaries, where you require them. These need to be installed at a place where \tex can find them.

\CMDI{\ProvidesDictionary}

The dictionary has to be saved in a specific format that relates to the \cmd{\ProvidesDictionary} command. The second argument of the command must be appended to the file name; for the example the file is saved as\footnote{This  example is from the translator package bundle and is under the folder \texttt{base}}:

|translator-basic-dictionary-German|

The concepts take a bit of time to sink in, but once you have everything set up, it is quite easy and straight forward to incorporate it, into your package. 

\begin{teXXX}
\ProvidesDictionary{translator-basic-dictionary}{German}

\providetranslation{Abstract}{Zusammenfassung}
\providetranslation{Addresses}{Adressen}
\providetranslation{addresses}{Adressen}
\providetranslation{Address}{Adresse}
\providetranslation{address}{Adresse}
\providetranslation{and}{und}
\providetranslation{Appendix}{Anhang}
\providetranslation{Authors}{Autoren}
\providetranslation{authors}{Autoren}
\providetranslation{Author}{Autor}
\providetranslation{author}{Autor}
\end{teXXX} 

This is in contrast to Babel and Polyglossia that define
commands for each string to be translated such as,

\begin{teXXX}
\def\captionsdutch{%
    \def\prefacename{Voorwoord}%
    \def\refname{Referenties}%
    \def\abstractname{Samenvatting}%
    \def\bibname{Bibliografie}%
    \def\chaptername{Hoofdstuk}%
    \def\appendixname{Bijlage}%
    \def\contentsname{Inhoudsopgave}%
    \def\listfigurename{Lijst van figuren}%
    \def\listtablename{Lijst van tabellen}%
    \def\indexname{Index}%
    \def\figurename{Figuur}%
    \def\tablename{Tabel}%
    \def\partname{Deel}%
    \def\enclname{Bijlage(n)}%
    \def\ccname{cc}%
    \def\headtoname{Aan}%
    \def\pagename{Pagina}%
    \def\seename{zie}%
    \def\alsoname{zie ook}%
    \def\proofname{Bewijs}%
    \def\glossaryname{Verklarende woordenlijst}%
    \def\today{\number\day~\ifcase\month%
      \or januari\or februari\or maart\or april\or mei\or juni\or
      juli\or augustus\or september\or oktober\or november\or
      december\fi
      \space \number\year}}
\end{teXXX}

\begin{macro}{\usedictionary}\marg{kind}
  This command tells the |translator| package, that at the beginning of
  the document it should load \textit{all} dictionaries of kind \meta{kind} for
  the languages used in the document. Note that the dictionaries are
  not loaded immediately, but only at the beginning of the document.

  If no dictionary of the given \emph{kind} exists for one of the
  language, nothing bad happens.

  Invocations of this command accumulate, that is, you can call it
  multiple times for different dictionaries.
\end{macro}

\Describe{\uselanguage}{\marg{list of languages}}
  This command tells the |translator| package that it should load the
  dictionaries for all languages in the \meta{list of languages}. The
  dictionaries are loaded at the beginning of the document.

\section{Fonts for All the World Scripts}

Many commercial as well as open source fonts exist that can be used to typeset text the world's scripts and languages. The aim of this section of the documentation is to present an overview of the most common scripts represented in the Unicode~7.0 standard. All the examples require the use of the \XeTeX\ engine. In addition you need to have a copy of the font on your own system. If you do not have them, the font loading mechanism of \XeTeX\ will take some time to search all the directories and slows compilation tremendously. 




\section{Pan-Unicode Fonts}

Thousands of fonts exist on the market, but fewer than a dozen fonts—sometimes described as "pan-Unicode" fonts—attempt to support the majority of Unicode's character repertoire. Instead, Unicode-based fonts typically focus on supporting only basic ASCII and particular scripts or sets of characters or symbols. Several reasons justify this approach: applications and documents rarely need to render characters from more than one or two writing systems; fonts tend to demand resources in computing environments; and operating systems and applications show increasing intelligence in regard to obtaining glyph information from separate font files as needed, i.e. font substitution. Furthermore, designing a consistent set of rendering instructions for tens of thousands of glyphs constitutes a monumental task; such a venture passes the point of diminishing returns for most typefaces.

The \texttt{NotSerif} font from Google\footnote{\protect\url{http://www.google.com/get/noto/}} has good support for many languages.

Another freeware pan-Unicode font is Titus\footnote{\protect\url{http://titus.fkidg1.uni-frankfurt.de/unicode/tituut.asp?Inp1=A&Inp2=B&Inp3=C&Inp4=d%40e.com&Inp6=0&Inp5=1}}
This is an extended version of this font is TITUS Cyberbit Unicode, includes 36,161 characters in v4.0.

\newfontfamily\titus[Scale=1.05]{TITUSCBZ.ttf}
\newfontfamily\noto{NotoSerif-Regular.ttf}

\begin{scriptexample}[]{Titus}
\titus

\lorem
\end{scriptexample}
\bigskip

\begin{scriptexample}[]{Noto}
\noto

\lorem
\end{scriptexample}


\section{The \texttt{ucharclasses} package}

For multilingual texts font switching can become cumbersome. The use of a pan-Unicode font as the default can help. However, if the languages are distinct enough to use different Unicode blocks, which are not covered by the \pkgname{polyglossia} package Mike Kamermans' package \pkgname{ucharclasses} can be used.

\begin{verbatim}
% and the font switching magic
\usepackage[CJK, Latin, Thai, Sinhala, Malayalam, DominoTiles, MahjongTiles]{ucharclasses}
\usepackage{fontspec}

% default transition uses the widest coverage font I know of
\setDefaultTransitions{\fontspec{Code2000.ttf}}{}

% overrides on the default rules for specific informal groups
\setTransitionsForLatin{\fontspec{Palatino Linotype}}{}
\setTransitionsForCJK{\fontspec{code2000.ttf}}{}%HAN NOM A
\setTransitionsForJapanese{\fontspec{code2000.ttf}}{}%Ume Mincho

% overrides on the default rules for specific unicode blocks
\setTransitionTo{CJKUnifiedIdeographsExtensionB}{\fontspec{SimSun-ExtB}}
\setTransitionTo{Thai}{\fontspec{IrisUPC}}
\setTransitionTo{Sinhala}{\fontspec{Iskoola Pota}}
\setTransitionTo{Malayalam}{\fontspec{Arial Unicode MS}}

\end{verbatim}

\bgroup
\begin{verbatim}
domino tiles, 🁇 🀼 🁐 🁋 🁚 🁝, and mahjong tiles: 🀑 🀑 🀑 🀒 🀒 🀒 🀕 🀕 🀕 🀗 🀗 🀗 🀅 🀅 (using FreeFont)
\end{verbatim}

domino tiles, 🁇 🀼 🁐 🁋 🁚 🁝, and mahjong tiles: 🀑 🀑 🀑 🀒 🀒 🀒 🀕 🀕 🀕 🀗 🀗 🀗 🀅 🀅 (using FreeFont)
\egroup

\section{PhD Settings}

\def\test{}
\cxset{language/.code=\test}
\cxset{language=greek}
\cxset{languages/.code=\test}
\cxset{languages={english,greek,spanish,chinese}}
\cxset{greek font/.code=\test}
\cxset{greek font=code2000.ttf}

\begin{key}{/chapter/language=\meta{language name}}  
The key language sets the main language for the document. This language will be used for the sectioning commands and common string translations.

If the language is English Polyglossia or Babel are not loaded automatically. If the language is other than English we load either Babel or Polyglossia depending on the engine used.
\end{key}


\begin{key}{/chapter/languages=\meta{language1, language2, language3}}  
The key |languages|, determines all the other scripts available for typesetting. For each language default font commands are create automatically. The aim is to be able to run a fully multilingual system with the minimum of upfront settings. These we leave to customize in the style template files.
\end{key}

\begin{key}{/chapter/greek font=\meta{options}\meta{font file}}  
The package comes with numerous language and appropriate default fonts
for each operating system. 
\end{key}

\section{Ancient and Historic Scripts}

Unicode encodes a number of ancient scripts, which have not been in normal use for a millennium or more, as well as historic scripts, whose usage ended in recent centuries. Although these scripts are no longer used to write living languages, documents and inscriptions using these languages exist, both for extinct languages and for precursors of modern languages. The primary user communities for these scripts are scholars, interested in studying the scripts and the languages written in them. A few, such as Coptic, also have contemporary liturgical or other special purposes. Some of the historic scripts are related to each other as well as to modern alphabets. The following are provides as of Unicode version~6.2.

\begin{center}
\begin{tabular}{lll}
Ogham.     &Ancient Anatolian Alphabets. &Avestan.\\
Old Italic. &Old South Arabian. &Ugaritic\\
Runic &Phoenician. &Old Persian\\
Gothic &Imperial Aramaic &Sumero-Akkadian\\
Old Turkic. &Mandaic &Egyptian Hieroglyphs.\\
Linear B &Inscriptional Parthian &Meroitic.\\
Cypriot Syllabary &Inscriptional Pahlavi&\\
\end{tabular}
\end{center}

The following scripts are also encoded but following the Unicode
convention are described in other sections

\begin{center}
\begin{tabular}{llllll}
Coptic &Glagolithic &Phags-pa. &Kaithi &Kharoshi &Brahmi.\\
\end{tabular}
\end{center}


^^A\subsection{Ogham}

\newfontfamily\ogham{code2000.ttf}

Ogham was added to the Unicode Standard in September 1999 with the release of version 3.0.
The spelling of the names given is a standardisation dating to 1997, used in Unicode Standard and in Irish Standard 434:1999.
The Unicode block for ogham is \texttt{U+1680–U+169F}.

\begin{scriptexample}[]{Ogham}
\bgroup
\ogham
0	1	2	3	4	5	6	7	8	9	A	B	C	D	E	F\\
U+168x	   	ᚁ	ᚂ	ᚃ	ᚄ	ᚅ	ᚆ	ᚇ	ᚈ	ᚉ	ᚊ	ᚋ	ᚌ	ᚍ	ᚎ	ᚏ\\
U+169x	ᚐ	ᚑ	ᚒ	ᚓ	ᚔ	ᚕ	ᚖ	ᚗ	ᚘ	ᚙ	ᚚ	᚛	᚜	\\

\titus

0	1	2	3	4	5	6	7	8	9	A	B	C	D	E	F\\
U+168x	   	ᚁ	ᚂ	ᚃ	ᚄ	ᚅ	ᚆ	ᚇ	ᚈ	ᚉ	ᚊ	ᚋ	ᚌ	ᚍ	ᚎ	ᚏ\\
U+169x	ᚐ	ᚑ	ᚒ	ᚓ	ᚔ	ᚕ	ᚖ	ᚗ	ᚘ	ᚙ	ᚚ	᚛	᚜
\egroup		
\end{scriptexample}
^^A\section{Ancient Anatolian Alphabets}

The Anatolian scripts described in this section all date from the first millenium BCE, and were used to write various ancient Indo-European languages of western and southwestern Anatolia (now Turkey). All are related to the Greek script and are probably adaptations of it. 

\newfontfamily\lycian{Aegean.ttf}
\let\lydian\lycian
\let\carian\lydian

\begin{description}
\item [Lycian] The Lycian alphabet was used to write the Lycian language. It was an extension of the Greek alphabet, with half a dozen additional letters for sounds not found in Greek. It was largely similar to the Lydian and the Phrygian alphabets.
 
\bgroup
\lydian
\obeylines
0	1	2	3	4	5	6	7	8	9	A	B	C	D	E	F
U+1028x	𐊀	𐊁	𐊂	𐊃	𐊄	𐊅	𐊆	𐊇	𐊈	𐊉	𐊊	𐊋	𐊌	𐊍	𐊎	𐊏
U+1029x	𐊐	𐊑	𐊒	𐊓	𐊔	𐊕	𐊖	𐊗	𐊘	𐊙	𐊚	𐊛	𐊜

Typeset with the \idxfont{Aegean.ttf} and the command \cmd{\lydian}
\egroup

\item[Lydian] Lydian script was used to write the Lydian language. That the language preceded the script is indicated by names in Lydian, which must have existed before they were written. Like other scripts of Anatolia in the Iron Age, the Lydian alphabet is a modification of the East Greek alphabet, but it has unique features. The same Greek letters may not represent the same sounds in both languages or in any other Anatolian language (in some cases it may). Moreover, the Lydian script is alphabetic.
Early Lydian texts are written both from left to right and from right to left. Later texts are exclusively written from right to left. One text is boustrophedon. Spaces separate words except that one text uses dots. Lydian uniquely features a quotation mark in the shape of a right triangle.
The first codification was made by Roberto Gusmani in 1964 in a combined lexicon (vocabulary), grammar, and text collection.


\bgroup
\lycian
\obeylines
	0	1	2	3	4	5	6	7	8	9	A	B	C	D	E	F
U+1092x	𐤠	𐤡	𐤢	𐤣	𐤤	𐤥	𐤦	𐤧	𐤨	𐤩	𐤪	𐤫	𐤬	𐤭	𐤮	𐤯
U+1093x	𐤰	𐤱	𐤲	𐤳	𐤴	𐤵	𐤶	𐤷	𐤸	𐤹						𐤿
Typeset with the \idxfont{Aegean.ttf} and the command \cmd{\lycian}

Examples of words

𐤬𐤭𐤠  - Ora - "Month"

𐤬𐤳𐤦𐤭𐤲𐤬𐤩  - Laqrisa - "Wall"

𐤬𐤭𐤦𐤡  - "House, Home"

\egroup

\item [Carian] The Carian alphabets are a number of regional scripts used to write the Carian language of western Anatolia. They consisted of some 30 alphabetic letters, with several geographic variants in Caria and a homogeneous variant attested from the Nile delta, where Carian mercenaries fought for the Egyptian pharaohs. They were written left-to-right in Caria (apart from the Carian–Lydian city of Tralleis) and right-to-left in Egypt. Carian was deciphered primarily through Egyptian–Carian bilingual tomb inscriptions, starting with John Ray in 1981; previously only a few sound values and the alphabetic nature of the script had been demonstrated. The readings of Ray and subsequent scholars were largely confirmed with a Carian–Greek bilingual inscription discovered in Kaunos in 1996, which for the first time verified personal names, but the identification of many letters remains provisional and debated, and a few are wholly unknown.

\begin{scriptexample}[]{Carian}
\bgroup
\carian
\obeylines
 	0	1	2	3	4	5	6	7	8	9	A	B	C	D	E	F
U+102Ax	𐊠	𐊡	𐊢	𐊣	𐊤	𐊥	𐊦	𐊧	𐊨	𐊩	𐊪	𐊫	𐊬	𐊭	𐊮	𐊯
U+102Bx	𐊰	𐊱	𐊲	𐊳	𐊴	𐊵	𐊶	𐊷	𐊸	𐊹	𐊺	𐊻	𐊼	𐊽	𐊾	𐊿
U+102Cx	𐋀	𐋁	𐋂	𐋃	𐋄	𐋅	𐋆	𐋇	𐋈	𐋉	𐋊	𐋋	𐋌	𐋍	𐋎	𐋏
U+102Dx	𐋐
\egroup
\end{scriptexample}

\newfontfamily\oldpunctuation{code2000.ttf}

Word dividers are infrequent, \emph{scriptio continua}\footnote{a style of writing without word dividers, that is, without spaces or other marks between words or sentences} is common. Words dividers which are attested are U+00B7 (\char"00B7) \textsc{MIDLE DOT} (or U+2E31 word separator middle dot), U+205A TWO DOT PUNCTUATION, and U+205D TRICOLON ({\oldpunctuation\char"205D}). In modern editions U+0020 SPACE may be found.

\end{description}
^^A

\section{Avestan script}
\label{s:avestan}
The Avestan alphabet is a writing system developed during Iran's Sassanid era (AD 226–651) to render the Avestan language.
As a side effect of its development, the script was also used for Pazend, a method of writing Middle Persian that was used primarily for the Zend commentaries on the texts of the Avesta. In the texts of Zoroastrian tradition, the alphabet is referred to as \emph{din dabireh} or \emph{din dabiri}, Middle Persian for "the religion's script".

The Avestan alphabet was replaced by the Arabic alphabet after Persia converted to Islam during the 7th century CE. 


Notable Features

The alphabet is written from right to left, in the same way as Syriac, Arabic and Hebrew.
See more at: \url{http://www.iranchamber.com/scripts/avestan_alphabet.php#sthash.ZRu7AkEb.dpuf}

\newfontfamily\avestan{NotoSansAvestan-Regular.ttf}



\begin{scriptexample}[]{Avestan}
\ifxetex\TeXXeTstate=1
\beginR\fi
\avestan\raggedleft
𐬄	
𐬅	
𐬆	
𐬇	
𐬈	
𐬉	
𐬊	
𐬋	
𐬌	
𐬍	
𐬎	
𐬏	
𐬐	
	
𐬒	
𐬓	
𐬔	
	
𐬖	
𐬗	
𐬘	
𐬙	
𐬚	
𐬛	
𐬜	
𐬝	
𐬞	
𐬟	
𐬠	
𐬡	
𐬢	
𐬣	
𐬤	
𐬥	
𐬦	
𐬧	
𐬨	
𐬩	
𐬪	
𐬫	
𐬬	
𐬭	
𐬮	
𐬯	
𐬰	
𐬱	
𐬲	
𐬳	
𐬴	
𐬵	
\ifxetex\endR
\TeXXeTstate=0\fi
\end{scriptexample}

The recent Google font \url{NotoSansAvestan-Regular_0.ttf} can be used to typeset the Avestan script, but really not suitable for any serious study of the language.
^^A\subsection{Old Turkic}

\newfontfamily\oldturkic{Segoe UI Symbol}
\begin{scriptexample}[]{Old Turkish}
\oldturkic
\obeylines
Orkhon	Yenisei
variants	Transliteration / transcription
Old Turkic letter  𐰀	𐰁 𐰂	a, ä
Old Turkic letter  𐰃	𐰄 𐰅	y, i (e)
Old Turkic letter  𐰆		o, u
Old Turkic letter  𐰇	𐰈	ö, ü

	0	1	2	3	4	5	6	7	8	9	A	B	C	D	E	F
U+10C0x	𐰀	𐰁	𐰂	𐰃	𐰄	𐰅	𐰆	𐰇	𐰈	𐰉	𐰊	𐰋	𐰌	𐰍	𐰎	𐰏
U+10C1x	𐰐	𐰑	𐰒	𐰓	𐰔	𐰕	𐰖	𐰗	𐰘	𐰙	𐰚	𐰛	𐰜	𐰝	𐰞	𐰟
U+10C2x	𐰠	𐰡	𐰢	𐰣	𐰤	𐰥	𐰦	𐰧	𐰨	𐰩	𐰪	𐰫	𐰬	𐰭	𐰮	𐰯
U+10C3x	𐰰	𐰱	𐰲	𐰳	𐰴	𐰵	𐰶	𐰷	𐰸	𐰹	𐰺	𐰻	𐰼	𐰽	𐰾	𐰿
U+10C4x	𐱀	𐱁	𐱂	𐱃	𐱄	𐱅	𐱆	𐱇	𐱈	

\hfill  Typeset with \texttt{Segoe UI Symbol} \cmd{\oldturkic} 
\end{scriptexample}

Irk Bitig or Irq Bitig (Old Turkic: {\bfseries\Large\oldturkic 𐰃𐰺𐰴 𐰋𐰃𐱅𐰃𐰏}), known as the Book of Omens or Book of Divination in English, is a 9th-century manuscript book on divination that was discovered in the "Library Cave" of the Mogao Caves in Dunhuang, China, by Aurel Stein in 1907, and is now in the collection of the British Library in London, England. The book is written in Old Turkic using the Old Turkic script (also known as "Orkhon" or "Turkic runes"); it is the only known complete manuscript text written in the Old Turkic script.[1] It is also an important source for early Turkic mythology.

The Old Turkic text does not have any sentence punctuation, but uses two black lines in a red circle as a word separation mark in order to indicate word boundaries as shown in Figure~{\ref{omen}}

\begin{figure}[htb]
\includegraphics[width=0.7\textwidth]{./images/omen.jpg}
\caption{Omen 11 (4-4-3 dice) of the Irk Bitig (folio 13a): "There comes a messenger on a yellow horse (and) an envoy on a dark brown horse, bringing good tidings, it says. Know thus: (The omen) is extremely good."}
\label{omen}
\end{figure}
^^A\section{Phoenician}
\label{s:phoenician}
\arial

The Phoenician alphabet and its successors were widely used over a broad area surrounding the Mediterranean Sea.

\let\phoenician\lycian

\begin{scriptexample}[]{Phoenician}

\unicodetable{phoenician}{"10900,"10910}

\end{scriptexample}

The Phoenician alphabet, called by convention the Proto-Canaanite alphabet for inscriptions older than around 1200 BCE, is the oldest verified consonantal alphabet, or abjad.[1] It was used for the writing of Phoenician, a Northern Semitic language, used by the civilization of Phoenicia. It is classified as an abjad because it records only consonantal sounds (matres lectionis were used for some vowels in certain late varieties).

Phoenician became one of the most widely used writing systems, spread by Phoenician merchants across the Mediterranean world, where it evolved and was assimilated by many other cultures. The Aramaic alphabet, a modified form of Phoenician, was the ancestor of modern Arabic script, while Hebrew script is a stylistic variant of the Aramaic script. The Greek alphabet (and by extension its descendants such as the Latin, the Cyrillic, and the Coptic) was a direct successor of Phoenician, though certain letter values were changed to represent vowels.

\begin{figure}[ht]
\includegraphics[width=\textwidth]{./images/phoenician.jpg}
\captionof{figure}{
Phoenician votive inscription from Idalion (Cyprus), 390 BC. BM 125315 from The Early Alphabet by John F. Healy.}
\end{figure}

As the letters were originally incised with a stylus, most of the shapes are angular and straight, although more cursive versions are increasingly attested in later times, culminating in the Neo-Punic alphabet of Roman-era North Africa. Phoenician was usually written from right to left, although there are some texts written in boustrophedon.


\printunicodeblock{./languages/phoenician.txt}{\phoenician}


\newpage
\section{Palmyrene}
\idxlanguage{Palmyrene}
\arial

Palmyrene is the very widely attested Aramaic dialect and script
of Palmyra in the Syrian desert. The texts date from the midfirst century to the destruction of Palmyra by the Romans in AD 272. Palmyra in the Roman period was a major trading centre.
\medskip

\begin{figure}[ht]
\centering

\includegraphics[width=0.9\textwidth]{./images/palmyrene.jpg}
\captionof{figure}{\protect\arial Limestone bust with Palmyrene inscription. Palmyra late 2nd century AD. BM WA 102612}

\end{figure}

\medskip
The longest of the Palmyrene texts, is the bilingual  taxation tariff written for the year 137 AD in Palmyrene Aramaic and Greek.\footnote{For more details see:MILIK J.T., Dédicaces faites par des dieux (Palmyre, Hatra, 
Tyr) et de thiases sémitiques à l'époque romaine, Paris 1972; ROSENTHAL R., Die 
Sprache der palmyrenischen Inschriften, Leipzig 1936; STARK J.K., Personal Names in 
Palmyrene Inscriptions, Oxford 1971; DRIJVERS H.J.W., The Religion of Palmyra, 
Leiden 1976; TEIXIDOR J., 'Palmyre et son commerce d'Auguste à Caracalla', in 
Semitica 34, (1984) 1-127.  } Trade connections 
took the Palmyrene script to other places, some not far away, such as Dura Europos on the Euphrates, butothers at a great distance. A particular inscription is from South Shields, Roman Arbeia, in the north-east of England, carved on behalf of a Palmyrene mechant for his deceased wife and probably dating to the early third century AD. 

The Palmyrene script existed in two main varieties, a monumental and a cursive one, though the latter is little known and the evidence  mostly from Palmyra itself. The Syriac script of Edessa in southern Turkey, is often regarded as derived or closely related to the Palmyrene---similarities are found in the letters: ', b, g, d, w, h, y, k, l, m, n, `, r and t---though a strong case can also be made for connecting Syriac with a northern Mesopotamian script-family represented principally in texts from Hatra, a city more or less contemporary with Palmyra in Upper Mesopotamia. 


\begin{figure}[ht]
\includegraphics[width=\textwidth]{./images/regina-epigraph.jpg}
\caption{It was customary for Palmyrenes to offer bilingual texts (Greek or Latin) on funerary monuments. The final line of Regina's epitaph is Barates' personal lament in Palmyrene: Regina, freedwoman of Barate, alas. (See \href{http://www2.cnr.edu/home/araia/regina.html}{regina}.)}
\end{figure}

A good article on the classification of Aramaic languages can be found in \textit{The Aramaic language and Its Classification} by Efrem Yildiz.\footnote{\url{http://www.jaas.org/edocs/v14n1/e8.pdf}}








^^A\newfontfamily\aegyptus{AegyptusR.ttf}

\chapter{Aegyptian Hieroglyphics}

\index{fonts>Aegyptus}\index{Aegyptus (font)}
\index{fonts>Hieroglyphics}\index{languages>hieroglyphics}

\newfontfamily\hiero{NotoSansEgyptianHieroglyphs-Regular.ttf}

Hieroglyphic writing appeared in Egypt at the end of the fourth millennium bce. The writing
system is pictographic: the glyphs represent tangible objects, most of which modern
scholars have been able to identify. A great many of the pictographs are easily recognizable
even by nonspecialists. Egyptian hieroglyphs represent people and animals, parts of the
bodies of people and animals, clothing, tools, vessels, and so on.

Hieroglyphs were used to write Egyptian for more than 3,000 years, retaining characteristic
features such as use of color and detail in the more elaborated expositions. Throughout the
Old Kingdom, the Middle Kingdom, and the New Kingdom, between 700 and 1,000 hieroglyphs
were in regular use. During the Greco-Roman period, the number of variants, as
distinguished by some modern scholars, grew to somewhere between 6,000 and 8,000.

Hieroglyphs were carved in stone, painted on frescoes, and could also be written with a reed
stylus, though this cursive writing eventually became standardized in what is called \emph{hieratic}
writing. Unicode does not encode the hieratic forms separately, but ust considers them as cursive forms of the hieroglyphs encoded block.

The Demotic script and then later the Coptic script replaced the earlier hieroglyphic and
hieratic forms for much practical writing of Egyptian, but hieroglyphs and hieratic continued
in use until the fourth century ce. An inscription dated August 24, 394 ce has been
found on the Gateway of Hadrian in the temple complex at Philae; this is thought to be
among the latest examples of Ancient Egyptian writing in hieroglyphs

\begin{figure}[htb]
\includegraphics[width=\textwidth]{./images/bookofthedead.jpg}
\end{figure}

In hieroglyphic texts, these drawings are not only simply arranged in sequential order, but also grouped on top of and next to each other. This rather complicates matters trying to register and reproduce hieroglyphic texts using a computer.

\section{Computer Typesetting}

Typesetting hieroglyphics with computers presents a number of problems. First is the method of inputting the characters and second the various methods required to stack hieroglyphics, the direction of writing which can be one of four different directions.

When the first computers were introduced in Egyptology in the late 1970s and the beginning of the 1980s, the graphical capacity of the machines was still in its infancy. Early attempts to register the hieroglyphic pictorial writing on computer therefore chose an encoding system to do this, using alphanumeric codes to represent or replace the graphics. To prevent many people from reinventing the wheel, during the first "Table Ronde Informatique et Egyptologie" in 1984 a committee was charged with the task to develop a uniform system for the encoding of hieroglyphic texts on computer. The resulting Manual for the Encoding of Hieroglyphic Texts for Computer-input (Jan Buurman, Nicolas Grimal, Jochen Hallof, Michael Hainsworth and Dirk van der Plas, Informatique et Egyptologie 2, Paris 1988), simply called Manuel de Codage, presents an easy to use and intuitive way of encoding hieroglyphic writing as well as the abbreviated hieroglyphic transcription (transliteration). The system proposed by the Manuel de Codage has since been adopted by international Egyptology as the official common standard for registering hieroglyphic texts on computer. Mark-Jan Nederhof proposed an enhanced encoding scheme to remove many of the limitations in the Manuel de Codage.

\pkgname{HieroTeX} is a \latexe package developed by to typeset hieroglyphic texts and still works well. The advantages of using \tex is of course its excellent typesetting capabilities and the usage of macros. Although inputting the texts as MdC codes is not that difficult, repeating the same codes over and over can be avoided with easily constructed simple substitution macros. 

\subsection{fonts}

One of the best fonts I came across is \idxfont{Aegyptus} from \url{http://users.teilar.gr/~g1951d/}\footnote{The site also has fonts for Aegean Numbers, Ancient Greek Musical Notation, Ancient Greek Numbers, Ancient Roman Symbols, Arkalochori Axe, Carian, Cypriot Syllabary, Dispilio tablet, Linear A, Linear B Ideograms, Linear B Syllabary, Lycian, Lydian, Old Italic, Old Persian, Phaistos Disc, Phoenician, Phrygian, Sidetic, Troy vessels’ signs and Ugaritic. Cretan Hieroglyphs and Cypro-Minoan script(s) are offered in separate files.}. The font provides all the unicode characters and also offers an additional number of glyphs that are not in the Unicode standard. The font uses the Unicode Private Use Areas to encode the glyphs. 

Another font is the Noto Egyptian Hieroglyphics from Google. This is a lightweight font with the symbols in their proper unicode slots. Mark-Jan Nederhof's \idxfont{NewGardiner} font is another one with support only for the Gardiner set. The codepoint mappings are incorrect, as the font has been  
encoded to EGPZ. The font is similar to the Aegyptus font, however it is just transposed and not recommended unless it is transposed. 

The editor software JSesh\footnote{\protect\url{http://jsesh.qenherkhopeshef.org/}} also provides a free font |JSeshFont.ttf|. This offers a correctly mapped unicode and is another good alternative. The symbols are drawn somewhat simpler and is just a matter of taste what you want to use.

My recommendation is for short demonstration purposes, the Noto font is to be preferred while for more serious work the Aegyptus font will be more useful. Using Lua the font can be transposed automatically to allow the use of commands that refer to unicode numbers. Another advantage of the Aegyptus font is that the glyphs are named with their Gardiner numbers, so it is somewhat easier to programmatically access them by name.\footnote{Unicode does not name the glyphs, but simply calls the Egyptian Hieroglyph $n$. } 

\medskip

\ifxetex
\bgroup
\centering 
\font\myfont = "Aegyptus"
\scalebox{7}{\myfont\XeTeXglyph 201}
\scalebox{7}{\myfont\XeTeXglyph 203}
\scalebox{7}{\myfont\XeTeXglyph 163}
\scalebox{7}{\myfont\XeTeXglyph 164}
\scalebox{7}{\myfont\XeTeXglyph 165}
\scalebox{7}{\myfont\XeTeXglyph 168}
\captionof{table}{Example of Egyptian Hieroglyphics typeset with the \textit{Aegyptus} font.} 
\egroup
\fi

\ifluatex
\bgroup
\centering 
\aegyptus
\scalebox{7}{\char"F300C}
\scalebox{7}{\char"F3001}
\scalebox{7}{\char"F3010}
\scalebox{7}{\char"F308B}
\scalebox{7}{\char"F3097}
\scalebox{7}{\char"F3091}
\captionof{table}{Example of Egyptian Hieroglyphics typeset with the \textit{Aegyptus} font.} 
\egroup

\fi


\subsection{Unicode Block}

Egyptian hieroglyphs is a Unicode block containing the Gardiner's sign list of Egyptian hieroglyphics.
The code points, in the range |0x13000| to |0x1342E|, are available starting from
\href{http://unicode.org/charts/PDF/U13000.pdf}{Unicode 5.2}

\begin{scriptexample}[]{Hieroglyphic}
\bgroup
\unicodetable{hiero}{"13000,"13010,"13020,"13030,"13040,"13050,"13060,"13070,%
"13080,%
"13090,"130A0,"130B0,"130C0,"130D0,"130E0,"130F0,%
"13100,"13110,"13120,"13130,"13140,"13150,"13060,"13070,"13080,"13090}
\egroup
\end{scriptexample}

\subsection{Gardiner's classification}

The standard reference on Egyptian hieroglyphics is Gartiner's Sign List, which lists common Egyptian hieroglyphs. These are grouped in categories from A-Aa. Each category represents a theme for example category A, is "man and his occupations". Based on this list ``Queen with flower" is denoted as \texttt{B7}. 

\subsubsection{Character Names} 

Egyptian hieroglyphic characters have traditionally been designated in
several ways:

\begin{enumerate}
\item  By complex description of the pictographs: \texttt{GOD WITH HEAD OF IBIS}, and so forth.
\item By standardized sign number: C3, E34, G16, G17, G24.
\item For a minority of characters, by transliterated sound.
\end{enumerate}

The characters in the Unicode Standard make use of the standard Egyptological catalog
numbers for the signs. Thus, the name for {\hiero\char"130F9} |U+13049| egyptian hieroglyph e034 refers
uniquely and unambiguously to the Gardiner list sign E34, described as a “{\aegean DESERT HARE}” ({\hiero \char"130FA}) and used for the sound “wn”. The Unicode catalog values are padded to three places with
zeros, so where the Gardiner classification is shown as \texttt{E34}, the unicode value is \texttt{E034}. 

Names for hieroglyphic characters identified explicitly in Gardiner 1953 or other sources as
variants for other hieroglyphic characters are given names by appending “A”, “B”, ... to the sign number. In the sources these are often identified using asterisks. Thus Gardiner’s G7,
G7*, and G7** correspond to U+13146 egyptian sign g007 {\hiero \char"13147}, U+13147 egyptian sign g007a, and U+13148 egyptian sign g007b, respectively.

\def\texthiero#1{{\color{black!95}\hiero #1}}

\begin{longtable}{>{\Large}lll>{\ttfamily}l}
{\hiero \char"13000}&A1-A70 & Man and his occupations &U+13000-1304F\\
{\hiero \char"13050}&B1-B9  &Woman and her occupations &U+13050-13059\\
{\hiero \char"1305A} &C1-C24 &Anthropomorphic Deities &U+1305A-13075\\
{\hiero \char"13076} &D1-D67 &Parts of the Human Body &U+13076-130D1\\
{\hiero \char"130D2} &E1-E38 &Mammals &U+13076-130D1\\
{\hiero \char"130FE}  &F1-F53	&Parts of Mammals &U+130FE-1313E\\
{\hiero\char"1313F}	&G1-G54	&Birds &U+1313F-1317E\\
{\hiero \char"1317F}	&H1-H8	&Parts of Birds &U+1317F-13187\\
\texthiero{\char"13188}	&I1-I15	&Amphibious Animals, Reptiles, etc. &U+13188-1319A\\
\texthiero{\char"1319B}	&K1-K8	&Fishes and Parts of Fishes &U+1319B-131A2\\
\texthiero{\char"131A3}	&L1-L8	&Invertebrata and Lesser Animals &U+131A3-131AC\\
\texthiero{\char"131AD}	&M1-M44	&Trees and Plants &U+13AD-131EE\\
\texthiero{\char"131EF}	&N1-N42	&Sky, Earth, Water &U+131EF-1321F\\
\texthiero{\char"13250}	&O1-O51	&Buildings and Parts of Buildings &U+13250-1329A\\
\texthiero{\char"1329B}	&P1-P11	&Ships and Parts of Ships &U+1329B-132A7\\
\texthiero{\char"132A8}	&Q1-Q7	& Domestic and Funerary Furniture &U+132A8-132AE\\
\texthiero{\char"132AF}	&R1-R29	&Temple Furniture and Sacret Emblems &U+132AF-132D0\\
\texthiero{\char"132D1}	&S1-S46	&Crowns, Dress, Staves, etc. &U+132D1-13306\\
\texthiero{\char"13307}	&T1-T36	&Warfare, Hunting, Butchery &U+13307-13332\\
\texthiero{\char"13333}	&U1-42	&Agriculture, Crafts and Professions &U+13333-13361\\
\texthiero{\char"13362}	&V1-V40a	&Rope, Fibre, Baskets, Bags, etc. &U+13362-133AE\\
\texthiero{\char"133AF}	&W1-W25	&Vessels of Stone and Earthenware &U+133AF-133CE\\
\texthiero{\char"133CF}	&X1-X8a	&Loaves and Cakes &U+133CF-133DA\\
\texthiero{\char"133DB}	&Y1-Y8	&Writing, Games, Music &U+133DB-133E3\\
\texthiero{\char"133E4}	&Z1-Z16H	&Strokes, Geometrical Figures, etc. &U+133E4-1340C\\
\texthiero{\char"1340D}	&Aa1-Aa32	&Unclassified &U+1340D-1342E\\
\end{longtable}

I particularly like the crocodile sign \def\crocodile{\color{teal}{\Huge\texthiero{\char"13188}}} {\crocodile}, as it is applicable to describe people in my field of work. 

\begin{scriptexample}[]{Woman and her occupations}
\unicodetable{hiero}{"13050}
\end{scriptexample}

\section{Positioning}

One of the core assumptions of any hieroglyphic encoding or mark-up scheme following the MdC is that signs and groups of signs maybe positioned next to each other or above each other. The former is indicated by the operator * and the latter by :. One may also use -, which functions as * for horizontal texts and as : for vertical text. 

In some dialects of the MdC relative positioning has been extended by the use of the |&| operator. This is used to form a kind of ligature, such as |D&t| can be defined to represent the \textit{Cobra at rest} sign I10 with sign X1 underneath, as follows:

\begin{center}
{\hiero\HUGE
       \mbox{\rlap{\char"133CF}\char"13193\hfill\hfill}\\
       {\large|insert[bs](I10,X1)|}

\mbox{\rlap{\scalebox{0.5}{\char"133E3}}\char"13193\hfill\hfill}\\
 	
}
\end{center}

This is only a partial solution and to automate it via kerning tables, will require hundreds of entries in the kerning tables. It will also need constant modifications as researchers discover new combinations. A better approach and which is easily applied to \tex based systems would be to adopt Nederhof's method of creating a new command |insert[bs](I10,X1)|. 

In \tex one could simply define a command \cmd{\insert} with one optional argument to handle the positioning. The positioning uses the letters [b,t,s,e] to position the glyph. the letters s and e stand for start and end, whereas b,t for bottom and top respectively. When there are only two symbols involved, this is not such a difficult operation, but when three or more symbols are to be grouped and kerned together, inserting with some form of scaling is necessary.

\subsection{Enclosures}

Enclosures. The two principal names of the king, the \emph{nomen} and \emph{prenomen}, were normally
written inside a \emph{cartouche}: a pictographic representation of a coil of rope.

In the Unicode representation of hieroglyphic text, the beginning and end of the cartouche
are represented by separate paired characters, somewhat like parentheses. The Unicode manual states that `rendering of a full cartouche surrounding a name requires specialized layout software', which is of course an easy task for \tex.

\begin{macro}{\cartouche}
The commands \cmd{\cartouche} and \cmd{\cartouche}, from Peter Wilson's \pkgname{hierglyph} package have been used for many years to demonstrate the use of hieroglyphics with \latexe. 
\end{macro}

There are a several characters for these start and end cartouche characters, reflecting various styles for the enclosures.

\cartouche{{\hiero \char"13147}$sin^{2} x + cos^{2} x = 1$}
\Cartouche{{\hiero \char"13147}$sin^{2} x + cos^{2} x = 1$}

Unicode:{\hiero 𓇓𓏏𓊵𓏙𓊩𓁹𓏃𓋀𓅂𓊹𓉻𓎟𓍋𓈋𓃀𓊖𓏤𓄋𓈐𓎟𓇾𓈅𓏤𓂦𓈉 }

\textpmhg{\HQ} 

\cartouche{\pmglyph{K:l-i-o-p-a-d:r-a}}
%\translitpmhg{\HK\Hl\Hi\Ho\Hp\Ha\Hd\Hr\Ha}

\printunicodeblock{./languages/hieroglyphics.txt}{\hiero}
\printunicodeblock{./languages/hieroglyphics-13100.txt}{\hiero}
\printunicodeblock{./languages/hieroglyphics-13200.txt}{\hiero}
\printunicodeblock{./languages/hieroglyphics-13300.txt}{\hiero}
\printunicodeblock{./languages/hieroglyphics-13400.txt}{\hiero}
\section{Numerals}

Egyptian numbers are encoded following the same principles used for the
encoding of Aegean and Cuneiform numbers. Gardiner does not supply a full set of
numerals with catalog numbers in his Egyptian Grammar, but does describe the system of
numerals in detail, so that it is possible to deduce the required set of numeric characters.

Two conventions of representing Egyptian numerals are supported in the Unicode Standard.
The first relates to the way in which hieratic numerals are represented. Individual
signs for each of the 1s, the 10s, the 100s, the 1000s, and the 10,000s are encoded, because in
hieratic these are written as units, often quite distinct from the hieroglyphic shapes into
which they are transliterated. The other convention is based on the practice of the \emph{Manual
de Codage}, and is comprised of five basic text elements used to build up Egyptian numerals.
There is some overlap between these two systems.

%% Needs some work to get it into LuaLaTeX
%% omitted for the time being
%\ifxetex
%\begin{texexample}{TeXeXglyph}{ex:xetexglyph}
%\raggedright
%\font\myfont = "Aegyptus"
%\setcounter{glyphcount}{136}
%
%\whiledo
%{\value{glyphcount}<\XeTeXcountglyphs\myfont}
%{\arabic{glyphcount}:~
%{\myfont\XeTeXglyph\arabic{glyphcount}}\quad
%\stepcounter{glyphcount}}
%\end{texexample}
%\fi

\section{Input Methods}

If you writing a document with a lot of hieroglyphics inputting of hieroglyphics can be problematic. Most researchers in the field will use special keyboards or editors. They also use MS/Word or OpenOffice. They can both be coerced to produce reasonable documents, but with \tex obviously better results can be achieved. One such editor is \href{http://jsesh.qenherkhopeshef.org/}{jsesh}. 


\begin{luacode*}
    local h = {}
          h = dofile("hiero.lua")
    local options = {style="block",
                     echo=true,
                     direction="RL",
                     size = "\\Huge",
                     color = "green",
                     headings = "captionof{figure}"  -- section/tablecaption/figurecaption
                     }
   -- prints full symbol list
   h.printgardiner(t,options)

   tex.print("\\par")
   local options = {style="block",
                     echo=true,
                     heading="\\par",
                     direction="RL",
                     color = "teal",
                     scale = 8}

   h.printhierochar("hiero","1317D",options)
   h.printhierochar("hiero","13000",{direction="RL",
                                        color = "teal",
                                        scale = 8})
   h.printhierochar("hiero","13003",{direction="LR",
                                        color = "teal",
                                        scale = 1})
   h.parseMdC([[M23-X1-R4-X8-Q2-D4-W17-R14-G4-R8-O29-
               V30-U23-N26-D58-O49-Z1-F13-N31-V30-N16-
               N21-Z1-D45-N25!]])

   tex.print("\\par")
   h.printgardinercat("B")

\end{luacode*}

\newcommand\hierochar[2][direction = "LR",
                         color     = "teal",
                         scale     = 1]{% 
               \luaexec{
                h = h or {}
                h = require("hiero.lua")  
                h.parseMdC(#2,{#1})}}
               
\newcommand\printhierochar[3][direction = "LR",
                              color     = "teal",
                              scale     = 4]{% 
               \luaexec{
                h = h or {}
                h = require("hiero.lua")  
                h.printhierochar(#2,#3,{#1})}}

This file just tests the various commands available for manipulating hieroglyphics. We tried to 
generalize the commands, so they can be re-used for other type of hieroglyphics.

{
\hierochar{"A1-A2-A3!"}

\centering 

\def\options{direction = "LR",
             color     = "teal",
             scale     = 7}

\def\fontname{"hiero"}

\def\hierochar#1{\printhierochar[\options]{\fontname}{#1}}
}


\begin{scriptexample}[]{Some Example}
Sometimes kerning might be required, especially if the
glyphs are scaled.This is easily achieved with a \cmd{\kern}
command and a suitable skip dimension.

\medskip

\bgroup
\fboxsep=0pt\fboxsep.4pt
\def\options{direction = "RL",
             color     = "black!95",
             scale     = 5}
\centering

\color{teal}
\fbox{\hierochar{"13051"}}
\kern-4mm
\hierochar{"13003"}
\def\options{direction = "LR",
             color     = "black!95",
             scale     = 5}
\fbox{\hierochar{"13003"}}\color{red}
\kern-4mm
\hierochar{"13051"}
\color{black!95}
\egroup
\begin{verbatim}
\centering
\hierochar{"13051"}
\kern-4mm
\hierochar{"13003"}
\def\options{direction = "RL",
             color     = "black!95",
             scale     = 5}
\hierochar{"13003"}
\kern-4mm
\hierochar{"13051"}
\end{verbatim}
\end{scriptexample}

A bit of a diversion is appropriate at this point. Our attempt after the historical overview, is to provide some routines for the capturing and display of hieroglyphic texts using LuaTeX. This involves getting low level information from the system regarding fonts. 

\begin{figure}[ht]
\begin{minipage}{0.45\textwidth}
\centering
\includegraphics[width=0.6\textwidth]{./images/fontforge.jpg}
\end{minipage}
\begin{minipage}[t]{0.45\textwidth}
\caption{Viewing font information with fontforge.}
\end{minipage}
\end{figure}

For each glyph, we are interested to get its unicode number, the position in the font table, its name and most importantly the font metrics. The font metrics are a set of parameters that are used to measure the bounding box, any ascenders or descenders and similar information. Using fontforge, these parameters can easily be viewed. However, we are not interested to make any modifications manually; what we are interested is to programmatically obtain this information using Lua. Lua's philosophy and a mantra repeated often by the developers, is that it provides the tools and not the solutions. What this means to the LuaTeX programmer, is that we need to reach very low level  to get this information, which is a road with many bumps. Luckily the tools have been provided by the LuaTeX developers. This comes with a lot of benefits as we can also do our own on the fly mapping, such as creating an index table holding all the Gardiner numbers. 

The |fontloader.open| function loads a font, but it's not usable by itself; the result should be turned into a table with
\textbf{fontloader.to\_table}, as follows:

\begin{verbatim}
  local f = fontloader.open
     ("c:/windows/fonts/NotSansEgyptianHieroglyphics-
       Regulat.ttf")
  fonttable = fontloader.to_table(f)
  fontloader.close(f)
\end{verbatim}

We will use the Google No Tofu Egyptian Hieroglyphic font to experiment with our hieroglyphics. I have used a full path to load the font, which resides on my windows machine in the fonts folder. Once we load all the information in the |fonttable| we use |fontloader.close| to discard the userdata from which the table is extracted. 

What makes OpenType fonts special is that they describe every aspect that you might be able to think of when you think of putting letters together to form words. In addition to the obvious "this is what letters look like" information, OpenType fonts also specify things like the name of each letter that is available in the font, how much of the Unicode standard the font implements, which horizontal and vertical metrics apply to which letters, exactly how the letters are arranged inside the font so that they can quickly be read out, what kind of font classifications apply (is it a fantasy font? is it bold face? is it fixed width? etc), what kind of memory allocation a printer needs to perform in order to be able to even load the font, etc. etc. etc. All these are stored in tables upon tables, similat to a collection of Russian dolls.

To view the values in the fonttable, we will first iterate over the \textbf{fonttable} and extract all the first level keys.

\begin{texexample}{Iterating through a font table}{}
\begin{luacode*}
local z={}
tf=fontloader.to_table(fontloader.open("c:/windows/fonts/NotoSansEgyptianHieroglyphs-Regular.ttf"))

-- we sort the keys to create a table
-- important keys to us are tf.glyphs

for k,v in pairs (tf) do
   --tex.print(k.."\\par")
   table.insert(z, k)
end

table.sort(z)
tex.print("\\begin{multicols}{3}\\raggedright")
for k,v in pairs (z) do
   z[k] = string.gsub(z[k],"%_","\\textunderscore ")
   local s = tf[v]
   tex.print("\\textbullet\\hskip3pt\\hangindent2em " .. z[k].." [\\textit{"..type(s).."}] ","\\par")
end
tex.print("\\end{multicols}")
\end{luacode*}
\end{texexample}

We iterate through the \textbf{fonttable} using the Lua  "pair" iterator and we simply print all the keys and the type of the values in a human readable form as shown in the example. Note the use of |\textunderscore| that replaces all underscores in the fields with its text equivalent to sanitize the output. This is a quick and dirty way to avoid the use of catcodes. Many of the keys, bear intuitive names and are not difficult to discern: \textit{version}, \textit{copyright} and the like. Getting the type of Lua variables is important in order to use them for error trapping. When you attempt for example to print a nil value an error will occur.

Now that we have peeked under the font we will iterate and capture the information of interest, which we will put into another table with two keys \textbf{info}  and \textbf{metrics}. In the metrics file we will get the bounding box related metrics of each and every glyph in the font and save it, into our own table. 

\begin{texexample}{More Metrics}{}
  \begin{luacode*}
   tex.print("units per em = ", tf.units_per_em,"\\par")
   for i,j in ipairs (tf.glyphs[6].boundingbox) do
      tex.print("bounding box["..i.."]".." = ", j,"\\par")
   end 
   local w = (tf.glyphs[6].boundingbox[3]-tf.glyphs[6].boundingbox[1])/tf.units_per_em
   local h = tf.glyphs[6].boundingbox[4]/tf.units_per_em
   tex.print("glyph width = ", w,"em\\par")
   tex.print("glyph height = ", h,"em\\par")

-- presents a nicely typeset table 

local rep, write = string.rep, tex.print
function ExploreTable (tab, offset)
    offset = offset or ""
    for k, v in pairs (tab) do
        local newoffset = offset .. "\\mbox{.}"
        if type(v) == "table" then
           -- if k == "boundingbox" then write("BB") end
           write(offset..k .. " = \\{\\par ")
           ExploreTable(v, newoffset)
           write(offset..newoffset .. "\\}\\par")
         else
           write(offset..k .. " = "..tostring(v),"\\par")
         end
      end
end

write("\\par{\\ttfamily ")
ExploreTable(tf.glyphs[38],"\\mbox{.}")
write("}")
  \end{luacode*}
\end{texexample}

The OpenType fonts standard, provides for so much information that we will ignore most of the items and focus on only a few tables and fields. A small utility after Paul Isambert's article is necessary to enable us to view tables easily within this book,


\begin{texexample}{ExploreTable utility}{}
\begin{luacode*}
-- presents a nicely typeset table 

local rep, write = string.rep, tex.print
function ExploreTable (tab, offset)
    offset = offset or ""
    for k, v in pairs (tab) do
        local newoffset = offset .. "\\mbox{.}"
        if type(v) == "table" then
           -- if k == "boundingbox" then write("BB") end
           write(offset..k .. " = \\{\\par ")
           ExploreTable(v, newoffset)
           write(offset..newoffset .. "\\}\\par")
         else
           write(offset..k .. " = "..tostring(v),"\\par")
         end
      end
end

write("\\par{\\ttfamily ")
ExploreTable(tf.glyphs[38],"\\mbox{.}")
write("}")
  \end{luacode*}
\end{texexample}

A good utility also is |TTX| that will convert an OTF font to XML and back. This requires that you have python installed.\footnote{See some good guidelines as to how to install it at \url{http://www.glyphrstudio.com/ttx/}.} The utility uses python to do the conversion. The archive can be downloaded from \url{http://sourceforge.net/projects/fonttools/files/latest/download}. This is a three prong attack. You need to have python install, the numpy library and then the TTX package. The |TTX| program was written by the font designer Just van Rossum, brother of the creator of the Python language, Guido van Rossum. The tool converts TrueType into human-readable |XML| format. The most attractive feature of this tool is that it also perform the opposite operation that is create a TruType font from an |XML| file. The |XML| format makes the hierarchy of the format clearer. Since SVG fonts are also described in |XML| it becomes an easier task to convert an |SVG| font to a TrueType font. To convert |bar.ttf| into |bar.ttx| you simply write:

\begin{verbatim}
ttx bar.ttf
\end{verbatim}

Similarly for the opposite conversion, from |.ttx| to |.ttf|

\begin{verbatim}
ttx bar.ttx
\end{verbatim}

The generated ttx file is approximately ten times larger than the original |.ttf| file. The files generated are huge affairs and difficult to manage.The command line option |-l| prints a list of the tables in the font. |TTX| is indispensable in the ``humanization'' of TrueType fonts. The details of the tables and what each field represents are eloquently described in that indispensable book by Yannis Haralambous \textit{Fonts \& Encodings.} Although the book is now somewhat dated, it is still the best source of information on many esoteric topics related to fonts. 






^^A\input{./languages/meroitic}

\subsection{Old Italic}

\newfontfamily\olditalic{seguisym.ttf}

Old Italic refers to any of several now extinct alphabet systems used on the Italian Peninsula in ancient times for various Indo-European languages (predominantly Italic) and non-Indo-European (e.g. Etruscan) languages. The alphabets derive from the Euboean Greek Cumaean alphabet, used at Ischia and Cumae in the Bay of Naples in the eighth century BC.

Various Indo-European languages belonging to the Italic branch (Faliscan and members of the Sabellian group, including Oscan, Umbrian, and South Picene, and other Indo-European branches such as Celtic, Venetic and Messapic) originally used the alphabet. Faliscan, Oscan, Umbrian, North Picene, and South Picene all derive from an Etruscan form of the alphabet.

The Germanic runic alphabet was derived from one of these alphabets by the 2nd century.
Old Italic is a Unicode block containing a unified repertoire of the three stylistic variants of pre-Roman Italic scripts.

\begin{scriptexample}[]{}
\unicodetable{olditalic}{"10300,"10310,"10320}
\end{scriptexample}

\subsection{Old South Arabian}

\newfontfamily\oldsoutharabian{NotoSansOldSouthArabian-Regular.ttf}

The ancient Yemeni alphabet (Old South Arabian ms3nd; modern Arabic: {\arabicfont المُسنَد‎}  musnad) branched from the Proto-Sinaitic alphabet in about the 9th century BC. It was used for writing the Old South Arabian languages of the Sabaic, Qatabanic, Hadramautic, Minaic (or Madhabic), Himyaritic, and proto-Ge'ez (or proto-Ethiosemitic) in Dʿmt. The earliest inscriptions in the alphabet date to the 9th century BC in Akkele Guzay, Eritrea[3] and in the 10th century BC in Yemen. There are no vowels, instead using the \emph{mater lectionis} to mark them.

Its mature form was reached around 500 BC, and its use continued until the 6th century AD, including Old North Arabian inscriptions in variants of the alphabet, when it was displaced by the Arabic alphabet.[4] In Ethiopia and Eritrea it evolved later into the Ge'ez alphabet,[1][2] which, with added symbols throughout the centuries, has been used to write Amharic, Tigrinya and Tigre, as well as other languages (including various Semitic, Cushitic, and Nilo-Saharan languages).

It is usually written from right to left but can also be written from left to right. When written from left to right the characters are flipped horizontally (see the photo).
The spacing or separation between words is done with a vertical bar mark (\textbar).
Letters in words are not connected together.

Old South Arabian script does not implement any diacritical marks (dots, etc.), differing in this respect from the modern Arabic alphabet.

\begin{scriptexample}[]{South Arabian}
\unicodetable{oldsoutharabian}{"10A60,"10A70}
\end{scriptexample}

Support in \latexe is provided via Peter Wilson's package \pkgname{sarabian}. The package provides all the |metafont| sources as well as transliteration commands and other utilities \seedocs{SARAB}.

\def\SAtdu{\oldsoutharabian\char"10A77}

A comparison between  the unicode and the rendering (scaled 5) \pkgname{sarabian} is shown below.

\centerline{\scalebox{3}{\SAtdu} \scalebox{3}{\textsarab{\SAtd}}}

There is no real advantage in using unicode fonts, if all you interested is to write some South Arabian text for inscriptions. 

\begin{symtable}[SARAB]{\SARAB\ South Arabian Letters}
\index{South Arabian alphabet}
\index{alphabets>South Arabian}
\label{sarabian}
\begin{tabular}{*4{ll@{\qquad}}ll}
\K[\textsarab{\SAa}]\SAa   & \K[\textsarab{\SAz}]\SAz   & \K[\textsarab{\SAm}]\SAm   & \K[\textsarab{\SAsd}]\SAsd & \K[\textsarab{\SAdb}]\SAdb \\
\K[\textsarab{\SAb}]\SAb   & \K[\textsarab{\SAhd}]\SAhd & \K[\textsarab{\SAn}]\SAn   & \K[\textsarab{\SAq}]\SAq   & \K[\textsarab{\SAtb}]\SAtb \\
\K[\textsarab{\SAg}]\SAg   & \K[\textsarab{\SAtd}]\SAtd & \K[\textsarab{\SAs}]\SAs   & \K[\textsarab{\SAr}]\SAr   & \K[\textsarab{\SAga}]\SAga \\
\K[\textsarab{\SAd}]\SAd   & \K[\textsarab{\SAy}]\SAy   & \K[\textsarab{\SAf}]\SAf   & \K[\textsarab{\SAsv}]\SAsv & \K[\textsarab{\SAzd}]\SAzd \\
\K[\textsarab{\SAh}]\SAh   & \K[\textsarab{\SAk}]\SAk   & \K[\textsarab{\SAlq}]\SAlq & \K[\textsarab{\SAt}]\SAt   & \K[\textsarab{\SAsa}]\SAsa \\
\K[\textsarab{\SAw}]\SAw   & \K[\textsarab{\SAl}]\SAl   & \K[\textsarab{\SAo}]\SAo   & \K[\textsarab{\SAhu}]\SAhu & \K[\textsarab{\SAdd}]\SAdd \\
\end{tabular}

\bigskip
\begin{tablenote}
  \usefontcmdmessage{\textsarab}{\sarabfamily}.  Single-character
  shortcuts are also supported: Both
  ``\verb+\textsarab{\SAb\SAk\SAn}+'' and ``\verb+\textsarab{bkn}+''
  produce ``\textsarab{bkn}'', for example.  \seedocs{\SARAB}.
\end{tablenote}
\end{symtable}


\section{South East Asian Scripts}

This section documents the facilities offered to typeset Southeast Asian Scripts. These scripts are used in most of Southeast Asia, Indonesia and the Philippines.

\begin{table}[htb]
\centering
\begin{tabular}{lll}
Thai. & Tai Tham &Balinese.\\
Lao.  &Tai Viet  &Javanese.\\
Myanmar &Kayah Li &Rejang\\
Khmer. &Cham &Batak\\
Tai Le &Philippine Scripts &Sundanese.\\
New Tai Lue & Buginese\\
\end{tabular}
\end{table}

\subsection{Thai}

\newfontfamily\thai[Scale=1.0,Script=Thai]{IrisUPC}

\def\thaitext#1{{\thai#1}}

\begin{scriptexample}[]{Thai}
\centerline{\LARGE\thaitext{◌ะ; ◌ัวะ; เ◌ะ; เ◌อะ; เ◌าะ; เ◌ียะ; เ◌ือะ; แ◌ะ; โ◌ะ}}


\hfill Typeset with \idxfont{IrisUPC} and the command \cmd{\thai}
\end{scriptexample}
\subsection{Balinese}

The Balinese script, natively known as Aksara Bali and Hanacaraka, is an abugida used in the island of Bali, Indonesia, commonly for writing the Austronesian Balinese language, Old Javanese, and the liturgical language Sanskrit. With some modifications, the script is also used to write the Sasak language, used in the neighboring island of Lombok.[1] The script is a descendant of the Brahmi script, and so has many similarities with the modern scripts of South and Southeast Asia. The Balinese script, along with the Javanese script, is considered the most elaborate and ornate among Brahmic scripts of Southeast Asia.[2]

Though everyday use of the script has largely been supplanted by the Latin alphabet, the Balinese script has significant prevalence in many of the island's traditional ceremonies and is strongly associated with the Hindu religion. The script is mainly used today for copying lontar or palm leaf manuscripts containing religious texts.[2][3]

\newfontfamily\balinese{AksaraBali.ttf}
\newfontfamily\indicative{code2000.ttf}

{\indicative ◌ }

\newcounter{under}
\setcounter{under}{"1B00}

\def\cb#1 {
\hspace*{2.5pt}
 \large
 $\text{◌#1}_{\pgfmathparse{Hex(\theunder)}\pgfmathresult}$
\stepcounter{under}
\vskip5pt\par
}
\begin{scriptexample}[]{Balinese}


\balinese
	 
᭐	᭑	᭒	᭓	᭔	᭕	᭖	᭗	᭘	᭙	᭚	᭛	᭜	᭝	᭞	᭟\\\
 
\def\columnseprulecolor{\color{thegray}}
\columnseprule.4pt
\begin{multicols}{8}

\texttt{U+1B0x}	

\cb{ᬀ }  \cb{ ᬁ } 	\cb{ ᬂ } 	\cb ᬃ	\cb ᬄ 	\cb ᬅ	\cb ᬆ	\cb ᬇ	\cb ᬈ	\cb ᬉ	\cb ᬊ	\cb ᬋ	\cb ᬌ	\cb ᬍ	\cb ᬎ	\cb ᬏ

\columnbreak

\texttt{U+1B1x}	 

\cb ᬐ	 \cb ᬑ 	\cb ᬒ 	\cb ᬓ	\cb ᬔ	\cb ᬕ	\cb ᬖ \cb ᬗ 	\cb ᬘ 	\cb ᬙ 	\cb ᬚ	\cb ᬛ 	\cb ᬜ 	\cb ᬝ 	\cb ᬞ	\cb ᬟ 

\columnbreak

U+1B2x	 

\cb ᬠ◌ 	\cb ᬡ	\cb ᬢ	\cb ᬣ	\cb ᬤ	\cb ᬥ	\cb ᬦ	\cb ᬧ	\cb ᬨ	\cb ᬩ	\cb ᬪ	\cb ᬫ	\cb ᬬ	\cb ᬭ	\cb ᬮ	\cb ᬯ

\columnbreak
U+1B3x 

\cb ᬰ	\cb ᬱ	\cb ᬲ	\cb ᬳ	\cb ᬴	\cb ᬵ	\cb ᬶ	\cb ᬷ	\cb ᬸ	\cb ᬹ	\cb ᬺ	\cb ᬻ	\cb ᬼ	\cb ᬽ	\cb ᬾ	\cb ᬿ


\columnbreak
U+1B4x	 

\cb ᭀ	 \cb ᭁ	\cb ᭂ	\cb ᭃ	\cb ᭄	\cb ᭅ	\cb ᭆ	\cb ᭇ	\cb ᭈ	\cb ᭉ	\cb ᭊ	\cb ᭋ

\columnbreak				
U+1B5x	 

\cb ᭐	\cb ᭑	\cb ᭒	\cb ᭓	\cb ᭔	\cb ᭕	\cb ᭖	\cb ᭗	\cb ᭘	\cb ᭙	\cb ᭚	\cb ᭛	\cb ᭜	\cb ᭝	\cb ᭞	\cb ᭟\\

\columnbreak

U+1B6x 

\cb ᭠	\cb ᭡	\cb ᭢	\cb ᭣	\cb ᭤	\cb ᭥	\cb ᭦	\cb ᭧	\cb ᭨◌ 	\cb ᭩◌ 	\cb ᭪◌ 	\cb ᭫	\cb ᭬	\cb ᭭	\cb ᭮	\cb ᭯

\columnbreak
U+1B7x	 

\cb ᭰	 \cb ᭱  \cb ᭲  \cb ᭳	 \cb ᭴	\cb ᭵	\cb ᭶	\cb ᭷	\cb ᭸	\cb ᭹	\cb ᭺	\cb ᭻	\cb ᭼


\end{multicols}

\end{scriptexample}
\defaulttext

One of the most comprehensive fonts is Aksara Bali\footnote{\url{http://www.alanwood.net/downloads/index.html}}. This is obtainable at Alan Wood's website.
\parindent1em
\section{Lao Alphabet}

\def\laotext#1{{\lao#1}}

The Lao alphabet, Akson Lao (Lao: \laotext{ອັກສອນລາວ} [ʔáksɔ̌ːn láːw]), is the main script used to write the Lao language and other minority languages in Laos. It is ultimately of Indic origin, the alphabet includes 27 consonants (\laotext{ພະຍັນຊະນະ} [pʰāɲánsānā]), 7 consonantal ligatures (\laotext{ພະຍັນຊະນະປະສົມ} [pʰāɲánsānā pá sǒm]), 33 vowels (\laotext{ສະຫລະ} [sálā]) (some based on combinations of symbols), and 4 tone marks (\laotext{ວັນນະຍຸດ} [ván nā ɲūt]). 



According to Article 89 of Amended Constitution of 2003 of the Lao People's Democratic Republic, the Lao alphabet is the official script to the official language, but is also used to transcribe minority languages in the country, but some minority language speakers continue to use their traditional writing systems while the Hmong have adopted the Roman Alphabet.[1] An older version of the script was also used by the ethnic Lao of Thailand's Isan region, who make up a third of Thailand's population, before Isan was incorporated into Siam, until its use was banned and supplemented with the very similar Thai alphabet in 1871, although the region remained distant culturally and politically until further government campaigns and integration into the Thai state (Thaification) were imposed in the 20th century.[2] The letters of the Lao Alphabet are very similar to the Thai alphabet, which has the same roots. They differ in the fact, that in Thai there are still more letters to write one sound and the more circular style of writing in Lao.

Lao, like most indic scripts, is traditionally written from left to right. Traditionally considered an \emph{abugida} script, where certain 'implied' vowels are unwritten, recent spelling reforms make this definition somewhat problematic, as all vowel sounds today are marked with diacritics when written according the Lao PDR's propagated and promoted spelling standard. However most Lao outside of Laos, and many inside Laos, continue to write according to former spelling standards, which continues the use of the implied vowel maintaining the Lao script's status as an \emph{abugida}. Vowels can be written above, below, in front of, or behind consonants, with some vowel combinations written before, over and after. Spaces for separating words and punctuations were traditionally not used, but a space is used and functions in place of a comma or period. The letters have no \emph{majuscule} or \emph{minuscule} (upper and lower case) differentiations

The Unicode block for the Lao script is U+0E80–U+0EFF, added in Unicode version 1.0. The first 10 characters of the row U+0EDx are the Lao numerals 0 through 9. Throughout the chart grey (unassigned) code points are shown, because the assigned Lao characters intentionally match the relative positions of the corresponding Thai characters. This has created the anomaly that the Lao letter \laotext{ສ} is not in alphabetical order, since it occupies the same codepoint as the Thai letter \laotext{ส}.

\begin{scriptexample}[]{}
\unicodetable{lao}{"0E80,"0E90,"0EA0,"0EB0,"0EC0,"0ED0,"0EE0,"0EF0}
\end{scriptexample}

\subsubsection{Numerals}
\bgroup
\lao
\begin{tabular}{rllllllllllll}
Hindu-Arabic numerals	&0	&1	&2	&3	&4	&5	&6	&7	&8	&9	&10 &	20\\
Lao numerals	&໐	&໑	&໒	&໓	&໔	&໕	&໖	&໗	&໘	&໙	&໑໐	&໒໐\\
Lao names	&ສູນ	&ນຶ່ງ	&ສອງ	&ສາມ	&ສີ່	&ຫ້າ 	&ຫົກ	&ເຈັດ	&ແປດ	&ເກົ້າ	&ສິບ	&ຊາວ\\
\end{tabular}
\egroup




\newfontfamily\javanese{Noto Sans Javanese}

%\newfontfamily\javanese{TuladhaJejeg_gr.ttf}

\section{Javanese}
\label{s:javanese}
\index{scripts>Javanese}


The Javanese (Ngoko Javanese: {\javanese ꦮꦺꦴꦁꦗꦮ},[3] Madya Javanese: {\javanese\   ꦠꦶꦪꦁꦗꦮꦶ},[4] Krama Javanese: ꦥꦿꦶꦪꦤ꧀ꦠꦸꦤ꧀ꦗꦮꦶ,[4] Ngoko Gêdrìk: wòng Jåwå, Madya Gêdrìk: tiyang Jawi, Krama Gêdrìk: priyantun Jawi, Indonesian: suku Jawa)[5] are an ethnic group native to the Indonesian island of Java. With approximately 100 million people (as of 2011), they form the largest ethnic group in Indonesia. They are predominantly located in the central to eastern parts of the island. There are also significant numbers of people of Javanese descent in most provinces of Indonesia, Malaysia, Singapore, Suriname, Saudi Arabia and the Netherlands.

The Javanese ethnic group has many sub-groups, such as the Mataram, Cirebonese, Osing, Tenggerese, Samin, Naganese, Banyumasan, etc.[6]

A majority of the Javanese people identify themselves as Muslims, with a minority identifying as Christians and Hindus. However, Javanese civilization has been influenced by more than a millennium of interactions between the native animism Kejawen and the Indian Hindu—Buddhist culture, and this influence is still visible in Javanese history, culture, traditions, and art forms. With a sizeable global population, the Javanese are considered significant as they are the fourth largest ethnic group among Muslims, in the world, after the Arabs,[7] Bengalis[8] and Punjabis.[9]


\paragraph{Javanese} is one of the Austronesian languages, but it is not particularly close to other languages and is difficult to classify. Its closest relatives are the neighbouring languages such as Sundanese, Madurese and Balinese. Most speakers of Javanese also speak Indonesian, the standardized form of Malay spoken in Indonesia, for official and commercial purposes as well as a means to communicate with non-Javanese-speaking Indonesians.

There are speakers of Javanese in Malaysia (concentrated in the states of Selangor and Johor) and Singapore. Some people of Javanese descent in Suriname (the Dutch colony of Suriname until 1975) speak a creole descendant of the language.

\begin{figure}[htbp]
\includegraphics[width=\textwidth]{javanese-people}
\end{figure}

The language is spoken in Yogyakarta, Central and East Java, as well as on the north coast of West Java. It is also spoken elsewhere by the Javanese people in other provinces of Indonesia, which are numerous due to the government-sanctioned transmigration program in the late 20th century, including Lampung, Jambi, and North Sumatra provinces. In Suriname, creolized Javanese is spoken among descendants of plantation migrants brought by the Dutch during the 19th century. In Madura, Bali, Lombok, and the Sunda region of West Java, it is also used as a literary language. It was the court language in Palembang, South Sumatra, until the palace was sacked by the Dutch in the late 18th century.

Javanese is written with the Latin script, Javanese script, and Arabic script.[5] In the present day, the Latin script dominates writings, although the Javanese script is still taught as part of the compulsory Javanese language subject in elementary up to high school levels in Yogyakarta, Central and East Java.

Javanese is the tenth largest language by native speakers and the largest language without official status. It is spoken or understood by approximately 100 million people. At least 45\% of the total population of Indonesia are of Javanese descent or live in an area where Javanese is the dominant language. All seven Indonesian presidents since 1945 have been of Javanese descent.[6] It is therefore not surprising that Javanese has had a deep influence on the development of Indonesian, the national language of Indonesia.

There are three main dialects of the modern language: Central Javanese, Eastern Javanese, and Western Javanese. These three dialects form a dialect continuum from northern Banten in the extreme west of Java to Banyuwangi Regency in the eastern corner of the island. All Javanese dialects are more or less mutually intelligible.


\paragraph{The Javanese script} (Hanacaraka/Carakan) is a script for writing the Javanese language, the native language of one of the peoples of the Island of Java. It is a descendent of the ancient Brahmi script of India, and so has many similarities with modern scripts of South Asia and Southeast Asia. The Javanese script is also used for writing Sanskrit, Old Javanese, and transcriptions of Kawi, as well as the Sundanese language, and the Sasak language.

\begin{figure}[htbp]
\hspace*{-1.5cm}\includegraphics[width=1.2\textwidth]{java-palm-leave-manuscript}
\end{figure}





\begin{scriptexample}[]{Javanese}
\bgroup
\javanese

꧋ꦱꦧꦼꦤ꧀ꦮꦺꦴꦁꦏꦭꦲꦶꦂꦲꦏꦺꦏꦤ꧀ꦛꦶꦩꦂꦢꦶꦏꦭꦤ꧀ꦢꦂꦧꦺꦩꦂꦠꦧꦠ꧀ꦭꦤ꧀ꦲꦏ꧀ꦲꦏ꧀ꦏꦁꦥꦝ꧉

꧋ ꦲꦮꦶꦠ꧀ꦲꦶꦏꦁꦄꦱ꧀ꦩꦄꦭ꧀ꦭꦃ꧈ ꦏꦁꦩꦲꦩꦸꦫꦃꦠꦸꦂ ꦩꦲꦲꦱꦶꦃ꧉ 	 
 ۝꧋ ꦄꦭꦶꦥꦃ꧀ ꦭ ꦩ꧀ ꦫ ꧌ ꦏꦁ — — ꦥꦿꦶꦏ꧀ꦱ ꦏꦉꦪꦥ꧀ꦥꦩꦸꦁꦄꦭ꧀ꦭꦃꦥꦶꦪꦺꦩ꧀ꦧꦏ꧀ ꧌꧉ ꦩꦁꦪꦏꦴꦪꦤꦴ ꦲꦶꦏꦸꦄꦪꦺꦪꦠꦴꦏꦶꦠꦧ꧀ꦑꦸꦂꦄꦤ꧀ꦏꦁꦥꦿꦪꦠꦭ꧉ 	 
᭐	᭑	᭒	᭓	᭔	᭕	᭖	᭗	᭘	᭙	᭚	᭛	᭜	᭝	᭞	᭟
 
\egroup
\end{scriptexample}


The Javanese script was added to Unicode Standard in version 5.2 on the code points \texttt{A980 - A9DF}. There are 91 code points for Javanese script: 53 letters, 19 punctuation marks, 10 numbers, and 9 vowels:
\medskip

\unicodetable{javanese}{"A980,"A990,"A9A0, "A9B0, "A9C0,"A9D0}

\medskip



As of the writing of this document (2017), there are several widely published fonts able to support Javanese, ANSI-based Hanacaraka/Pallawa by Teguh Budi Sayoga,[21] Adjisaka by Sudarto HS/Ki Demang Sokowanten,[22] JG Aksara Jawa by Jason Glavy,[23] Carakan Anyar by Pavkar Dukunov,[24] and Tuladha Jejeg by R.S. Wihananto,[25] which is based on Graphite (SIL) smart font technology. Other fonts with limited publishing includes Surakarta made by Matthew Arciniega in 1992 for Mac's screen font,[26] and Tjarakan developed by AGFA Monotype around 2000.[27] There is also a symbol-based font called Aturra developed by Aditya Bayu in 2012–2013.[28]

Due to the script's complexity, many Javanese fonts have different input method compared to other Indic scripts and may exhibit several flaws. \docFont{JG Aksara Jawa}, in particular, may cause conflicts with other writing system, as the font use code points from other writing systems to complement Javanese's extensive repertoire. This is to be expected, as the font was made before Javanese implementation in Unicode.[29]

Arguably, the most "complete" font, in terms of technicality and glyph count, is \docFont{TuladhaJejeg}. It comes with keyboard facilities, displaying complex syllable structure, and support extensive glyph repertoire including non-standard forms which may not be found in regular Javanese texts, by utilizing Graphite (SIL) smart font technology. |Tuladha Jejeg| uses variable stroke widths on its glyphs with serifs on some glyphs\footnote{\protect\url{https://sites.google.com/site/jawaunicode/main-page}}.

However, as not many writing systems require such complex feature, use is limited to programs with Graphite technology, such as Firefox browser, Thunderbird email client, and several OpenType word processor and of course XeLaTeX. The font was chosen for displaying Javanese script in the Javanese Wikipedia.[16]

\paragraph{jawaTeX} Jawa\TeX{} project is initial effort to make Javanese characters typesetting program using \TeX{}/\LaTeX{}. This project is aimed to make Javanese widely used. The main project is developing transliteration models to transliterate Latin document into Javanese document. Perl and \TeX{}/\LaTeX{} are use in this project, the program are develop to run in text mode (console) both Linux and Windows but not limit on it. Web based program also developed, and automatic embedded Javanese characters in HTML See \href{http://jawatex.org/jawa/jawatex}{jawatex}.


\section{Khmer}
\newfontfamily\normaltext{Arial Unicode MS}
\normaltext

\def\khmerdefaultfont#1{\newfontfamily\khmer[Scale=MatchUppercase]{#1}}
\def\khmertext#1{{\khmer#1}}

\cxset{khmer font/.code=\khmerdefaultfont{#1}}

\cxset{khmer font/.default=Khmer}

\cxset{language=khmer, 
       khmer font = Khmer UI}

\begin{key}{/chapter/khmer font=\meta{font name} (Khmer  UI)} Loads the font
command \cmd{\khmer}. When the command is used it typesets text in
khmer unicode. There is no need to load the language, unless it is the main document language. For windows the default font is \texttt{DaunPenh} this font is in general too small to read; a better font to use is Khmer UI.
\end{key}

\begin{key}{/tikz/turtle/right=\meta{angle} (default 90)}
  Turns the turtle right by the given angle. 
\end{key}


The Khmer script (Khmer: {\Large\khmertext{អក្សរខ្មែរ}}; IPA: [ʔaʔsɑː kʰmaːe]) [2] is an \textit{abugida} (alphasyllabary) script used to write the Khmer language (the official language of Cambodia). It is also used to write Pali among the Buddhist liturgy of Cambodia and Thailand.

It was adapted from the Pallava script, a variant of Grantha alphabet descended from the Brahmi script of India, which was used in southern India and South East Asia during the 5th and 6th Centuries AD.[3] The oldest dated inscription in Khmer was found at Angkor Borei District in Takéo Province south of Phnom Penh and dates from 611.[4] The modern Khmer script differs somewhat from precedent forms seen on the inscriptions of the ruins of Angkor.

Not all Khmer consonants can appear in syllable-final position. The most common syllable-final consonants include {\khmer កងញតនបមល}. The pronunciation of the consonant in final position may differ from it's normal pronunciation.


\begin{tabular}{llp{9cm}}
\khmertext{ំ}	&nĭkkôhĕt (\khmertext{និគ្គហិត})	&niggahita; nasalizes the inherent vowels and some of the dependent vowels, see anusvara, sometimes used to represent [aɲ] in Sanskrit loanwords\\
\khmertext{ះ}	&reăhmŭkh (\khmertext{រះមុខ})	&"shining face"; adds final aspiration to dependent or inherent vowels, usually omitted, corresponds to the visarga diacritic, it maybe included as dependent vowel symbol\\
\khmertext{ៈ}	&yŭkôleăkpĭntŭ (\khmertext{យុគលពិន្ទុ})	&yugalabindu ("pair of dots"); adds final glottalness to dependent or inherent vowels, usually omitted\\
\khmertext{៉}	 &musĕkâtônd (\khmertext{មូសិកទន្ត})	&mūsikadanta ("mouse teeth"); used to convert some o-series consonants (\khmertext{ង ញ ម យ រ វ}) to the a-series\\
\khmertext{៊}	&treisâpt (\khmertext{ត្រីសព្ទ})	trīsabda; used to convert some a-series consonants (\khmertext{ស ហ ប អ}) to the o-series\\
\end{tabular}




ុ	kbiĕh kraôm (ក្បៀសក្រោម)	also known as bŏkcheung (បុកជើង); used in place of the diacritics treisâpt and musĕkâtônd when they would be impeded by superscript vowels
់	bântăk (បន្តក់)	used to shorten some vowels; the diacritic is placed on the last consonant of the syllable
៌	rôbat (របាទ)
répheăk (រេផៈ)	rapāda, repha; behave similarly to the tôndâkhéat, corresponds to the Devanagari diacritic repha, however it lost its original function which was to represent a vocalic r
 ៍	tôndâkhéat (ទណ្ឌឃាដ)	daṇḍaghāta; used to render some letters as unpronounced
៎	kakâbat (កាកបាទ)	kākapāda ("crow's foot"); more a punctuation mark than a diacritic; used in writing to indicate the rising intonation of an exclamation or interjection; often placed on particles such as /na/, /nɑː/, /nɛː/, /vəːj/, and the feminine response /cah/
៏	âsda (អស្តា)	denotes stressed intonation in some single-consonant words[5]
័	sanhyoŭk sannha (សំយោគសញ្ញា)	represents a short inherent vowel in Sanskrit and Pali words; usually omitted
៑	vĭréam (វិរាម)	a mostly obsolete diacritic, corresponds to the virāma
្	cheung (ជើង)	a.w. coeng; a sign developed for Unicode to input subscript consonants, appearance of this sign varies among fonts
\section{Sundanese}
\newfontfamily\sundanese{SundaneseUnicode-1.0.5.ttf}
^^A\newfontfamily\sundanese{Arial Unicode MS}
\def\ublock#1{\texttt{{\arial #1}}}

The Sundanese script (Aksara Sunda, {\sundanese ᮃᮊ᮪ᮞᮛ ᮞᮥᮔ᮪ᮓ}) is a writing system which is used by the Sundanese people. It is built based on Old Sundanese script (Aksara Sunda Kuno) which was used by the ancient Sundanese between the 14th and 18th centuries.

\begin{scriptexample}[]{Sundanese}
\unicodetable{sundanese}{"1B80,"1B90,"1BA0,"1BB0}

\sundanese
\obeylines
\bgroup
᮱ {\arial= 1}	᮲ {\arial= 2}	᮳{\arial = 3}
᮴ {\arial= 4}	᮵ {\arial = 5} 	᮶ {\arial= 6}
᮷ {\arial= 7}	᮸ {\arial= 8}	᮹ {\arial= 9}
᮰ {\arial= 0}

\egroup
\end{scriptexample}

\begin{scriptexample}[]{Sundanese}
\bgroup
\sundanese
\centering

◌ᮃᮄᮅᮆᮇᮈᮉᮊᮋᮌᮍᮎᮏᮐᮕᮔᮓᮑᮖᮗᮚᮛᮜᮝᮞᮟᮠᮠ


\egroup
\end{scriptexample}

\bgroup
\def\1{\sundanese ᮱}
\TextOrMath\1\1

$\1$
\egroup

In text In texts, numbers are written surrounded with dual pipe sign \textbar \ldots \textbar. Example: {\textbar \sundanese ᮲᮰᮱᮰\textbar} = 2010













^^A\subsection{Oriya alphabet}
\newfontfamily\oriya[Scale=1.1,Script=Oriya]{code2000.ttf}

\def\oriyatext#1{{\oriya#1}}
The Oriya script or Utkala Lipi (Oriya: \oriyatext{ଉତ୍କଳ ଲିପି}) or Utkalakshara (Oriya: \oriyatext{ଉତ୍କଳାକ୍ଷର}) is used to write the Oriya language, and can be used for several other Indian languages, for example, Sanskrit.

\centerline{\Huge\oriyatext{ଉତ୍କଳ ଲିପି}}

\bgroup
\oriya
୦୧୨୩୪୫୬୭୮୯
ଅ ଆ ଇ ଈ ଉ ଊ ଋ ୠ ଌ ୡ ଏ ଐ ଓ ଔ କ ଖ ଗ ଘ ଙ ଚ ଛ ଜ ଝ ଞ ଟ ଠ ଡ ଢ ଣ ତ ଥ ଦ ଧ ନ ପ ଫ ବ ଵ ଭ ମ ଯ ର ଳ ୱ ଶ ଷ ସ ହ ୟ ଲ
\egroup

\begin{quotation}
Oṛiyā is encumbered with the drawback of an excessively awkward and cumbrous written character. ... At first glance, an Oṛiyā book seems to be all curves, and it takes a second look to notice that there is something inside each.(G. A. Grierson, Linguistic Survey of India, 1903)
\end{quotation}

Comparison of Oṛiyā script with its neighbours[edit]
At a first look the great number of signs with round shapes suggests a closer relation to the southern neighbour Telugu than to the other neighbours Bengali in the north and Devanāgarī in the west. The reason for the round shapes in Oriya and Telugu (and also in Kannaḍa and Malayāḷam) is the former method of writing using a stylus to scratch the signs into a palm leaf. These tools do not allow for horizontal strokes because that would damage the leaf.

Oriya letters are mostly round shaped whereas in Devanāgarī and Bengali have horizontal lines. So in most cases the reader of Oṛiyā will find the distinctive parts of a letter only below the hoop. Considering this the  closer relation to Devanāgarī and Bengali exists than to any southern script, though both northern and southern scripts have the same origin, Brāhmī.

Oriya (\oriyatext{ଓଡ଼ିଆ} oṛiā), officially spelled Odia,[3][4] is an Indian language belonging to the Indo-Aryan branch of the Indo-European language family. It is the predominant language of the Indian states of Odisha, where native speakers comprise 80\% of the population,[5] and it is spoken in parts of West Bengal, Jharkhand, Chhattisgarh and Andhra Pradesh. Oriya is one of the many official languages in India; it is the official language of Odisha and the second official language of Jharkhand. [6][7][8] Oriya is the sixth Indian language to be designated a Classical Language in India, on the basis of having a long literary history and not having borrowed extensively from other languages.

^^A
^^A\subsection{Mongolian Script}

\newfontfamily\mongolian[Language=Mongolian, Scale=1.3]{code2000.ttf}

The classical Mongolian script (in Mongolian script: {\mongolian  ᠮᠣᠩᠭᠣᠯ ᠪᠢᠴᠢᠭ᠌} Mongγol bičig; in Mongolian Cyrillic: Монгол бичиг Mongol bichig), also known as Uyghurjin Mongol bichig, was the first writing system created specifically for the Mongolian language, and was the most successful until the introduction of Cyrillic in 1946. Derived from Uighur, Mongolian is a true alphabet, with separate letters for consonants and vowels. The Mongolian script has been adapted to write languages such as Oirat and Manchu. Alphabets based on this classical vertical script are used in Inner Mongolia and other parts of China to this day to write Mongolian, Sibe and, experimentally, Evenki.
\medskip

\bgroup\par
\noindent
\colorbox{graphicbackground}{\color{black}^^A
\begin{minipage}{\textwidth}^^A
\parindent1pt
\vskip10pt
\leftskip10pt \rightskip\leftskip
\mongolian
\large
ᠬᠦᠮᠦᠨ ᠪᠦᠷ ᠲᠥᠷᠥᠵᠦ ᠮᠡᠨᠳᠡᠯᠡᠬᠦ ᠡᠷᠬᠡ ᠴᠢᠯᠥᠭᠡ ᠲᠡᠢ᠂ ᠠᠳᠠᠯᠢᠬᠠᠨ ᠨᠡᠷ᠎ᠡ ᠲᠥᠷᠥ ᠲᠡᠢ᠂ ᠢᠵᠢᠯ ᠡᠷᠬᠡ ᠲᠡᠢ ᠪᠠᠢᠠᠭ᠃ ᠣᠶᠤᠨ ᠤᠬᠠᠭᠠᠨ᠂ ᠨᠠᠨᠳᠢᠨ ᠴᠢᠨᠠᠷ ᠵᠠᠶᠠᠭᠠᠰᠠᠨ ᠬᠦᠮᠦᠨ ᠬᠡᠭᠴᠢ ᠥᠭᠡᠷ᠎ᠡ ᠬᠣᠭᠣᠷᠣᠨᠳᠣ᠎ᠨ ᠠᠬᠠᠨ ᠳᠡᠭᠦᠦ ᠢᠨ ᠦᠵᠢᠯ ᠰᠠᠨᠠᠭᠠ ᠥᠠᠷ ᠬᠠᠷᠢᠴᠠᠬᠥ ᠤᠴᠢᠷ ᠲᠠᠢ᠃
\par
\vspace*{10pt}
\end{minipage}
}
\medskip
^^A
^^A\subsection{Tibetan}

^^A\newfontfamily\tibetan{TibMachUni.ttf}

^^A\newfontfamily\tibetan{Qomolangma-Chuyig.ttf}

^^A should pick it up automatically \tibetan

Fonts described in this section can be obtained from The Tibetan \& Himalayan Library
\footnote{\url{http://www.thlib.org/tools/scripts/wiki/tibetan%20machine%20uni.html}  }

I have tried a few \texttt{Tibetan Machine Uni (TMU)} seems to be used by a number of scholars. 

A tip when you are trying to locate fonts is to find a related article in Wikipedia, such as Tibetan alphabet and inspect the element using your browser to see what fonts are being used.


|style="font-family:'Jomolhari','Tibetan Machine Uni','DDC Uchen', 'Kailash';| 


If you cannot see the script and rather than boxes or question marks then you can search and download one of the fonts in |font-family|.

\def\tibetandefaultfont#1{\newfontfamily\tibetan[Language=Tibetan]{#1}}


\cxset{language=tibetan} 
\cxset{tibetan font/.code=\tibetandefaultfont{#1}}


^^A\cxset{tibetan font = TibMachUni.ttf}




\begin{key}{/chapter/language = tibetan} The key |language=tibetan| sets the default language as Tibetan, using the main font given by the key |tibetan font=TibMachUni.ttf|.
\end{key}

\begin{key}{/chapter/tibetan font = TibMachUni.ttf} The key |tibetan font=font-name| sets the default font for the Tibetan language. It will also create the switch \cmd{\tibetan} for typesetting text in Tibetan.
\end{key}

\begin{texexample}{Tibetan language setttings}{ex:tibetan}
\cxset{language=tibetan, tibetan font = TibMachUni.ttf}
\tibetan

\tibetan Tibetan: དབུ་ཅན
\end{texexample}


The Tibetan alphabet is an \emph{abugida} of Indic origin used to write the Tibetan language as well as Dzongkha, the Sikkimese language, Ladakhi, and sometimes Balti. 

The printed form of the alphabet is called \textit{uchen} script (Tibetan: དབུ་ཅན་, Wylie: dbu-can; "with a head") while the hand-written cursive form used in everyday writing is called umê script (Tibetan: དབུ་མེད་, Wylie: dbu-med; "headless").
\uccoff
The alphabet is very closely linked to a broad ethnic Tibetan identity. Besides Tibet, it has also been used for Tibetan languages in Bhutan, India, Nepal, and Pakistan.[1] The Tibetan alphabet is ancestral to the Limbu alphabet, the Lepcha alphabet,[2] and the multilingual 'Phags-pa script.[2]
\uccon

The Tibetan alphabet is romanized in a variety of ways.[3] This article employs the Wylie transliteration system.

The Tibetan alphabet has thirty basic letters, sometimes known as "radicals", for consonants.[2]

ཀ ka /ká/	ཁ kha /kʰá/	ག ga /kà, kʰà/	ང nga /ŋà/
ཅ ca /tʃá/	ཆ cha /tʃʰá/	ཇ ja /tʃà/	ཉ nya /ɲà/
ཏ ta /tá/	ཐ tha /tʰá/	ད da /tà, tʰà/	ན na /nà/
པ pa /pá/	ཕ pha /pʰá/	བ ba /pà, pʰà/	མ ma /mà/
ཙ tsa /tsá/	ཚ tsha /tsʰá/	ཛ dza /tsà/	ཝ wa /wà/ (not originally part of the alphabet)[5]
ཞ zha /ʃà/[6]	ཟ za /sà/	འ 'a /hà/[7]
ཡ ya /jà/	ར ra /rà/	ལ la /là/
ཤ sha /ʃá/[6]	ས sa /sá/	ཧ ha /há/[8]
ཨ a /á/

\subsubsection{Unicode Block Tibetan}


\bgroup\large
\begin{tabular}{llllllllllllllll l}
\toprule
	           &|0|	&|1|	&|2|	&|3|	&|4|	&|5|	&|6|	&|7|	&|8|	&|9|	&|A|	&|B|	&|C|	&|D|	&|E|	&|F|\\
\midrule
\texttt{U+0F0x}	&ༀ	&༁	&༂	&༃	&༄	&༅	&༆	&༇	&༈	&༉	&༊	&་	&༌  &	།	&༎	&༏\\
\midrule
\texttt{U+0F1x} &༐	&༑	&༒	&༓	&༔	&༕	&༖	&༗	&༘&	༙	&༚	&༛	&༜	&༝	&༞	&༟\\
\midrule
\texttt{U+0F2x} &༠	&༡	&༢	&༣	&༤	&༥	&༦	&༧	&༨	&༩	&༪	&༫	&༬	&༭	&༮	&༯\\
\midrule
\texttt{U+0F3x}	&༰ &༱	 &༲ &༳	&༴ &༵	&༶ & ༷	&༸&	༹	&༺&	༻	&༼&	༽	&༾	&༿\\
\midrule
\texttt{U+0F4x} &ཀ	&ཁ	&ག	&གྷ	&ང	&ཅ	&ཆ	&ཇ	&	&ཉ	&ཊ	&ཋ	&ཌ	&ཌྷ	&ཎ	&ཏ\\
\midrule
\texttt{U+0F5x}	 &ཐ	&ད	&དྷ	&ན	&པ	&ཕ	&བ	&བྷ	&མ	&ཙ	&ཚ	&ཛ	&ཛྷ	&ཝ	&ཞ	&ཟ\\
\midrule
\texttt{U+0F6x} &འ	&ཡ	&ར	&ལ	&ཤ	&ཥ	&ས	&ཧ	&ཨ	&ཀྵ	&ཪ	&ཫ	&ཬ	&&&\\
^^A\texttt{U+0F7x}&&	ཱ &	& &ི	ཱི&	ུ&	ཱུ&	ྲྀ&	ཷ&	ླྀ&	ཹ&	ེ&	ཻ&	ོ&	ཽ&	&ཾ	&ཿ\\
\midrule
\texttt{U+0F8x}&    ྀ   & 	ཱྀ&	ྂ&	&ྃ &	྄	&྅&	྆	&྇	ྈ&	ྉ&	ྊ&	ྋ&	ྌ&	ྍ&	ྎ&	ྏ\\
\midrule
\texttt{U+0F9x} &	ྐ&	ྑ   & 	ྒ &	ྒྷ &	ྔ &	ྕ &	ྖ &	ྗ &		ྙ &	ྚ &	ྛ &	ྜ &	ྜྷ &	ྞ &	ྟ\\
\texttt{U+0FAx} &	ྠ &	ྡ &	ྡྷ &	ྣ &	ྤ &	ྥ &		&ྦ	&ྦྷ	ྨ&	ྩ&	ྪ&	ྫ&	ྫྷ&	ྭ&	ྮ&	ྯ\\
\midrule
\texttt{U+0FBx} 
&	  ྰ 
&	
& ྱ  	 
&ྲ	
&ླ	
&ྴ
&	ྵ
&	ྶ
&	ྷ
&ྸ
&
&
&
&	
&྾	
&྿\\
\midrule
\texttt{U+0FCx}	 &࿀&	࿁&	࿂&	࿃&	࿄&	࿅&	&࿇	&࿈	&࿉	&࿊	&࿋	&࿌	&&	࿎	&࿏\\
\midrule
\texttt{U+0FDx}	&࿐	&࿑	&࿒	&࿓	&࿔	&࿕	&࿖	&࿗	&࿘	&࿙	&࿚	&&&&&\\
\midrule
\texttt{U+0FEx} &&&&&&&&&&&&&&&&\\
\midrule
\texttt{U+0FFx}  &&&&&&&&&&&&&&&&\\
\bottomrule
\end{tabular}
\egroup




\subsubsection{Fonts for Tibetan}

Fonts for Tibetan need to be downloaded one set of fonts are the \texttt{Qomolangma}. They come in different flavours, but they appear
to offer advantages as compared to the Tibetan Machine Uni.
\medskip


\newfontfamily\betsu{Qomolangma-Betsu.ttf}
\newfontfamily\drutsa{Qomolangma-Drutsa.ttf}
\newfontfamily\chuyig{Qomolangma-Chuyig.ttf}
\newfontfamily\tsumachu{Qomolangma-Tsumachu.ttf}
\newfontfamily\uchensutung{Qomolangma-UchenSutung.ttf}
\newfontfamily\uchensuring{Qomolangma-UchenSuring.ttf}
\newfontfamily\uchensarchen{Qomolangma-UchenSarchen.ttf}
\newfontfamily\uchensarchung{Qomolangma-UchenSarchung.ttf}
\newfontfamily\tsuring{Qomolangma-Tsuring.ttf}
\newfontfamily\TMU{TibMachUni.ttf}
\newfontfamily\himalaya{Microsoft Himalaya}
\uccoff

{
\centering

\renewcommand{\arraystretch}{1.5}

\begin{tabular}{lr}
\toprule
|Qomolangma-Betsu.ttf| & {\betsu  དབུ་མེད }\\
\midrule
|Qomolangma-Chuyig.ttf| &{\chuyig  དབུ་མེད}\\
\midrule
|Qomolangma-Drutsa.ttf| &{\drutsa  དབུ་མེད}\\
\midrule
|Qomolangma-Tsumachu.ttf|&{\tsumachu  དབུ་མེད}\\
\midrule
|Qomolangma-Tsuring.ttf| &{\tsuring  དབུ་མེད}\\
\midrule
|Qomolangma-UchenSarchen.ttf| &{\uchensarchen དབུ་མེད}\\
\midrule
|Qomolangma-UchenSarchung.ttf|&{\uchensarchung དབུ་མེད }\\
\midrule
|Qomolangma-UchenSuring.ttf|&{\uchensuring དབུ་མེད}\\
\midrule
|Qomolangma-UchenSutung.ttf|&{\uchensutung དབུ་མེད }\\
\midrule
|TibMachUni.ttf| &{\TMU དབུ་མེད }\\
\midrule
|Microsoft Himalaya| &{\himalaya དབུ་མེད ཽ}\\
\bottomrule
\end{tabular}

}
\bigskip

\bgroup
\LARGE\tsuring
\noindent༆ །ཨ་ཡིག་དཀར་མཛེས་ལས་འཁྲུངས་ཤེས་བློ  འི་\par
གཏེར༑ །ཕས་རྒོལ་ཝ་སྐྱེས་ཟིལ་གནོན་གདོང་ལྔ་བཞིན།།\par
ཆགས་ཐོགས་ཀུན་བྲལ་མཚུངས་མེད་འཇམ་དབྱངསམཐུས།།\par
མཧཱ་མཁས་པའི་གཙོ་བོ་ཉིད་འགྱུར་ཅིག། །མངྒལཾ༎\par
\egroup

\subsubsection{Tibetan numbers}
\cxset{language=tibetan, tibetan font = TibMachUni.ttf}

{
\obeylines
\small
TIBETAN DIGIT ZERO	༠
TIBETAN DIGIT ONE	༡	
TIBETAN DIGIT TWO	༢	
TIBETAN DIGIT THREE	༣	
TIBETAN DIGIT FOUR	༤	
TIBETAN DIGIT FIVE	༥	
TIBETAN DIGIT SIX	༦	
TIBETAN DIGIT SEVEN	༧	
TIBETAN DIGIT EIGHT	༨	
TIBETAN DIGIT NINE	༩	
TIBETAN DIGIT HALF ONE	\tibetan༪	
TIBETAN DIGIT HALF TWO	༫	
TIBETAN DIGIT HALF THREE	༬
TIBETAN DIGIT HALF FOUR ༭	
TIBETAN DIGIT HALF FIVE ༯	
TIBETAN DIGIT HALF SIX	 ༯	
TIBETAN DIGIT HALF SEVEN	༰	
TIBETAN DIGIT HALF EIGHT	༱	
TIBETAN DIGIT HALF NINE	༲	
TIBETAN DIGIT HALF ZERO	༳	
}


Tibetan numbers

The usage is not certain. By some interpretations, this has the value of 9.5. Used only in some traditional contexts, these appear as the last digit of a multidigit number, eg. ༤༬ represents 42.5. These are very rarely used, however, and other uses have been postulated.

\defaulttext

^^A
^^A
^^A

^^A\section{Tamil}
\newfontfamily\tamil[Scale=1.1,Script=Tamil]{code2000.ttf}

\def\tamiltext#1{{\tamil#1}}

The Tamil script (\tamiltext{தமிழ் அரிச்சுவடி} tamiḻ ariccuvaṭi) is an abugida script that is used by the Tamil people in India, Sri Lanka, Malaysia and elsewhere, to write the Tamil language, as well as to write the liturgical language Sanskrit, using consonants and diacritics not represented in the Tamil alphabet.[1] Certain minority languages such as Saurashtra, Badaga, Irula, and Paniya are also written in the Tamil script

The Tamil script has 12 vowels (\tamiltext{உயிரெழுத்து} uyireḻuttu "soul-letters"), 18 consonants (\tamiltext{மெய்யெழுத்து} meyyeḻuttu "body-letters") and one character, the āytam \tamiltext{ஃ (ஆய்தம்)}, which is classified in Tamil grammar as being neither a consonant nor a vowel (\tamiltext{அலியெழுத்து} aliyeḻuttu "the hermaphrodite letter"), though often considered as part of the vowel set (\tamiltext{உயிரெழுத்துக்கள்} uyireḻuttukkaḷ "vowel class"). The script, however, is syllabic and not alphabetic.[3] The complete script, therefore, consists of the thirty-one letters in their independent form, and an additional 216 combinant letters representing a total 247 combinations (\tamiltext{உயிர்மெய்யெழுத்து} uyirmeyyeḻuttu) of a consonant and a vowel, a mute consonant, or a vowel alone. These combinant letters are formed by adding a vowel marker to the consonant. Some vowels require the basic shape of the consonant to be altered in a way that is specific to that vowel. Others are written by adding a vowel-specific suffix to the consonant, yet others a prefix, and finally some vowels require adding both a prefix and a suffix to the consonant. In every case the vowel marker is different from the standalone character for the vowel.
The Tamil script is written from left to right.

Tamil is a Unicode block containing characters for the Tamil, Badaga, and Saurashtra languages of Tamil Nadu India, Sri Lanka, Singapore, and Malaysia. In its original incarnation, the code points U+0B02..U+0BCD were a direct copy of the Tamil characters A2-ED from the 1988 ISCII standard. The Devanagari, Bengali, Gurmukhi, Gujarati, Oriya, Telugu, Kannada, and Malayalam blocks were similarly all based on their ISCII encodings.

\begin{scriptexample}[]{Tamil}
\unicodetable{tamil}{"0B80,"0B90,"0BA0,"0BB0,"0BC0,"0BE0,"0BF0}

\hfill  Typeset with \cmd{\tamil} and \texttt{code2000.ttf}
\end{scriptexample}

\subsection{Tamil Numbers and Numerals}

Originally, Tamils did not use zero, nor did they use positional digits (having separate 
symbols for the numbers 10, 100 and 1000). Symbols for the numbers are similar to 
other Tamil letters, with some minor changes. 

For example, the number 3782 is not written as \tamiltext{௩௭௮௨} as in modern usage. Instead it 
is written as \tamiltext{௩ ௲ ௭ ௱ ௮ ௰ ௨}. This would be read as they are written as 
Three Thousands, Seven Hundreds, Eight Tens, Two; or in Tamil as 
\tamiltext{௩௲௭௱௮௰௨ž}.\footnote{https://cloud.github.com/downloads/raaman/Tamil-Numeral/tamilnumbers.html}

\subsection{Dates}

Once the script is loaded the day, month and year can be loaded using the command  \cmd{\tamildate}, which returns the |\today| formatted as per custom Tamil. 

\begin{center}
\bgroup
\tamil
\begin{tabular}{lll}
day	 &month	&year	\\

௳	&௴	      &௵	\\

u	&mee	      &wa	\\
\egroup
\end{center}











^^A\chapter{Armenian}

\label{s:armenian}\index{Armenian}\index{scripts>Armenian}

As we present the scripts in alphabetic order, the first script we will typeset is in Armenian. There are many fonts available for the language. We use two in the example, the first one is \textit{FreeSans} and the second is \textit{Sylphaen} which is found on Windows Operating systems. The language is not supported by the \pkg{Babel} and partially supported by the \pkgname{Polyglossia}. \tcbdocmarginnote{china revision}

\def\ucfirst#1#2;{\MakeUppercase#1#2}


\def\armeniantest#1#2{
  {\parindent0pt
  \topline \vskip3pt
  \noindent\mbox{
     \ucfirst#1;\hfill\hbox{[\texttt{U+0530-U+058F}]}
  }}
 \nobreak

\begin{minipage}{0.45\textwidth}
\bgroup
%\setotherlanguage{#1}
\begin{#1}
#2
[\today]
\end{#1}
\egroup
\end{minipage}\hspace*{1em}
\begin{minipage}{0.45\textwidth}
\bgroup
  \parindent0pt
  \ttfamily\raggedright
  \string\documentclass\{article\}\par
  \string\usepackage[no-math]\{fontspec\}\\
  \string\newfontfamily\textbackslash#1font[Script=\ucfirst #1;,\\   ~~~~~~~Scale=MatchLowercase]
\{FreeSans\}\par
  \string\begin\{document\}\\
  \string\setotherlanguage\{#1\}\\
  \string\begin\{#1\}\\
  \egroup
\begin{#1}
\hskip10pt\vbox{#2}
\end{#1}
\bgroup
  \ttfamily[\detokenize{\today}]\\
  \string\end\{#1\}\\
  \string\end\{document\}
\egroup
\end{minipage}


\textit{FreeSans}: \url{ http://www.gnu.org/software/freefont/}
}

\armeniantest{armenian}{Բոլոր մարդիկ ծնվում են ազատ ու հավասար իրենց
արժանապատվությամբ ու իրավունքներով։       
Նրանք ունեն բանականություն ու խիղճ և միմյանց
պետք է եղբայրաբար վերաբերվեն։}

The Armenian script was invented around 407 AD, by Mesrop Maštoc, a cleric who wanted to 
translate Greek and Syriac scriptures and liturgical texts into Armenian. The system he devised 
is a pure alphabet, closely modelled on the traditional order of Greek phonetic values, with 
additional graphemes to represent Armenian sounds not found in Greek. The orthography is, 
phonetically, a near perfect representation of the Armenian language, and has remained almost 
entirely unchanged since its invention. In recent times, the letterforms in many Armenian 
typefaces have consciously modelled Latin types in their treatment of serifs, stroke weight and 
stress, and other details. This is the approach that Geraldine adopted for the Sylfaen Armenian, 
in order to harmonise the different scripts within the font. 

This kind of harmonisation has to be 
very carefully handled, as there is, of course, a point at which one can corrupt the normative 
letterforms and produce something which will be unacceptable to native readers. Once again, 
we sought expert review of the design, this time from Manvel Shmavonyan, an Armenian type designer, and his Russian colleague Vladimir Yefimov at 
ParaType in Moscow.

\bgroup
\medskip
\fontspec[Script=Armenian,Scale=1.7]{Sylfaen}
\centering

Աա Բբ Գգ Դդ Եե Զզ Էէ Ըը Թթ Ժժ Իի \\
Լլ Խխ Ծծ Կկ Հհ Ձձ Ղղ Ճճ Մմ Յյ Նն \\
Շշ Ոո Չչ Պպ Ջջ Ռռ Սս Վվ Տտ Րր Ցց \\
Ււ Փփ Քք Օօ Ֆֆ / և ﬓ ﬔ ﬕ ﬖ ﬗ\\
\egroup
\captionof{table}{Armenian, showing the basic alphabet (typeset using the \textit{Sylfaen} font.}
\medskip

\bgroup
\def\m#1 #2 #3\\{\makebox[2em]{#1}\makebox[2em]{{\fontspec{code2000.ttf}#2}}\makebox[2em]{\hfill#3 \\ }}
\fontspec[Script=Armenian,Scale=1.1]{Sylfaen}

\begin{multicols}{4}
\m Ա	A	1\\
\m Բ	B	2\\
\m Գ	G	3\\
\m Դ	D	4\\
\m Ե	E	5\\
\m Զ	Z	6\\
\m Է	ē	7\\
\m Ը	ə	8\\
\m Թ	tʿ	9\\
\m Ժ	ž	10\\
\m Ի	I	20\\
\m Լ	L	30\\
\m Խ	X	40\\
\m Ծ	C	50\\
\m Կ	K	60\\
\m Հ	H	70\\
\m Ձ	J	80\\
\m Ղ	ł	90\\
\m Ճ	č	100\\
\m Մ	M	200\\
\m Յ	Y	300\\
\m Ն	N	400\\
\m Շ	š	500\\
\m Ո	O	600\\
\m Չ	čʿ	700\\
\m Պ	P	800\\
\m Ջ	ǰ	900\\
\m Ռ	ṙ	1000\\ 
\m Ս	S	2000\\
\m Վ	V	3000\\
\m Տ	T	4000\\
\m Ր	R	5000\\
\m Ց	cʿ	6000\\
\m Ւ	W	7000\\
\m Փ	pʿ	8000\\
\m Ք	kʿ	9000\\

\end{multicols}
\captionof{table}{Armenian Numerals \textit{(from Wikipedia).}
The first column is the classical Armenian numeral, the second the transliteration and the third the arabic numeral it represents.}

\medskip

Numbers in the Armenian numeral system are obtained by simple addition. Armenian numerals are written left-to-right (as in the Armenian language). Although the order of the numerals is irrelevant since only addition is performed, the convention is to write them in decreasing order of value.

\begin{align*}
\text{ՌՋՀԵ} &= 1975 = 1000 + 900 + 70 + 5\\
\text{ՍՄԻԲ} &= 2222 = 2000 + 200 + 20 + 2\\
\text{ՍԴ}   &= 2004 = 2000 + 4\\
\text{ՃԻ}   &= 120 = 100 + 20\\
\text{Ծ}    &= 50
\end{align*}

To write numbers greater than 9999, it is necessary to have numerals with values greater than 9000. This is done by drawing a line over them, indicating their value is to be multiplied by 10000:

\begin{align*}
\overline{\text{Ա}} &= 10000\\
\overline{\text{Ջ}} &= 9000000\\
\overline{\text{ՌՃԽԳ}}\text{ՌՄԾԵ} &= 11431255
\end{align*}
\egroup

^^A

\section{Bopomofo}
\label{s:bopomofo}
Bopomofo is the colloquial name of the \textit{zhuyin fuhao} or \textit{zhuyin} system of phonetic notation for the transcription of spoken Chinese, particularly the Mandarin dialect. Consisting of 37 characters and four tone marks, it transcribes all possible sounds in Mandarin. 

Bopomofo was introduced in China by the Republican Government, in the 1910s and used alongside the Wade-Giles system, which used a modified Latin alphabet. The Wade system was replaced by \textit{Hanyu Pinyin} in 1958 by the Government of the People's Republic of China,[1] at the International Organization for Standardization (ISO) in 1982 (ISO 7098:1982). Bopomofo remains widely used as an educational tool and electronic input method in Taiwan. On Windows the font Microsoft JhengHei can be used. 

Windows fonts that can be used \texttt{Microsoft JhengHei} and \texttt{SimSun}.

U+3100–U+312F
\newfontfamily\bopomofo{Microsoft JhengHei}

\begin{scriptexample}[]{Bopomofo}
{\centering\bopomofo 

伯帛勃脖舶博渤霸壩灞

}

\hfill \texttt{Typeset with \cmd{\bopomofo} and Microsoft JhengHei font }
\end{scriptexample}

\begin{scriptexample}[]{Bopomofo}

{\centering\bopomofo

伯帛勃脖舶博渤霸壩灞

}
\hfill \texttt{Typeset with \cmd{\bopomofo} and JhengHei font }
\end{scriptexample}


The Bopomofo Extended block, running from \unicodenumber{U+31A0-U31BF}, contains less universally recognized Bopomofo characters used to write various non-Mandarin Chinese languages. A few additional tone marks are unified with characters in the Spacing Modifier Letters block. 










^^A\newfontfamily\georgian[Script=Georgian,Scale=1.2]{code2000.ttf}

\newfontfamily\georgianarial[Script=Georgian,Scale=1.2]{Arial Unicode MS}
\section{Georgian}
\label{sec:georgian}
The Georgian scripts are the three writing systems used to write the Georgian language: Asomtavruli, Nuskhuri and Mkhedruli. Their letters are equivalent, sharing the same names and alphabetical order and all three are unicameral (make no distinction between upper and lower case). Although each continues to be used, Mkhedruli (see below) is taken as the standard for Georgian and its related Kartvelian languages\footnote{Unicode Standard, V. 6.3. U10A0, p. 3}. 

\bgroup
\topline



\begin{scriptexample}[]{}
\georgian 

\centering
 
ყველა ადამიანი იბადება თავისუფალი და თანასწორი თავისი ღირსებითა და უფლებებით. მათ მინიჭებული აქვთ გონება და სინდისი და ერთმანეთის მიმართ უნდა იქცეოდნენ ძმობის სულისკვეთებით.
\medskip

\georgianarial
ყველა ადამიანი იბადება თავისუფალი და თანასწორი თავისი ღირსებითა და უფლებებით. მათ მინიჭებული აქვთ გონება და სინდისი და ერთმანეთის მიმართ უნდა იქცეოდნენ ძმობის სულისკვეთებით.
\bottomline
\captionof{table}{Article 1 of the Universal Declaration of Human Rights in Georgian, typeset in \texttt{code2000} (top) and \texttt{Arial Unicode MS } (bottom).}

\end{scriptexample}

The scripts originally had 38 letters. Georgian is currently written in a 33-letter alphabet, as five of the letters are obsolete in that language. The Mingrelian alphabet uses 36: the 33 of Georgian, one letter obsolete for that language, and two additional letters specific to Mingrelian and Svan. That same obsolete letter, plus a letter borrowed from Greek, are used in the 35-letter Laz alphabet. The fourth Kartvelian language, Svan, is not commonly written, but when it is it uses the letters of the Mingrelian alphabet, with an additional obsolete Georgian letter and sometimes supplemented by diacritics for its many vowels.

^^A
^^A\section{Malayalam}
\label{sec:malayam}
\newfontfamily\malayam[Scale=1.1]{Lohit-Malayalam.ttf}

\def\malamtext#1{{\malayam#1}}

The Malayalam script (Malayalam: \malamtext{മലയാളലിപി}, Malayāḷalipi, IPA: [mɐləjaːɭɐ lɪβɪ], also known as Kairali script (Malayalam: \malamtext{കൈരളീലിപി}), is a Brahmic script used commonly to write the Malayalam language—which is the principal language of the Indian state of Kerala, spoken by 35 million people in the world.[3] Like many other Indic scripts, it is an alphasyllabary (\textit{abugida}), a writing system that is partially “alphabetic” and partially syllable-based. The modern Malayalam alphabet has 15 vowel letters, 41 consonant letters, and a few other symbols. The Malayalam script is a Vattezhuttu script, which had been extended with Grantha script symbols to represent Indo-Aryan loanwords.[4] The script is also used to write several minority languages such as Paniya, Betta Kurumba, and Ravula.[5] The Malayalam language itself was historically written in several different scripts.

\begin{scriptexample}[]{Malayalam}
\centerline{\Huge\malamtext{കൈരളീലിപി}}
\end{scriptexample}
^^A\subsection{Greek}
\index{languages>Greek}\index{Herodotus}\index{alphabets>Greek}
\newfontfamily\greek[Script=Greek,Scale=1.02]{NotoSerif-Regular.ttf}
\def\greektext#1{\greek{#1}}

`The Phoenicians who came with Kadmos,' wrote Herodotus in the fifth century BC of the legendary Phoenician prince of Tyre and brother of Europa, `\ldots introduced into Greece, after their settlement in the country, a number of accomplishments of which the most important was writing, an art which probably was unknown to the Greeks until then'. 

The Greek alphabet is the script that has been used to write the Greek language since the 8th century BC.[2] It was derived from the earlier Phoenician alphabet, and was in turn the ancestor of numerous other European and Middle Eastern scripts, including Cyrillic and Latin.[3] Apart from its use in writing the Greek language, both in its ancient and its modern forms, the Greek alphabet today also serves as a source of technical symbols and labels in many domains of mathematics, science and other fields.

In its classical and modern forms, the alphabet has 24 letters, ordered from alpha to omega. Like Latin and Cyrillic, Greek originally had only a single form of each letter; it developed the letter case distinction between upper-case and lower-case forms in parallel with Latin during the modern era.

\bgroup
\greek\obeyspaces

Α	ἄλφα	aleph	alpha	[alpʰa]	[ˈalfa]	Listeni/ˈælfə/
Β	βῆτα	beth	beta	[bɛːta]	[ˈvita]	/ˈbiːtə/, US /ˈbeɪtə/
Γ	γάμμα	gimel	gamma	[ɡamma]	[ˈɣama]	/ˈɡæmə/
Δ	δέλτα	daleth	delta	[delta]	[ˈðelta]	/ˈdɛltə/
Η	ἦτα	  heth	   eta	 [hɛːta], [ɛːta]	[ˈita]	/ˈiːtə/, US /ˈeɪtə/
Θ	θῆτα	teth	theta	[tʰɛːta]	[ˈθita]	/ˈθiːtə/, US Listeni/ˈθeɪtə/
Ι	ἰῶτα	yodh	iota	[iɔːta]	[ˈʝota]	Listeni/aɪˈoʊtə/
Κ	κάππα	kaph	kappa	[kappa]	[ˈkapa]	Listeni/ˈkæpə/
Λ	λάμβδα	lamedh	lambda	[lambda]	[ˈlamða]	Listeni/ˈlæmdə/
Μ	μῦ	mem	mu	[myː]	[mi]	Listeni/ˈmjuː/; occasionally US /ˈmuː/
Ν	νῦ	nun	nu	[nyː]	[ni]	/ˈnjuː/ (US /ˈnuː/)
Ρ	ῥῶ	reš	rho	[rɔː]	[ro]	Listeni/ˈroʊ/
Τ	ταῦ	taw	tau	[tau]	[taf]	/ˈtaʊ/ or /ˈtɔː/

\topline
\begin{quote}
Ἡροδότου Ἁλικαρνησσέος ἱστορίης ἀπόδεξις ἥδε, ὡς μήτε τὰ γενόμενα ἐξ ἀνθρώπων τῷ χρόνῳ ἐξίτηλα γένηται, μήτε ἔργα μεγάλα τε καὶ θωμαστά, τὰ μὲν Ἕλλησι, τὰ δὲ βαρβάροισι ἀποδεχθέντα, ἀκλεᾶ γένηται, τὰ τε ἄλλα καὶ δι' ἣν αἰτίην ἐπολέμησαν ἀλλήλοισι.[2]

Herodotus of Halicarnassus, his Researches are set down to preserve the memory of the past by putting on record the astonishing achievements of both the Greeks and the Barbarians; and more particularly, to show how they came into conflict.[3]
\end{quote}
\bottomline

\symbol{"1F00}
\symbol{"1F01}
\egroup
^^A
^^A\subsection{Kannada alphabet}

\newfontfamily\kannada[Scale=1.0,Script=Kannada]{Lohit-Kannada.ttf}

\def\kannadatext#1{{\kannada#1}}

The Kannada alphabet (\kannadatext{ಕನ್ನಡ ಲಿಪಿ}) is an abugida of the Brahmic family,[2] used primarily to write the Kannada language, one of the Dravidian languages of southern India. Several minor languages, such as Tulu, Konkani, Kodava, and Beary, also use alphabets based on the Kannada script.[3] The Kannada and Telugu scripts share high mutual intellegibility with each other, and are often considered to be regional variants of single script. Similarly, Goykanadi, a variant of Old Kannada, has been historically used to write Konkani in the state of Goa.[4]

\begin{scriptexample}[]{Kannada}
\centerline{\LARGE\kannadatext{ಙ	ಙ್ಕ	ಙ್ಖ	ಙ್ಗ	ಙ್ಘ	ಙ್ಙ	ಙ್ಚ	ಙ್ಛ	ಙ್ಜ	ಙ್ಝ	ಙ್ಞ	ಙ್ಟ	ಙ್ಠ	ಙ್ಡ	ಙ್ಢ}}
\end{scriptexample}

\medskip

The Kannada script (aksharamale or varnamale) is a phonemic abugida of forty-nine letters, and is written from left to right. The character set is almost identical to that of other Brahmic scripts. Consonantal letters imply an inherent vowel. Letters representing consonants are combined to form digraphs (ottaksharas) when there is no intervening vowel. Otherwise, each letter corresponds to a syllable.
The letters are classified into three categories: swara (vowels), vyanjana (consonants), and yogavaahaka (part vowel, part consonant).
The Kannada words for a letter of the script are akshara, akkara, and varna. Each letter has its own form (ākāra) and sound (shabda), providing the visible and audible representations, respectively. Kannada is written from left to right.[7]
^^A\section{Myanmar}
\label{s:myanmar}
\index{Myanmar}\index{Burmese}\index{Mon}\index{Unicode>Myanmar}\index{Fonts>Padauk}

%\newfontfamily\myanmar{Padauk}

The Burmese script (Burmese:{\myanmar မြန်မာအက္ခရာ}; MLCTS: mranma akkha.ra; pronounced: [mjəmà ʔɛʔkʰəjà]) is an abugida in the Brahmic family, used for writing Burmese. It is an adaptation of the Old Mon script[2] or the Pyu script. In recent decades, other alphabets using the Mon script, including Shan and Mon itself, have been restructured according to the standard of the now-dominant Burmese alphabet. Besides the Burmese language, the Burmese alphabet is also used for the liturgical languages of Pali and Sanskrit.

The characters are rounded in appearance because the traditional palm leaves used for writing on with a stylus would have been ripped by straight lines.[3] It is written from left to right and requires no spaces between words, although modern writing usually contains spaces after each clause to enhance readability.

The earliest evidence of the Burmese alphabet is dated to 1035, while a casting made in the 18th century of an old stone inscription points to 984.[1] Burmese calligraphy originally followed a square format but the cursive format took hold from the 17th century when popular writing led to the wider use of palm leaves and folded paper known as parabaiks.[3] The alphabet has undergone considerable modification to suit the evolving phonology of the Burmese language.

Mon/Burmese script was added to the Unicode Standard in September, 1999 with the release of version 3.0. It was extended in October, 2009 with the release of version 5.2 and again in June, 2014 with the release of version 7.0.

\begin{docKey}[phd]{myanmar font}{=\meta{font name}}{default none initial Padauk}
Loads the font and creates associated environments and commands.
\end{docKey}

\begin{scriptexample}[]{Myanmar}
\unicodetable{myanmar}{"1000,"1010,"1020,"1030,"1040,"1050,"1060,"1070,"1080,"1090}
\end{scriptexample}







^^A
^^A\subsection{Osmanian Alphabet}

\bgroup
\newfontfamily\osmanian{code2001.ttf}
\osmanian
𐒚𐒁𐒖𐒄 𐒚𐒐 𐒚 𐒎𐒚𐒍𐒚𐒐 𐒑𐒚𐒒𐒠𐒚𐒐 𐒎𐒚𐒑𐒁𐒗 𐒚𐒁𐒖𐒄 𐒚𐒌𐒖𐒄 𐒚𐒁𐒖𐒄𐒖 𐒚
𐒌𐒜
\egroup
^^A\newfontfamily\hanunoo{NotoSansHanunoo-Regular.ttf}

\section{Hanunó’o}

Hanunó’o is one of the indigenous scripts of the Philippines and is used by the Mangyan peoples of southern Mindoro to write the Hanunó'o language.[1] 

It is an \emphasis{abugida} descended from the Brahmic scripts, closely related to Baybayin, and is famous for being written vertical but written upward, rather than downward as nearly all other scripts (however, it's read horizontally left to right). It is usually written on bamboo by incising characters with a knife.[2][3] Most known Hanunó'o inscriptions are relatively recent because of the perishable nature of bamboo. It is therefore difficult to trace the history of the script



\begin{scriptexample}[width=2cm]{Hanunoo}
\hanunoo

{\Large
\obeylines
ᜠ 
ᜫ
ᜨᜲ
ᜫᜲ
ᜰ
ᜮ
ᜥ
ᜦ᜴}

Typeset with \texttt{NotoSansHanunoo-Regular.ttf} and the command \cmd{\hanunoo}
\end{scriptexample}

Vertically positionning the text is not currently supported by \pkgname{fontspec} and the manual says \textsc{Todo!}. You are your own here, or you can just put the characters in a box and give it a try.

\begin{minipage}[t]{2cm}
\begin{tcolorbox}[width=2cm,colback=graphicbackground,
boxrule=0pt,toprule=0pt,colframe=white]
\Large\hanunoo
ᜩ\\
ᜤ\\
ᜮ\\
ᜥᜳ\\
ᜨ᜴ \\
ᜨ᜴\\
ᜫᜳ\\
ᜥ\\
\end{tcolorbox}
\end{minipage}
\begin{minipage}[t]{2cm}
\begin{tcolorbox}[width=2cm,colback=graphicbackground,
boxrule=0pt,toprule=0pt,colframe=white]
\LARGE\hanunoo
ᜩ\\
ᜤ\\
ᜮ\\
ᜥᜳ\\
ᜨ᜴ \\
ᜨ᜴\\
ᜫᜳ\\
ᜥ\\
\end{tcolorbox}
\end{minipage}
\begin{minipage}[t]{\textwidth-6cm}

The script is written from bottom to top. Typesetting this type of script automatically is not without its problems. One way is to use the build-in features of the font if they are available, but currently this gives problems---at least with the fonts that I have tried. Entering the text is also problematic as you will more than likely see little boxes rather than the actual glyph with most text editors common to \latexe. If you only need a couple of characters or a short sentence, an easy solution is to use |\rotatebox|. Another solution is to use a macro that can add the letters onto a stack, then place them in a box with a limited width. We can use |\@tfor| for this.  
\end{minipage}
^^A
^^A\newfontfamily\glagolitic{MPH 2B Damase}

\section{Glagolitic}

\epigraph{The average Ph.D. thesis is nothing but a transference of bones from one graveyard to another.}{%
J. Frank Dobie (1888-1964)}


\label{s:glagolitic}
\fboxrule0pt\fboxsep0pt

\noindent
The Glagolitic alphabet /{\glagolitic ˌɡlæɡɵˈlɪtɨk/}, also known as Glagolitsa, is the oldest known Slavic alphabet, from the 9th century.

It was created in the 9th century by Saint Cyril, a Byzantine monk from Thessaloniki. He and his brother, Saint Methodius, were sent by the Byzantine Emperor Michael III in 863 to Great Moravia to spread Christianity among the West Slavs in the area. The brothers decided to translate liturgical books into the Old Slavic language that was understandable to the general population, but as the words of that language could not be easily written by using either the Greek or Latin alphabets, Cyril decided to invent a new script, Glagolitic, which he based on the local dialect of the Slavic tribes from the Byzantine Salonika region.
After the deaths of Cyril and Methodius, the Glagolitic alphabet ceased to be used in Moravia, but their students continued to propagate it in the west and south. 

After a long career, Glagolitic writing stopped being used, except for
religious purposes in certain dioceses of Bosnia and Dalmatia (Croatia).
The Cyrillic alphabet was adopted by all Orthodox Slays and served to note
their literary language. Most of the Slays who rallied to Rome rejected it,
however, which created the paradoxical situation in ex-Yugoslavia, where
two peoples who speak the same language write in different scripts, the
Serbs in Cyrillic and the Croats with Roman characters. Finally, as is
known, the ex-Soviet Union did much to put into writing the languages
spoken by the peoples within its borders, for the most part noting them in
adaptations of the Cyrillic alphabet, while Russian became the language of
culture throughout the Soviet Union.\cite{henri1994}

Slavic printing in Glagolitic characters originated in Venice, where a
\textit{Sluzebnik} (or \textit{Leitourgikon}) was published in 1483, followed by missals and
breviaries, all printed by Andrea Torresani, the future father-in-law and
associate of Aldus Manutius. After 1494 some attempts were made to create
printshops in Croatia itself, first in Senj in 1508, then, after 1530, in
Rijeka (Fiume). The work of these firms was almost totally liturgical (religious,
at any rate), and it had strong competition from manuscript works
that were better adapted to the diversity of local liturgical customs. Religion
also dictated the output of a printshop founded to provide Protestant propaganda
that was set up in Tubingen between 1560 and 1564 by Baron
Hans von Ungnad and that printed the great Lutheran texts in Glagolitic
characters.\footfullcite{henri1994}

Figure~\ref{fig:zograf} illustrates an example of the language.\footnote{\url{https://en.wikipedia.org/wiki/Glagolitic_script\#/media/File:ZographensisColour.jpg}}

\begin{figure}[htbp]
\centering

\includegraphics[width=0.45\linewidth]{glagolitic}
\caption[The first page of the Gospel of Mark from the 10th–11th century Codex Zographensis, found in the Zograf Monastery in 1843.]{The first page of the Gospel of Mark from the 10th–11th century Codex Zographensis, found in the Zograf Monastery in 1843.}
\label{fig:zograf}
\end{figure}

\section{Unicode Support}
The Glagolitic alphabet was added to the Unicode Standard in March 2005 with the release of version 4.1.
The Unicode block for Glagolitic is U+2C00–U+2C5F.



\begin{scriptexample}[]{glacolitic}

\unicodetable{glagolitic}{%
"2C00,"2C10,"2C20,"2C30,"2C40,"2C50}

\texttt{typeset with Damase version 2.0 MPH 2B Damase}
\end{scriptexample}
\bgroup
\glagolitic

The name was not coined until many centuries after its creation, and comes from the Old Church Slavonic glagolъ "utterance" (also the origin of the Slavic name for the letter G). The verb glagoliti means "to speak". It has been conjectured that the name glagolitsa developed in Croatia around the 14th century and was derived from the word glagolity, applied to adherents of the liturgy in Slavonic.[1]

In Old Church Slavonic the name is {\glagolitic ⰍⰫⰓⰊⰎⰎⰑⰂⰋⰜⰀ}, Кѷрїлловица.
The name Glagolitic in Bulgarian, Russian, Macedonian глаголица (glagolica), Belarusian is глаголіца (hłaholica), Croatian glagoljica, Serbian глагољица / glagoljica, Czech hlaholice, Polish głagolica, Slovene glagolica, Slovak hlaholika, and Ukrainian глаголиця (hlaholyća).



\egroup

\section{Additional Modern Scripts}

\begin{center}
\begin{tabular}{lp{5cm}l}
Ethiopic. &Vai. &Deseret.\\
Mongolian. &Bamum. &Shavian.\\
Osmanya.   &Cherokee. &Lisu.\\
Tifinagh.  &Canadian Aboriginal Syllabics. &Miao.\\
N’Ko.&&\\
\end{tabular}
\end{center}

Ethiopic, Mongolian, and Tifinagh are scripts with long histories. Although their roots can
be traced back to the original Semitic and North African writing systems, they would not
be classified as Middle Eastern scripts today

The Cherokee script is a syllabary developed between 1815 and 1821, to write the Cherokee
language, still spoken by small communities in Oklahoma and North Carolina. Canadian
Aboriginal Syllabics were invented in the 1830s for Algonquian languages in Canada. The
system has been extended many times, and is now actively used by other communities, including speakers of Inuktitut and Athapascan languages.

Deseret is a phonemic alphabet devised in the 1850s to write English. It saw limited use for
a few decades by members of The Church of Jesus Christ of Latter-day Saints. Shavian is
another phonemic alphabet, invented in the 1950s to write English. It was used to publish
one book in 1962, but remains of some current interest




\subsection{Ethiopic}
Ge'ez (ግዕዝ Gəʿəz), (also known as Ethiopic) is a script used as an abugida (syllable alphabet) for several languages of Ethiopia and Eritrea. It originated as an abjad (consonant-only alphabet) and was first used to write Ge'ez, now the liturgical language of the Ethiopian Orthodox Tewahedo Church and the Eritrean Orthodox Tewahedo Church. In Amharic and Tigrinya, the script is often called fidäl (ፊደል), meaning "script" or "alphabet".

The Ge'ez script has been adapted to write other, mostly Semitic, languages, particularly Amharic in Ethiopia, and Tigrinya in both Eritrea and Ethiopia. It is also used for Sebatbeit, Me'en, and most other languages of Ethiopia. In Eritrea it is used for Tigre, and it has traditionally been used for Blin, a Cushitic language. Tigre, spoken in western and northern Eritrea, is considered to resemble Ge'ez more than do the other derivative languages.[citation needed] Some other languages in the Horn of Africa, such as Oromo, used to be written using Ge'ez, but have migrated to Latin-based orthographies.
For the representation of sounds, this article uses a system that is common (though not universal) among linguists who work on Ethiopian Semitic languages. This differs somewhat from the conventions of the International Phonetic Alphabet. See the articles on the individual languages for information on the pronunciation.

There are a number of fonts available and we have selected the Google \idxfont{NotoSansEthiopic}
\newfontfamily\ethiopic{NotoSansEthiopic-Bold.ttf}

\begin{scriptexample}[]{Ethiopic}
\unicodetable{ethiopic}{"1200,"1210,"1220,"1230,"1240,"1250,"1260,"1270,"1280,"1290,^^A
"12A0,"12B0,"12C0,"12E0,"12F0,"1300,"1310,"1330,"1340,"1350,"1360,"1370}
\end{scriptexample}
\section{Vai}
\label{s:vai}

The Vai syllabary is a syllabic writing system devised for the Vai language by Momolu Duwalu Bukele of Jondu, in what is now Grand Cape Mount County, Liberia.[1] [2] Bukele is regarded within the Vai community, as well as by most scholars, as the syllabary's inventor and chief promoter when it was first documented in the 1830s. It is one of the two most successful indigenous scripts in West Africa.

\newfontfamily\vai{code2000.ttf}
\begin{scriptexample}[]{Vai}
\unicodetable{vai}{"A500,"A510,"A520,"A530,"A540,"A550,"A560,"A570,^^A
"A580,"A590,"A5A0,"A5B0,^^A
"A5C0,"A5D0,"A5E0,"A5F0,"A610,"A620,"A630}
\end{scriptexample}

In the 1920s ten decimal digits were devised for Vai; these were “Vai-style” glyph variants of
European digits (see Figure 11). They were not popular with Vai people  even for historical purposes. All
the modern literature uses European digits.


\begin{scriptexample}[]{Vai}
\bgroup
\vai
\obeylines\Large
0	1	2	3	4	5	6	7	8	9
꘠	꘡	꘢	꘣	꘤	꘥	꘦	꘧	꘨	꘩
\vai
\egroup
\end{scriptexample}



\printunicodeblock{./languages/vai.txt}{\vai}
\section{Deseret script}
\newfontfamily\deseret{code2001.ttf}

The Deseret alphabet (dɛz.əˈrɛt.) (Deseret: {\deseret 𐐔𐐯𐑅𐐨𐑉𐐯𐐻 or 𐐔𐐯𐑆𐐲𐑉𐐯𐐻}) is a phonemic English spelling reform developed in the mid-19th century by the board of regents of the University of Deseret (later the University of Utah) under the direction of Brigham Young, second president of The Church of Jesus Christ of Latter-day Saints.

In public statements, Young claimed the alphabet was intended to replace the traditional Latin alphabet with an alternative, more phonetically accurate alphabet for the English language. This would offer immigrants an opportunity to learn to read and write English, he said, the orthography of which is often less phonetically consistent than those of many other languages. Similar experiments were not uncommon during the period, the most well-known of which is the Shavian alphabet.

Young also prescribed the learning of Deseret to the school system, stating "It will be the means of introducing uniformity in our orthography, and the years that are now required to learn to read and spell can be devoted to other studies".[2]


Deseret script {\deseret 𐐔𐐯𐑅𐐨𐑉𐐯𐐻}  [U+10400-U+1044F]
\medskip

\bgroup
\par
\noindent
\colorbox{graphicbackground}{\color{black}^^A
\begin{minipage}{\textwidth}^^A
\parindent1pt
\vskip10pt
\leftskip10pt \rightskip\leftskip
\deseret
\large

𐐂 𐑌𐐲𐑉𐑅𐐨𐑉𐐮 𐐮𐑆 𐐪 𐐹𐐨𐑅 𐐱𐑂 𐑊𐐰𐑌𐐼 𐐱𐑌 𐐸𐐶𐐮𐐽 𐑁𐑉𐐭𐐻𐐻𐑉𐐨𐑆 𐐪𐑉 𐑅𐐻𐐪𐑉𐐻𐐯𐐼,


\par
\vspace*{10pt}
\end{minipage}
}

Text: Deseret alphabet http://www.omniglot.com/writing/deseret.htm
\medskip
\egroup

\PrintUnicodeBlock{./languages/deseret.txt}{\deseret}

\chapter{Bamum}
\label{s:bamum}
\epigraph{"No known alphabet was ever invented by a European."}{Jeffreys' translation from the Royal script.}

\label{s:bamum}
\index{scripts>Bamum}
\newfontfamily\bamum{NotoSansBamum-Regular.ttf}

The Bamum scripts are an evolutionary series of six scripts created for the Bamum language by King Njoya of Cameroon at the turn of the 20th century. They are notable for evolving from a pictographic system to a partially alphabetic syllabic script in the space of 14 years, from 1896 to 1910. Bamum type was cast in 1918, but the script fell into disuse around 1931.

\begin{figure}[htbp]
\parindent=0pt

\centering

\includegraphics[width=\textwidth]{bamum}

\caption{King Njoya of Bamum receiving an oil painting of Kaiser Wilhelm II. The gift was in return for his support in the German campaign against the Nso'.}
\end{figure}

The Bamum, sometimes called Bamoum, Bamun, Bamoun, or Mum, are a Bantoid ethnic group of Cameroon with around 215,000 members.



\begin{scriptexample}[]{Bamum}
\unicodetable{bamum}{"A6A0,"A6B0,"A6C0,"A6D0,"A6E0,"A6F0}
\end{scriptexample}
\section{Shavian}
\label{s:shavian}
\def\shaviansetup#1{}
\newfontfamily\shavian{code2001.ttf}
^^A\newfontfamily\shavian{NotoSansShavian-Regular.ttf}
\cxset{shavian font/.code=\shaviansetup{#1}}
\cxset{shavian font=shavian}




\begin{scriptexample}[]{shavian}
\shavian

𐑳 𐑡𐑻𐑯𐑰 𐑑 𐑞 𐑕𐑧𐑯𐑑𐑻 𐑝 𐑞 𐑻𐑔
𐑚𐑲 - ·𐑡𐑵𐑤𐑟 ·𐑝𐑻𐑯

𐑗𐑩𐑐𐑑𐑻 1 - 𐑥𐑲 𐑳𐑙𐑒𐑳𐑤 𐑥𐑱𐑒𐑕 𐑳 𐑜𐑮𐑱𐑑 𐑛𐑦𐑕𐑒𐑳𐑝𐑻𐑰

     𐑤𐑫𐑒𐑦𐑙 𐑚𐑩𐑒 𐑑 𐑷𐑤 𐑞𐑩𐑑 𐑣𐑩𐑟 𐑳𐑒𐑻𐑛 𐑑 𐑥𐑰 𐑕𐑦𐑯𐑕 𐑞𐑩𐑑 𐑦𐑝𐑧𐑯𐑑𐑓𐑳𐑤 𐑛𐑱, 𐑲 𐑩𐑥 𐑕𐑒𐑧𐑮𐑕𐑤𐑰 𐑱𐑚𐑳𐑤 𐑑 𐑚𐑦𐑤𐑰𐑝 𐑦𐑯 𐑞 𐑮𐑰𐑩𐑤𐑳𐑑𐑰 𐑝 𐑥𐑲 𐑩𐑛𐑝𐑧𐑯𐑗𐑻𐑟. 𐑞𐑱 𐑢𐑻 𐑑𐑮𐑵𐑤𐑰 𐑕𐑴 𐑢𐑳𐑯𐑛𐑻𐑓𐑳𐑤 𐑞𐑩𐑑 𐑰𐑝𐑦𐑯 𐑯𐑬 𐑲 𐑩𐑥 𐑚𐑦𐑢𐑦𐑤𐑛𐑻𐑛 𐑢𐑧𐑯 𐑲 𐑔𐑦𐑙𐑒 𐑝 𐑞𐑧𐑥.
     𐑥𐑲 𐑳𐑙𐑒𐑳𐑤 𐑢𐑪𐑟 𐑳 𐑡𐑻𐑥𐑳𐑯, 𐑣𐑩𐑝𐑦𐑙 𐑥𐑧𐑮𐑰𐑛 𐑥𐑲 𐑥𐑳𐑞𐑻𐑟 𐑕𐑦𐑕𐑑𐑻, 𐑩𐑯 𐑦𐑙𐑜𐑤𐑦𐑖𐑢𐑫𐑥𐑳𐑯. 𐑚𐑰𐑦𐑙 𐑝𐑧𐑮𐑰 𐑥𐑳𐑗 𐑳𐑑𐑩𐑗𐑑 𐑑 𐑣𐑦𐑟 𐑓𐑪𐑞𐑻𐑤𐑳𐑕 𐑯𐑧𐑓𐑘𐑵, 𐑣𐑰 𐑦𐑯𐑝𐑲𐑑𐑳𐑛 𐑥𐑰 𐑑 𐑕𐑑𐑳𐑛𐑰 𐑳𐑯𐑛𐑻 𐑣𐑦𐑥 𐑦𐑯 𐑣𐑦𐑟 𐑣𐑴𐑥 𐑦𐑯 𐑞 𐑓𐑪𐑞𐑻𐑤𐑩𐑯𐑛. 𐑞𐑦𐑕 𐑣𐑴𐑥 𐑢𐑪𐑟 𐑦𐑯 𐑳 𐑤𐑪𐑮𐑡 𐑑𐑬𐑯, 𐑯 𐑥𐑲 𐑳𐑙𐑒𐑳𐑤 𐑳 𐑐𐑮𐑳𐑓𐑧𐑕𐑻 𐑝 𐑓𐑳𐑤𐑪𐑕𐑳𐑓𐑰, 𐑒𐑧𐑥𐑳𐑕𐑑𐑮𐑰, 𐑡𐑰𐑪𐑤𐑳𐑡𐑰, 𐑥𐑦𐑯𐑻𐑪𐑤𐑳𐑡𐑰, 𐑯 𐑥𐑧𐑯𐑰 𐑳𐑞𐑻 𐑳𐑤𐑴𐑡𐑰𐑕.

\arial

\hfill Excerpt from Jules Vern,  \textit{Journey to the Center of the Earth from \href{http://shavian.weebly.com/}{shavian}}
\end{scriptexample}

The example is typeset using \texttt{code2001.ttf}. There are numerous fonts that provide Shavian glyphs. \texttt{ESL Gothic Unicode} font by Ethan Lamoreaux\footnote{\url{http://www.fontspace.com/ethan-lamoreaux/esl-gothic-unicode}}. The Noto fonts also have a Shavian font. 

You can activate typesetting in Shavian using the key:

\begin{key}{/chapter/shavian font = \meta{font name}} The key will setup the
default font for the Shavian script and define the commands \cmd{\shavian} and \cmd{\textshavian}. 
\end{key}

\PrintUnicodeBlock{./languages/shavian.txt}{\shavian}





\subsection{Osmanya}

\newfontfamily\osmanya{NotoSansOsmanya-Regular.ttf}

\begin{scriptexample}[]{Osmanya}
\unicodetable{osmanya}{"10480,"10490,"104A0}
\end{scriptexample}

The Osmanya alphabet (Somali: Cismaanya; Osmanya: {\osmanya 𐒋𐒘𐒈𐒑𐒛𐒒𐒕𐒀}), also known as Far Soomaali ("Somali writing"), is a writing script created to transcribe the Somali language. It was invented between 1920 and 1922 by Osman Yusuf Kenadid of the Majeerteen Darod clan, the nephew of Sultan Yusuf Ali Kenadid of the Sultanate of Hobyo.

While Osmanya gained reasonably wide acceptance in Somalia and quickly produced a considerable body of literature, it proved difficult to spread among the population mainly due to stiff competition from the long-established Arabic script as well as the emerging Somali alphabet developed by the Somali linguist, Shire Jama Ahmed, which was based on the Latin script.

As nationalist sentiments grew and since the Somali language had long lost its ancient script,[1] the adoption of a universally recognized writing script for the Somali language became an important point of discussion. After independence, little progress was made on the issue, as opinion was divided over whether the Arabic or Latin scripts should be used instead.

In October 1972, due to its simplicity, the fact that it lent itself well to writing Somali since it could cope with all of the sounds in the language, and the already widespread existence of machines and typewriters designed for its use,[2][3] the government of Somali president Mohamed Siad Barre unilaterally elected to use only the Latin script for writing Somali instead of the Arabic or Osmanya scripts.[4] Barre's administration subsequently launched a massive literacy campaign designed to ensure its sole adoption. This led to a sharp decline in use of Osmanya.
\section{Cherokee}
\index{scripts>Cherokee}
\index{scripts>Cherokee>fonts}
\label{sec:cherokee}
Windows comes with |Plantagenet Cherokee| font. The |code2000| also has good support for the alphabet. The \texttt{SIL font Charis SIL} also has good support and can be downloaded at \href{http://scripts.sil.org/cms/scripts/page.php?item_id=CharisSIL_download}{scripts.sel.org}, the latest version gave me problems when used with Windows. 

  
\def\textcherokee#1{{\cherokee   #1}}


\begin{docKey}[phd]{cherokee font}{ = \meta{font name}} {default none, initial=code2000}
 Loads the font
command \cmd{\cherokee}. When the command is used it typesets text in
cherokee unicode. There is no need to load the language, unless it is the main document language. For windows the default font is  |Plantagenet Cherokee|. Another font is FreeSerif, which we are using here.
\end{docKey}

\begin{scriptexample}[]{Cherokee}
{\cherokee
\begin{tabular}{lp{8.5cm}}
Translation	  &John (ᏣᏂ) 3:16\\
American Bible Society 1860	&ᎾᏍᎩᏰᏃ ᏂᎦᎥᎩ ᎤᏁᎳᏅᎯ ᎤᎨᏳᏒᎩ ᎡᎶᎯ, ᏕᏅᏲᏒᎩ ᎤᏤᎵᎦ ᎤᏪᏥ ᎤᏩᏒᎯᏳ ᎤᏕᏁᎸᎯ, ᎩᎶ ᎾᏍᎩ ᏱᎪᎯᏳᎲᏍᎦ ᎤᏲᎱᎯᏍᏗᏱ ᏂᎨᏒᎾ, ᎬᏂᏛᏉᏍᎩᏂ ᎤᏩᏛᏗ.\\

(Transliteration)	& nasgiyeno nigavgi unelanvhi ugeyusvgi elohi, denvyosvgi utseliga uwetsi uwasvhiyu udenelvhi, gilo nasgi yigohiyuhvsga uyohuhisdiyi nigesvna, gvnidvquosgini uwadvdi.\\
\end{tabular}}
\end{scriptexample}

\begin{texexample}{Using text...}{cherokee}
\bgroup
\cherokee \large\textbf{ᎾᏍᎩᏰᏃ}
\textcherokee{ᎾᏍᎩᏰᏃ}
\egroup
\end{texexample}

If you have trouble getting them to work\footnote{\url{http://tex.stackexchange.com/questions/132087/displaying-cherokee-text}}

\url{http://www.cherokee.org/AboutTheNation/Language/CherokeeFont.aspx}




\section{Tifnagh}

\newfontfamily\tifinagh{code2000.ttf}

Tifinagh (Berber pronunciation: [tifinaɣ]; also written Tifinaɣ in the Berber Latin alphabet, {\tifinagh  ⵜⵉⴼⵉⵏⴰⵖ} in Neo-Tifinagh, and تيفيناغ in the Berber Arabic alphabet) is a series of abjad and alphabetic scripts used by Berber peoples to write Berber languages.[1]
A modern derivate of the traditional script, known as Neo-Tifinagh, was introduced in the 20th century. A slightly modified version of the traditional script, called Tifinagh Ircam, is used in a number of Moroccan elementary schools in teaching the Berber language to children as well as a number of publications.[2][3]

The word tifinagh is thought to be a Berberized feminine plural cognate of Punic, through the Berber feminine prefix ti- and Latin Punicus; thus tifinagh could possibly mean "the Phoenician (letters)"[4][5] or "the Punic letters".

\bgroup

\noindent\tifinagh
\colorbox{thecodebackground}{\color{black}^^A
\begin{minipage}{\textwidth}
\parindent1pt
\vskip10pt
\leftskip10pt \rightskip\leftskip
Tifnagh     ⵜⵉⴼⵉⵏⴰⵖ [U+2D30-U+2D7F]

ⴰⴳⵍⴷⵓⵏ ⴰⵎⵥⵥⴰ

ⵙ ⵡⴰⵡⴰⵍ ⴳⵔⵉ ⵉⴷⵙ, ⵙⵙⵏⵖ ⵢⴰⵜ ⵜⵖⴰⵡⵙⴰ ⵜⵉⵙⵙ ⵙⵏⴰⵜ  ⵉⵅⴰⵜⵔⵏ: ⵉⵜⵔⵉ ⵙⴳ ⴷⴷ ⵉⴷⴷⴰ ⵓⵔ ⵉⵎⵇⵇⵓⵔ, ⵉⵍⵍⴰ ⵖⴰⵙ ⴰⵏⵛⵜ ⵏ ⵢⴰⵜ ⵜⴰⴷⴷⴰⵔⵜ !

ⴰⵢⴰ ⵓⴽⵣⵖ ⵜ. ⵙⵙⵏⵖ ⵉⵙ ⴱⵕⵕⴰ ⵏ ⵉⵜⵔⴰⵏ ⵣⵓⵏⴷ ⴰⴽⴰⵍ, ⵊⵓⴱⵉⵜⵔ, ⵎⴰⵔⵙ, ⴱⵉⵏⵓⵙ – ⵉⵜⵔⴰⵏ ⵎⵉ ⵏⴽⴼⴰ ⵉⵙⵎⴰⵡⵏ – ⵍⵍⴰⵏ ⴷⵉⵖ ⵉⵜⵔⴰⵏ ⵢⴰⴹⵏ ⵎⵥⵥⵉⵢⵏⵉⵏ, ⵡⵉⵏⵏⴰ ⵓⵔ ⵏⵣⵎⵉⵔ ⴰⴷ ⵏⵥⵔ ⵙ ⵓⵜⵉⵍⵉⵙⴽⵓⴱ. ⴰⴷⴷⴰⵢ ⵢⵓⴼⴰ ⵓⴰⵙⵜⵕⵓⵏⵓⵎ ⵢⴰⵏ ⴷⵉⴳⵙⵏ, ⴷⴰ ⵢⴰⵙ ⵉⵜⵜⴳⴰ ⵙ ⵢⵉⵙⵎ ⵢⴰⵏ ⵡⵓⵜⵜⵓⵏ. ⴷⴰ ⵢⴰⵙ ⵉⵇⵇⴰⵔ ⵙ ⵓⵎⴷⵢⴰⵜ : « ⴰⵙⵜⵔⵓⵉⴷ 3251 ».

ⵓⴽⵣⵖ ⵉⵙ ⴷⴷ ⵉⴷⴷⴰ ⵓⴳⵍⴷⵓⵏ ⵎⵥⵥⵉⵢⵏ ⵙⴳ ⵉⵜⵔⵉ ⵎⵉ ⵇⵇⴰⵔⵏ ⴰⵙⵜⵔⵓⵉⴷ ⴱ612. ⴰⵙⵜⵔⵓⵉⴷ ⴰ, ⵓⵔ ⵉⵜⵓⵥⵔⴰ ⴰⵔ 1909 ⵙ ⵓⵜⵉⵍⵉⵙⴽⵓⴱ. ⵉⵥⵔⴰ ⵜ ⵢⴰⵏ ⵓⴰⵙⵜⵕⵓⵏⵓⵎ ⴰⵜⵓⵔⴽⵉⵢ. ⵉⵙⵙⴽⵏ ⵜⵓⴼⴰⵢⵜ ⵏⵏⵙ ⴳ ⵢⴰⵏ ⵓⴳⵔⴰⵡ ⴰⴳⵔⴰⵖⵍⴰⵏ ⵏ ⵍⴰⵙⵜⵕⵓⵏⵓⵎⵢ. ⵎⴰⵛⴰ, ⴰⴽⴷ ⵢⵉⵡⵏ ⵓⵔ ⵜ ⵢⵓⵎⵏ ⴰⵛⴽⵓ ⵉⵍⵍⴰ ⵉⵍⵙⴰ ⵢⴰⵜ ⵎⵍⵙⵉⵡⵜ ⵓⵔ ⵉⴳⵉⵏ ⴰⵎⵎ ⵜⵉⵏ ⵎⴷⴷⵏ. ⵎⴷⴷⵏ ⵉⵎⵇⵔⴰⵏⴻⵏ, ⴰⵎⴽⴰ ⴰⴽⴽ ⴰⵢ ⴳⴰⵏ.

ⵎⴰⵛⴰ ⵙ ⵓⵎⴷⴰⵣ ⵏ ⵜⵓⵙⵙⵏⴰ ⵏ ⴰⵙⵜⵔⵓⵉⴷ ⴱ612, ⵉⴽⴽⵔ ⵢⴰⵏ ⵓⴷⵉⴽⵜⴰⵜⵓⵔ ⴰⵜⵓⵔⴽⵢ, ⵉⴳⴳ ⴰⵙⵏ ⵛⵛⵉⵍ ⵉ ⵎⴷⴷⵏ ⴰⴷ ⵍⵙⵙⴰⵏ ⵎⵍⵙⵉⵡⵜ ⵏ ⵓⵔⵓⴱⵉⵢⵏ, ⵡⴰⵏⵏⴰ ⵢⴰⴳⵉⵏ ⵉⵏⵖ ⵜ. ⴰⵙⵜⵔⵓⵏⵓⵎ ⵏⵏⴰⵖ, ⵢⵓⵍⵙ ⴷⵉⵖ ⵉ ⵜⵎⵙⴽⴰⵏⵜ ⵏⵏⵙ ⴰⵙⴳⴳⴰⵙ ⵏ 1920, ⵜⵉⴽⴽⵍⵜ ⵏⵏⴰⵖ ⵉⵍⵍⴰ ⵉⵍⵙⴰ ⵢⴰⵜ ⵎⵍⵙⵉⵡⵜ ⵢⵖⵓⴷⴰⵏ ⵛⵉⴳⴰⵏ. ⵜⵉⴽⴽⵍⵜ ⵏⵏⴰⵖ, ⵎⴷⴷⵏ ⴰⴽⴽ ⵓⵎⴻⵏ ⴰⵡⴰⵍ ⵏⵏⵙ.
\par
\vspace*{10pt}
\end{minipage}
}

\subsection{Unified Canadian Aboriginal Syllabics}

Unified Canadian Aboriginal Syllabics is a Unicode block containing characters for writing Inuktitut, Carrier, several dialects of Cree, and Canadian Athabascan languages. Additions for some Cree dialects, Ojibwe, and Dene can be found at the Unified Canadian Aboriginal Syllabics Extended block.
\medskip

\newfontfamily\aboriginal{code2000.ttf}
\bgroup
\par
\noindent
\colorbox{graphicbackground}{\color{black}^^A
\begin{minipage}{\textwidth}^^A
\parindent1pt
\vskip10pt
\leftskip10pt \rightskip\leftskip

\aboriginal
ᒥᓯᐌ ᐃᓂᓂᐤ ᑎᐯᓂᒥᑎᓱᐎᓂᐠ ᐁᔑ ᓂᑕᐎᑭᐟ ᓀᐢᑕ ᐯᔭᑾᐣ ᑭᒋ ᐃᔑ
\bfseries ᑲᓇᐗᐸᒥᑯᐎᓯᐟ ᑭᐢᑌᓂᒥᑎᓱᐎᓂᐠ ᓀᐢᑕ ᒥᓂᑯᐎᓯᐎᓇ᙮
Unicode Block: Unified Canadian Aboriginal Syllabics, UCAS Extended
Text: UDHR: Cree, Swampy ᐯᔭᐠ ᐱᐢᑭᑕᓯᓇᐃᑲᐣ ᐁᐢᐱᑕᐢᑲᒥᑲᐠ ᐊᐢᑭᐠ ᑭᒋ ᐃᑗᐎᐣ ᐃᓂᓂᐎ ᒥᓂᑯᐎᓯᐎᓇ ᐅᒋ
\par
\vspace*{10pt}
\end{minipage}
}
\medskip
\egroup
\subsection{Miao}

The Pollard script, also known as Pollard Miao (Chinese: 柏格理苗文 Bó Gélǐ Miao-wen) or Miao, is an abugida loosely based on the Latin alphabet and invented by Methodist missionary Sam Pollard. Pollard invented the script for use with A-Hmao, one of several Miao languages. The script underwent a series of revisions until 1936, when a translation of the New Testament was published using it. The introduction of Christian materials in the script that Pollard invented caused a great impact among the Miao. Part of the reason was that they had a legend about how their ancestors had possessed a script but lost it. According to the legend, the script would be brought back some day. When the script was introduced, many Miao came from far away to see and learn it.[1][2]

Pollard credited the basic idea of the script to the Cree syllabics designed by James Evans in 1838–1841, “While working out the problem, we remembered the case of the syllabics used by a Methodist missionary among the Indians of North America, and resolved to do as he had done” (1919:174). He also gave credit to a Chinese pastor, “Stephen Lee assisted me very ably in this matter, and at last we arrived at a system” (1919:174). In listing the phrases he used to describe devising the script, there is clear indication of intellectual work, not revelation: “we looked about”, “resolved to attempt”, “adapting the system”, “solved our problem” (Pollard 1919:174,175).

Changing politics in China led to the use of several competing scripts, most of which were romanizations. The Pollard script remains popular among Hmong in China, although Hmong outside China tend to use one of the alternative scripts. A revision of the script was completed in 1988, which remains in use.

As with most other abugidas, the Pollard letters represent consonants, whereas vowels are indicated by diacritics. Uniquely, however, the position of this diacritic is varied to represent tone. For example, in Western Hmong, placing the vowel diacritic above the consonant letter indicates that the syllable has a high tone, whereas placing it at the bottom right indicates a low tone.

A still experimental font, that supports Graphite technology is \idxfont{Mia Unicode}\footnote{\url{http://phjamr.github.io/miao.html\#intro}}. The font is licenced under the SIL terms and we are using it in the |phd| package as the default font for the Miao script.

\newfontfamily\miao{MiaoUnicode-Regular.ttf}

\begin{scriptexample}[]{Miao}
\unicodetable{miao}{"16F00,"16F10,"16F20,"16F30,"16F40,"16F70,"16F80,"16F90}
\end{scriptexample}

{\miao 𖼴	𖼵	𖼶	𖼷	𖼸	𖼹	𖼺	}

Features for Miao
There are three features currently available for the Miao script:
\bgroup
\miao
Chuxiong ‘wart’ variant
Stylistic alternates for 𖼳 and 𖼴
Aspiration marker always on right
The ‘wart’ (a translated technical term!) is the small circle in characters like 𖼁, 𖼅, and 𖼾. In the Chuxiong orthography, it is rendered not as a circle but as a dot on the right of the letter, as shown in point 5 here (pdf).

Miao Unicode has a feature called “chux” for handling this. In LibreOffice you can use this style by typing “Miao Unicode:chux=1” into the font field.
\section{N'ko}

\newfontfamily\nko{NotoSansNKo-Regular.ttf}

N'Ko {\nko(ߒߞߏ)} is both a script devised by Solomana Kante in 1949 as a writing system for the Manding languages of West Africa, and the name of the literary language itself written in the script. The term N'Ko means ``I say'' in all Manding languages.

The script has a few similarities to the Arabic script, notably its direction (right-to-left) and the connected letters. It obligatorily marks both tone and vowels.


\begin{scriptexample}[]{N'ko}
\unicodetable{nko}{"07C0,"07D0,"07E0,"07F0}
\end{scriptexample}

The N'Ko alphabet is written from right to left, with letters being connected to one another.

The script is principally used in Guinea and Côte d'Ivoire (respectively by Maninka and Dioula-speakers), with an active user community in Mali (by Bambara-speakers). Publications include a translation of the Qur'an, a variety of textbooks on subjects such as physics and geography, poetic and philosophical works, descriptions of traditional medicine, a dictionary, and several local newspapers. It has been classed as the most successful of the West African scripts.[3] The literary language used is intended as a koine blending elements of the principal Manding languages (which are mutually intelligible), but has a particularly strong Maninka flavour.

The Latin script with several extended characters (phonetic additions) is used for all Manding languages to one degree or another for historical reasons and because of its adoption for "official" transcriptions of the languages by various governments. In some cases, such as with Bambara in Mali, promotion of literacy using this orthography has led to a fair degree of literacy in it. Arabic transcription is commonly used for Mandinka in The Gambia and Senegal.


\subsection{Mongolian}
\newfontfamily\mongolian{NotoSansMongolian-Regular.ttf}

The classical Mongolian script (in Mongolian script:{\mongolian ᠮᠣᠩᠭᠣᠯ ᠪᠢᠴᠢᠭ᠌} Mongγol bičig; in Mongolian Cyrillic: Монгол бичиг Mongol bichig), also known as Uyghurjin Mongol bichig, was the first writing system created specifically for the Mongolian language, and was the most successful until the introduction of Cyrillic in 1946. Derived from Uighur, Mongolian is a true alphabet, with separate letters for consonants and vowels. The Mongolian script has been adapted to write languages such as Oirat and Manchu. Alphabets based on this classical vertical script are used in Inner Mongolia and other parts of China to this day to write Mongolian, Sibe and, experimentally, Evenki.

\begin{scriptexample}[]{Mongolian}
\unicodetable{mongolian}{"1820,"1830,"1840,"1850,"1860,"1870,"1880,"1890,"18A0}
\end{scriptexample}



\section{Middle Eastern Scripts}

The scripts in this section have a common origin in the ancient Phoenician alphabet. They include:

\begin{center}
\begin{tabular}{ll}
Hebrew & Samaritan\\
Arabic & Thaana\\
Syriac &\\
\end{tabular}
\end{center}

The Hebrew script is used in Israel and for languages of the Diaspora. The Arabic script is
used to write many languages throughout the Middle East, North Africa, and certain parts
of Asia. The Syriac script is used to write a number of Middle Eastern languages. These
three also function as major liturgical scripts, used worldwide by various religious groups.

The Samaritan script is used in small communities in Israel and the Palestinian Territories
to write the Samaritan Hebrew and Samaritan Aramaic languages. The Thaana script is
used to write Dhivehi, the language of the Republic of Maldives, an island nation in the
middle of the Indian Ocean. 

Text in these scripts is written from right to left. Arabic and Syriac are cursive scripts even when typeset, unlike Hebrew, Samaritan  and Thaana, where letters are unconnected. Most letters in Arabic and Syriac assume different forms depending on their position in a word. Shaping rules are not required for Hebrew because only five letters have position-dependent forms, and these forms are separately encoded.

Historically, Middle Eastern  scripts did not write short vowels. In modern scripts they are represented  by marks positioned above or below a consonantal letter. Vowels and other
marks of pronunciation (“vocalization”) are encoded as combining characters, so support
for vocalized text necessitates use of composed character sequences. Yiddish, Syriac, and
Thaana are normally written with vocalization; Hebrew, Samaritan, and Arabic are usually written unvocalized. 

\section{Hebrew}
\newfontfamily\hebrew{Miriam}
\fontspec{Arial Unicode MS}
To properly typeset Hebrew texts you first need to choose an appropriate font and also set the directionality of the text. This
is done using the etex commands:

\CMDI{\beginL} and \CMDI{\beginR} 

For \XeTeX\ you also need to add near the top of your document |\TeXXeTstate=1|. The package \pkgname{bidi} can be used to set all parameters. Be warned that it redefines almost all of \latexe's commands, so for short mixed texts, I wouldn't recommend its usage. 



The Hebrew alphabet (Hebrew: אָלֶף־בֵּית עִבְרִי[a], alefbet ʿIvri ), known variously by scholars as the Jewish script, square script, block script, is used in the writing of the Hebrew language, as well as other Jewish languages, most notably Yiddish, Ladino, and Judeo-Arabic. There have been two script forms in use; the original old Hebrew script is known as the paleo-Hebrew script (which has been largely preserved, in an altered form, in the Samaritan script), while the present "square" form of the Hebrew alphabet is a stylized form of the Assyrian script. Various "styles" (in current terms, "fonts") of representation of the letters exist. There is also a cursive Hebrew script, which has also varied over time and place. On Windows you can use the \texttt{Miriam} font or \texttt{Arial Unicode MS} or \texttt{Miriam Fixed}.
\medskip

\topline

\bgroup\TeXXeTstate=1
\raggedleft\hebrew{}\beginR

הכתב הכנעני הקדום הלך והתפשט וסימניו היו מוכרים כל כך, עד כי המשתמשים בו התחילו "להתעצל" בהשלמת הציורים, והניחו כי הקורא יבין גם מתוך שרטוטים סכמתיים באיזו אות מדובר. כך, למשל, הפך הראש למשולש עם צוואר; כף היד מלאת האצבעות הפכה לשרטוט דל, ומהדג נותר רק הזנב. כשהעברים אמצו את הכתב הכנעני הם התקשו לזהות חלק מהציורים המקוריים והניחו למשל כי הסימן המתאר את המילה "זהה" הוא כלי נשק; שזנב הדג המשולש הוא דלת, ושדווקא הנחש הוא דג. כך נולדו שמותיהם העבריים של האותיות זי"ן, דל"ת ונו"ן (נון הוא דג, כמו אמנון, שפמנון וכו'). הציורים שהפכו לסימנים התגלגלו לכתבים נוספים, ואפילו ליוונית וללטינית. גם בכתב העברי המודרני ניתן לזהות המשך התפתחותי ברור מן הכתב הכנעני הקדום, והשתמרות שמות האותיות מקלה מאוד על פענוח המקור.


בתקופת בית שני, אומץ האלפבית הארמי לשימוש השפה העברית במקום האלפבית העברי העתיק, כאשר בזה האחרון נעשה שימוש מועט כגון כתיבת השמות הקדושים והטבעת מטבעות. עם הזמן, נעלם גם שימוש זה של הכתב העתיק. האלפבית העברי של ימינו הוא אפוא פיתוח של האלפבית הארמי ולא של הכתב העברי העתיק.	
{}

 לֹ֥א תִשָּׂ֛א

\endR


\egroup
\bottomline
\medskip

To make all paragraphs  RL use the \cmd{\everypar}\footnote{See discussions at \url{http://tex.stackexchange.com/questions/141867/minimal-bidi-for-typesetting-rl-text} and \url{http://www.tug.org/pipermail/xetex/2004-August/000697.html}}. 

\begin{verbatim}
\newbox\mybox \everypar{\setbox\mybox\lastbox\beginR\box\mybox}
\everypar={% at the start of each paragraph, do....
    \setbox0=\lastbox % save the paragraph indent, if any
    \beginR % set R-L direction
    \box0 % then re-insert the indent
	}
\end{verbatim}

The Hebrew alphabet has 22 letters, of which five have different forms when used at the end of a word. Hebrew is written from right to left. Originally, the alphabet was an abjad consisting only of consonants. Like other \textit{abjads}, such as the Arabic alphabet, means were later devised to indicate vowels by separate vowel points, known in Hebrew as niqqud. In rabbinic Hebrew, the letters א ה ו י are also used as matres lectionis to represent vowels. When used to write Yiddish, the writing system is a true alphabet (except for borrowed Hebrew words). In modern usage of the alphabet, as in the case of Yiddish (except that ע replaces ה) and to some extent modern Israeli Hebrew, vowels may be indicated. Today, the trend is toward full spelling with these letters acting as true vowels.

\section{Samaritan}
\newfontfamily\samaritan{NotoSansSamaritan-Regular.ttf}

The Samaritan alphabet is used by the Samaritans for religious writings, including the Samaritan Pentateuch, writings in Samaritan Hebrew, and for commentaries and translations in Samaritan Aramaic and occasionally Arabic.

The Samaritans are, consider themselves to be the descendants of the Northern Tribes of Israel that were not sent into Assyrian captivity, and have continuously resided in the land of Israel.

The Torah Scroll of the Samaritans uses an alphabet that is very different from the one used on Jewish Torah Scrolls. According to the Samaritans themselves and Hebrew scholars, this alphabet is the original "Old Hebrew" alphabet.

Even as far back as 1691, this connection between the Samaritan and the "Old" Hebrew alphabets was made by Henry Dodwell; "[the Samaritans] still preserve [the Pentateuch] in the Old Hebrew characters."

Samaritan is a direct descendant of the Paleo-Hebrew alphabet, which was a variety of the Phoenician alphabet in which large parts of the Hebrew Bible were originally penned. All these scripts are believed to be descendants of the Proto-Sinaitic script. That script was used by the ancient Israelites, both Jews and Samaritans. The better-known "square script" Hebrew alphabet traditionally used by Jews is a stylized version of the Aramaic alphabet which they adopted from the Persian Empire (which in turn adopted it from the Arameans). 

After the fall of the Persian Empire, Judaism used both scripts before settling on the Aramaic form. For a limited time thereafter, the use of paleo-Hebrew (proto-Samaritan) among Jews was retained only to write the Tetragrammaton, but soon that custom was also abandoned.



ShofarRegular StamAshkenazCLM.ttf

\begin{scriptexample}[]{Samaritan}
\bgroup
\TeXXeTstate=1
\unicodetable{samaritan}{"0800,"0810,"0820,"0830}
\egroup
\TeXXeTstate=0
\end{scriptexample}

I battled to get an appropriate font for the Samaritan script and had to use the \idxfont{Noto Sans Samaritan} from Google


^^A\printunicodeblock{./languages/samaritan.txt}{\samaritan}


\url{http://www.ancient-hebrew.org/ahh/ahh.htm#_Toc314842274}



\section{Arabic}

\newfontfamily\arabian{Scheherazade-R.ttf}

The Arabic script is a writing system used for writing several languages of Asia and Africa, such as Arabic, Sorani and Luri Dialects of Kurdish language, Persian, Pashto and Urdu.[1] Even until the 16th century, it was used to write some texts in Spanish.[2] After the Latin script, Chinese characters, and Devanagari, it is the fourth-most widely used writing system in the world.[3]
The Arabic script is written from right to left in a cursive style. In most cases the letters transcribe consonants, or consonants and a few vowels, so most Arabic alphabets are abjads.

The script was first used to write texts in Arabic, most notably the Qurʼān, the holy book of Islam. With the spread of Islam, it came to be used to write languages of many language families, leading to the addition of new letters and other symbols, with some versions, such as Kurdish, Uyghur, and old Bosnian being abugidas or true alphabets. It is also the basis for a rich tradition of Arabic calligraphy.

\begin{verbatim}
\begin{Arabic}
ّ هو إذ الغاية؛ شريف الفوائد، جم المذهب، عزيز فنّ التاريخ فنّ أنّ اعلم
والملوك سيرهم، في والأنبياء أخلاقهم، في الأمم من الماضين أحوال على يوقفنا
ّ أحوال في يرومه لمن ذلك في الإقتداء فائدة تتم حتّى وسياستهم؛ دولهم في
والدنيا. الدين
\end{Arabic}
\end{verbatim}




As of Unicode 7.0, the Arabic script is contained in the following blocks:
Arabic (0600—06FF, 255 characters)
Arabic Supplement (0750—077F, 48 characters)
Arabic Extended-A (08A0—08FF, 39 characters)
Arabic Presentation Forms-A (FB50—FDFF, 608 characters)
Arabic Presentation Forms-B (FE70—FEFF, 140 characters)
Rumi Numeral Symbols (10E60—10E7F, 31 characters)
Arabic Mathematical Alphabetic Symbols (1EE00—1EEFF, 143 characters)[1][2]

The basic Arabic range encodes the standard letters and diacritics, but does not encode contextual forms (U+0621–U+0652 being directly based on ISO 8859-6); and also includes the most common diacritics and Arabic-Indic digits. The Arabic Supplement range encodes letter variants mostly used for writing African (non-Arabic) languages. The Arabic Extended-A range encodes additional Qur'anic annotations and letter variants used for various non-Arabic languages. The Arabic Presentation Forms-A range encodes contextual forms and ligatures of letter variants needed for Persian, Urdu, Sindhi and Central Asian languages. The Arabic Presentation Forms-B range encodes spacing forms of Arabic diacritics, and more contextual letter forms. The presentation forms are present only for compatibility with older standards, and are not currently needed for coding text.[3] 

The Arabic Mathematical Alphabetical Symbols block encodes characters used in Arabic mathematical expressions.

\begin{multicols}{3}
\printunicodeblock{./languages/arabic.txt}{\arabian}
\end{multicols}








\section{Thaana}

\newfontfamily\thaana{MV Boli}
Thaana, Taana or Tāna ({\thaana  ތާނަ}‎ in Tāna script) is the modern writing system of the Maldivian language spoken in the Maldives. Thaana has characteristics of both an abugida (diacritic, vowel-killer strokes) and a true alphabet (all vowels are written), with consonants derived from indigenous and Arabic numerals, and vowels derived from the vowel diacritics of the Arabic abjad. Its orthography is largely phonemic.

The Thaana script first appeared in a Maldivian document towards the beginning of the 18th century in a crude initial form known as Gabulhi Thaana which was written scripta continua. This early script slowly developed, its characters slanting 45 degrees, becoming more graceful and spaces were added between words. 

As time went by it gradually replaced the older Dhives Akuru alphabet. The oldest written sample of the Thaana script is found in the island of Kanditheemu in Northern Miladhunmadulu Atoll. It is inscribed on the door posts of the main Hukuru Miskiy (Friday mosque) of the island and dates back to 1008 AH (AD 1599) and 1020 AH (AD 1611) when the roof of the building were built and the renewed during the reigns of Ibrahim Kalaafaan (Sultan Ibrahim III) and Hussain Faamuladeyri Kilege (Sultan Hussain II) respectively.

\begin{scriptexample}[]{Thaana}
\unicodetable{thaana}{"0780,"0790,"07A0,"07B0}

\hfill Typeset with MV Boli and the command \cmd{\thaana}.
\end{scriptexample}


^^A\printunicodeblock{./languages/thaana.txt}{\thaana}

\subsection{Syriac}

\newfontfamily\syriac{Estrangelo Edessa}

Syriac /ˈsɪriæk/ ({\syriac{ܠܫܢܐ ܣܘܪܝܝܐ}} Leššānā Suryāyā) is a dialect of Middle Aramaic that was once spoken across much of the Fertile Crescent and Eastern Arabia.[1][2][5] Having first appeared as a script in the 1st century AD after being spoken as an unwritten language for five centuries,[6] Classical Syriac became a major literary language throughout the Middle East from the 4th to the 8th centuries,[7] the classical language of Edessa, preserved in a large body of Syriac literature.
It became the vehicle of Syriac Christianity and culture, spreading throughout Asia as far as the Indian Malabar Coast and Eastern China,[8] and was the medium of communication and cultural dissemination for Arabs and, to a lesser extent, Persians. Primarily a Christian medium of expression, Syriac had a fundamental cultural and literary influence on the development of Arabic,[9] which largely replaced it towards the 14th century.[3] Syriac remains the liturgical language of Syriac Christianity.
Syriac is a Middle Aramaic language, and, as such, it is a language of the Northwestern branch of the Semitic family. It is written in the Syriac alphabet, a derivation of the Aramaic alphabet.

\begin{scriptexample}[]{Syriac}
\unicodetable{syriac}{"0700,"0710,"0720,"0730,"0740}
\end{scriptexample}

The Syriac Abbreviation (a type of overline) can be represented with a special control character called the Syriac Abbreviation Mark (U+070F {\syriac \char"070F ܘ}).


\cxset{steward,
  numbering=arabic,
  custom=stewart,
  offsety=0cm,
  image={asia.jpg},
  texti={An introduction to the use of font related commands. The chapter also gives a historical background to font selection using \tex and \latex. },
  textii={In this chapter we discuss keys that are available through the \texttt{phd} package and give a background as to how fonts are used
in \latex.
 },
 pagestyle = empty
}

\arial


\chapter{South Asian Scripts}

The scripts of South Asia share so many characteristics that a side by side comparison of a few often reveal structural similarities even in the 
modern letterforms.
\medskip

\begin{center}
\begin{tabular}{lll}
Devanagari. &Gujarati &Telugu\\
Bengali   &Oriya &Kannada\\
Gurmukhi &Tamil  &Malayalam\\
Sinhala &Kaithi  &Meetei Mayek\\
Tibetan &Saurashtra &Ol Chiki.\\
Lepcha  &Sharada &Sora Sompeng\\
Phags-pa &Takri &Kharoshthi\\
Limbu &Chakma & Brahmi\\
Syloti Nagri & &\\
\end{tabular}
\end{center}

The sections that follow describe the scripts briefly and the |phd| settings
to activate the relevant commands and load appropriate fonts. 

\section{Devanagari}
\parindent1em

Devanagari is part of the Brahmic family of scripts of India, Nepal, Tibet, and South-East Asia.[2] It is a descendant of the Gupta script, along with Siddham and Sharada.[2] Eastern variants of Gupta called nāgarī are first attested from the 7th century CE; from c. 1200 CE these gradually replaced Siddham, which survived as a vehicle for Tantric Buddhism in East Asia, and Sharada, which remained in parallel use in Kashmir. An early version of Devanagari is visible in the Kutila inscription of Bareilly dated to Vikram Samvat 1049 (i.e. 992 CE), which demonstrates the emergence of the horizontal bar to group letters belonging to a word.[3]

Sanskrit nāgarī is the feminine of nāgara "relating or belonging to a town or city". It is feminine from its original phrasing with lipi ("script") as nāgarī lipi "script relating to a city", that is, probably from its having originated in some city.[4]

The use of the name devanāgarī is relatively recent, and the older term nāgarī is still common.[2] The rapid spread of the term devanāgarī may be related to the almost exclusive use of this script to publish Sanskrit texts in print since the 1870s.[2]

On Windows use \texttt{Arial Unicode MS}. 
\medskip

\newfontfamily\devanagari[Script=Devanagari,Scale=1.5]{Arial Unicode MS}

\begin{scriptexample}[]{Devanagari}
{\begin{center}\parindent0pt\devanagari

ंःअआइईउऊऋऌऍऎएऐऑऒओऔऔँ \par 

ी	ु	ू	ृ	ॄ	ॅ	ॆ	े	ै	ॉ	ॊ	ो	ौ	्	\par

\bigskip		
\begin{tabular}{lll lll lll l}
०	&१	&२	&३	&४	&५	&६	&७	&८	&९\\
0	&1	&2	&3	&4	&5	&6	&7	&8	&9\\
\end{tabular}
\end{center}	
}
\end{scriptexample}


On Linux \texttt{Lohit} is a font family designed to cover Indic scripts and released by Red Hat. The Lohit fonts currently cover 11 languages: Assamese, Bengali, Gujarati, Hindi, Kannada, Malayalam, Marathi, Oriya, Punjabi, Tamil, Telugu.[1] The fonts were supplied by Modular Infotech and licensed under the GPL. In September 2011, they were retroactively relicensed under the OFL.[2] The Lohit fonts are used as web fonts by some Wikimedia Foundation sites, like Wikipedia, since March 2012.The font currently support 21 Indian languages. 

\newfontfamily\devanagarilohit[Script=Devanagari,Scale=1.5]{Lohit-Devanagari.ttf}

\begin{scriptexample}[]{Devanagari}
\begin{center}\parindent0pt\devanagarilohit

ंःअआइईउऊऋऌऍऎएऐऑऒओऔऔँ \par 

ी	ु	ू	ृ	ॄ	ॅ	ॆ	े	ै	ॉ	ॊ	ो	ौ	्	\par

\bigskip		
\begin{tabular}{lll lll lll l}
०	&१	&२	&३	&४	&५	&६	&७	&८	&९\\
0	&1	&2	&3	&4	&5	&6	&7	&8	&9\\
\end{tabular}
\end{center}
\end{scriptexample}

\subsubsection{Punctuation} 
The end of a sentence or half-verse may be marked with a dot known as a pūrna virām or a vertical line danda: \textbar. The end of a full verse may be marked with two vertical lines: \textbar\textbar. A comma, or alpa virām, is used to denote a natural pause in speech. With expansion of English speakers in India, the full stop is also sometimes used.

\subsection{LaTeX support}

\latex2e support can be found in the \pkgname{sanskrit}. The package contains the font files and pre-processor for printing Sanskrit
text in both devanāgarī and transliterated Roman with diacritics. Another package that can be used with \XeTeX\ is support \pkgname{devnag}.  This was originally developed by Frans Velthuis for the University of Groningen, The Netherlands, and it was the first system to provide
support for the script for \tex. The package was  extended by Anshuman Pandey. The package provides both fonts as well as tranliteration macros.


\subsection{Gujarati}


Gujarati has its own writing system, distinct but related to several other Indian languages' writing systems, such as the one used to write Hindi. Strictly speaking, the Gujarati writing system is what is called an \emph{abugida} (and not an \textit{alphabet}), because the consonant characters all contain an inherent vowel, and other vowels are written as accents added on to the consonant characters. There are also symbols for stand-alone vowels.

The Gujarati script ({\gujarati{ગુજરાતી લિપિ }} Gujǎrātī Lipi), which like all Nāgarī writing systems is strictly speaking an abugida rather than an alphabet, is used to write the Gujarati and Kutchi languages. It is a variant of Devanāgarī script differentiated by the loss of the characteristic horizontal line running above the letters and by a small number of modifications in the remaining characters.
With a few additional characters, added for this purpose, the Gujarati script is also often used to write Sanskrit and Hindi.
Gujarati numerical digits are also different from their Devanagari counterparts.
\medskip

\bgroup
\newfontfamily\gujaratilohit[Script=Gujarati,Scale=1.5]{Lohit-Gujarati.ttf}
\gujarati

\centering

English/Hindi/Gujarati Alphabets

\begin{tabular}{lllllllllllllllllllll}
A &B &bh &C &ch &chh &D &dh &E &F &G &gh &H &I &J &K &kh &L &M &N &O\\

अ &ब &भ &क &च &छ &ड/द &ध/ढ़ &इ &फ &ग &घ &ह &ई &ज &क &ख &ल &म &न/ण &ऑ\\

અ &બ &ભ &ક &ચ &છ &ડ/દ &ધ /ઢ &ઇ &ફ &ગ &ઘ &હ &ઈ &જ &ક &ખ &લ &મ &ન/ણ &ઓ\\

\end{tabular}
\egroup

\medskip

Gujarati has its own set of numeric signs (placed alongside their Hindu-Arabic [or Indo-Arabic] counterparts in the tables below), they are employed in much the same way as English;  that is to say, they are put together in the same manner in order to express larger numbers. It is quite possible to simply substitute the Gujarati numerals for the Hindu-Arabic ones.

The Gujarati words for 1-10 are as follows:
\medskip

\bgroup
\begin{center}
\gujarati
\begin{tabular}{ccl}
Arabic & Gujarati &Name\\
Numeral &Numeral  &\\
0	&૦	&mīṇḍuṃ or shunya\\
1	&૧	&ekaṛo or ek\\
2	&૨	&bagaṛo or bay\\
3	&૩	&tragaṛo or tran\\
4	&૪	&chogaṛo or chaar\\
5	&૫	&pāchaṛo or paanch\\
6	&૬	&chagaṛo or chah\\
7	&૭	&sātaṛo or sāt\\
8	&૮	&āṭhaṛo or āanth\\
9	&૯	&navaṛo or nav\\
10 &૧૦ &દસ das\\

\end{tabular}
\end{center}
\egroup

\subsection{Bengali}

There are two Windows fonts that can be used with Windows \textit{Shonar Bangla} and \textit{Vrinda}. For open source fonts one can use, \textit{code2000}.
\bigskip

\bgroup
\newfontfamily\bengali[Script=Bengali,Scale=4]{Shonar Bangla}


\bengali
\centering

  অ  আ ই  ঈ  উ  ঊ  ঋ  এ  ঐ\par

\fontspec[Script=Bengali,Scale=3.2]{Vrinda}

\centering

  অ  আ ই  ঈ  উ  ঊ  ঋ  এ  ঐ\par


\fontspec[Script=Bengali,Scale=3.2]{code2000.ttf}

\centering

  অ  আ ই  ঈ  উ  ঊ  ঋ  এ  ঐ\par

\captionof{table}{The consonant{\protect\bengal{} ক (kô)} along with the diacritic form of the vowels {\protect\bengal{} অ, আ, ই, ঈ, উ, ঊ, ঋ, এ, ঐ, ও and ঔ} \textit{from Wikipedia}.}
\egroup

\subsection{Saurashtra}

\newfontfamily\saurashtra{code2000.ttf}

Saurashtra or Sourashtra or {\saurashtra ꢱꣃꢬꢵꢰ꣄ꢜ꣄ꢬꢵ} or Palkar or Patkar (Sanskrit: सौराष्ट्र, Tamil: சௌராட்டிரம்) is an Indo-Aryan language[3] spoken by the Saurashtrian community native to Gujarat, who migrated and settled in Southern India. Madurai in Tamil Nadu has the highest number of people belonging to this community and also remains as their cultural center.

The language is largely only in spoken form even though the language has its own script. The lack of schools teaching Saurashtra script and the language is often cited as a reason for the very few number of people who actually know to read and write in Saurashtra script. Latin, Devanagari or Tamil script is used as alternative for Saurashtra Script by many Saurashtrians.

Census of India places the language under Gujarati. Official figures show the number of speakers as 185,420 (2001 census).[4]



\begin{scriptexample}[]{Saurashtra}
\bgroup
\saurashtra

ꢮꢶꢯ꣄ꢮ ꢱꣃꢬꢵꢰ꣄ꢜ꣄ꢬꢪ꣄ ꢦꢡ꣄ꢬꢶꢒꢾ ꢱꢵꢡ꣄ꢡꢒꢸ ꢂꢮꢬꢾ
ꢮꣁꢭꢱ꣄ꢢꢵꢥꢪꢸꢒ꣄(ꣀꢵꢮꢾꢔꢹ ꢂꢮ꣄ꢬꢶꢫꣁ


\arial

Text: Vishwa Sourashtram \url{http://www.sourashtra.info/ghEr.htm}
\egroup
\end{scriptexample}

\subsection{Ol Chiki script}

The Ol Chiki script, also known as Ol Cemetʼ (Santali: ol 'writing', cemet' 'learning'), Ol Ciki, Ol, and sometimes as the Santali alphabet, was created in 1925 by Raghunath Murmu for the Santali language.

Previously, Santali had been written with the Latin alphabet. But because Santali is not an Indo-Aryan language (like most other languages in the south of India), Indic scripts did not have letters for all of Santali's phonemes, especially its stop consonants and vowels, which made writing the language accurately in an unmodified Indic script difficult. The detailed analysis was given by Dr. Byomkes Chakrabarti in his 'Comparative Study of Santali and Bengali'. Missionaries (first of all Paul Olaf Bodding, a Norwegian) brought the Latin script, which is better at representing Santali stops, phonemes and nasal sounds with the use of diacritical marks and accents. Unlike most Indic scripts, which are derived from Brahmi, Ol Chiki is not an abugida, with vowels given equal representation with consonants. Additionally, it was designed specifically for the language, but one letter could not be assigned to each phoneme because the sixth vowel in Ol Chiki is still problematic.
Ol Chiki has 30 letters, the forms of which are intended to evoke natural shapes. Linguist Norman Zide said "The shapes of the letters are not arbitrary, but reflect the names for the letters, which are words, usually the names of objects or actions representing conventionalized form in the pictorial shape of the characters."[1] It is written from left to right.

\newfontfamily\olchiki{code2000.ttf}

\begin{scriptexample}[]{olchiki}
\bgroup
\olchiki
\obeylines

U+1C5x 	᱐	᱑	᱒	᱓	᱔	᱕	᱖	᱗	᱘	᱙	ᱚ	ᱛ	ᱜ	ᱝ	ᱞ	ᱟ
U+1C6x	   ᱠ	ᱡ	ᱢ	ᱣ	ᱤ	ᱥ	ᱦ	ᱧ	ᱨ	ᱩ	ᱪ	ᱫ	ᱬ	ᱭ	ᱮ	ᱯ
U+1C7x  	ᱰ	ᱱ	ᱲ	ᱳ	ᱴ	ᱵ	ᱶ	ᱷ	ᱸ	ᱹ	ᱺ	ᱻ	ᱼ	ᱽ	᱾	᱿
\egroup
\end{scriptexample}

\subsection{Lepcha}
\newfontfamily\lepcha{Mingzat-R.ttf}

The Lepcha script, or Róng script is an abugida used by the Lepcha people to write the Lepcha language. Unusually for an abugida, syllable-final consonants are written as diacritics.

The Mingzat font is still under development by SIL so I am not too sure if the rendering is correct\footnote{\url{http://scripts.sil.org/cms/scripts/page.php?site_id=nrsi&id=Mingzat}}.

\begin{scriptexample}[]{Lepcha}
\bgroup
\lepcha
\obeylines
 	    0	1	2	3	4	5	6	7	8	9	A	B	C	D	E	F
U+1C0x	 ᰀ	ᰁ	ᰂ	ᰃ	ᰄ	ᰅ	ᰆ	ᰇ	ᰈ	ᰉ	ᰊ	ᰋ	ᰌ	ᰍ	ᰎ	ᰏ
U+1C1x	 ᰐ	ᰑ	ᰒ	ᰓ	ᰔ	ᰕ	ᰖ	ᰗ	ᰘ	ᰙ	ᰚ	ᰛ	ᰜ	ᰝ	ᰞ	ᰟ
U+1C2x	 ᰠ	ᰡ	ᰢ	ᰣ	ᰤ	ᰥ	ᰦ	ᰧ	ᰨ	ᰩ	ᰪ	ᰫ	ᰬ	ᰭ	ᰮ	ᰯ
U+1C3x	 ᰰ	ᰱ	ᰲ	ᰳ	ᰴ	ᰵ	ᰶ	᰷	x	x	x	᰻	᰼	᰽	᰾	᰿
U+1C4x	 ᱀	᱁	᱂	᱃	᱄	᱅	᱆	᱇	᱈	᱉	x	x	x	ᱍ	ᱎ	ᱏ

\egroup
\end{scriptexample}

\subsection{Sharada}

The Śāradā, or Sharada, script (शारदा) is an abugida writing system of the Brahmic family of scripts, developed around the 8th century. It was used for writing Sanskrit and Kashmiri. The Gurmukhī script was developed from Śāradā. Originally more widespread, its use became later restricted to Kashmir, and it is now rarely used except by the Kashmiri Pandit community for ceremonial purposes. Śāradā is another name for Saraswati, the goddess of learning.
Śāradā script was added to the Unicode Standard in January, 2012 with the release of version 6.1.

The Unicode block for Śāradā script, called Sharada, is U+11180–U+111DF: Unable to locate font in unicode.


\subsection{Sora Sompeng}

Sorang Sompeng script is used to write in Sora, a Munda language with 300,000 speakers in India. The script was created by Mangei Gomango in 1936 and is used in religious contexts.[1] He was familiar with Oriya, Telugu and English, so the parent systems of the script are Brahmi and Latin.[2]
The Sora language is also written in the Latin alphabet and the Telugu script.

Sorang Sompeng script was added to the Unicode Standard in January, 2012 with the release of version 6.1. Nirmala UI.ttf (Windows 8.1)



\unicodetable{arial}{"110D0,"110E0,"110F0}
 	
This did not work with Windows 7, and the experiment failed. 

\subsection{Phags-pa}

The 'Phags-pa script,[1], (Mongolian: дөрвөлжин үсэг "Square script") was an alphabet designed by the Tibetan monk and vice-king Drogön Chögyal Phagpa for the Mongol Yuan emperor Kublai Khan as a unified script for the literary languages of the Yuan. Widespread use was limited to about a hundred years during the Yuan Dynasty, and it fell out of use with the advent of the Ming dynasty. The documentation of its use provides clues about the changes in the varieties of Chinese, the Tibetic languages, Mongolian and other neighboring languages during the Yuan era.

\newfontfamily\phagspa{code2000.ttf}

\begin{scriptexample}[]{Phags-pa}
\bgroup
\obeylines
\phagspa

 	0	1	2	3	4	5	6	7	8	9	A	B	C	D	E	F
U+A84x	ꡀ	ꡁ	ꡂ	ꡃ	ꡄ	ꡅ	ꡆ	ꡇ	ꡈ	ꡉ	ꡊ	ꡋ	ꡌ	ꡍ	ꡎ	ꡏ
U+A85x	ꡐ	ꡑ	ꡒ	ꡓ	ꡔ	ꡕ	ꡖ	ꡗ	ꡘ	ꡙ	ꡚ	ꡛ	ꡜ	ꡝ	ꡞ	ꡟ
U+A86x	ꡠ	ꡡ	ꡢ	ꡣ	ꡤ	ꡥ	ꡦ	ꡧ	ꡨ	ꡩ	ꡪ	ꡫ	ꡬ	ꡭ	ꡮ	ꡯ
U+A87x	ꡰ	ꡱ	ꡲ	ꡳ	꡴	꡵	꡶	


ꡏꡟ ꡋꡞ ꡏꡟ ꡋꡞ ꡏ ꡜꡖ ꡏꡟ ꡋꡞ ꡓꡞ ꡏꡟ
ꡈꡋ ꡋꡋ ꡓꡘ ꡈ ꡭ ꡏ ꡏ ꡝ ꡭꡟꡘ ꡓꡋ ꡮꡟꡊ
\egroup
\bgroup
\raggedright

\setcounter{glyphcount}{"A840}

\topline
\phagspa
\newcount\n
\n="A840

\def\htable{^^A
  \def\fm##1{\makebox[2em]##1}^^A
  U+A840\fm 0\fm1\fm2\fm3\fm4\fm5\fm 6\fm 7\fm 8\fm	9\fm A\fm B\fm C\fm D\fm E\fm F}

\htable\par
U+A840^^A 
\loop^^A
  \makebox[2em]{\char\n }^^A   
   \advance\n by1 ^^A
   \ifnum\n<"A850^^A
\repeat
\par U+A850^^A
\loop^^A
  \makebox[2em]{\char\n }^^A   
   \advance\n by1 ^^A
  \ifnum\n<"A860^^A
\repeat
\par U+A860^^A
\loop^^A
  \makebox[2em]{\char\n }^^A   
   \advance\n by1 ^^A
  \ifnum\n<"A870^^A
\repeat
\par U+A870^^A
\loop^^A
  \makebox[2em]{\char\n }^^A   
   \advance\n by1 ^^A
  \ifnum\n<"A878^^A
\repeat

\bottomline

\arial
\hfill Typeset with \texttt{code2000.ttf} and \cmd{\phagspa}

Text: \href{http://babelstone.blogspot.com/2006/12/phags-pa-fonts-1-babelstone-phags-pa.html}{babelstone}
\egroup
\end{scriptexample}

Phags-pa is a historical script related to Tibetan that was created as the national script of
the Mongol empire. Even though Phags-pa was used mostly in Eastern and Central Asia for
writing text in the Mongolian and Chinese languages, it is discussed in this chapter because
of its close historical connection to the Tibetan script. The script has very limited modern use. It bears similarity to Tibetan and has no case distinctions. It is written vertically in columns running for left to right, like Mongolian. Units are often composed of several syllables and sometimes are separated by whitespace.


\subsection{Syloti Nagri}
\index{languages>Sylheti Nagari}
Sylheti Nagari or Syloti Nagri (Silôṭi Nagôri) is the original script used for writing the Sylheti language. It is an almost extinct script, this is because the Sylheti Language itself was reduced to only dialect status after Bangladesh gained independence and because it did not make sense for a dialect to have its own script, its use was heavily discouraged. The government of the newly formed Bangladesh did so to promote a greater "Bengali" identity. This led to the informal adoption of the Eastern Nagari script also used for Bengali and Assamese. It is also known as Jalalabadi Nagri, Mosolmani Nagri, Ful Nagri etc.

\newfontfamily\syloti{NotoSansSylotiNagri-Regular.ttf}
\newfontfamily\damase{damase_v.2.ttf}
\bgroup
\damase
\obeylines
	0	1	2	3	4	5	6	7	8	9	A	B	C	D	E	F
U+A80x	ꠀ	ꠁ	ꠂ	ꠃ	ꠄ	ꠅ	꠆	ꠇ	ꠈ	ꠉ	ꠊ	ꠋ	ꠌ	ꠍ	ꠎ	ꠏ
U+A81x	ꠐ	ꠑ	ꠒ	ꠓ	ꠔ	ꠕ	ꠖ	ꠗ	ꠘ	ꠙ	ꠚ	ꠛ	ꠜ	ꠝ	ꠞ	ꠟ
U+A82x	ꠠ	ꠡ	ꠢ	ꠣ	ꠤ	ꠥ	ꠦ	ꠧ	꠨	꠩	꠪	꠫
\egroup

\subsection{Chakma}

\newfontfamily\chakma{RibengUni.ttf}

\bgroup
\chakma
𑄇𑄳𑄇 Kkā = 𑄇 Kā + 𑄳 VIRAMA + 𑄇 Kā
𑄇𑄳𑄑 Ktā = 𑄇 Kā + 𑄳 VIRAMA + 𑄑 Tā
𑄇𑄳𑄖 Ktā = 𑄇 Kā + 𑄳 VIRAMA + 𑄖 Tā
𑄇𑄳𑄟 Kmā = 𑄇 Kā + 𑄳 VIRAMA + 𑄟 Mā
𑄇𑄳𑄌 Kcā = 𑄇 Kā + 𑄳 VIRAMA + 𑄌 Cā
𑄋𑄳𑄇 ńkā = 𑄋 ńā + 𑄳 VIRAMA + 𑄇 Kā
𑄋𑄳𑄉 ńkā = 𑄋 ńā + 𑄳 VIRAMA + 𑄉 Gā
𑄌𑄳𑄌 ccā = 𑄌 cā + 𑄳 VIRAMA + 𑄌 Cā

\egroup

\subsection{Limbu}

The Limbu script is used to write the Limbu language. The Limbu script is an abugida derived from the Tibetan script. Limbu is a Tibeto-Burman language spoken mainly in Nepal,[3] significant communities in Bhutan, Sikkim, Darjeeling district, India by the Limbu community. Virtually all Limbus are bilingual in Nepali.

\newfontfamily\limbu{code2000.ttf}
\bgroup
\obeylines
\limbu
0	1	2	3	4	5	6	7	8	9	A	B	C	D	E	F
U+190x	ᤀ	ᤁ	ᤂ	ᤃ	ᤄ	ᤅ	ᤆ	ᤇ	ᤈ	ᤉ	ᤊ	ᤋ	ᤌ	ᤍ	ᤎ	ᤏ
U+191x	ᤐ	ᤑ	ᤒ	ᤓ	ᤔ	ᤕ	ᤖ	ᤗ	ᤘ	ᤙ	ᤚ	ᤛ	ᤜ	ᤝ	ᤞ	
U+192x	ᤠ	ᤡ	ᤢ	ᤣ	ᤤ	ᤥ	ᤦ	ᤧ	ᤨ	ᤩ	ᤪ	ᤫ				
U+193x	ᤰ	ᤱ	ᤲ	ᤳ	ᤴ	ᤵ	ᤶ	ᤷ	ᤸ	᤹	᤺	᤻				
U+194x	᥀				᥄	᥅	᥆	᥇	᥈	᥉	᥊	᥋	᥌	᥍	᥎	᥏
\egroup

\subsection{Brahmi}



Brāhmī is the modern name given to one of the oldest writing systems used in the Indian subcontinent and in Central Asia during the final centuries BCE and the early centuries CE. Like its contemporary, Kharoṣṭhī, which was used in what is now Afghanistan and Western Pakistan, Brahmi (native to north and central India) was an \emph{abugida}.

The best-known Brahmi inscriptions are the rock-cut edicts of Ashoka in north-central India, dated to 250–232 BCE. The script was deciphered in 1837 by James Prinsep, an archaeologist, philologist, and official of the East India Company.[1] The origin of the script is still much debated, with current Western academic opinion generally agreeing (with some exceptions) that Brahmi was derived from or at least influenced by one or more contemporary Semitic scripts, but a current of opinion in India favors the idea that it is connected to the much older and as-yet undeciphered Indus script

\subsection{Unicode [U+11000-U+1107F]}


\newfontfamily\brahmi{code2000.ttf}

\begin{scriptexample}[]{Brahmi}
\bgroup
\raggedleft
\brahmi

         
   

\arial
\hfill Text: Asokan Edict typeset with \texttt{NotoSansBrahmi-Regular.ttf} 
\egroup
\end{scriptexample}


\begin{description}
\item[Abkhazia] (Abkhaz: Аҧсны́ Apsny [apʰsˈnɨ]; Georgian: აფხაზეთი Apkhazeti; Russian: Абхазия Abkhaziya) is a disputed territory and partially recognised state controlled by a separatist government on the eastern coast of the Black Sea and the south-western flank of the Caucasus.

\item[Achinese] Acehnese language (Achinese) is a Malayo-Polynesian language spoken by Acehnese people natively in Aceh, Sumatra, Indonesia. This language is also spoken in some parts in Malaysia by Acehnese descendents there, such as in Yan, Kedah.

Formerly, Acehnese language was written in Arabic script called Jawoë or Jawi in Malay language. The script is less common nowadays.[citation needed] Now, Acehnese language is written in Latin script since colonization by the Dutch; with the addition of supplementary letters. The additional letters are é, è, ë, ö and ô.[8] The sound ɨ is represented by 'eu' and the sound ʌ is represented by 'ö' respectively. The letter 'ë' is used to represent the schwa sound which forms the second part in the diphthongs.

\item[Adyghe] Adyghe (/ˈædɨɡeɪ/ or /ˌɑːdɨˈɡeɪ/;[3] Adyghe: Адыгэбзэ adyghabze), also known as West Circassian (КӀахыбзэ), is one of the two official languages of the Republic of Adygea in the Russian Federation, the other being Russian. It is spoken by various tribes of the Adyghe people: Abzekh,[4] Adamey, Bzhedug;[5] Hatuqwai, Temirgoy, Mamkhegh; Natekuay, Shapsug;[6] Zhaney, Yegerikuay, each with its own dialect. The language is referred to by its speakers as Adygebze or Adəgăbză, and alternatively spelled in English as Adygean, Adygeyan or Adygei. The literary language is based on the Temirgoy dialect.
There are apparently around 128,000 speakers of the language on the native territory in Russia, almost all of them native speakers. In the whole world, some 300,000 speak the language. The largest Adyghe-speaking community is in Turkey, spoken by the post Russian–Circassian War (circa 1763–1864) diaspora; in addition to that, the Adyghe language is spoken by the Cherkesogai in Krasnodar Krai.

Ублапӏэм ыдэжь Гущыӏэр щыӏагъ. Ар Тхьэм ыдэжь щыӏагъ, а Гущыӏэри Тхьэу арыгъэ. Ублапӏэм щегъэжьагъэу а Гущыӏэр Тхьэм ыдэжь щыӏагъ. Тхьэм а Гущыӏэм зэкӏэри къыригъэгъэхъугъ. Тхьэм къыгъэхъугъэ пстэуми ащыщэу а Гущыӏэм къыримыгъгъэхъугъэ зи щыӏэп. Мыкӏодыжьын щыӏэныгъэ а Гущыӏэм хэлъыгъ, а щыӏэныгъэри цӏыфхэм нэфынэ афэхъугъ. Нэфынэр шӏункӏыгъэм щэнэфы, шӏункӏыгъэри нэфынэм текӏуагъэп.

Translation: In the beginning was the Word, and the Word was with God, and the Word was God. The same was in the beginning with God. All things were made by him, and without him was not any thing made that was made. In him was life, and the life was the light of men. And the light shineth in darkness, and the darkness comprehended it not.

\item[Albanian]Albanian (shqip [ʃcip] or gjuha shqipe [ˈɟuha ˈʃcipɛ], meaning Albanian language) is an Indo-European language spoken by approximately 7.6 million people,[3] primarily in Albania, Kosovo, the Republic of Macedonia and Greece, but also in other areas of Southeastern Europe in which there is an Albanian population, including Montenegro and Serbia (Presevo Valley). Centuries-old communities speaking Albanian-based dialects can be found scattered in Greece, southern Italy,[4] Sicily, and Ukraine.[5] As a result of a modern diaspora, there are also Albanian speakers elsewhere in those countries and in other parts of the world, including Scandinavia, Switzerland, Germany, Austria and Hungary, United Kingdom, Turkey, Australia, New Zealand, Netherlands, Singapore, Brazil, Canada, and the United States.

Letter:	A	B	C	Ç	D	Dh	E	Ë	F	G	Gj	H	I	J	K	L	Ll	M	N	Nj	O	P	Q	R	Rr	S	Sh	T	Th	U	V	X	Xh	Y	Z	Zh\\
IPA value:	a	b	t͡s	t͡ʃ	d	ð	e	ə	f	ɡ	ɟ	h	i	j	k	l	ɫ	m	n	ɲ	o	p	c	ɾ	r	s	ʃ	t	θ	u	v	d͡z	d͡ʒ	y	z	ʒ\\

\end{description}

\begin{multicols}{5}
\raggedright
Abkhazian\\
Abron\\
Achinese\\
Acoli\\
Adyghe\\
Afar\\
Afrikaans\\
Aghem\\
Akan\\
Akoose\\
Albanian\\
Albay\\
Bikol\\
Amo\\
Asturian\\
Asu\\
Atikamekw
Atsam
Avaric
Aymara
Azerbaijani (Cyrillic script)\\
Azerbaijani (Latin script)\\
Bafia\\
Bafut\\
Balinese\\
Balkan Gagauz Turkish
Bambara (Latin script)
Banjar
Baoulé
Basaa
Bashkir
Basque
Batak
Batak Toba
Belarusian
Bemba
Bena
Betawi
Bikol
Bini
Bislama
Bomu
Bosnian (Cyrillic script)
Bosnian (Latin script)
Breton
Bube
Buginese
Buhid
Bulgarian
Bulu
Buriat
Bushi
Catalan
Cebaara Senoufo
Cebuano
Central Atlas Tamazight (Latin script)
Central-Eastern Niger Fulfulde
Central Huasteca Nahuatl
Central Mazahua
Chamorro
Chechen
Chiga
Chipewyan
Church Slavic
Chuukese
Chuvash
Colognian
Congo Swahili
Cornish
Corsican
Croatian
Czech
Dan
Danish
Dargwa
Dogrib
Duala
Dutch
Dyula
Eastern Huasteca Nahuatl
East Futuna
Efik
Embu
English
Erzya
Esperanto
Estonian
Ewe
Ewondo
Fang
Faroese
Fijian
Filipino
Finnish
Fon
French
Friulian
Fulah
Ga
Gagauz
Galician
Ganda
German
Ghomala
Gilbertese
Gorontalo
Greek
Gronings
Guajajára
Guarani
Guianese Creole French
Gusii
Gwichʼin
Haitian
Hanunoo
Hausa (Latin script)
Hawaiian
Hiligaynon
Hiri Motu
Hungarian
Ibibio
Icelandic
Igbo
Iloko
Inari Sami
Indonesian
Ingush
Interlingua
Inuinnaqtun
Inuktitut (Latin script)
Inupiaq
Irish
Italian
Javanese
Jenaama Bozo
Jju
Jola-Fonyi
Kabardian
Kabuverdianu
Kabyle
Kaingang
Kako
Kalaallisut
Kalanga
Kalenjin
Kalo Finnish Romani
Kamba
Karachay-Balkar
Kara-Kalpak
Karelian
Kashubian

Kazakh (Cyrillic script)

Kerinci
Khasi
Kʼicheʼ
Kikuyu
Kimbundu
Kinyarwanda
Kita Maninkakan
Kom
Komering
Komi
Komi-Permyak
Kongo
Koro
Koro Wachi
Kosraean
Koyraboro Senni
Koyra Chiini
Kpelle
Krio
Kuanyama
Kumyk
Kurdish (Latin script)

Kwasio

Kyrgyz (Cyrillic script)

Kyrgyz (Latin script)

Lak\\
Lakota\\
Lampung Api\\
Langi\\
Lango\\
Latin\\
Latvian\\
Lezghian\\
Limburgish\\
Lingala\\
Lithuanian\\
Lombard
Lomwe
Lower Sorbian
Low German
Lozi
Luba-Katanga
Luba-Lulua
Lule Sami
Luo
Luxembourgish
Luyia
Maasina Fulfulde
Macedonian
Machame
Madurese
Mafa
Maguindanaon
Makasar
Makhu
Makhuwa-Meetto
Makonde
Malagasy
Malay (Latin script)
Maltese
Mandar
Mandingo (Latin script)
Manx
Manyika
Maori
Mapuche
Mari
Marshallese
Masaaba
Masai
Mbunga
Medumba
Mende
Meru
Meta’
Minangkabau
Mohawk
Moksha
Mongo
Mongolian (Cyrillic script)
Montagnais
Morisyen
Mossi
Mundang
Nama
Nauru
Navajo
Naxi
Ndau
Ndonga
Neapolitan
Negeri Sembilan Malay
Ngaju
Ngiemboon
Ngomba
Nigerian Fulfulde
Nigerian Pidgin
Niuean
Northern Sami
Northern Sotho
North Ndebele
North Slavey
Norwegian Bokmål
Norwegian Nynorsk
Nuer
Nyamwezi
Nyanja
Nyankole
Occitan
Oromo
Ossetic
Palauan
Pampanga
Pangasinan
Papiamento
Pohnpeian
Pökoot
Polish
Portuguese
Punu
Quechua
Rajasthani
Rejang
Réunion Creole French
Riang
Rinconada Bikol
Romanian
Romansh
Rombo
Ronga
Rundi
Russian
Rusyn
Rwa
Safaliba
Saho
Sakha
Samburu
Samoan
Sangir
Sango
Sangu
Santali
Sasak
Scots
Scottish Gaelic
Sena
Serbian (Cyrillic script)
Serbian (Latin script)
Serer
Seselwa Creole French
Shambala
Shona
Sicilian
Sidamo
Sinte Romani
Skolt Sami
Slave
Slovak
Slovenian
Soga
Somali
Soninke
Southern Altai
Southern Sami
Southern Sotho
South Ndebele
Spanish
Sranan Tongo
Sukuma
Sundanese
Susu
Swahili
Swati
Swedish
Swiss German
Tachelhit (Latin script)
Tae’
Tagbanwa
Tahitian
Taita
Tajik (Cyrillic script)
Tamashek
Taroko
Tasawaq
Tatar
Tausug
Tavringer Romani
Teso
Tetum
Timne
Tiv
Tokelau
Tok Pisin
Tolaki
Tomo Kan Dogon
Tongan
Tooro
Tornedalen Finnish
Tsonga
Tswana
Tumbuka
Turkish
Turkmen (Latin script)
Tuvalu
Tuvinian
Tyap
Uab Meto
Udmurt
Ukrainian
Ulithian
Umbundu
Unknown Language
Uyghur (Cyrillic script)
Uzbek (Cyrillic script)
Uzbek (Latin script)
Vai (Latin script)
Venda
Vietnamese
Virgin Islands Creole English
Vunjo
Wallisian
Walloon
Walser
Waray
Welsh
Western Frisian
Western Huasteca Nahuatl
Western Mari
Wolof
Xaasongaxango
Xavánte
Xhosa
Yangben
Yao
Yapese
Yemba
Yoruba
Yucatec Maya
Zarma
Zaza
Zeelandic
Zhuang
Zulu
\end{multicols}





\end{document}



%\chapter{Calendars and Dates}

\epigraph{Take the Traders' method of timekeeping. The frame corrections were incredibly complex - and down at the very bottom of it was a little program that ran a counter. Second by second, the Qeng Ho counted from the instant that a human had first set foot on Old Earth's moon. But if you looked at it still more closely ... the starting instant was actually about fifteen million seconds later, the 0-second of one of Humankind's first computer operating systems.}{Vernor Vinge, in \textit{A Deepness in the Sky}}

\section{Measuring time}
Civil Calendars The alterration of day and night is a clcar physical phenomenon that is repetitive and countable; so the solar day is the basic unit of all
calendars. Some calendars use the lunar month and/or solar year for longcr units,
but these periods arc neither fixed nor made up of integral mrmbers of days. Nloreover,
thc length of thc lunar month is not a simple rational fraction of the solar
year. However, the cycle of weeks, each of seven namcd days, is very widely used,
and it continues independently of the enumeration of days in the calendar.

The development of calendars varied, depending largely on religion, culture, politics
and economics. Religious practices and holidays along with agricultural cycles
have been defi ned in terms of lunar sightings, solar motion, and the appearance
of the stars in the sky. Hence, calendars have been based on lunar or solar motions,
or a combination of the two. Unfortunately, years, months, and days are not integral
multiples of each other, and this fact has led to complications in creating
calendars. For example, the year as measured by the length of time for the Sun to
return to the same place along its path in the sky (ecliptic) is currently equal to
365.242 189 7 days of 86 400 seconds, and the length of the month measured by
the Moon ’ s phases is 29.530 59 days. The year cannot be composed of an integral
number of months or days. Historically, the counting of years in different calendars
has been based on the reigns of rulers, the lives of religious leaders, and the
traditional beginnings of cultures.

Today, while a number of calendars continue in use for religious or national
reasons, the Gregorian calendar, initially introduced by Pope Gregory XIII in 1582
and adopted by various countries over the next 340 years, is the calendar used
internationally for civil purposes ( Explanatory Supplement to The Astronomical
Ephemeris and The American Ephemeris and Nautical Almanac , 1961 ). It will be of
satisfactory accuracy for thousands of years to come. There are a number of books
and references on calendars and the computer programs for converting between
them (Richards, 1998 ; Explanatory Supplement to the Astronomical Almanac , 1992 ).
A set of chronological ‘ eras ’ exists based on the various calendars, and these along
with current years of various calendars are tabulated each year in The Astronomical
Phenomena .

\subsection{The year}
\indexmany[year]{solar,tropical}

The tropical or solar year, properly, and by way of eminence so-called, is the space of time in which the sun moves 
through the twelve signs of the zodiac. This, by observations of the best modern astronomers, contains \printtime[5]{365}{5}{48}{46.14912}. The quantity assumed by the authors of the Gregorian calendar was \printtime[0]{365}{ 5}{49}{0} which  corresponds exactly with the observations of Bianchini, and  de La Hire, in the next century. In the civil, or popular account, the year. 

The excess of the solar year over 360 days has been given by different astronomers as follows:---
 
\def\daytime#1#2#3#4{%
 #1\textsuperscript{d}%
 #2\textsuperscript{h}%
 #3\textsuperscript{m}%
 #4\textsuperscript{s}%
  }
 
 \indexmany[calendric calculations]{Meton,Euctemon,%
   Hipparchos,Sosigenes,Albategnius,Copernicus,Tycho %  
   Brahe,Kepler,Halley,Lalande,Delambre,Laplace,Hind}

 
 \begin{longtable}{l r l}
Meton and Euctemon  &5th Century BC  &\printtimeinterval{6}{18}{57}{0}\\
Hipparchos          &2 Century BC        &\printtimeinterval{5}{55}{12}{0}\\
Sosigenes           &1 Century BC        &\printtimeinterval{6}{0}{0}{0}\\
Albategnius         &9th Century AD    &\printtimeinterval{5}{46}{24}{0}\\ 
Alphonsine Tables   &13th Century AD  &\printtimeinterval{5}{49}{16}{0}\\
Copernicus          &16th Century AD  &\printtimeinterval{5}{46}{6}{0}\\
Tycho Brahe     	  &16th Century AD  &\printtimeinterval{5}{48}{45.5}{0}\\
Kepler 				    &17th Century AD  &\printtimeinterval{5}{48}{57.65}{0}\\
Halley 				    &17th Century AD  &\printtimeinterval{5}{48}{54.691}\\
Lalande 			      &18th Century AD  &\PrintTimeInterval{5}{48}{35.5}\\
Delambre			      &18th Century AD  &\printtimeinterval {5}{48}{51.6}{0}\\ 
Laplace				    &18th Century AD  &\printtimeinterval {5}{48}{49.7}{0} \\
Hind, 1850			    &19th Century AD  &\printtimeinterval {5}{41}{46.2}{0} \\
\end{longtable} 


The oldest references to the Greek word τροπή [turn, soltice] are from Hesiod and Homer. Evidence
exists that from the earliest times the Chinese, the Hindus and the Greeks, and others did measure the
length of the tropical year, also called the seasonal year. According to Delabre this year is so called
`because the first astronomers did deduce it from the return of the Sun to the same tropic’.\footnote{Delabre, J., \textit{Histoire de l’Astronomie}, Paris, 1817.}

According to Ptolemy, Hipparchus wrote: \enquote{I composed a book about the length of the year, in which
I show that this is the time required for the Sun to travel from a tropic to the same tropic again, or from
an equinox to that same equinox, and that it is equal to 365.25 days minus 1/300 of a day-and-night, and not to a fourth of a day as the mathematicians believed}.\footnote{\textit{Almagest}, Book III. The citation has been reported by Ptolemy.}


The accepted current tropical year value on January 1, 2000 was 365.2421897 or \PrintTimeInterval{365}{5}{48}{45.19} . This changes slowly; an expression suitable for calculating the length in days for the distant past is:

\begin{equation}
Y = 365.2421896698 - 6.15359\times10^{-6}T- 7.29\times^{-10} T^2 + 2.64 \times10^{-10} T^3
\end{equation}

where $T$ is in Julian centuries of 36,525 days measured from noon January 1, 2000 TT (Terrestial Time) (in negative numbers for dates in the past). (McCarthy \& Seidelmann, 2009, p. 18.; Laskar, 1986)

\begin{texexample}{Tropical Year}{ex:tropical year}
\begin{luacode*}
local T = -1000
local T1, T2, T3 = -6.15359e-6*T, -7.29e-10*T*T, 2.64e-10*T*T*T
tex.print("\\numprint{"..tostring(T1).."}")
tex.print("\\numprint{"..tostring(T2).."}")
tex.print("\\numprint{"..tostring(T3).."}")
tex.print("")
tex.print(365.2421896698 + T1 + T2 + T3)
\end{luacode*}
\end{texexample}


Modern astronomers define the tropical year as time for the Sun's mean longitude to increase by 360°. The process for finding an expression for the length of the tropical year is to first find an expression for the Sun's mean longitude (with respect to {\panunicode ♈}), such as Newcomb's expression given above, or Laskar's expression (1986, p. 64). When viewed over a 1 year period, the mean longitude is very nearly a linear function of Terrestrial Time. To find the length of the tropical year, the mean longitude is differentiated, to give the angular speed of the Sun as a function of Terrestrial Time, and this angular speed is used to compute how long it would take for the Sun to move 360$\circ$. (Meeus \& Savoie, 1992, p. 42).

The length of the tropical year accordinng to Leverrier:
\begin{equation}
365.24219647 - 0.00000624 T \text{days}
\end{equation}
while Newcomb's well known expression, derived from his solar theory, is
\begin{equation}
\PrintTimeInterval {365}{5}{46}{0} - \PrintTimeInterval {0}{0}{0}{.530}T
\end{equation}

In these two expressions, $T$ is the time in Julian centuries of 36525 days measured from 1900 January 0.5 Ephemeris time. 

\subsection{Month}
\index{month}\index{month, astronomical}

The next convenient measure for the division of time, which is marked by the revolution of the
celestial objects is the month. The astronomical month is the period of time in which the moon 
performs a complete revolution round the heavens, and is either \textit{periodical} or \textit{synodical}. The periodical
month is the time in which the moon moves from one point the same point again, and is equal to \printtime{27}{7}{ 43}{47}; and the synodical month, or lunation, as it is sometimes called, is that portion of time which
elapses between two successive new moons, or between two succesive conjuctions of the moon with the sun, and is equal to \printtime{29}{12}{44}{3.19}.  The solar month is that portion of time in which the sun moves through an
entire sign of the zodiac, the mean quantity of which is \printtime{30}{10}{29}{3.84576}, being the twelfth of the
solar year.


\subsection{Week}

The origins of the seven day week is thought to have originated with Sumeria, who gave the name
of one of the seven planets to each hour of the day, and deisgnated each day by the name of that planet, which corresponded with the first hour of the day. 

The Latins adopted these designations in their names of 
the days of the week. They are to be found in old law books 
and MSS. For a very interesting discussions as to how the seven
day week, survived and passed to us see The Economist.\footnote{\protect\url{http://www.economist.com/node/895542?fsrc=scn/fb/wl/ar/thepowerofseven}}

Occasionally, the signs only of the planets were used, for 
the sake of brevity, particularly in diaries and journals. This 
is notably the case in the original MS. field-book of Mason 
and Dixon's survey of the boundary line between Pennsylvania and Maryland, 1763 to 1768, 
in possession of the Historical 
Society of Pennsylvania. In this book the name of 
each day of the week is represented by the sign, in addition 
to the usual dates, for a period of over four years. See, also, 
" Minutes of the Provincial Council of Pennsylvania" (Colonial Records), vol. ii. pages 90 to 96, etc. etc. In the latter 
part of vol i. (same Records) the Latin names of the days 
were used. 

\begin{center}
\begin{tabular}{l c l l}
\toprule
Latin              &Signs                   &English      &Anglo-Saxon\\
\midrule
Dies Saturni   &{\panunicode\char"2644}  &Saturday     &Saetern-daeg\\
Dies Solis       &{\panunicode\char"2609}  &Sunday       &Sunnan-daeg\\
Dies Lunae     &{\panunicode\char"263D} &Monday      &Monan-daeg\\
Dies Martis     &{\panunicode\char"2642} &Tuesday       &Tiwes-daeg\\
Dies Mercurii  &{\panunicode\char"263F}  &Wednesday &Wodnes-daeg\\
Dies Jovis       &{\panunicode\char"2643}  &Thursday     &Thors-daeg\\
Dies Veneris   &{\panunicode\char"2640}  &Friday           &Frigas-daeg\\
\bottomrule
\end{tabular}
\end{center}

The |IAU| discourages the use of the planetary symbols in articles and we only show them in the above table for historical reasons.  

The Aztecs had a ritual cycle of 260 days, known as Tonalpohuali, which was divided
into 20 weeks of 13 days known as Trecena. They also divided the solar year of 365 days, into 18
periods of 20 days and five nameless days known as Nemontemi. Although some consider this 20-day
grouping a month, it has no relation to lunation. It was divided into four ``weeks'' of five days.

In more recent times, the Soviet Union between 1929 and 1931 changed from the seven-day week to a five-day week. There
were 72 weeks and an additional five national holidays inserted within three of them totaling a year of
365 days.\index{Soviet week}

\begin{figure}[ht]
\includegraphics[width=\textwidth]{./images/soviet-calendar.jpg}
\caption{The Soviet Calendar for 1930}
\end{figure}

The five day week was a social disaster and this was then changed to a six day week, which later was
abandoned by a decree issued on 27 June 1940.

\begin{figure}[ht]
\centering
\includegraphics[width=0.5\textwidth]{./images/soviet-calendar-1939.jpg}
\caption{The six day Soviet Calendar for 1939}
\end{figure}

From the summer of 1931 until 26 June 1940, each Gregorian month was usually divided into five six-day weeks, more and less (as shown by the 1933 and 1939 calendars displayed here).[2] The sixth day of each week was a uniform day off for all workers, that is days 6, 12, 18, 24 and 30 of each month. 

The last day of 31-day months was always an extra work day in factories, which, when combined with the first five days of the following month, made six successive work days. But some commercial and government offices treated the 31st day as an extra day off. To make up for the short fifth week of February, 1 March was a uniform day off followed by four successive work days in the first week of March (2–5). The partial last week of February had four work days in common years (25–28) and five work days in leap years (25–29). But some enterprises treated 1 March as a regular work day, producing nine or ten successive work days between 25 February and 5 March, inclusive. The dates of the five national holidays did not change, but they now converted five regular work days into holidays within three six-day weeks rather than splitting those weeks into two parts (none of these holidays was on a ``sixth day")


The Unicode |CLDR| specification dictates that for each language a set of files are provided for calendar related information. This for example enables the printing of calendars in a specific language such as Greek, but using an islamic calendar.  The internationalization tables do not provide any conversion or calculation routines. They just represent how the traslated string would look. 

\begin{texexample}{i18n}{i18-3}
\begin{luacode}
local c = require("i18n.el.caislamic")
local s1 = c.el.dates.calendars.islamic.months.format.abbreviated["1"]
local s2 = c.el.dates.calendars.islamic.months.format.abbreviated["2"]
tex.print('typeof :', type(c), '\\par')
tex.print(s1, '\\par', s2)
\end{luacode}
\end{texexample}

\subsection{Available calendar translations}

The available calendar translations for each language, they are provided as submodules to the |i18n| module. They are listed in Table    . The prefix is the 2-letter name of the language so for English it will be ca-buddhist.

\begin{table}[ht]

\begin{multicols}{2}
ca-buddhist\\
ca-chinese\\
ca-coptic\\
ca-dangi\\
ca-ethiopic\\
ca-ethiopic-amete-alem\\
ca-generic\\
ca-gregorian\\
ca-hebrew\\
ca-indian\\
ca-islamic\\
ca-islamic-civil\\
ca-islamic-rgsa\\
ca-islamic-tba\\
ca-islamic-umalqura\\
ca-japanese\\
ca-persian\\
ca-roc\\
\end{multicols}
\caption{Available calendars for each language.}
\end{table}

\subsection{Historical Roman Calendars}

\subsection{Julian Calendar}

Julius Caesar in 46 BC (708 AUC\footnote{A.U.C. is from the Latin ab urbe condita, which translates from the founding of the City (Rome).}) reformed the then current calendar.   It was the predominant calendar in the Roman world, most of Europe, and in European settlements in the Americas and elsewhere, until it was refined and superseded by the Gregorian calendar. The difference in the average length of the year between Julian (365.25 days) and Gregorian (365.2425 days) is 0.002%.

The Julian calendar has a regular year of 365 days divided into 12 months, as listed in Table of months. A leap day is added to February every four years. The Julian year is, therefore, on average 365.25 days long. It was intended to approximate the tropical (solar) year. Although Greek astronomers had known, at least since Hipparchus, a century before the Julian reform, that the tropical year was a few minutes shorter than 365.25 days, the calendar did not compensate for this difference. As a result, the calendar year gained about three days every four centuries compared to observed equinox times and the seasons. This discrepancy was corrected by the Gregorian reform of 1582. The Gregorian calendar has the same months and month lengths as the Julian calendar, but inserts leap days according to a different rule. Consequently, the Julian calendar is currently 13 days behind the Gregorian calendar; for instance, 1 January in the Julian calendar is 14 January in the Gregorian. Old Style (O.S.) and New Style (N.S.) are sometimes used with dates to indicate either whether the start of the Julian year has been adjusted to start on 1 January (N.S.) even though documents written at the time use a different start of year (O.S.), or whether a date conforms to the Julian calendar (O.S.) rather than the Gregorian (N.S.). Dual dating uses two consecutive years because of differences in the starting date of the year, or includes both the Julian and Gregorian dates.

The Julian calendar has been replaced as the civil calendar by the Gregorian calendar in all countries which formerly used it, although it continued to be the civil calendar of some countries into the 20th century. Among the last countries to convert to the Gregorian Calendar were Greece (in 1924), Turkey (in 1926) and Egypt (in 1928).[2] As of 1930, all countries that were using the Julian calendar had discontinued it. Most Christian denominations in the West and areas evangelized by Western churches have also replaced the Julian calendar with the Gregorian as the basis for their liturgical calendars. However, most branches of the Eastern Orthodox Church still use the Julian calendar for calculating the dates of moveable feasts, including Easter (Pascha). Some Orthodox churches have adopted the Revised Julian calendar for the observance of fixed feasts, while other Orthodox churches retain the Julian calendar for all purposes.[3] The Julian calendar is still used by the Berber people of North Africa, and on Mount Athos. In the form of the Alexandrian calendar, it is the basis for the Ethiopian calendar, which is the civil calendar of Ethiopia.

The ordinary years in the previous Roman calendar consisted of 12 months, for a total of 355 days. In addition a 27-day intercalary month, the \textit{Mensis Intercalaris}, was sometimes inserted between February and March. This extra month was formed by inserting 22 days after the first 23 or 24 days of February, which counted down toward the start of March, 
became the last five days of Intercalaris. The result was to add2 or 23 days to the year forming an intercalary year of 377 or 
378 days.

Caesar returned to Rome in 46 BC and, according to Plutarch, called in the best philosophers and mathematicians of his time to solve the problem of the calendar.[16] Pliny says that Caesar was aided in his reform by the astronomer Sosigenes of Alexandria[17] who is generally considered the principal designer of the reform. Sosigenes may also have been the author of the astronomical almanac published by Caesar to facilitate the reform.[18] Eventually, it was decided to establish a calendar that would be a combination between the old Roman months, the fixed length of the Egyptian calendar, and the 365¼ days of the Greek astronomy. According to Macrobius, Caesar was assisted in this by a certain Marcus Flavius.[19]

Since the Julian and Gregorian calendars were long used simultaneously, although in different places, calendar dates in the transition period are often ambiguous, unless it is specified which calendar was being used. In some circumstances, double dates might be used, one in each calendar. The notation ``Old Style" (O.S.) is sometimes used to indicate a date in the Julian calendar, as opposed to ``New Style" (N.S.), which either represents the Julian date with the start of the year as 1 January or a full mapping onto the Gregorian calendar. This notation is used to clarify dates from countries which continued to use the Julian calendar after the Gregorian reform, such as Great Britain, which did not switch to the reformed calendar until 1752, or Russia, which did not switch until 1918.

Throughout the long transition period, the Julian calendar has continued to diverge from the Gregorian. This has happened in whole-day steps, as leap days which were dropped in certain centennial years in the Gregorian calendar continued to be present in the Julian calendar. Thus, in the year 1700 the difference increased to 11 days after February 28 (Gregorian); in 1800, 12; and in 1900, 13. Since 2000 was a leap year according to both the Julian and Gregorian calendars, the difference of 13 days did not change in that year: 29 February 2000 (Gregorian) fell on 16 February 2000 (Julian). This difference will persist through the last day of February, 2100 (Gregorian), since 2100 is not a Gregorian leap year, but is a Julian leap year. Monday 1 March 2100 (Gregorian) falls on Monday 16 February 2100 (Julian).[81]


\subsection{Gregorian calendar}

The routines for calculating and displaying Gregorian Calendar dates are provided by 
the \pkgname{phd} as lua interfaced code. They are also provided (with limited functionality) for the other \TeX\ engines.

The Gregorian calendar is the most widely calendar use today. The calendar was designed by a commission instructed by Pope~Gregory~XIII in the sixteenth century. This is strictly a solar calendar based on a 365-day common year divided into twelve months. Leap years have 366 days when one extra day is added to February.

For a computer implementation, the easiest way to reckon time is simply to count
days: Establish an arbitrary starting point as day 1 and specify a date by giving
a day number relative to that starting point;\footnote{See Leslie Lamport's \textit{On the Proof of Correctness of a Calendar Program}, Communications of ACM, Vol 22, Number 10, October 1979.} a single thirty-two bit integer allows
the representation of more than 11.7 million years. Such a reckoning of time is, evidently, extremely awkward for human beings and is not in common use, except among astronomers who use Julian day numbers to specify dates.

Thus a date is normally a triplet of three integers starting from an era. We specify a date such as (13, December, 2014) where we let "January", \ldots, ``December'' be the names for the integers $1,\ldots,12$. A calendar is the assignment of dates to days.  More precisely.

Let us define an era to be an infinite sequence of days. A calendar for that era is an assignment of a date to each day in the era. 
\[gregorian[n] = (day[n], month[n], year[n])\]
where the integer-valued functions $day, month$ and $year$ are defined inductively.

We will also need to define the range that we need to print calendar, as calendars are printed normally over a 42 day interval.

\begin{figure}[h]
\centering
\includegraphics[width=0.7\textwidth]{./images/calendar-01.jpg}
\end{figure}

The full page calendar is illustrated in Figure~\ref{girlcalendar}.
\begin{figure}[htb]
\centering
\includegraphics[width=\textwidth]{./images/calendar-02.jpg}
\caption{The full page rendering of the makeCalendar command}
\label{girlcalendar}
\end{figure}

The calendar holds the images and styles in  a Lua table, which is easily configurable to use it, you need to
use the \cmd{\makeCalendar}\meta{year}, which creates a pdf. Care must be used in selecting the right size images and with an aspect ratio that suits the paper dimensions. 

\clearpage

\subsection{Coptic calendar}

The Coptic calendar or Alexandrian calendar, which is a descendant of the Ancient Egyptian calendar, is still used by the Coptic Orthodox Church and still used in Egypt.   This calendar is based on the ancient Egyptian calendar. To avoid the calendar creep of the latter, a reform of the ancient Egyptian calendar was introduced at the time of Ptolemy III (Decree of Canopus, in 238 BC) which consisted of the intercalation of a sixth epagomenal day every fourth year. However, this reform was opposed by the Egyptian priests, and the idea was not adopted until 25 BC, when the Roman Emperor Augustus formally reformed the calendar of Egypt, keeping it forever synchronized with the newly introduced Julian calendar. To distinguish it from the Ancient Egyptian calendar, which remained in use by some astronomers until medieval times, this reformed calendar is known as the Coptic calendar. Its years and months coincide with those of the Ethiopian calendar but have different numbers and names.

\subsubsection{Coptic Year}

The Coptic Year retained the ancient Egyptian civil year and its divisions of three seasosn, four months each.  The three seasons are commemorated by special prayers in the Coptic Divine Liturgy. This calendar is still in use all over Egypt by farmers to keep track of the various agricultural seasons. The Coptic calendar has 13 months, 12 of 30 days each and an intercalary month at the end of the year of 5 or 6 days depending whether the year is a leap year or not. The year starts on 11 September in the Gregorian Calendar or on the 12th in the year before (Gregorian) Leap Years. The Coptic Leap Year follows the same rules as the Gregorian so that the extra month always has 6 days in the year before a Gregorian Leap Year. The Coptic Year is divided as follows:


\subsubsection{Coptic Months}  The Coptic calendar has thirteen months. 

\begin{table}[p]
\begin{scriptexample}[\arial]{Script}

\captionof{table}{Coptic calendar months.}
{\small
\begin{tabular}{l >{\pan}p{1.3cm} >{\pan\raggedright }p{1.3cm} >{\raggedright}p{1.3cm} l l >{\RaggedRight}p{3cm}}
\toprule
      &Bohairic & Sahidic & Coptic &Arabic &Date &Origin\\
\midrule      
1	& Ⲑⲱⲟⲩⲧ
      &{\pan Ⲑⲟⲟⲩⲧ} 	
      &Thout	  
      &Tout	
      &11 Sept – 10 Oct	
      &Akhet (Inundation)	Thoth, god of Wisdom \& Science\\
      
 2	&Ⲡⲁⲟⲡⲓ	&Ⲡⲁⲱⲡⲉ	&Paopi	&Baba	&11 Oct – 10 Nov	 &Akhet (Inundation)	Hapi, god of the Nile (Vegetation)\\
  
 
3	&Ⲁⲑⲱⲣ	&Ϩⲁⲑⲱⲣ	&Hathor	&Hatour	&10 Nov – 9 Dec	&Akhet (Inundation)	Hathor, goddess of beauty and love (the land is lush and green)\\  

4	&Ⲭⲟⲓⲁⲕ	&Ⲕⲟⲓⲁⲕ	&Koiak	&Kiahk	&10 Dec – 8 Jan	&Akhet (Inundation)	Ka Ha Ka = Good of Good, the sacred Apis Bull\\   
     
5	&Ⲧⲱⲃⲓ	&Ⲧⲱⲃⲉ	&Tobi	&Touba	&9 Jan – 7 Feb 	&Proyet, Peret, or Poret (Growth)	Amso Khem, a form of Amun-Ra (growth of nature and rain)\\
     
6	&Ⲙⲉϣⲓⲣ	&Ⲙϣⲓⲣ 	&Meshir	&Amshir	&8 Feb – 9 Mar &Proyet, Peret, or Poret (Growth)	Mechir, genius of wind (month of storms and wind) \\   

 7	&Ⲡⲁⲣⲉⲙϩⲁⲧ	&Ⲡⲁⲣⲙ̀ϩⲟⲧⲡ	&Paremhat	&Baramhat	&10 Mar – 8 Apr	 &Proyet, Peret, or Poret (Growth)	Mont, god of war (high temperatures; month of the sun)\\

8	&Ⲫⲁⲣⲙⲟⲩⲑⲓ	&Ⲡⲁⲣⲙⲟⲩⲧⲉ	&Parmouti	&Baramouda	&9 Apr – 8 May	&Proyet, Peret, or Poret (Growth)	Renno, severe wind and death (vegetation ends; earth is dry)\\ 

9	&Ⲡⲁϣⲟⲛⲥ	&Ⲡⲁϣⲟⲛⲥ	&Pashons	&Bashans	&9 May – 7 Jun	&Shomu or Shemu (Harvest)	Khenti, a form of Horus, god of metals\\

10	&Ⲡⲁⲱⲛⲓ	&Ⲡⲁⲱⲛⲉ	&Paoni	&Ba'ouna	&8 Jun – 7 Jul   &	Shomu or Shemu (Harvest)	p3-n-In = valley festival\\


11	&Ⲉⲡⲓⲡ	 &Ⲉⲡⲓⲡ	  &Epip	&Abib	&8 Jul – 6 Aug	&Shomu or Shemu (Harvest)	Apida, the serpent that Horus, son of Osiris, killed\\

12	&Ⲙⲉⲥⲱⲣⲓ	&Ⲙⲉⲥⲱⲣⲏ	&Mesori	&Mesra	 &7 Aug – 5 Sept	&Shomu or Shemu (Harvest)	Mesori, birth of the sun\\

13	&Ⲡⲓⲕⲟⲩϫⲓ  ⲛ̀ⲁ̀ⲃⲟⲧ	&Ⲕⲟⲩϫⲓ  ⲛ̀ⲁ̀ⲃⲟⲧ	 &Pi Kogi Enavot	&Nasie	&6 Sep – 10 Sep &	Shomu or Shemu (Harvest)	The Little Month  \\                

\bottomrule    
\end{tabular}

}
\vspace*{3cm}
\end{scriptexample}

\end{table}









                  
                  
                  
                  

\section{Ethiopic calendar}

The Ethiopic calendar, is very similar to the Coptic. “The day starts with sunrise” is the conceptual basis for the clock in Ethiopia and many of its neighbors. Being near the equator this translates to roughly 6 AM each day with an even 12 hours of light and darkness with only a little seasonal drifting. A twelve hour clock is used that begins at “12 AM” with sunrise (aka 6 AM in the West), reaches “noon” at “6 AM”, followed by “12 PM” 6 hours later and “6 PM” at “midnight”. Think of it as a clock or watch with the “6” at the top and the “12” at the bottom.

The calendar in Ethiopia has 13 months and the year is 7 years 8 months and 11 days behind the Gregorian calendar (12 days when a leap year occurs). Which means that the year 2000 has only just occurred on September 12th of this year! But why? Many references will state that the Ethiopic calendar is based on the Julian, but this is only partly true. The Ethiopic calendar descends more directly from the Coptic which in turn is a reformation of the ancient Egyptian solar calendar with respect to the Julian scheme also known as the “Alexandrian Calendar”.

The ancient Egyptian solar calendar used a 365 day year with the year divided into 3 seasons of 120 days and each season into 4 months of 30 days. Five corrective, or epagomenal, days were added at the end of the year. The months were only numbered initially but later took on the corresponding month names from a second, lunar based calendar of Egypt. The month names under the lunar calendar derived their names from the major feast that would occur during the respective month. The problem with “calendar creep” was not addressed until the arrival of the Julian calendar in 46 BC with the introduction of an extra day for a “leap year”. The Coptic calendar applied the Julian leap year in 25 BC thus forever fixing the date synchronization between the two calendar systems.

However, the Coptic and Ethiopic calendars do not apply the leap year correction rule where leap year is skipped every 100 years, except every 400 years, except (maybe) every 16,000 years. So “calendar creep” will continue between the Coptic, Ethiopic and Gregorian calendars. Leap years however do not occur when the year is a multiple of 4, as with the Gregorian system, but will occur on the year prior. Thus 1999 was a leap year as will be 2003 and so on. The four year cycle is also enumerated with the names of the four Evangelists: Mateos, Markos, Luqas and Yohannes as they are known in Ethiopian Orthodox Church. Yohannes is the leap year and is considered the end of a four year cycle.

Where the Coptic and Ethiopic calendars differ will be in the month names, which are language specific even within Ethiopia, as are the days of the week and day divisions. The other major difference is that the year in the Coptic calendar is presently 1724 some 276 years behind the Ethiopic. The modern Coptic calendar’s origin, or epoch, is counted from the year 284 AD when many Coptic Christians were martyr under the rein of Roman Emperor Diocletian. So what accounts for the difference in the Ethiopic calendar versus the Gregorian? Popular legend has it that the 7 year, 8 month and 11 day difference is the time it took for the news of the Birth of Christ to reach Ethiopia. More likely the answer is that the Ethiopic calendar went through fewer reformations than did other calendar systems (notably skipping the reformation by Dionysius Exiguus in AD 525) which potentially makes it more in keeping with “actual time” since the birth of Christ. The truth is, we may never know.

\begin{luacode}
ethiopicMonths = {
"Meskerem",
"Tekemt",
"Hedar",
"Tahsas",
"Ter",
"Yekatit",
"Megabit",
"Miazia",
"Genbot",
"Sene",
"Hamle",
"Nehasse",
"Pagumen"
}

for i=1,13 do 
   tex.print(ethiopicMonths[i]..', ')
end
\end{luacode}


\section{The French Revolutionary Calendar}

The French Revolution, accompanied by a zeal to change all traditional things in France, induced the rulers to change their calendar along with their government.  It was decreed by the National Convention, in the autumn of 1793, that the old calendar was to be abolished and that the new French era should be reckoned from the foundation of the republic, September 22, 1792, of the Common Era, on the day of the true autumnal equinox; that each year should begin on the midnight of the day on which the autumnal equinox falls; and that the first year of the French Republic had begun immediately after 12 o’clock P.M. of the 21st of September, 1792, and had terminated on the midnight between the 21st and 22nd of September, 1793.

As the French months consisted of 30 days each, making in all 360 days, the remaining five days required to complete the year were called \textit{complementary days} and \textit{sans-culottides}. They were named as follows:

\begin{tabular}{llll}
1. Primedi  & Fe\^ete de la Vertu  & The Virtues & Sept. 17th \\
\end{tabular}

The intercalary day of every fourth year was called La  sans-culottide, and was to be the Festival of 
the Revolution,  to be dedicated to a grand solemnity, in which the French should celebrate the period of their enfranchisement, and the  institution of the Republic. The National oath, \enquote{To live  free or die,} was to be renewed. 

Each day was divided according to the decimal system, 
into ten parts or hours, and these into ten others, and so on. 
Each month was divided into three decades, each consisting of ten days ; 
the names of which were taken from the 
Latin numerals. The first was called Primedi, 2nd Duodi, 3rd 
Iridi, 4th Quartidi, 5th Quintidi, 6th Sextidi, 7th Septidi, 8th 
Octidi, 9th Nonidi, and 10th Decadi. The last was the day 
of rest, and superseded the former Sunday. \index{Calendars>French Revolutionary}

\begin{figure}[htp]
\centering
\includegraphics[width=\textwidth]{./images/republican-calendar.jpg}
\caption{The months illustrated the Republican Calendar  
Author: Salvatore Tresca in 1794. 
Paris, Musée Carnavalet. }
\end{figure}

\subsection{Revolutionary Calendar to Gregorian Calendar}

The year can easily be reckoned by taking the Gregorian year and adding 1. 

\begin{texexample}{Revolutionary year example}{}
\begin{luacode}
  local year, revyear
  year = 1792
  revyear = 1
  tex.print((year-revyear)+1)
\end{luacode}
\end{texexample}

\section{Epoch}
Every calendar has an \textit{epoch} or starting date. This date is virtually never teh date the calendar was adopted but rather a hypothetical starting point for the first day of the calendar. The best example is the Gregorian calendar which was adopted in the sixteenth century, but is epoch is January 1, 1.



\section{Julian Day Numbers}
Julian day is the continuous count of days since the beginning of the Julian Period and is used primarily by astronomers.

The Julian Day Number (JDN) is the integer assigned to a whole solar day in the Julian day count starting from noon Universal time, with Julian day number 0 assigned to the day starting at noon on Monday, January 1, 4713 BC, proleptic Julian calendar (November 24, 4714 BC, in the proleptic Gregorian calendar),[1][2][3] a date at which three multi-year cycles started (which are: Indiction, Solar, and Lunar cycles) and which preceded any dates in recorded history.[4] For example, the Julian day number for the day starting at 12:00 UT on January 1, 2000, was 2 451 545.[5]

The Julian date (JD) of any instant is the Julian day number plus the fraction of a day since the preceding noon in Universal Time. Julian dates are expressed as a Julian day number with a decimal fraction added.[6] For example, the Julian Date for 00:30:00.0 UT January 1, 2013, is 2 456 293.520 833.[7]

The Julian Period is a chronological interval of 7980 years; year 1 of the Julian Period was 4713 BC.[8] It has been used by historians since its introduction in 1583 to convert between different calendars. The Julian calendar year 2018 is year 6731 of the current Julian Period. The next Julian Period begins in the year AD 3268.


\begin{tabular}{l l} 
JD 0 &= Noon on Monday, January 1, 4713 BCE (Julian)\\
     &= Noon on Monday, November 24, -4713 (Gregorian)\\
\end{tabular}     
     
The Julian day number can be calculated using the following formulas (integer division is used exclusively, that is, the remainder of all divisions are dropped)

The months (M) January to December are 1 to 12. For the year (Y) [[astronomical year numbering]] is used, thus 1 BC is 0, 2 BC is −1, and 4713 BC is −4712. D is the day of the month. JDN is the Julian Day Number, which pertains to the noon occurring in the corresponding calendar date.

\begin{texexample}{fmod}{}
\begin{luacode}
local a,b = 17,3
local c = (a - math.fmod(a,b))/b
tex.print(c)

math.div = function(a,b) 
    local c = (a - math.fmod(a,b))/b
    return c
end

tex.print(math.div(17,3))   

\end{luacode}
\end{texexample}


\vfill





\parindent1em


\chapter{Internationalization and Globalization}


\section{Introduction}

In this Chapter we discuss the requirements for localization of software and how this can be applied to \latex. In a way this chapter overlaps the one on languages. However, here we focus mostly on LuaTeX solutions. We also extend the discussion to calendric and solar calculations.

Internationalization is the process of designing a software application so that it can potentially be adapted to various languages and regions without engineering changes. Localization is the process of adapting internationalized software for a specific region or language by adding locale-specific components and translating text. Localization (which is potentially performed multiple times, for different locales) uses the infrastructure or flexibility provided by internationalization (which is ideally performed only once, or as an integral part of ongoing development).\index{internationalization}\index{globalization}

The development of routines for software internationalization and globalization has been an ongoing effort for many years. Currently the accepted method for building such software is the use of i18n. This is an abbreviation of the first letter and last letter of the word internationalization and the 18 is the number of characters in the word.

Internationalization based on i18n is not an easy task for \LaTeX. To an extend some of the issues have been removed with the use of Babel and Polyglossia that provide translation strings for many of the worlds scripts. The de facto resource for internationalization is the Unicode Consortium’s \href{http://cldr.unicode.org/}{CLDR} project.\index{i18n}

\section{Enforcing local styles}

To understand the magnitude of the problem let us look at some of the easier parts of localizing. Consider the Greek days of the week.
\medskip
\begin{trivlist}\item[]\panunicode
\begin{tabular}{llll}
\toprule
Day &Normal Form &Abbreviation &Narrow\\
Monday &Δευτέρα &Δευ. &Δ. \\
\midrule
\end{tabular}
\end{trivlist}

In Greek the abbreviated form, is always capitalized and a stop is provided. The same is true for the month. The narrow form can give problems, unless it is for calendars, where the content is clear. This is because "{\panunicode Π}" are the initials for both ``{\panunicode Πέμπτη}" (Thursday), and ``{\panunicode Παρασκευή}" (Friday). 

In date formats with long month format, that do not include the day, the full month form should be used.
In date formats with long month format, that also include the day, the long date format should be used.

If limited space is available, it is possible to omit the period in the abbreviated form of months, but this should be used only when there is a serious technical restriction

Ultimately, we are aiming at providing the necessary rules to build an automated style that can be used by the system.
                

\section{Locales}
\index{locale}

In computing, a \emph{locale} is a set of parameters that defines the user's language, country and any special variant preferences that the user wants to see in their user interface. Usually a locale identifier consists of at least a \textit{languag}e identifier and a \textit{region} identifier.

On POSIX platforms such as Unix, Linux and others, locale identifiers are defined similar to the BCP 47 definition of language tags, but the locale variant modifier is defined differently, and the character set is included as a part of the identifier. It is defined in this format: |[language[_territory][.codeset][@modifier]]|. (For example, Australian English using the UTF-8 encoding is en\_AU.UTF-8.)

For \latex these ``locales'' can be thought of as the settings of language keys through Babel and Polyglossia. These settings have served the community well for many years, but a litany of duct taping through other packages are a testimony to their limitations. Packages for dates, time and number formatting have been developed to assist. Here is my attempt to put the solution on a better footing and to start providing mechanisms via LuaTeX for a 'plugin'
architecture to find improve solutions. 

\section{Common Locale Data Repository}

The Common Locale Data Repository Project, is a project of the Unicode Consortium to provide locale data in the XML format for use in computer applications. CLDR contains locale specific information that an operating system will typically provide to applications. CLDR is written in LDML (Locale Data Markup Language). The information is currently used in International Components for Unicode, Apple's Mac OS X, OpenOffice.org, and IBM's AIX, among other applications and operating systems

\begin{enumerate}
\item Translations for language names.
\item Translations for territory and country names.
\item Translations for currency names, including singular/plural modifications.
\item Translations for weekday, month, era, period of day, in full and abbreviated forms.
\item Translations for timezones and example cities (or similar) for timezones.
\item Translations for calendar fields. This is useful especially in conjuction with PGF presentational forms.
\item Patterns for formatting/parsing dates or times of day.
\item Examplar sets of characters used for writing the language.
\item Patterns for formatting/parsing numbers.
\item Rules for language adapted collation. \label{collation}
\item Rules for formatting numbers in traditional numeral systems (like Roman numerals, Armenian numerals, ...).
\item Rules for spelling out numbers as words.
\item Rules for transliteration between scripts. A lot of it is based on BGN/PCGN romanization.
\item Rules for \emph{delimiters} such as quotations and question marks.
\end{enumerate}

Currently the consortium’s distribution make the data available in both json and xml formats.  These files hold data for a specific \emph{locale}. Sadly missing are any document sectioning information that would have enabled the incorporation of the above into LaTeX and overcoming some of the Babel and Polyglossia limitations.

We do not need many of the files provided by the CLDR unicode consortium and others we are missing. Take for example the |delimiters| file. 

\bgroup
\scriptsize
\begin{phdverbatim}
  "main" = {
    "ff": {
      "identity": {
        "version": {
          "_cldrVersion": "26",
          "_number": "$Revision: 10739 $"
        },
        "generation": {
          "_date": "$Date: 2014-08-07 12:54:13 -0500 (Thu, 07 Aug 2014) $"
        },
        "language": "ff"
      },
      "delimiters": {
        "quotationStart": "„",
        "quotationEnd": "”",
        "alternateQuotationStart": "‚",
        "alternateQuotationEnd": "’"
      }
    }
  }
}
\end{phdverbatim}
\egroup

Of course the |Json| format as it is, is not readable by Lua a format such as:

\begin{verbatim}
delimiters = {
        quotationStart = "«",
        quotationEnd = "»",
        alternateQuotationStart = "\"",
        alternateQuotationEnd = "\""
      }
\end{verbatim}

\begin{texexample}{i18n}{i18-1}

\panunicode
\begin{luacode*}
-- mock the delimiters from the json
-- file
greekname = 'el'
delimiters = {
        quotationStart = "«",
        quotationEnd = "»",
        alternateQuotationStart = [["]],
        alternateQuotationEnd = [["]]
      }
tex.print(delimiters.quotationStart .. 'test' .. delimiters.quotationEnd)
tex.print ([[\gdef\]] .. greekname .. [[quote#1{\directlua{tex.sprint(delimiters.quotationStart .. '#1' .. delimiters.quotationEnd)}}]])
\end{luacode*}

\def\elquote#1{%
  \directlua {tex.sprint(delimiters.quotationStart .. '#1' .. delimiters.quotationEnd)}
}
\end{texexample}



This is of course a much more simplified way of what one needs to program for a full system. The advantage
of producing the \tex definition also through LuaTeX is that we can keep all the code in one place and econd, we can avoid |\csname| costructs.
\begin{texexample}{elquote}{}
\elquote{This is some longer text in Greek quotes.}
\end{texexample}

I have opted to incorporate these files in the |json| format and provide routines for interfacing via the \pkgname{phd} package.  The reason for opting for a json format, is my other attempts to interface the package with |couchdb|.  My preference for a Nosql type of database, is that  they are better suited in handling data that is commonly  found in documents and also many of the routines will be interchangeable for web applications. I am also hoping that the collation information (see \ref{collation}), will eventually lead to better indices, a subject left untouched in the current distribution.\index{json}

\section{The package phd approach}

The package |phd| packge takes an approach to use only json resource files for the provision of language dependent information, rather than TeX commands alone, as is done by Babel and Polyglossia. 

\section{Language and Region Tags}
\index{tags>regions}\index{tags>language}

Languages are represented by tags such as "en"  for English or "el" for Greek. Other languages have no significant variation and are represented by a language subtag such as "en-US".  The names are mostly intuitive, but in many case bear no relationship to their English names, for example Armenian is coded as \textbf{hy}. There is a useful utility at the SIL website for viewing these codes.\footnote{\protect\url{http://www-01.sil.org/iso639-3/codes.asp?order=reference_name&letter=\%25}.} Note that the CLDR database does not cover all the languages listed in the ISO-639.\footcite{iso639} \index{ISO-639}

The language tags are based on the BGN which is mapped to languages based on ISO-639-1.

ISO 639-2 is the alpha-3 code in Codes for the representation of names of languages-- Part 2. There are 21 languages that have alternative codes for bibliographic or terminology purposes. In those cases, each is listed separately and they are designated as "B" (bibliographic) or "T" (terminology). In all other cases there is only one ISO 639-2 code. Multiple codes assigned to the same language are to be considered synonyms. ISO 639-1 is the alpha-2 code.

We will describe the tables using the English language, which is normally the default and Greek as a second language, as the script is distinctive enough to demonstrate their use. We will also explain Lua routines available via the \pkgname{phd} that are provided as alternatives to Babel and Polyglossia.

{layout.lua}

{layout.orientation.characterOrder} = |left_to_right| or |right_to_left|

layout.orientation.lineOrder = |top_to_bottom|

Example \ref{i18-1} loads the Greek internationalization file |layout| and prints the two fields. Before we send it to
the TeX typesetter we sanitize the string underscores using |gsub|. For illustration purposes we have used |gsub| both as an object method and as a function.

\begin{texexample}{i18n}{i18-1}
\begin{luacode}
local c = require("i18n.el.layout")
local s1 = string.gsub(c.el.layout.orientation.characterOrder, '_', '\\textunderscore ')
local s2 = c.el.layout.orientation.lineOrder:gsub('_', '\\textunderscore ')
tex.print('typeof :', type(c))
tex.print(s1, '\\par', s2)
\end{luacode}
\end{texexample}

Of course for Greek the above information is hardly necessary, but at the level of Lua programming, if we are automating the switching of text direction Greek text might signal a change in direction. Let us have another try using the same code for arabic text. All we have to change is the \textbf{el} to \textbf{ar}.

\begin{texexample}{i18n}{i18-2}
\begin{luacode}
local c = require("i18n.ar.layout")
local s1 = string.gsub(c.ar.layout.orientation.characterOrder, '_', '\\textunderscore ')
local s2 = c.ar.layout.orientation.lineOrder:gsub('_', '\\textunderscore ')
tex.print('typeof :', type(c), '\\par')
tex.print(s1, '\\par', s2)
\end{luacode}
\end{texexample}

In the next example we get the string for the first month of the year in the ``abbreviated'' style. I have changed the json
strings directly to Lua for this file to speed up processing.

\begin{texexample}{i18n}{i18-2}
\begin{luacode}
local c = require("i18n.el.cagregorian")
local months = c.el.dates.calendars.gregorian.months.formats
local days = c.el.dates.calendars.gregorian.days.formats

tex.print("\\begin{tabular}{ll} ")
for i=1,12 do
  tex.sprint(i.." &"..months.wide[i].."\\\\ ")
end
tex.print("\\end{tabular}")
\end{luacode}
\end{texexample}

Printing directly to the document has many benefits but does slow developemnt, both of the code as well as the document. Another distraction is transferring arguments from \tex to Lua and vice versa.

Similarly we can print the months in the Italian language by loading the \textbf{i18n.italian} module and iterating through the month strings. I am still thinking about the interface and the best way forward to provide an easy to use and remember interface. 


Let us now develop a longer example. We will load a number of languages and typeset a table for the different months.
Since we are running the example directly in the document, some patience is required. 

\bigskip 

\begin{texexample}{Month string in various languages}{ex:transl}
\bgroup
\parindent0pt
\newfontfamily\langtable{code2000}
\langtable
\scriptsize
\begin{luacode} 

c = require("i18n.irish")
d = require("i18n.russian")
e = require("i18n.latin")
f = require("i18n.german")
g = require("i18n.kannada")
h = require("i18n.lao")
j = require("i18n.turkish")
k = require("i18n.albanian")

local count=0

local months_irish = c.irish.months
local months_russian = d.russian.months
local months_latin = e.latin.months 
local months_german = f.german.months
local months_kannada = g.kannada.months
local months_lao    = h.lao.months
local months_turkish = j.turkish.months
local months_albanian = k.albanian.months
local centering = function()
                     tex.print("\\centering")
end

local par = function()
               tex.print("\\par")   
end

local tabular = function() 
	tex.print("\\begin{tabular}{clllllll}")
	tex.sprint("\\toprule")
end	


local endtabular = function()
	tex.print("\\bottomrule")
	tex.print("\\end{tabular}")
	tex.print("\\medskip")
end

local eol = function()
  return("\\\\")
end


-- center the table
centering()
tabular()
tex.sprint("Month","&Irish", "&Russian", "&Latin", "&Kannada", "&Lao","&Turkish","&Albanian", eol())
tex.sprint("\\midrule")
for i = 1,12 do
   count = i
   tex.sprint(i.."&", months_irish[i],
                 "&",months_russian[i], 
                 "&",months_latin[i], 
                 "&", months_kannada[i], 
                 "&"..months_lao[i], 
                 "&"..months_turkish[i],
                 "&"..months_albanian[i],
                 eol() )
end  
endtabular()
par()

\end{luacode} 
 
\egroup
\end{texexample}

Now some explanation for the code. We started by loading the necessary libraries for the languages that we wanted to print the month strings and allocated them to local variables.

We then iterated through the twelve months of the gregorian table and typeset them. We could have put the languages in a Lua table and iterated over them. I haven't done it so that the code is clearer. I tried to keep the API functions separate as much as possible. We also defined a font using \docAuxCommand{newfontfamily} of the \pkg{fontspec} package to ensure that we can print the Asian and Cyrillic scripts.

The long javanesque object notations make it difficult to work, but once they are set in functions and locals, development is fast. After the detour to explore the i18n tables and available information, we are now ready to tackle the production of multi-lingual calendars and to complete are library on internationalization. Before we do that a detour to understand
the complexity of calendrical calculations and some historical information is required.
\vfill




%\chapter{Directionality of Scripts}

Most of the languages around the world are printed horizontally, from left-to-right, with the first
line of the page at the top of the page. In the Middle East, Arabic, Hebrew and several other scripts are
written horizontally, from right-to-left. In East Asia, traditional writing and printing is done vertically,
with the first line of the page on the right-hand side; in Japan, this practice is still common for literature.
Uighur and Mongolian writing is also vertical, but the first line of the page is on the left.

The general problem does not consist in simply printing a text in each of the different directions.
A typesetting system to be used for multilingual documents needs to be able to print, on the same
page, multiple languages using different direction combinations, as needed. This means that headers,
footers, columns, tables, paragraphs, and everything else appearing on a page can potentially appear in
the different directions.

Genkō yōshi {\panunicode (原稿用紙}, \enquote{manuscript paper}) is a type of Japanese paper used for writing. It is printed with squares, typically 200 or 400 per sheet, each square designed to accommodate a single Japanese character or punctuation mark. Genkō yōshi may be used with any type of writing instrument (pencil, pen or ink brush), and with or without a shitajiki (protective \enquote{under-sheet}).

\begin{figure}[htbp]
\includegraphics[width=0.9\textwidth]{genkoyoshi}

\bgroup
\footnotesize
Correct use of genkō yōshi (400 square sheet shown):
\narrower\narrower
\begin{enumerate}
\item Title on the 1st column, first character in the 4th square.
\item Author's name on the 2nd column, with 1 square between the family name and the given name, and 1 empty square below.
\item First sentence of the essay begins on the 3rd column, in the 2nd square. Each new paragraph begins on the 2nd square.
\item Subheadings have 1 empty column before and after, and begin on the 3rd square of a new column.
\item Punctuation marks normally occupy their own square, except when they will occur at the top of a column, in which case they share a square with the last character of the previous column.
\end{enumerate}
\egroup

\caption{Correct use of genkō yōshi (credit: wikimedia)}
\end{figure}

The difficulty of both entering text, as well as integrating such a system with \tex is obvious. \xetex and \luatex both provide capabilities of switchin the text direction based on three letter combinations, following a text direction command.

\section{Historical overview}
A historical overview is given by Plaice\footcite{plaice2013} and summarizes the state of the art as of 2013. Most of what follows in this overview is based on this article.

\subsection{TeX–XeT}
In the TEX world, the first work [5] in multidirectionality
was made by Donald Knuth and Pierre
MacKay, who developed TEX– XET, which allowed
the mixing of left-to-right and right-to-left horizontal
texts in the same paragraph. This was done by using
nested |\beginL–\endL| and |\beginR–\endR| pairs
to, respectively, embed left-to-right and right-to-left
texts. However, their work supposes that all pages
are left-to-right, so it is not suitable for true right-to-left
documents.

\subsection{pTEX}

Mixed horizontal and vertical typesetting was introduced
in the TEX world with pTEX [2], a tool
developed at ASCII Corporation in Japan. Still used
in Japan, pTEX allows a vertical or a horizontal list
to be begun with either of the |\yoko| and |\tate|
primitives. In |\yoko| mode, horizontal and vertical
boxes have the same meaning as in standard TEX. In
|\tate| mode, |\hboxes| are vertical and |\vboxes| are
horizontal.

\subsection{CSS}

The work on supporting multiple directions in Cascading
Style Sheets (CSS) for HTML [1] has introduced
some useful terminology:

\textbf{Block flow direction}: from first to last line;\\
\textbf{Inline base direction}: from first to last glyph;\\
\textbf{Line orientation}: direction towards “top” of line.

\subsection{Omega}

Omega was the first successor to \tex to attempt to solve the multidirectional problem in its generality [3,
4, 8]. It assumes that a box or a font’s direction can be designated by three characters, where each is one
of Top, Bottom, Left, and Right. These characters absolutely designate one of the edges of the physical
page. Then a writing direction must designate:

\begin{description}
\item [Primary part] The \enquote{top} of each page.
\item [Secondary part] The \enquote{left} of each line.
\item [Tertiary part] The \enquote{top} of each character.
\end{description}

The secondary direction must be orthogonal to the
primary direction. The tertiary direction can take all
four values. Hence there are 32 possible directions.
Here are the most common ones:

TLT — Left-right (LR) scripts, horizontal CJK.

TRT — Right-left (RL) scripts.

RTT — Vertical CJK, upright LR scripts in vertical
CJK.

LTL — Mongolian and Uighur (MU).

RTL —MU scripts in vertical CJK.

RTR — Rotated LR scripts in vertical CJK.

LTR — Rotated LR scripts in MU.

LTT — Vertical CJK in MU.

Notwithstanding the impressive number of possible
writing directions, the proposed solution was not
sufficiently general, as it did not make provisions for
phenomena such as typesetting to a curve. Furthermore,
it required different fonts for different writing
directions, despite the fact that many of them simply
involved rotating text.




\section{LuaTeX's Direction identifiers}

The \luatex manual is terse and short in providing examples and difficult sometimes to understand how to use. Most of the discussion that follows is based on an explanation by David Carlisle.

The luatex system distinguishes four different directions,
TLT, TRT, RTT, LTL.
The three letters in the name denote three aspects of the typesetting direction behavior.

\begin{enumerate}
\item
The direction towards the \enquote{top} of the paragraph
(that is, the start of the vertical mode direction)
is one of 

T(top), L(left), R(right).

For English and Arabic, the beginning of the paragraph is \textbf{T};

for Japanese vertical text it is \textbf{R};

for Mongolian it is \textbf{L}.


\item 
The direction towards the beginning of the line
(that is, the start of the horizontal mode direction)
is one of
T(top), L(left), R(right).

Defines  where  each  line  begins.

For  English, it is L;

for Arabic, it is R;

for Japanese and Mongolian, it is T.


\item 
The top of the glyphs within the line
is one of
T(top), L(left).

\bigskip

These result in the following typical settings:

TLT for English,

TRT for  Arabic,

RTT for  Japanese,

LTL for  Mongolian.

\end{enumerate}



\section{Direction registers}

The direction state is stored in five registers as listed below

\halign{&# \hfil\cr
page &|tex.pagedir|& |\pagedir|\cr
text &|tex.textdir|& |\textdir|, |\linedir|\cr
mathematics &|tex.mathdir|& |\mathdir|\cr
body &|tex.bodydir|& |\bodydir|\cr
paragraph &|tex.pardir|& |\pardir|\cr}



Each of these primitives takes as primitive one of the above four writing directions.



\begin{docCommand}{pagedir}{\meta{direction}}
\end{docCommand}


Can |\bodydir| be different to |\pagedir|? If it is different get warning
warning  (backend):
  pagedir differs from bodydir, the output may be placed wrongly on the page.


\begin{docCommand}{pardir}{\meta{writing direction code}}
Sets the paragraph direction.
\end{docCommand}


This defines the direction of the paragraph building.\par
In the default |\pagedir| TLT |\bodydir| TLT |\textdir| TLT then

TLT:
paragraph indentation left of first line, at top.
|\rightskip| fills from the right and |\parfillskip| fills the bottom line, from the right

TRT:
paragraph indentation left of first line, at top.
|\rightskip| fills from the left and |\parfillskip| fills the bottom line from the left,

LTL
paragraph indentation left of first line, at top.
|\rightskip| is a vertical skip after each line

RTT
paragraph indentation vertical above first line, at top.
|\rightskip| is a vertical skip after each line


\begin{docCommand}{textdir}{\meta{direction}}
\end{docCommand}

This primitive can appear anywhere in a text.
Grouping is respected, so it is possible to have inserts within a
paragraph.

\begin{docCommand}{linedir}{\meta{direction}}
\end{docCommand}

The |\linedir| primitive sets the same text direction parameter but
with a modified positioning of the direction nodes with respect to white
space, which is convenient in some cases. Compare the two examples below.


|abc {\textdir TRT xyz \textdir TLT 123} abc|

abc {\textdir TRT xyz \textdir TLT 123} abc

|abc {\linedir TRT xyz \linedir TLT 123} abc|

abc {\linedir TRT xyz \linedir TLT 123} abc


Note how in the first case the space after |xyz| in the source
is affected by the direction and so ends up visually adjacent to the
space after |abc|, with no space visible between xyz and 123.
|\linedir| adjusts the position of inserted direction nodes relative
to adjacent space characters so that text runs remain separated by the
spaces.


\begin{texexample}{Lua text direction}{ex:textdir}
\bgroup
\panunicode
\linedir TRT
\pardir TRT
كنت أريد أن أقرأ كتابا عن تاريخ المرأة في فرنسا‬

\begin{luacode*}
  tex.print(tex.pagedir, tex.linedir)
    
\end{luacode*}
\egroup
\end{texexample}

\begin{docCommand}{mathdir}{\meta{direction}}
Sets the direction for math typesetting.
\end{docCommand}
Normally mathematics is done in the same direction as English, namely
TLT There have been situations where it has been written TRT.

TLT: left to right\par
TRT: Right to left\par
LTL: down with superscripts to the left\par
RTT: down with superscripts to the right\par



\section{Box directions}

\begin{description}

\item [\cs{boxdir}]

The |\boxdir| primitive allows the direction of a previously
constructed box register to be altered.

For example:

|\newbox\bxA|\par
|\setbox\bxA\hbox{abc}|\par
|1 \box\bxA\ 2 \boxdir\bxA TRT\ box\bxA|

Produces
%  1 abc 2 cba

\newbox\bxA
\setbox\bxA\hbox{abc}
1 \copy\bxA\ 2 \boxdir\bxA TRT\ \copy\bxA

Note that this only specifies the initial direction, any direction
nodes that were saved in the box are retained with their original values.

|\newbox\bxB|\par
|\setbox\bxB\hbox{\textdir TLT abc}|\par
|1 \copy\bxB\ 2 \boxdir\bxB TRT\ \copy\bxB|

Produces
% 1 abc 2 abc

\newbox\bxB
\setbox\bxB\hbox{\textdir TLT abc}
1 \copy\bxB\ 2 \boxdir\bxB TRT\ \copy\bxB

\end{description}

\section{Direction nodes}

to be added\dots

\section{Balancing the direction stack}

Something about how direction nodes are constructed to add |+| or |-|
prefixed direction nodes to generate a balanced stack for the back end
processing\dots


\section{ local\textunderscore par nodes}

Perhaps, add something, as far as they relate to directionality\dots

\section{Paragraph Shape}

Something about |\parshape| and |\shapemode| to be added\dots

%\cxset{steward,
  numbering=arabic,
  custom=stewart,
  offsety=0cm,
  image={europa.jpg},
  texti={An introduction to the use of font related commands. The chapter also gives a historical background to font selection using \tex and \latex. },
  textii={In this chapter we discuss keys that are available through the \texttt{phd} package and give a background as to how fonts are used
in \latex.
 },
 pagestyle = empty,
}

\chapter{European Alphabetic Scripts}

\section{Introduction}

Modern European alphabetic scripts are derived from or influenced by the Greek script,
which itself was an adaptation of the Phoenician alphabet. A Greek innovation was writing
the letters from left to right, which is the writing direction for all the scripts derived from or
inspired by Greek

The European alphabetic scripts and additional characters described in this chapter follow the Unicode blocks:
\medskip


\begin{center}
\begin{tabular}{lll}
Latin &Cyrillic &Georgian.\\
Greek. &Glagolitic. &Modifier letters\\
Coptic &Armenian. &Combining marks\\
\end{tabular}
\end{center}

\section{Latin Script}

Latin script, or Roman script, is an alphabet based on the letters of the classical Latin alphabet. It is used as the standard method of writing in most Western and Central European languages, as well as many languages from other parts of the world. Latin script is the basis for the largest number of alphabets of any writing system[1] and is the most widely adopted writing system in the world (commonly used by about 70\% of world's population). It is also the basis of the International Phonetic Alphabet. The 26 most widespread letters are the letters contained in the ISO basic Latin alphabet.

The script is either called Roman script or Latin script, in reference to its origin in ancient Rome. In the context of transliteration the term "romanization" or "romanisation" is often found.[2][3] Unicode uses the term "Latin"[4] as does the International Organization for Standardization (ISO).[5] The numerals are called Roman numerals.


\subsection{Ligatures}

\newfontfamily\pan{code2000.ttf}

Ligatures for the Latin script are found in the Unicode block Alphabetic Presentation Forms which contains standard ligatures for the Latin, Armenian, and Hebrew scripts.

\begin{scriptexample}[]{Ligatures}
\unicodetable{pan}{"FB00,"FB10,"FB20,"FB30,"FB40}
\end{scriptexample}

\newfontfamily\georgian[Script=Georgian,Scale=1.2]{code2000.ttf}

\newfontfamily\georgianarial[Script=Georgian,Scale=1.2]{Arial Unicode MS}
\section{Georgian}
\label{sec:georgian}
The Georgian scripts are the three writing systems used to write the Georgian language: Asomtavruli, Nuskhuri and Mkhedruli. Their letters are equivalent, sharing the same names and alphabetical order and all three are unicameral (make no distinction between upper and lower case). Although each continues to be used, Mkhedruli (see below) is taken as the standard for Georgian and its related Kartvelian languages\footnote{Unicode Standard, V. 6.3. U10A0, p. 3}. 

\bgroup
\topline



\begin{scriptexample}[]{}
\georgian 

\centering
 
ყველა ადამიანი იბადება თავისუფალი და თანასწორი თავისი ღირსებითა და უფლებებით. მათ მინიჭებული აქვთ გონება და სინდისი და ერთმანეთის მიმართ უნდა იქცეოდნენ ძმობის სულისკვეთებით.
\medskip

\georgianarial
ყველა ადამიანი იბადება თავისუფალი და თანასწორი თავისი ღირსებითა და უფლებებით. მათ მინიჭებული აქვთ გონება და სინდისი და ერთმანეთის მიმართ უნდა იქცეოდნენ ძმობის სულისკვეთებით.
\bottomline
\captionof{table}{Article 1 of the Universal Declaration of Human Rights in Georgian, typeset in \texttt{code2000} (top) and \texttt{Arial Unicode MS } (bottom).}

\end{scriptexample}

The scripts originally had 38 letters. Georgian is currently written in a 33-letter alphabet, as five of the letters are obsolete in that language. The Mingrelian alphabet uses 36: the 33 of Georgian, one letter obsolete for that language, and two additional letters specific to Mingrelian and Svan. That same obsolete letter, plus a letter borrowed from Greek, are used in the 35-letter Laz alphabet. The fourth Kartvelian language, Svan, is not commonly written, but when it is it uses the letters of the Mingrelian alphabet, with an additional obsolete Georgian letter and sometimes supplemented by diacritics for its many vowels.

\chapter{Armenian}

\label{s:armenian}\index{Armenian}\index{scripts>Armenian}

As we present the scripts in alphabetic order, the first script we will typeset is in Armenian. There are many fonts available for the language. We use two in the example, the first one is \textit{FreeSans} and the second is \textit{Sylphaen} which is found on Windows Operating systems. The language is not supported by the \pkg{Babel} and partially supported by the \pkgname{Polyglossia}. \tcbdocmarginnote{china revision}

\def\ucfirst#1#2;{\MakeUppercase#1#2}


\def\armeniantest#1#2{
  {\parindent0pt
  \topline \vskip3pt
  \noindent\mbox{
     \ucfirst#1;\hfill\hbox{[\texttt{U+0530-U+058F}]}
  }}
 \nobreak

\begin{minipage}{0.45\textwidth}
\bgroup
%\setotherlanguage{#1}
\begin{#1}
#2
[\today]
\end{#1}
\egroup
\end{minipage}\hspace*{1em}
\begin{minipage}{0.45\textwidth}
\bgroup
  \parindent0pt
  \ttfamily\raggedright
  \string\documentclass\{article\}\par
  \string\usepackage[no-math]\{fontspec\}\\
  \string\newfontfamily\textbackslash#1font[Script=\ucfirst #1;,\\   ~~~~~~~Scale=MatchLowercase]
\{FreeSans\}\par
  \string\begin\{document\}\\
  \string\setotherlanguage\{#1\}\\
  \string\begin\{#1\}\\
  \egroup
\begin{#1}
\hskip10pt\vbox{#2}
\end{#1}
\bgroup
  \ttfamily[\detokenize{\today}]\\
  \string\end\{#1\}\\
  \string\end\{document\}
\egroup
\end{minipage}


\textit{FreeSans}: \url{ http://www.gnu.org/software/freefont/}
}

\armeniantest{armenian}{Բոլոր մարդիկ ծնվում են ազատ ու հավասար իրենց
արժանապատվությամբ ու իրավունքներով։       
Նրանք ունեն բանականություն ու խիղճ և միմյանց
պետք է եղբայրաբար վերաբերվեն։}

The Armenian script was invented around 407 AD, by Mesrop Maštoc, a cleric who wanted to 
translate Greek and Syriac scriptures and liturgical texts into Armenian. The system he devised 
is a pure alphabet, closely modelled on the traditional order of Greek phonetic values, with 
additional graphemes to represent Armenian sounds not found in Greek. The orthography is, 
phonetically, a near perfect representation of the Armenian language, and has remained almost 
entirely unchanged since its invention. In recent times, the letterforms in many Armenian 
typefaces have consciously modelled Latin types in their treatment of serifs, stroke weight and 
stress, and other details. This is the approach that Geraldine adopted for the Sylfaen Armenian, 
in order to harmonise the different scripts within the font. 

This kind of harmonisation has to be 
very carefully handled, as there is, of course, a point at which one can corrupt the normative 
letterforms and produce something which will be unacceptable to native readers. Once again, 
we sought expert review of the design, this time from Manvel Shmavonyan, an Armenian type designer, and his Russian colleague Vladimir Yefimov at 
ParaType in Moscow.

\bgroup
\medskip
\fontspec[Script=Armenian,Scale=1.7]{Sylfaen}
\centering

Աա Բբ Գգ Դդ Եե Զզ Էէ Ըը Թթ Ժժ Իի \\
Լլ Խխ Ծծ Կկ Հհ Ձձ Ղղ Ճճ Մմ Յյ Նն \\
Շշ Ոո Չչ Պպ Ջջ Ռռ Սս Վվ Տտ Րր Ցց \\
Ււ Փփ Քք Օօ Ֆֆ / և ﬓ ﬔ ﬕ ﬖ ﬗ\\
\egroup
\captionof{table}{Armenian, showing the basic alphabet (typeset using the \textit{Sylfaen} font.}
\medskip

\bgroup
\def\m#1 #2 #3\\{\makebox[2em]{#1}\makebox[2em]{{\fontspec{code2000.ttf}#2}}\makebox[2em]{\hfill#3 \\ }}
\fontspec[Script=Armenian,Scale=1.1]{Sylfaen}

\begin{multicols}{4}
\m Ա	A	1\\
\m Բ	B	2\\
\m Գ	G	3\\
\m Դ	D	4\\
\m Ե	E	5\\
\m Զ	Z	6\\
\m Է	ē	7\\
\m Ը	ə	8\\
\m Թ	tʿ	9\\
\m Ժ	ž	10\\
\m Ի	I	20\\
\m Լ	L	30\\
\m Խ	X	40\\
\m Ծ	C	50\\
\m Կ	K	60\\
\m Հ	H	70\\
\m Ձ	J	80\\
\m Ղ	ł	90\\
\m Ճ	č	100\\
\m Մ	M	200\\
\m Յ	Y	300\\
\m Ն	N	400\\
\m Շ	š	500\\
\m Ո	O	600\\
\m Չ	čʿ	700\\
\m Պ	P	800\\
\m Ջ	ǰ	900\\
\m Ռ	ṙ	1000\\ 
\m Ս	S	2000\\
\m Վ	V	3000\\
\m Տ	T	4000\\
\m Ր	R	5000\\
\m Ց	cʿ	6000\\
\m Ւ	W	7000\\
\m Փ	pʿ	8000\\
\m Ք	kʿ	9000\\

\end{multicols}
\captionof{table}{Armenian Numerals \textit{(from Wikipedia).}
The first column is the classical Armenian numeral, the second the transliteration and the third the arabic numeral it represents.}

\medskip

Numbers in the Armenian numeral system are obtained by simple addition. Armenian numerals are written left-to-right (as in the Armenian language). Although the order of the numerals is irrelevant since only addition is performed, the convention is to write them in decreasing order of value.

\begin{align*}
\text{ՌՋՀԵ} &= 1975 = 1000 + 900 + 70 + 5\\
\text{ՍՄԻԲ} &= 2222 = 2000 + 200 + 20 + 2\\
\text{ՍԴ}   &= 2004 = 2000 + 4\\
\text{ՃԻ}   &= 120 = 100 + 20\\
\text{Ծ}    &= 50
\end{align*}

To write numbers greater than 9999, it is necessary to have numerals with values greater than 9000. This is done by drawing a line over them, indicating their value is to be multiplied by 10000:

\begin{align*}
\overline{\text{Ա}} &= 10000\\
\overline{\text{Ջ}} &= 9000000\\
\overline{\text{ՌՃԽԳ}}\text{ՌՄԾԵ} &= 11431255
\end{align*}
\egroup

\subsection{Greek}
\index{languages>Greek}\index{Herodotus}\index{alphabets>Greek}
\newfontfamily\greek[Script=Greek,Scale=1.02]{NotoSerif-Regular.ttf}
\def\greektext#1{\greek{#1}}

`The Phoenicians who came with Kadmos,' wrote Herodotus in the fifth century BC of the legendary Phoenician prince of Tyre and brother of Europa, `\ldots introduced into Greece, after their settlement in the country, a number of accomplishments of which the most important was writing, an art which probably was unknown to the Greeks until then'. 

The Greek alphabet is the script that has been used to write the Greek language since the 8th century BC.[2] It was derived from the earlier Phoenician alphabet, and was in turn the ancestor of numerous other European and Middle Eastern scripts, including Cyrillic and Latin.[3] Apart from its use in writing the Greek language, both in its ancient and its modern forms, the Greek alphabet today also serves as a source of technical symbols and labels in many domains of mathematics, science and other fields.

In its classical and modern forms, the alphabet has 24 letters, ordered from alpha to omega. Like Latin and Cyrillic, Greek originally had only a single form of each letter; it developed the letter case distinction between upper-case and lower-case forms in parallel with Latin during the modern era.

\bgroup
\greek\obeyspaces

Α	ἄλφα	aleph	alpha	[alpʰa]	[ˈalfa]	Listeni/ˈælfə/
Β	βῆτα	beth	beta	[bɛːta]	[ˈvita]	/ˈbiːtə/, US /ˈbeɪtə/
Γ	γάμμα	gimel	gamma	[ɡamma]	[ˈɣama]	/ˈɡæmə/
Δ	δέλτα	daleth	delta	[delta]	[ˈðelta]	/ˈdɛltə/
Η	ἦτα	  heth	   eta	 [hɛːta], [ɛːta]	[ˈita]	/ˈiːtə/, US /ˈeɪtə/
Θ	θῆτα	teth	theta	[tʰɛːta]	[ˈθita]	/ˈθiːtə/, US Listeni/ˈθeɪtə/
Ι	ἰῶτα	yodh	iota	[iɔːta]	[ˈʝota]	Listeni/aɪˈoʊtə/
Κ	κάππα	kaph	kappa	[kappa]	[ˈkapa]	Listeni/ˈkæpə/
Λ	λάμβδα	lamedh	lambda	[lambda]	[ˈlamða]	Listeni/ˈlæmdə/
Μ	μῦ	mem	mu	[myː]	[mi]	Listeni/ˈmjuː/; occasionally US /ˈmuː/
Ν	νῦ	nun	nu	[nyː]	[ni]	/ˈnjuː/ (US /ˈnuː/)
Ρ	ῥῶ	reš	rho	[rɔː]	[ro]	Listeni/ˈroʊ/
Τ	ταῦ	taw	tau	[tau]	[taf]	/ˈtaʊ/ or /ˈtɔː/

\topline
\begin{quote}
Ἡροδότου Ἁλικαρνησσέος ἱστορίης ἀπόδεξις ἥδε, ὡς μήτε τὰ γενόμενα ἐξ ἀνθρώπων τῷ χρόνῳ ἐξίτηλα γένηται, μήτε ἔργα μεγάλα τε καὶ θωμαστά, τὰ μὲν Ἕλλησι, τὰ δὲ βαρβάροισι ἀποδεχθέντα, ἀκλεᾶ γένηται, τὰ τε ἄλλα καὶ δι' ἣν αἰτίην ἐπολέμησαν ἀλλήλοισι.[2]

Herodotus of Halicarnassus, his Researches are set down to preserve the memory of the past by putting on record the astonishing achievements of both the Greeks and the Barbarians; and more particularly, to show how they came into conflict.[3]
\end{quote}
\bottomline

\symbol{"1F00}
\symbol{"1F01}
\egroup

\newfontfamily\glagolitic{MPH 2B Damase}

\section{Glagolitic}

\epigraph{The average Ph.D. thesis is nothing but a transference of bones from one graveyard to another.}{%
J. Frank Dobie (1888-1964)}


\label{s:glagolitic}
\fboxrule0pt\fboxsep0pt

\noindent
The Glagolitic alphabet /{\glagolitic ˌɡlæɡɵˈlɪtɨk/}, also known as Glagolitsa, is the oldest known Slavic alphabet, from the 9th century.

It was created in the 9th century by Saint Cyril, a Byzantine monk from Thessaloniki. He and his brother, Saint Methodius, were sent by the Byzantine Emperor Michael III in 863 to Great Moravia to spread Christianity among the West Slavs in the area. The brothers decided to translate liturgical books into the Old Slavic language that was understandable to the general population, but as the words of that language could not be easily written by using either the Greek or Latin alphabets, Cyril decided to invent a new script, Glagolitic, which he based on the local dialect of the Slavic tribes from the Byzantine Salonika region.
After the deaths of Cyril and Methodius, the Glagolitic alphabet ceased to be used in Moravia, but their students continued to propagate it in the west and south. 

After a long career, Glagolitic writing stopped being used, except for
religious purposes in certain dioceses of Bosnia and Dalmatia (Croatia).
The Cyrillic alphabet was adopted by all Orthodox Slays and served to note
their literary language. Most of the Slays who rallied to Rome rejected it,
however, which created the paradoxical situation in ex-Yugoslavia, where
two peoples who speak the same language write in different scripts, the
Serbs in Cyrillic and the Croats with Roman characters. Finally, as is
known, the ex-Soviet Union did much to put into writing the languages
spoken by the peoples within its borders, for the most part noting them in
adaptations of the Cyrillic alphabet, while Russian became the language of
culture throughout the Soviet Union.\cite{henri1994}

Slavic printing in Glagolitic characters originated in Venice, where a
\textit{Sluzebnik} (or \textit{Leitourgikon}) was published in 1483, followed by missals and
breviaries, all printed by Andrea Torresani, the future father-in-law and
associate of Aldus Manutius. After 1494 some attempts were made to create
printshops in Croatia itself, first in Senj in 1508, then, after 1530, in
Rijeka (Fiume). The work of these firms was almost totally liturgical (religious,
at any rate), and it had strong competition from manuscript works
that were better adapted to the diversity of local liturgical customs. Religion
also dictated the output of a printshop founded to provide Protestant propaganda
that was set up in Tubingen between 1560 and 1564 by Baron
Hans von Ungnad and that printed the great Lutheran texts in Glagolitic
characters.\footfullcite{henri1994}

Figure~\ref{fig:zograf} illustrates an example of the language.\footnote{\url{https://en.wikipedia.org/wiki/Glagolitic_script\#/media/File:ZographensisColour.jpg}}

\begin{figure}[htbp]
\centering

\includegraphics[width=0.45\linewidth]{glagolitic}
\caption[The first page of the Gospel of Mark from the 10th–11th century Codex Zographensis, found in the Zograf Monastery in 1843.]{The first page of the Gospel of Mark from the 10th–11th century Codex Zographensis, found in the Zograf Monastery in 1843.}
\label{fig:zograf}
\end{figure}

\section{Unicode Support}
The Glagolitic alphabet was added to the Unicode Standard in March 2005 with the release of version 4.1.
The Unicode block for Glagolitic is U+2C00–U+2C5F.



\begin{scriptexample}[]{glacolitic}

\unicodetable{glagolitic}{%
"2C00,"2C10,"2C20,"2C30,"2C40,"2C50}

\texttt{typeset with Damase version 2.0 MPH 2B Damase}
\end{scriptexample}
\bgroup
\glagolitic

The name was not coined until many centuries after its creation, and comes from the Old Church Slavonic glagolъ "utterance" (also the origin of the Slavic name for the letter G). The verb glagoliti means "to speak". It has been conjectured that the name glagolitsa developed in Croatia around the 14th century and was derived from the word glagolity, applied to adherents of the liturgy in Slavonic.[1]

In Old Church Slavonic the name is {\glagolitic ⰍⰫⰓⰊⰎⰎⰑⰂⰋⰜⰀ}, Кѷрїлловица.
The name Glagolitic in Bulgarian, Russian, Macedonian глаголица (glagolica), Belarusian is глаголіца (hłaholica), Croatian glagoljica, Serbian глагољица / glagoljica, Czech hlaholice, Polish głagolica, Slovene glagolica, Slovak hlaholika, and Ukrainian глаголиця (hlaholyća).



\egroup






%\subsection{Mongolian Script}

\newfontfamily\mongolian[Language=Mongolian, Scale=1.3]{code2000.ttf}

The classical Mongolian script (in Mongolian script: {\mongolian  ᠮᠣᠩᠭᠣᠯ ᠪᠢᠴᠢᠭ᠌} Mongγol bičig; in Mongolian Cyrillic: Монгол бичиг Mongol bichig), also known as Uyghurjin Mongol bichig, was the first writing system created specifically for the Mongolian language, and was the most successful until the introduction of Cyrillic in 1946. Derived from Uighur, Mongolian is a true alphabet, with separate letters for consonants and vowels. The Mongolian script has been adapted to write languages such as Oirat and Manchu. Alphabets based on this classical vertical script are used in Inner Mongolia and other parts of China to this day to write Mongolian, Sibe and, experimentally, Evenki.
\medskip

\bgroup\par
\noindent
\colorbox{graphicbackground}{\color{black}^^A
\begin{minipage}{\textwidth}^^A
\parindent1pt
\vskip10pt
\leftskip10pt \rightskip\leftskip
\mongolian
\large
ᠬᠦᠮᠦᠨ ᠪᠦᠷ ᠲᠥᠷᠥᠵᠦ ᠮᠡᠨᠳᠡᠯᠡᠬᠦ ᠡᠷᠬᠡ ᠴᠢᠯᠥᠭᠡ ᠲᠡᠢ᠂ ᠠᠳᠠᠯᠢᠬᠠᠨ ᠨᠡᠷ᠎ᠡ ᠲᠥᠷᠥ ᠲᠡᠢ᠂ ᠢᠵᠢᠯ ᠡᠷᠬᠡ ᠲᠡᠢ ᠪᠠᠢᠠᠭ᠃ ᠣᠶᠤᠨ ᠤᠬᠠᠭᠠᠨ᠂ ᠨᠠᠨᠳᠢᠨ ᠴᠢᠨᠠᠷ ᠵᠠᠶᠠᠭᠠᠰᠠᠨ ᠬᠦᠮᠦᠨ ᠬᠡᠭᠴᠢ ᠥᠭᠡᠷ᠎ᠡ ᠬᠣᠭᠣᠷᠣᠨᠳᠣ᠎ᠨ ᠠᠬᠠᠨ ᠳᠡᠭᠦᠦ ᠢᠨ ᠦᠵᠢᠯ ᠰᠠᠨᠠᠭᠠ ᠥᠠᠷ ᠬᠠᠷᠢᠴᠠᠬᠥ ᠤᠴᠢᠷ ᠲᠠᠢ᠃
\par
\vspace*{10pt}
\end{minipage}
}
\medskip
\cxset{chapter title font-shape=upshape,
          chapter format=hang}
\DocInput{\jobname.dtx}
%\bibliography{phd} worked
%\printindex worked
 \end{document}
 %
% \chapter{THE OUTPUT ROUTINE (OTR)}

\epigraph{Sherlock Holmes in "The sign of four": "'My mind," he said, "rebels at stagnation. Give me problems, give me work, give me the most abstruse cryptogram or the most intricate analysis, and I am in my own proper atmosphere.'" }{}
\normalsize
The output routine is one of the more mysterious pieces
of \tex.
and as  David Salomon noted\footnote{TUGboat/tb-11-1/tb27salomon.pdf}, advanced users hardly need to be convinced that an unerstanding of OTRs is important, since they must be used whenever, special output is desired.
 The chapter of the \texbook discussing output
routines claims that designing output routines makes one:

\begin{quotation}
achieve the level of a `\tex Grandmaster'.
As is so often the case, mastery of the concept of an
output routine in plain TEX will only barely prepare you
for the complexities awaiting you with LATEX’s variant of
an output routine.
\end{quotation}


The subject is considered complex for the following reasons:

\begin{enumerate}
\item OTRS are asynchronous with the
rest of TEX (this is explained later) and involve difficult concepts such as splitting boxes and insertions.
\item Certain features, which could be useful in OTRs are not supported by \tex. Specifically there are no commands to identify marks, rules and |whatsits| in a box and to break up a line of text into individual characters.
\end{enumerate}

\tex\ 's page breaking algorithm is simpler than the line breaking one. The reason for this is that global optimization
of page breakpoints, the way is done in the paragraph algorithm is prohibitively in terms of memory (especially in the 1980s).

Theoretically, page breaking is a more complicated \footnote{\href{test}{http://www.cs.utk.edu/~eijkhout/594-LaTeX/handouts/breaking/page-tutorial.pdf}}than line breaking. First we will briefly discuss the algoithms that \tex\ actually
uses.


\section{Page breaking algorithm}

The problem of page breaking has two components. One is that of stretching or shrinking
available glue (mostly around display math or section headings) to find typographically
desirable breakpoints. The other is that of placing ‘floating’ material, such as tables and
figures. These are typically placed at the top or the bottom of a page, on or after the first
page where they are referenced. These ‘inserts’, as they are called in TEX, considerably
complicate the page breaking algorithms, as well as the theory.

\subsection{Typographical constraints}

There are various typographical guidelines for what a page should look like, and TEX has
mechanisms that can encourage, if not always enforce, this behaviour.

\begin{enumerate}
\item The first line of every page should be at the same distance from the top. This changes
if the page starts with a section heading which is a larger type size.

\item The last line should also be at the same distance, this time from the bottom. This
is easy to satisfy if all pages only contain text, but it becomes harder if there are
figures, headings, and display math on the page. In that case, a ‘ragged bottom’ can
be specified.

\item  A page may absolutely not be broken between a section heading and the subsequent
paragraph or subsection heading.

\item It is desirable that

\begin{enumerate}
\item the top of the page does not have the last line of a paragraph started on the
preceding page

\item the bottom of the page does not have the first line of a paragraph that continues
on the next page.
\end{enumerate}

\end{enumerate}



For ordinary purposes you will probably find that \tex's automatic
method of page breaking is satisfactory. And when it occasionally gives unpleasant
results, you can force the machine to break at your favorite place by
typing |\eject|. But be careful: |eject| will cause \tex to stretch the page
out, if necessary, so that the top and bottom baselines agree with those on other
pages.  If you want to eject a short page, filling it with blank space at the bottom,
type | \vfill\eject|  instead.

\section{The current page and the recent contributions list}

The main vertical list of TEX is divided in two parts: the \emph{current page} and the list of \emph{recent
contributions}. Any material that is added to the main vertical list is appended to the recent
contributions; the act of moving the recent contributions to the current page is known as
\emph{exercising the page builder}.

Every time something is moved to the current page, TEX computes the cost of breaking the
page at that point. If it decides that it is past the optimal point, the current page up to the
best break so far is put in |box255| and the remainder of the current page is moved back
on top of the recent contributions. If the page is broken at a penalty, that value is recorded
in |outputpenalty|, and a penalty of size 10 000 is placed on top of the recent contributions;
otherwise, |outputpenalty| is set to 10 000.

If the current page is empty, discardable items that are moved from the recent contributions
are discarded. This is the mechanism that lets glue disappear after a page break and at the
top of the first page. When the first non-discardable item is moved to the current page, the
|topskip| glue is inserted; 



\section{When is the page builder activated?}


The page builder comes into play in the following circumstances.

\begin{enumerate}
\item  Around paragraphs: after the \cs{everypar} tokens have been inserted, and after the
paragraph has been added to the vertical list. See the end of this chapter for an
example.

\item  Around display formulas: after the \cs{everydisplay} tokens have been inserted, and after
the display has been added to the list.

\item  After \cs{par} commands, boxes, insertions, and explicit penalties in vertical mode.

\item  After an output routine has ended.
\end{enumerate}



In these places the page builder moves the recent contributions to the current page. Note that
\tex\  need not be in vertical mode when the page builder is exercised. In horizontal mode,
activating the page builder serves to move preceding vertical glue (for example, \cs{parskip},
\cs{abovedisplayskip}) to the page.

The \cs{end} command – which is only allowed in external vertical mode – terminates a TEX job,
but only if the main vertical list is empty and \cs{deadcycles} = 0. If this is not the case the
combination


|\hbox{}\vfill\penalty+ $-2^{30}$|

is appended, which forces the output routine to act.

\section{The depth of the current page}
The depth of the page is important since normally in good typesetting successive pages should have the same (or almost the same vertical size. (flushbottom). The height of a page is controlled and set exactly by \tex equal to |\vsize|. Consider a large |vbox| with lines of text, glue and penalties. The depth of this box, is the depth of the last component [80]. If the last component is a glue or penalty, the depth is zero. If it is a box, then its depth becomes the depth of the entire |\vbox|, except that it is limited to the value of parameter |\boxmaxdepth|.

If
|\boxmaxdepth=1pt| and the depth of the bottom box
is 1.94444pt, then the depth of the entire |\vbox|
will be 1pt and its height will be incremented
by .94444pt. This is equivalent to lowering the
reference point (or, equivalently, the baseline) of
the |\vbox| by .94444pt. In the plain format,
|\boxmaxdepth=\maxdimen| [348], so it has no effect
on the depths of boxes. However, |\boxmaxdepth|
can always be changed by the user \footnote{This \texttt{\textbackslash boxmaxdepth} setting is to ensure that deep footnotes do not overwrite the
footer (on account of the negative skip added later): it should use \texttt{\textbackslash @maxdepth}
otherwise the change is pointless when there are footnotes.
But see also its use when combining 
floats.  \latex uses a value of 5.5pt whereas plain a value of 4pt [348].}



If the last line on a page, contains letters that happen to not have any depth, the page depth will be zero. Try for example this:

\begin{teXXX}
....
\showthe\pagedepth
\bye
\end{teXXX}

You can also try it with a \latex minimal and will produce the same output.


\section{The height of a box of text}

Following the literature we denote the value of |\baselineskip| (which is normally 12pt) by $b$. 
A
large |\vbox| with text consists mainly of lines of
text, each an |\hbox|, separated by globs of glue,
normally in the (varying) amounts necessary to
separate baselines by exactly $b$, but sometimes just
the amount |\lineskip|. We assume a simple case
where no large characters or equations are used. In
such a case, all lines of text are separated by $b$. The
height of the box is thus:
\begin{gather}
b(n - 1) + \text{the height of the first line}
\end{gather}
where $n$ is the number of text lines. Remember that the first line is a special case and adjustments can be made using the value of |\topskip|.

\section{The height of \texttt{\textbackslash box255}}

In the case of |\box255|,
enough glue is placed above the first line of text
to reach to |\topskip| from the first baseline. We
denote the value of |\topskip| by $h$ (10pt in plain).
So if the baseline of the first line is now h below the
top of the page, the height H of |\box255| should
be b(n - 1) + h (Fig. 3). However, the height of
|\box255| is always set, by the page builder, to
|\vsize|. The difference between the two heights is
usually supplied by the flexible glues on the page,
the most common of which is |\parskip|

\begin{figure}[htp]
\includegraphics{./graphics/heightofpagebox}
\end{figure}


\subsection{Dead cycles.} An execution of the OTR without shipping any material is called a \texttt{dead cycle}. Dead cycles, have their uses and we will explain this a bit later on. However, long iterations that just return \textit{dead cycles} is an indication of an error somewhere. \tex counts the number of dead cycles in a counter named |\deadcycles| and stops the run if |\deadcycles >= \maxdeadcycles|.  In the \textit{plain} format |\maxdeadcycles| is set as 25 and in \latex as \the\deadcycles. |\maxdeadcycles = 100| is \the\maxdeadcycles. Each time |\shipout| is invoked, it resets |\deadcycles| to zero.

\begin{teXXX}
If the file is not included, reset \deadcycles, so that a long list of non-included
files does not generate an `Output loop' error.
115 \deadcycles\z@
116 \@nameuse{cp@#1}%
117 \fi
118 \let\@auxout\@mainaux}
\end{teXXX}


\subsection{\tex's Page Number.} The page number can come from any source. Salomon provides an example where the \textsc{OTR} typesets a page number from a |\count| variable. This is typeset centered below the printed area.

\begin{teXXX}
\newcount\pageNum
\output={
\shipout\vbox{
\box255\smallskip
\centerline{\tenrm\the\pageNum}}
\global\advance\pageNum by1}
\end{teXXX}

Notice that the output macro, just passes the contents of the box to |\shipout|. This is not actually a very good method, but is shown here to illustrate a point.

Note the |\tenrm| in the preceding example. It
is necessary because of the asynchronous nature of
the \otr. When the \otr is invoked, \tex can be
anywhere on the next page. Specifically, it could
be inside a group where a different font is used.
Without the |\tenrm|, that font (the current font)
would be used in the otr.
In the plain format, the |\count0| variable
serves as the page number, and the following two
macros are especially useful.




\subsection{The \texttt{\textbackslash vsplit} operation.} 

Supposed you have inserted the material required to go on a page on a big |\vbox|, but the material is a bit extra that what is required to fill a page exactly. You would need an operation to split the box in two. The |vsplit| operation does that. It is important to the understanding of OTR operations to have an intimate knowledge of |\vsplit|. Its syntax is: 

|\vsplit|\meta{box number} to \meta{dim}

The result of the operation is a box. Most often it appears in an assignment such as: |\setbox1=\vsplit0 to2.6in|. This sets |\box1| to a
height of 2.6in, moves material from the top of
|\box0| to |\box1|, and keeps the remainder in |\box0|.

\begin{macro}{\loremlines}
It is important to remember that most of \tex's commands work with \latex as well. In Example~\ref{ex:loremlines}, we define a box to hold |lipsum| text in a two column layout. We want to define a macro that can split the box in as many lines as we require. 
\end{macro}

\begin{texexample}{Splitting a vbox}{ex:loremlines}
\newbox\one
\newbox\two
\long\gdef\loremlines#1#2{%
   \setbox\one=\vbox {#2}
   \setbox\two=\vsplit\one to #1\baselineskip
   \unvbox\two
   \gdef\boxone{#2}
}
\begin{multicols}{2}
\small
\loremlines{16}{\lipsum[1-2]}
\end{multicols}
\boxone
\end{texexample}


\tex assumes that the new |\box1| may have to
be shipped out as part of the page. It therefore
places a glue similar to $h$ at the top of |\box1|.
This glue is called |\splittopskip| and has a plain
format value of 10pt [348].

One important thing to note is that a box can only be split \textit{between} lines of text. 
If we split a box to another size, |\box1| will come out underfull.

Here is an \otr which splits the page, ships
out the top part and returns the rest to the MVL
(actually, to the recent contributions):

\begin{teXXX}
\output={\setbox0=\vsplit255 to1in
\shipout\box0 \unvbox255}
\end{teXXX}






\section{Communicating with the OTR: Marks}

\begin{multicols}{2}
The user can pass information to the output routine through \textit{marks}. Marks have the syntax

\begin{teX}
\mark{mark text}
\end{teX}

which is put in a mark item on the current vertical list. The mark text is subject to expansion
as in \cs{edef}.
If the mark is given in horizontal mode it migrates to the surrounding vertical lists like an
insertion item (see page Text By Topic 77); however, if this is not the external vertical list, the output routine
will not find the mark.

Marks are the main mechanism through which the output routine can obtain information
about the contents of the currently broken-off page, in particular its top and bottom. TEX sets
three variables:

{\obeylines
\cs{botmark} the last mark occurring on the current page;
\cs{firstmark} the first mark occurring on the current page;
\cs{topmark} the last mark of the previous page, that is, the value of \cs{botmark} on the previous
page.
}



If no marks have occurred yet, all three are empty; if no marks occured on the current pagr, all three variables are equal to the \cs{botmark} of the previous page. 

Marks can be used to get a section heading into the headline or footline of the page.

\begin{verbatim}
\def\section#1{ ... \mark{#1} ... }
\def\rightheadline{\hbox to \hsize
    {\headlinefont \botmark\hfil\pagenumber}}
\def\leftheadline{\hbox to \hsize
   {\headlinefont \pagenumber\hfil\firstmark}}
\end{verbatim}

This places the title of the first section that starts on a left page in the left
headline, and the title of the last section that starts on the right page in
the right headline. Placing the headlines on the page is the job of the output
routine; see below.

It is important that no page breaks can occur in between the mark and the
box that places the title:

\emphasis{mark,nobreak}
\begin{teXXX}
\def\section#1{ ...
   \penalty\beforesectionpenalty
   \mark{#1}
   \hbox{ ... #1 ...}
   \nobreak
   \vskip\aftersectionskip
   \noindent}
\end{teXXX}
\end{multicols}



\section{Insertions}
Insertions are considered one of  the most  com- 
plex  topics in \tex. Many users master  topics  such 
as tokens,  file  I/O, macros,  and  even  OTRS  before 
they dare  tackle  insertions.  The  reason  is  that 
insertions  are  complex,  and  The \texbook, while 
covering all the relevant material, is somewhat cryp- 
tic regarding  insertions, and  lacks  simple examples. 
The  main  discussion  of  insertions takes  place  on 
[115-1251.  where \tex' s  registers  are also discussed. 
Examples  of  insertions are  shown, mostly  without 
explanations,  on  [363-364,  423-424].  A lot of what is described here is based on an article in TUGboat by David Salomon\footnote{ 
http://www.tug.org/TUGboat/Articles/tb11-4/tb30salomon.pdf}

Many users understand the idea of floats. Certain material to be typeset needs to be held in a buffer and inserted at different points on a page, for example a a figure that does not fit on a page it has to be inserted at the top of the next page. An \textit{insertion} is just a piece of a document that is generated at a certain point but appears at another point. Common examples are figures, footnotes and endnotes. Quoting Knuth:

\begin{quote}
  This  algorithm  is  admittedly  complicated, 
but  no  simpler  mechanism  seems  to  do  nearly 
as  much.
\end{quote}

\section{OTR Example}

\begin{figure}%
 \centering
  \includegraphics[width=0.37\linewidth]{./graphics/framedpage}
  \caption{A boxed page}
  \label{fig:framedpage}
\end{figure}

Here is an OTR for a \textit{framed} page. It surrounds the
page with double rules on all sides, and centers the
page number below the double box. Note that the
page shipped out is wider and taller than \cs{box255}.
The value of \cs{hsize} in this case is, therefore, not
the width of the final page shipped out, but the
width of the text lines in \cs{box255}.

Macro \cs{frameit} typesets text and surrounds it
with 4 rules (see [Ex. 21.3]). Parameter \#2 is the
space between the rules and the text. \#1 is a box
containing the text.

\emphasis{output,shipout}
\begin{teXXX}
\def\frameit#1#2{%
 \vbox{\hrule
  \hbox{%
    \vrule \kern#2pt
      \vbox{\kern#2pt #1
         \kern#2pt}%
      \kern#2pt\vrule}
\hrule}}

\output={
   \shipout\vbox{
   \boxit{\frameit{\box255}9}
      \medskip
      \centerline{Test Framed Page}}
  \advancepageno}
\end{teXXX}


Plain TeX has an output routine that takes care of  simple things like page numbering and insertions
using \cs{footnote} and \cs{topinsert}. 

\section{\LaTeX\  output routines}

So far we have examined the \tex OTR in detail. I hope it has given you enough understanding, not only to write your own output routine, but also to now be ready to study the \latex output routine, which is much more complicated. We have so far seen that  when \tex 
is typesetting pages of continuous text, it will gather material until it can find a least-cost page break intended to
make the gathered material fit the \cs{pagegoal size}. The
gathered material will then be placed into |\box255| and
the output routine stored in the token register \cs{output}
will be processed in a group of its own. 

Usually it will
arrange the gathered material in some way, add headers,
footlines and page numbers, and ship the gathered results out in typeset form with the \cs{shipout} command.
At the time of the \cs{shipout} command all \cs{open} and
\cs{write} commands stored in the box shipped out are expanded and written out. This is what makes it possible to have page labels corresponding to the actual page
numbers at the time of shipout: the corresponding info
is written to the |.aux| file at that time.
The output routine may decide to place material
back on the main vertical list instead of shipping it out.

\LaTeX\ output routine is described in \texttt{ltoutput.dtx}. You should also have a look at \texttt{ltfloat.dtx}. The algorithm is revisited i \latex3 and Frank Mittelbach, published a paper
\footnote{\protect\url{http://www.latex-project.org/papers/xo-pfloat.pdf}} in which he explains some of the problems facing the team, when dealing with the output routine.


Information on the output routine is rather scarce. Best source is a series of  articles in the TUGBoat by David Salomon.

\href{http://www.tug.org/TUGboat/Articles/tb11-1/tb27salomon.pdf}{Output Routines: Examples and Techniques. Part I: Introduction and Examples.}

\href{http://www.tug.org/TUGboat/Articles/tb11-2/tb28salomon.pdf}{Output Routines: Examples and Techniques. Part II: OTR Techniques}

\href{http://www.tug.org/TUGboat/Articles/tb11-4/tb30salomon.pdf}{Output Routines: Examples and Techniques. 
Part III: Insertions}

\href{http://www.tug.org/TUGboat/Articles/tb15-1/tb42salomon-output.pdf}{Output routines: Examples and techniques Part IV: Horizontal techniques}


David Kastrup's article \href{http://www.tug.org/TUGboat/Articles/tb24-3/kastrup.pdf}{Output Routine Requirements for Advanced Typesetting Tasks} (Proceedings of EuroTEX 2003) otlined some of the difficult areas and specifications for generic routines

The standard blocks are well described above and most tasks could be accomplished 
by rather working from
standard building blocks like \textit{insertion lists}, \textit{here points},
default mechanisms for \textit{margin notes} and so on.


\section*{Calling the output routine}

The output routine is called either by TeX's normal page-breaking
mechanism, or by a macro putting a penalty < or = -10000 in the output
list. In the latter case, the penalty indicates why the output
routine was called, using the following code.
penalty reason

\begin{tabular}{ll}
\toprule
penalty &reason\\
\midrule
-10000  &\ pagebreak\\
~       &\ newpage\\
-10001  &clearpage (\ penalty -10000 \ vbox{}| \ penalty -10001)|\\
-10002  &float insertion, called from horizontal mode\\
-10003 &float insertion, called from vertical mode.\\
-10004 &float insertion.\\
\bottomrule
\end{tabular}
\medskip

Note: A |float| or |marginpar| puts the following sequence in the output
list: 

\begin{enumerate}
\item a penalty of -10004,

\item a null |\vbox|

\item a penalty of -10002 or -10003.
\end{enumerate}

This solves two special problems:

\begin{enumerate}
\item If the float comes right after a |\newpage| or |\clearpage|,
then the first penalty is ignored, but the second one
invokes the output routine.

\item If there is a split footnote on the page, the second 'page'
puts out the rest of the footnote
\end{enumerate}

\latex first defines some helper routines and increase the \cs{maxdeadcycles}. The helper macros are for
manipulating lisst.

\begin{teX}
 \maxdeadcycles = 100
 \let\@elt\relax
 \def\@next#1#2#3#4{\ifx#2\@empty #4\else
   \expandafter\@xnext #2\@@#1#2#3\fi}
   \@next \CS \LIST {NONEMPTY}{EMPTY} == %% NOTE: ASSUME
\@elt = \relax
 BEGIN assume that \LIST == \@elt \B1 ... \@elt \Bn
 if n = 0
 then EMPTY
 else 
   \CS :=L \B1
   \LIST :=G \@elt \B2 ... \@elt \Bn
   NONEMPTY
 fi
END
\end{teX}


\begin{teX}
11 \def\@xnext \@elt #1#2\@@#3#4{\def#3{#1}\gdef#4{#2}}

12 \def\@testfalse{\global\let\if@test\iffalse}
13 \def\@testtrue {\global\let\if@test\iftrue}
14 \@testfalse}
   }

15 \def\@bitor#1#2{\@testfalse {\let\@elt\@xbitor
16   \@tempcnta #1\relax #2}}

17 \def\@xbitor #1{\@tempcntb \count#1
18    \ifnum \@tempcnta =\z@
19    \else
20      \divide\@tempcntb\@tempcnta
21    \ifodd\@tempcntb \@testtrue\fi
22   \fi}
\end{teX}

\begin{multicols}{2}
\subsection{Float boxes and lists.} 
A \textit{float list} consisting of the 
floats in boxes |\boxa ... \boxN| has
the form:

|\@elt \boxa ... \@elt \boxN|
where |\boxI| is defined by

|\newinsert\boxI|

Normally, |\@elt| is |\let| to |\relax|. A test can be performed on the
entire 
oat list by locally |\def|'ing |\@elt| appropriately and
executing the list.
This is a lot more efficient than looping through the list.
\LaTeX\ defines float boxes as |bx@A| to |bx@R| to make them available for 
inserts. These will be used later to define the lists that hold these boxes. 


\latex now defines the float boxes. Each one is defined as an insert.
\begin{teXXX}
\newinsert\bx@A
...
\newinsert\bx@I
\newinsert\bx@J
\newinsert\bx@K
\newinsert\bx@L
\newinsert\bx@M
\newinsert\bx@N
\newinsert\bx@O
\newinsert\bx@P
\newinsert\bx@Q
\newinsert\bx@R
\end{teXXX}


\end{multicols}
Once these boxes are defined they are inserted in the |@freelist|. At this point all the other lists are defined.

\emphasis{@freelist,@toplist,@botlist,@midlist,@currlist}
\begin{teXXX}
41 \gdef\@freelist{\@elt\bx@A\@elt\bx@B\@elt\bx@C\@elt\bx@D\@elt\bx@E
42                 \@elt\bx@F\@elt\bx@G\@elt\bx@H\@elt\bx@I\@elt\bx@J
43                 \@elt\bx@K\@elt\bx@L\@elt\bx@M\@elt\bx@N
44                 \@elt\bx@O\@elt\bx@P\@elt\bx@Q\@elt\bx@R}
\end{teXXX}

All the lists are defined initially to be empty.
\begin{teXXX}
45 \gdef\@toplist{}
46 \gdef\@botlist{}
47 \gdef\@midlist{}
48 \gdef\@currlist{}
49 \gdef\@deferlist{}
50 \gdef\@dbltoplist{}
51 \gdef\@dbldeferlist{}
\end{teXXX}


The lists are similar to those defined in \texttt{plain}.

\begin{description}
\item[\string\@freelist] : List of empty boxes for placing new 
floats.
\item[\string\@toplist] : List of 
floats to go at top of current column.
\item[\string\@midlist] : List of 
floats in middle of current column.
\item[\string\@botlist] : List of 
floats to go at bottom of current column.
\item[\string\@deferlist] : List of 
floats to go after current column.
\item[\string\@dbltoplist] : List of double-col. 
floats to go at top of current
page.
\item[\string\@dbldeferlist] : List of double-column 
floats to go on subsequent
pages.

\end{description}

\begin{multicols}{2}
Check was prudent when defining the newinsert boxes in order to reserve space and memory. The package \docpkg{morefloats} can be used to add more floats to this list. This should have definitely been included here in a revision.

\subsection{Defining Layout parameters} All the page layout parameters are defined next. 

\begin{teXXX}
52 \newdimen\topmargin
53 \newdimen\oddsidemargin
54 \newdimen\evensidemargin
55 \let\@themargin=\oddsidemargin
56 \newdimen\headheight
57 \newdimen\headsep
58 \newdimen\footskip
59 \newdimen\textheight
60 \newdimen\textwidth
61 \newdimen\columnwidth
62 \newdimen\columnsep
63 \newdimen\columnseprule
64 \newdimen\marginparwidth
65 \newdimen\marginparsep
66 \newdimen\marginparpush
\end{teXXX}

Remember  that TeX knows littel about a page. The problem is that TEX has no idea how
wide and tall the paper is. All it knows is the
left and top offsets, and the dimensions of the
printed area (|\hsize| and |\vsize|). All these dimensions need to be calculated and adjustments made within the \otr.

A document normally  starts by specifying:

\begin{teXXX}
\newdimen\paperheight
\newdimen\paperwidth
\paperheight=..in \paperwidth=..in
\end{teXXX}


\end{multicols}


\subsection*{The AtBeginDvi}
A box register is used  to put stuff that must appear before anything else
in the |.dvi| file.

The stuff in the box should not add any typeset material to the page when it
is unboxed.

\emphasis{AtBeginDvi,@begindvibox}

\begin{teXXX}
67 \newbox\@begindvibox
68 \def \AtBeginDvi #1{%
69 \global \setbox \@begindvibox
70 \vbox{\unvbox \@begindvibox #1}%
71 }
\end{teXXX}

\begin{teXXX}
72 \newdimen\@maxdepth
73 \@maxdepth = \maxdepth
\end{teXXX}


Some new registers for paperheight and paperwidth are defined:

\begin{teXXX}
74 \newdimen\paperheight
75 \newdimen\paperwidth
76 \newif \if@insert
These should definitely be global:
77 \newif \if@fcolmade
78 \newif \if@specialpage \@specialpagefalse
These should be global but are not always set globally in other les.
79 \newif \if@firstcolumn \@firstcolumntrue
80 \newif \if@twocolumn \@twocolumnfalse
Not sure about these: two questions. Should things which must apply to a whole
doument be local or global (they probably should be `preamble only' commands)?
Are these three such things?
81 \newif \if@twoside \@twosidefalse
82 \newif \if@reversemargin \@reversemarginfalse
83 \newif \if@mparswitch \@mparswitchfalse
This counter has been imported from `multicol'.
84 \newcount \col@number
85 \col@number \@ne
\end{teXXX}

and a lot of other internal registers

\begin{teX}
86 \newcount\@topnum
87 \newdimen\@toproom
88 \newcount\@dbltopnum
89 \newdimen\@dbltoproom
90 \newcount\@botnum
91 \newdimen\@botroom
92 \newcount\@colnum
93 \newdimen\@textmin
94 \newdimen\@fpmin
95 \newdimen\@colht
96 \newdimen\@colroom
97 \newdimen\@pageht
98 \newdimen\@pagedp
99 \newdimen\@mparbottom \@mparbottom\z@
100 \newcount\@currtype
101 \newbox\@outputbox
102 \newbox\@leftcolumn
103 \newbox\@holdpg
104 \def\@thehead{\@oddhead} % initialization
105 \def\@thefoot{\@oddfoot}
\end{teX}


\subsection{\texttt{\textbackslash clearpage}}

The clearpage macro is a bit complicated, as it needs to avoid a complete empty page after a |\twocolumn[..]|. This prevents the text from the argument
vanishing into a  float box, never to be seen again. We hope that it does not
produce wrong formatting in other cases.

\begin{teX}
106 \def\clearpage{%
107   \ifvmode
108   \ifnum \@dbltopnum =\m@ne
109     \ifdim \pagetotal <\topskip
110       \hbox{}%
111     \fi
112   \fi
113  \fi
114 \newpage
115 \write\m@ne{}%
116 \vbox{}%
117 \penalty -\@Mi
118 }
\end{teXXX}

\subsection{The \texttt{\textbackslash clearpagedoublepage} macro} 

This checks for odd and even pages by using the
page counter |c@page|.  It also provides switches of twoside printing. 
\TODO{Why not from auxiliary?}

\begin{teXXX}
119 \def\cleardoublepage{\clearpage\if@twoside \ifodd\c@page\else
120 \hbox{}\newpage\if@twocolumn\hbox{}\newpage\fi\fi\fi}
\end{teXXX}

Note the |\newpage| is defined a bit further on. This is a fairly simple definition, since most of the code that follows only gets a bit complicated with the twocolumn option. It sets the dimensions and the booleans to those appropriate for the |onecolumn| option. An important note we back to \tex's |\hsize|. Both the linewidth as well as the columnwidth are set to this.

\begin{teXXX}
123 \def\onecolumn{%
124   \clearpage
125   \global\columnwidth\textwidth
126   \global\hsize\columnwidth
127   \global\linewidth\columnwidth
128   \global\@twocolumnfalse
129   \col@number \@ne
130   \@floatplacement
     }
\end{teXXX}

\subsection{\string newpage.} 

The |\newpage| macro is programmed defensively. The two checks at the beginning ensure that an item label or run-in section title
immediately before a |\newpage| get printed on the correct page, the one before
the page break.
All three tests are largely to make error processing more robust; that is why
they all reset the 
flags explicitly, even when it would appear that this would be
done by a |\leavevmode|.

\begin{teXXX}
131 \def \newpage {%
132  \if@noskipsec
133    \ifx \@nodocument\relax
134      \leavevmode
135      \global \@noskipsecfalse
136    \fi
137 \fi
138 \if@inlabel
139   \leavevmode
140   \global \@inlabelfalse
141 \fi
142 \if@nobreak \@nobreakfalse \everypar{}\fi
143 \par
144 \vfil
145 \penalty -\@M}
\end{teXXX}

An empty cols is defined. There is a note here, that an invisible rule might have been a better idea.

\begin{teXXX}
146 \def \@emptycol {\vbox{}\penalty -\@M}
\end{teXXX}

\subsection{The \string twocolumn macro.} This is the longest definition so far. We will leave it for a while and then come back. There are several bug fixes to the two-column stuff here. Firstly, like the onecolumn the page parameters are set to the correct parameters.


\begin{teXXX}
147 \def \twocolumn {%
148 \clearpage
149 \global\columnwidth\textwidth
150 \global\advance\columnwidth-\columnsep
151 \global\divide\columnwidth\tw@
152 \global\hsize\columnwidth
153 \global\linewidth\columnwidth
154 \global\@twocolumntrue
155 \global\@firstcolumntrue
156 \col@number \tw@
\end{teXXX}



\section*{The output macro}

The setting of the \cs{output} is quite short but it belies its complexity.
After having checked verious parameters it redirects to |@specialoutput|. This is the heart of the routines. Notice that \latex just fills in the token list of \tex's |output| routine, it does not attempt to redefine it or save it. 
Should some hooks be defined here, life might have been made easier, however, what one can do is to first save the \latex output routine and then redefine the output as one may wish. Return to it can happen after it. If you take this approach, you should be careful of packages that redefine output, such as |multicol| and |longtable|. An approach such as this is taken by |revtex|.

\emphasis{ifnum,fi,else,ifdimen,@specialoutput}
\begin{teX}
204 \output {%
205 \let \par \@@par
206 \ifnum \outputpenalty<-\@M
207    \@specialoutput
208 \else
209    \@makecol
210    \@opcol
211    \@startcolumn
212    \@whilesw \if@fcolmade \fi
213      {%
218      \@opcol\@startcolumn}%
219 \fi
220 \ifnum \outputpenalty>-\@Miv
221 \ifdim \@colroom<1.5\baselineskip
222 \ifdim \@colroom<\textheight
223 \@latex@warning@no@line {Text page \thepage\space
224 contains only floats}%
225 \@emptycol
226 % \if@twocolumn
227 % \if@firstcolumn
228 % \else
229 % \@emptycol
230 % \fi
231 % \fi
232 \else
  233 \global \vsize \@colroom
234 \fi
235 \else
236   \global \vsize \@colroom
237 \fi
238 \else
239   \global \vsize \maxdimen
240 \fi
241 }
\end{teX}



\begin{teXXX}
244 \gdef\@specialoutput{%
245   \ifnum \outputpenalty>-\@Mii
246     \@doclearpage
247   \else
248     \ifnum \outputpenalty<-\@Miii
249         \ifnum \outputpenalty<-\@MM \deadcycles \z@ \fi
250                 \global \setbox\@holdpg \vbox {\unvbox\@cclv}%
251         \else
252         \global \setbox\@holdpg \vbox{%
253                 \unvbox\@holdpg
254                 \unvbox\@cclv
We must now remove the box added by the 
oat mechanism and the \topskip
glue therefore added above it by TEX.
255                \setbox\@tempboxa \lastbox
256                \unskip
257 }%
These two are needed as separate dimensions only by \@addmarginpar; for other
purposes we put the whole size into \@pageht (see below).
258                \@pagedp \dp\@holdpg
259                \@pageht \ht\@holdpg
260                \unvbox \@holdpg

261                \@next\@currbox\@currlist{%
262                \ifnum \count\@currbox>\z@
Putting the whole size into \@pageht (see above).
263                  \advance \@pageht \@pagedp
264                  \ifvoid\footins \else
265                    \advance \@pageht \ht\footins
266                    \advance \@pageht \skip\footins
267                    \advance \@pageht \dp\footins
268                \fi
\end{teXXX}



\subsection{The \string @doclearpage macro.} This is an emergency action. It dumps everything: footnotes first and then floats. 


\section*{The Kludgeins}

The kludgeins are simply inserts that fool \tex in enlarging a page by a small amount, normally used to allow one or two lines of text to go in the same page.

The two kludgeins mentioned in the kernel are are \cs{enlargethisspace} and its star version.\footnote{The Oxford English Dictionary (2nd ed., 1989) kludge entry cites one source for this word's earliest recorded usage, definition, and etymology: Jackson W. Granholm's 1962 "How to Design a Kludge" article, which appeared in the American computer magazine Datamation
kludge  Also kluge. [J. W. Granholm's jocular invention: see first quot.; cf. also bodge v., fudge v.]

'An ill-assorted collection of poorly-matching parts, forming a distressing whole' (Granholm); esp. in Computing, a machine, system, or program that has been improvised or 'bodged' together; a hastily improvised and poorly thought-out solution to a fault or 'bug'.

The word 'kludge' is...derived from the same root as the German Kluge..., originally meaning 'smart' or 'witty'.... 'Kludge' eventually came to mean 'not so smart' or 'pretty ridiculous'.}



\begin{teXX}
\gdef \enlargethispage{%
1198 \@ifstar
1199 {%
1203   \@enlargepage{\hbox{\kern\p@}}}%
1204 {%
1208   \@enlargepage\@empty}%
1209 }
\end{teXX}

Adds |<dim>| to the height of the current column only. On the printed page the
bottom of this column is extended downwards by exactly |<dim>| without having
any effect on the placement of the footer; this may result in an overprinting.
\cs{enlargethispage}.

Similar to |\enlargethispage| but it tries to squeeze the column to be printed
in as small a space as possible, ie it uses any shrinkability in the column. If the
column was not explicitly broken (e.g. with |\pagebreak|) this may result in an
overfull box message but except for this it will come out as expected (if you know
what to expect).
The star form of this command is dedicated to Leslie Lamport, the other we
need for ourselves (FMi, CAR).
These commands may well have unwanted if used soon before a\ldots

 




\section{Using packages to ease the pain}

OTR routines are notoriously difficult to debug and define. Some of the available packages at CTAN
can make the programming job easier.

The |everypage| package by Sergio Callegari provides hooks into the \latex\ internal commands to
to do actions on every page or on the current page. Specifically, actions  are performed \emph{before} the page is shipped, so they can be
used to put watermarks \emph{in the background} of a page, or to
set the page layout. 

The package provides two hooks:

\emphasis{AddEverypageHook,AddThisPageHook}
\begin{teXXX}
  \AddEverypageHook{Test}
  \AddThisPageHook
\end{teXXX}

The package reminds in some sense
\docpkg{bobhook} by Karsten Tinnefeld, but it differs in the way in
 which the hooks are implemented, as detailed in the following.
 In some sense it may also be related to the package
 \docpkg{everyshi} by Martin Schroeder, but again the implementation
 is different.

 
 This program adds two \LaTeX\ hooks that get run when document
 pages are finalized and output to the |.dvi| or |.pdf|
 file. Specifically, one hook gets executed on every page, while the
 other is executed for the current page. Hook actions are are performed
 \emph{before} the page is output on the medium, and this is
 important to be able to play with the page layout or to put things
 \emph{behind} the page contents (e.g., watermarks such as an image,
 framing, the ``DRAFT'' word, and the like).
 
 The package reminds in some sense \Lpack{bobhook} by Karsten
 Tinnefeld, but it differs in the way in which the hooks are
 implemented:
 


 \begin{enumerate}
 \item there is no formatting inherent in the hooks. If one wants to
   put some watermark on a page, it is his own duty to put in the
   hook the code to place the watermark in the right position. Also
   note that the hooks code should \emph{eat up no space} in the
   page.  Again, if the hooks are meant to place some material on the
   page, it is the duty of the hook programmer to put code in the
   hooks to pretend that the material has zero width and zero height.
   The implementation is \emph{lighter} than the \Lpack{bobhook} one,
   and possibly more flexible, since one is not limited by any
   pre-coded formatting for the hooks. On the other hand it is
   possibly more difficult to use. Nonetheless, it is easy to think
   of other packages relying on \Lpack{everypage} to deliver more
   user-friendly and \emph{task specific} interfaces. Already there
   are a couple of them: the package \Lpack{flippdf} produces
   mirrored pages in a PDF document and \Lpack{draftwatermark}
   watermarks document pages.
 \item similarly to \Lpack{bobhook} and \Lpack{watermark}, the
   package relies on the manipolatoin of the internal \LaTeX\ macro
   |\@begindvi| to do the job. However, the redefinition of
   |\@begindvi| is here postponed as much as possible, striving to
   avoid interference with other packages using |\AtBeginDvi| or
   anyway manipulating |\@begindvi|. Specifically \Lpack{everypage}
   makes no special assumption on the initial code that |\@begindvi|
   might contain.
 \end{enumerate}



Also in some sense \Lpack{everypage} can be related to package
 \Lpack{everyshi} by Martin Schroeder, but it differs radically from
 it in the implementation. In fact,\Lpack{everypage} operates by
 manipulation of the |\@begindvi| macro, rather than at the
 lower level |shipout| macro.


\section{How to place a background image}

One can use TikZ to place a background image on a page

First we define some utility macros:


\begin{teXXX}
  \def\bg@contents{Draft}
  \def\bg@color{red!45}
  \def\bg@angle{60}
  \def\bg@opacity{.5}
  \def\bg@scale{15}
  \def\bg@position{current page.center}
  \def\bg@anchor{}
  \def\bg@hshift{0}
  \def\bg@vshift{0}
\end{teXXX}

A new command is then developed to describe the background material

\begin{teX}
\newcommand\bg@material{%
   \begin{tikzpicture}[remember picture,overlay]
   \node [rotate=\bg@angle,scale=\bg@scale,opacity=\bg@opacity,%
   xshift=\bg@hshift,yshift=\bg@vshift,color=\bg@color]
   at (\bg@position) [\bg@anchor] {\bg@contents};
  \end{tikzpicture}}%
\end{teX}

Once the background material has been defined we can place it on the page by simply calling

\begin{teXXX}
   \newcommand\BgThispage{\AddThispageHook{\bg@material}}
\end{teXXX}

The background package has capitalized on two good packages the TikZ and the everypage. Similarly you can use your own ingenuity to design whatever you want




\section{hooking at shipout}


This package provides the hooks \cs{EveryShipout} and 
  \cs{AtNextShipout} whose arguments are executed after the output 
  routine has constructed \cs{box255}, and before \cs{shipout} is 
  called.

  An example application for this package would be a package for
  adding text to the bottom of each page.
  Such a package does exist: \docpkg{prelim2e}\cite{package!prelim2e}.

The solution  uses is based on code developed in  \textsf{quire.tex} by
 Marcel R.~van der Goot.  It is based upon \cs{afterassignment} and \cs{aftergroup}.



 







































% 
%</driver>
% \fi
% 
%  \CheckSum{0}
%  \CharacterTable
%  {Upper-case    \A\B\C\D\E\F\G\H\I\J\K\L\M\N\O\P\Q\R\S\T\U\V\W\X\Y\Z
%   Lower-case    \a\b\c\d\e\f\g\h\i\j\k\l\m\n\o\p\q\r\s\t\u\v\w\x\y\z
%   Digits        \0\1\2\3\4\5\6\7\8\9
%   Exclamation   \!     Double quote  \"     Hash (number) \#
%   Dollar        \$     Percent       \%     Ampersand     \&
%   Acute accent  \'     Left paren    \(     Right paren   \)
%   Asterisk      \*     Plus          \+     Comma         \,
%   Minus         \-     Point         \.     Solidus       \/
%   Colon         \:     Semicolon     \;     Less than     \<
%   Equals        \=     Greater than  \>     Question mark \?
%   Commercial at \@     Left bracket  \[     Backslash     \\
%   Right bracket \]     Circumflex    \^     Underscore    \_
%   Grave accent  \`     Left brace    \{     Vertical bar  \|
%   Right brace   \}     Tilde         \~}
%
%
%
% \changes{1.0}{2018/01/26}{Converted to DTX file}
%
% \DoNotIndex{\newcommand,\newenvironment}
% \GetFileInfo{phd.dtx}
% 
%  \def\fileversion{v1.0}          
%  \def\filedate{2012/03/06}
% \title{The \textsf{phd} package.
% \thanks{This
%        file (\texttt{phd-utils.dtx}) has version number \fileversion, last revised
%        \filedate.}
% }
% \author{Dr. Yiannis Lazarides \\ \url{yannislaz@gmail.com}}
% \date{\filedate}
% 
% ^^A\maketitle
% 
% ^^A\frontmatter
%  ^^A\coverpage{./images/hine02.jpg}{Book Design }{Camel Press}
% ^^A\secondpage
% \pagestyle{empty}
%  \begin{abstract}
%   This is the Ukrainian language module for the {datetime2}
%   package. If you want to use the settings in this module you must
%   install it in addition to installing \pkg{datetime2}. If you use
%   {babel} or {polyglossia}, you will need this module to
%   prevent them from redefining \cs{today}. The {datetime2}
%   {useregional} setting must be set to "text" or "numeric"
%   for the language styles to be set.
%   Alternatively, you can set the style in the document using
%   \cs{DTMsetstyle}, but this may be changed by \cs{dateukrainian}
%   depending on the value of the {useregional} setting.
% \end{abstract}
% 
% \chapter{The phd-i18n Package}
% One of the primary aims of the package was to simplify the user interface. At the 
% author level, if one has the appropriate stylesheet, nothing needs to be done.
%
% \section{Usage}
%
% To set the main language of the document use:
%  \begin{sverbatim}
%    \cxset{locale german}
%  \end{sverbatim}
%
% This is best done early in the preamble.
%
% To use any secondary language in the text, there is no need with LuaLaTeX to 
% do anything in the preamble. Just use the appropriate |\text| command.
%   \begin{quote}
%     |\textgerman{your text here}|
%   \end{quote}
%
% For longer texts you can use the environment type commands:
%   \begin{quote}
%     |\begin{ngerman}|\\
%     |\end{ngerman}|
%   \end{quote}
%
% All locale names can be inputted in multiple ways: a) Using thelowercase name of the language as found
% in polyglossia or Babel. Using a title case of the language name i.e., |Greek| or |greek|. You can also use 
% the ISO two code or BCF47 codes for greek it would be |el|. If there are dialects you can use |UKenglish|, as in Babel. Remember you only input the main language. If you do not and the package is included it will default
% to |UKenglish|. 
%
% ^^A \StopEventually{}
%  \OnlyDescription
%
%  ^^A\StopEventually{\printindex}
% \vfill
%<*package>
% \CodelineNumbered
% \pagestyle{headings}
% \def\chaptername{\appendixname}
% \appendix
% \chapter{Implementation}
%
% \section{Specification}
%
% \begin{enumerate}
%  \item  Provide translation strings for all languages listed in
%     the Babel and polyglossia packages and extend this to all the unicode languages.
%  \item  Handle date time.
%  \item Provide useful macros.
%  \item Set quotation marks for the language.
%  \item Provide lua code where appropriate.
%  \item Handle specific numerals, for languages that have their own
%        numerals.
%  \item Handle text directionality without having to load the bidi package.
%  \item Handle Asian languages.
%  \item Rationalize methodology and algorithms as far as possible.
%  \item Enable traditional typesetting rules for the language. I am not too sure if this
%        belongs here, as at least for sectioning commands this is provided by the sectioning
%        keys, however it makes no harm to re-intoduce it here.
%        \begin{enumerate}
%          \item Different enumerate environments (e.g. for French).
%          \item Indentation of paragraphs after sections (e.g. French, basque, Ukrainian).
%          \item French spacing after punctuation.
%          \item Spacing before colons.
%        \end{enumerate}
%  \item No active quotes will be provided by default, as we expect the author to enter
%        commands using unicode. However these will be provided as options and be compatible
%        with Babel.
%  \item Version 1.0 should work with \lualatex only. Higher versions will be
%        adapted to work with \latexe and \xetex, as far as this is feasible..
%  \item Provide a Go or Lua pre-processor utility to reduce the user mark-up for
%        quotations and other similar cases such as spacing before and after \cs{ldots}.
% \end{enumerate}
%
% \section{Data}
%
% \begin{enumerate}
% \item The CLDR data currently at version 33.1 will be used as the data source for 
%       translations available in the CLDR specification. A Go program will be developed
%       to download the data and transform it to \latex macros.
% \item The CLDR does not provide translations for sectioning and other common strings currently
%       required. These will be manually entered.
% \item Number formatting will use the CLDR data.
% \item Date formatting will use both the CLDR data as well as current conventional data formatting
%       as used in polyglossia and babel.
% \end{enumerate}
% \section{Preliminaries}
%
%   Standard file identification. We first announce the package 
%	 and require that it be used with \LaTeX2e.
%
%    \begin{macrocode}
\NeedsTeXFormat{LaTeX2e}[2017/04/15]%
\ProvidesPackage{phd-i18n}[2018/1/13 v1.0 i18N utilities (YL)]%
%    \end{macrocode}
% \section{Keys}
% The keys are defined using the \pgfname keys package. I diverted from the normal
% pattern used in other \pkg{phd} packages, using a higher number of 
% paths than normal. I did this, as it can give more flexibility to user
% extensions and also to have almost a one to one relationship with the nesting
% found in CLDR language files.
%
% We first define a generic command to generate keys for a language tag.
%
%    \begin{macrocode}
\ExplSyntaxOn
\def\setcaptions#1{
\cxset
 {
  locale/#1/captions/refname/.code                             = 
    \cs_set:cpn {refname}{##1}
    \cs_set:cpn {greekrefname} {##1},
  locale/#1/captions/abstractname/.code                        = 
    \def\abstractname{##1},
  locale/#1/captions/bibname/.code                             = 
    \def\bibname{##1},
  locale/#1/captions/prefacename/.code                         = 
    \def\prefacename{##1},
  locale/#1/captions/chaptername/.code                         = 
    \def\chaptername{\panunicode ##1},
  locale/#1/captions/appendixname/.code   = \cs_set:cpn {appendixname} {##1},
  locale/#1/captions/contentsname/.code   = \cs_set:cpn {contentsname} {##1},
  locale/#1/captions/listfigurename/.code = \cs_set:cpn {listfigurename} {##1},
  locale/#1/captions/listtablename/.code  = \cs_set:cpn {listtablename} {##1},
  locale/#1/captions/indexname/.code      = \cs_set:cpn {indexname} {##1},
  locale/#1/captions/figurename/.code     = \cs_set:cpn {figurename} {##1},
  locale/#1/captions/tablename/.code      = \cs_set:cpn {tablename} {##1},
  locale/#1/captions/partname/.code       = \cs_set:cpn {partname} {##1},
  locale/#1/captions/pagename/.code       = \cs_set:cpn {pagename} {##1},
  locale/#1/captions/seename/.code        = \cs_set:cpn {seename} {##1},
  locale/#1/captions/alsoname/.code       = \cs_set:cpn {alsoname} {##1},
  locale/#1/captions/enclname/.code       = \cs_set:cpn {enclname} {##1},
  locale/#1/captions/ccname/.code         = \cs_set:cpn {ccname} {##1},
  locale/#1/captions/headtoname/.code     = \cs_set:cpn {headtoname} {##1},
  locale/#1/captions/proofname/.code      = \cs_set:cpn {proofname} {##1},
  locale/#1/captions/glossaryname/.code   = \cs_set:cpn {glossaryname} {##1},
%    \end{macrocode}
% For dates we have a slightly different approach than Babel and Polyglossia,
% we just redefine \cs{today}. So far I don't see the need to define \cs{date}\meta{language}.
% If the language is set as main language it will work ok and for others it will work
% in a group. It will save +- 100 tokens. We can always add it, afterwards just for documentation
% if necessary.
%    \begin{macrocode}
  locale/#1/date/.code  = \cs_set:cpn {today} {##1}
  \cs_set:cpn {date#1}{##1},
%    \end{macrocode}
% \subsection{Numbers}
%  For numbers we define one to one keys to match the numbers.json of the CLDR specification
%  for the language.
%
%    \begin{macrocode}
  locale/#1/numbers/defaultnumberingsystem/.code         = 
    \def\defaultnumberingsystem{##1},
  locale/#1/numbers/othernumberingsystems/.code          = 
    \def\othernumberingsystems{##1},
  locale/#1/numbers/minimumgroupdigits/.code             = 
    \def\minimumgroupdigits{##1},
  locale/#1/numbers/symbols/decimal/.code                = 
    \def\symbolsdecimal{##1},
  locale/#1/numbers/symbols/group/.code                  = 
    \def\symbolsgroup{##1},
  locale/#1/numbers/symbols/list/.code                   = 
    \def\symbolslist{##1},
  locale/#1/numbers/symbols/percentsign/.code            = 
    \def\symbolspermille{##1},
  locale/#1/numbers/symbols/plussign/.code               = 
    \def\symbolsplussign{##1},
  locale/#1/numbers/symbols/minussign/.code              = 
    \def\symbolsminussign{##1},
  locale/#1/numbers/symbols/exponential/.code            = 
    \def\symbolsexponential{##1},
  locale/#1/numbers/symbols/superscriptingexponent/.code = 
    \def\symbolssuperscriptingexponent{##1},
  locale/#1/numbers/symbols/permille/.code               = 
    \def\symbolsnan{##1},
  locale/#1/numbers/symbols/infinity/.code               = 
    \def\symbolsinfinity{##1},
  locale/#1/numbers/symbols/nan/.code                    = 
    \def\symbolsnan{##1},
  locale/#1/numbers/symbols/timeseparator/.code          = 
    \def\symbolstimeseparator{##1},
% 
 }
 }
 
%    \end{macrocode} 
%    \begin{macrocode}
\setcaptions{asturian}
\setcaptions{amharic}
\setcaptions{greek} 
\setcaptions{german}
\setcaptions{french}
\setcaptions{italian}
\setcaptions{albanian}
\setcaptions{malayalam}
\setcaptions{basque}
\setcaptions{brazil}
\setcaptions{breton}
\setcaptions{bulgarian}
\setcaptions{catalan}
\setcaptions{croatian}
\setcaptions{czech}
\setcaptions{danish}
\setcaptions{dutch} %TODO
\setcaptions{estonian}
\setcaptions{finnish}
\setcaptions{friulan}
\setcaptions{galician}
\setcaptions{icelandic}
\setcaptions{irish}
\setcaptions{latin}
\setcaptions{latvian}
\setcaptions{lithuanian}
\setcaptions{lsorbian}
\setcaptions{magyar}
\setcaptions{marathi}
\setcaptions{nko}
\setcaptions{norsk}
\setcaptions{occitan}
\setcaptions{piedmontese}
\setcaptions{polish}
\setcaptions{portuges}
\setcaptions{romanian}
\setcaptions{romansh}
\setcaptions{samin}
\setcaptions{serbian}
\setcaptions{serbian~cyrillic}
\setcaptions{slovak}
\setcaptions{slovenian}
\setcaptions{swedish}
\setcaptions{tamil}
\setcaptions{telugu}
\setcaptions{turkish}
\setcaptions{turkmen}
\setcaptions{ukrainian}
\setcaptions{usorbian}
\setcaptions{hangul}
\setcaptions{welsh}
\setcaptions{russian}
%    \end{macrocode}
% \section{Functions generated via scripts}
% A number of \latex3 functions are generated automatically via a Go program. The Go script
% obtains data for a particular language tag via the CLDR database and transforms the data into
% suitable TeX commands. As TeX commands are stored in memory this is faster for execution.
%
% \begin{docCommand}{l_phd_months_wide_}{language\meta{month}}
%  Used to print the wide format month for a \meta{language}. 
% \end{docCommand}
%
% \begin{docCommand}{l_phd_months_abbreviated_}{language\meta{month}}
%  Used to print the wide format month for a \meta{language}. 
% \end{docCommand}
%
% \begin{docCommand}{l_phd_months_narrow_} {language\meta{month}}
%  Used to print the wide format month for a \meta{language}. 
% \end{docCommand}

% \section{Asturian}
%Astur-Leonese is the historical language of Asturias, portions of the Spanish provinces of León and Zamora and the area surrounding Miranda do Douro in northeastern Portugal.[8] Like the other Romance languages of the Iberian peninsula, it evolved from Vulgar Latin during the early Middle Ages. Asturian was closely linked with the Kingdom of Asturias (718–910) and the ensuing Leonese kingdom. The language had contributions from pre-Roman languages spoken by the Astures, an Iberian Celtic tribe, and the post-Roman Germanic languages of the Visigoths and Suevi. CLDR language code = ast
%    \begin{macrocode}
\cxset{locale~asturian/.style = {
  locale/asturian/captions/refname        = Referencies,
  locale/asturian/captions/abstractname   = Sumariu,
  locale/asturian/captions/bibname        = Bibliografía,
  locale/asturian/captions/prefacename    = Entamu,
  locale/asturian/captions/chaptername    = Capítulu,
  locale/asturian/captions/appendixname   = Apéndiz,
  locale/asturian/captions/contentsname   = Conteníu,
  locale/asturian/captions/listfigurename = Llista~de~figures,
  locale/asturian/captions/listtablename  = Llista~de~tables,
  locale/asturian/captions/indexname      = Índiz,
  locale/asturian/captions/figurename     = Figura,
  locale/asturian/captions/tablename      = Tabla,
  locale/asturian/captions/partname       = Parte,
  locale/asturian/captions/pagename       = Páxina,
  locale/asturian/captions/seename        = ver,
  locale/asturian/captions/alsoname       = ver~tamién,
  locale/asturian/captions/enclname       = incl.,
  locale/asturian/captions/ccname         = cc,
  locale/asturian/captions/headtoname     = Pa,
  locale/asturian/captions/proofname      = Demostración,
  locale/asturian/captions/glossaryname   = Glosariu,
  locale/asturian/date=\number\day~\ifcase\month\or
    de~xineru\or de~febreru\or de~marzu\or d'abril\or de~mayu\or de~xunu\or
    de~xunetu\or d'agostu\or de~setiembre\or d'ochobre\or de~payares\or
    d'avientu\fi\space de~\number\year,
}} 
\cxset{locale~Asturian/.alias = locale~asturian}
%    \end{macrocode}
% \CaptionsList{Asturian}
% \section{Amharic}
% CLDR is am
%    \begin{macrocode}
\cxset{locale~amharic/.style = {
  locale/amharic/captions/refname        = የነሥ~ጹሁፍ~ምንጭ,
  locale/amharic/captions/abstractname   = አኅጽተሮ~ጽሁፍ,
  locale/amharic/captions/bibname        = ቢዋ~መጽሃፍት,
  locale/amharic/captions/prefacename    = መቅድም,
  locale/amharic/captions/chaptername    = ክፍል,
  locale/amharic/captions/appendixname   = መድበል,
  locale/amharic/captions/contentsname   = ይዘት,
  locale/amharic/captions/listfigurename = የሥዕችሎ~ማውጫ,
  locale/amharic/captions/listtablename  = የሰንጠዥረ~ማውጫ,
  locale/amharic/captions/indexname      = ምህጻር~ቃል,
  locale/amharic/captions/figurename     = ሥዕል,
  locale/amharic/captions/tablename      = ሰንጠረዥ,
  locale/amharic/captions/partname       = ንዑስ ክፍል,
  locale/amharic/captions/pagename       = ገጽ,
  locale/amharic/captions/seename        = ይመልከቱ,
  locale/amharic/captions/alsoname       = ይህምን~ይመልከቱ,
  locale/amharic/captions/enclname       = አባሪዎች,
  locale/amharic/captions/ccname         = ግልባጭ,
  locale/amharic/captions/headtoname     = ለ,
  locale/amharic/captions/proofname      = ማረጋገጫ,
  locale/amharic/captions/glossaryname   = የቃላት~መፍቻ,
  locale/amharic/date=\number\day\space\ifcase\month\or
    de~xineru\or de~febreru\or de~marzu\or d'abril\or de~mayu\or de~xunu\or
    de~xunetu\or d'agostu\or de~setiembre\or d'ochobre\or de~payares\or
    d'avientu\fi\space de~\number\year,
}}
 
\cxset{locale~Amharic/.alias = locale~amharic}
%    \end{macrocode}
% \CaptionsList{Amharic}
% \section{Greek}
%    \begin{macrocode}
\cs_set:Npn \l_phd_months_wide_greek #1 {
  \if_case:w #1
    \or: Ιανουαρίου
    \or: Φεβρουαρίου
    \or: Μαρτίου
    \or: Απριλίου
    \or: Μαΐου
    \or: Ιουνίου
    \or: Ιουλίου
    \or: Αυγούστου
    \or: Σεπτεμβρίου
    \or: Οκτωβρίου
    \or: Νοεμβρίου
    \or: Δεκεμβρίου
  \fi:
}
\cxset{locale~greek/.style = {
  locale/greek/captions/refname        = Αναφορές,
  locale/greek/captions/abstractname   = Περίληψη,
  locale/greek/captions/bibname        = Βιβλιογραφία,
  locale/greek/captions/prefacename    = Πρόλογος,
  locale/greek/captions/chaptername    = Κεφάλαιο,
  locale/greek/captions/appendixname   = Παράρτημα,
  locale/greek/captions/contentsname   = Περιεχόμενα,
  locale/greek/captions/listfigurename = Κατάλογος~σχημάτων,
  locale/greek/captions/listtablename  = Κατάλογος~πινάκων,
  locale/greek/captions/indexname      = Ευρετήριο,
  locale/greek/captions/figurename     = Σχήμα,
  locale/greek/captions/tablename      = Πίνακας,
  locale/greek/captions/partname       = Μέρος,
  locale/greek/captions/pagename       = Σελίδα,
  locale/greek/captions/seename        = βλέπε,
  locale/greek/captions/alsoname       = βλέπε~επίσης,
  locale/greek/captions/enclname       = Συνημμένα,
  locale/greek/captions/ccname         = Κοινοποίηση,
  locale/greek/captions/headtoname     = Προς,
  locale/greek/captions/proofname      = Απόδειξη,
  locale/greek/captions/glossaryname   = Γλωσσάρι,
  locale/greek/date = {\number\day\space%
     \l_phd_months_wide_greek {\month}
      \space\number\year},
}} 
\cxset{locale~Greek/.alias = locale~Greek}  
%    \end{macrocode}
% \CaptionsList{Greek}
% \section{German}
%    \begin{macrocode}
\cs_set:Npn \l_phd_months_wide_german #1 {
  \if_case:w #1
    \or: Januar 
    \or: Februar
    \or: März
    \or: April
    \or: Mai
    \or: Juni
    \or: Juli
    \or: August
    \or: September
    \or: Oktober
    \or: November
    \or: Dezember
  \fi:
}
\cxset{locale~german/.style = {
  locale/german/captions/refname        = Literatur,
  locale/german/captions/abstractname   = Zusammenfassung,
  locale/german/captions/bibname        = Literaturverzeichnis,
  locale/german/captions/prefacename    = Vorwort,
  locale/german/captions/chaptername    = Kapitel,
  locale/german/captions/appendixname   = Anhang,
  locale/german/captions/contentsname   = Inhaltsverzeichnis,
  locale/german/captions/listfigurename = Abbildungsverzeichnis,
  locale/german/captions/listtablename  = Tabellenverzeichnis,
  locale/german/captions/indexname      = Index,
  locale/german/captions/figurename     = Abbildung,
  locale/german/captions/tablename      = Tabelle,
  locale/german/captions/partname       = Teil,
  locale/german/captions/pagename       = Seite,
  locale/german/captions/seename        = siehe,
  locale/german/captions/alsoname       = siehe~auch,
  locale/german/captions/enclname       = Anlage(n),
  locale/german/captions/ccname         = Verteiler,
  locale/german/captions/headtoname     = An,
  locale/german/captions/proofname      = Beweis,
  locale/german/captions/glossaryname   = Glossar,
  locale/german/date = {\number\day.%
    \space 
    \l_phd_months_wide_german{\month}
    \space \number\year},
}}  
\cxset{locale~German/.alias=locale~german} 
%    \end{macrocode}
%
% \CaptionsList{German}
%
% \section{French}
% \subsection{French months} Months for French are defined as per CLDR. 
%    \begin{macrocode}
\cs_set:Npn \l_phd_months_wide_french #1 {
  \if_case:w #1
     \or: janvier
     \or: février
     \or: mars
     \or: avril
     \or: mai
     \or: juin
     \or: juillet
     \or: août
     \or: septembre
     \or: octobre
     \or: novembre
     \or: décembre
  \fi:
}
% 
\cs_set:Npn \l_phd_months_wide_abbreviated #1 {
  \if_case:w #1
     \or: janv.,
     \or: févr.,
     \or: mars,
     \or: avr.,
     \or: mai,
     \or: juin,
     \or: juil.,
     \or: août,
     \or: sept.,
     \or: oct.,
     \or: nov.,
     \or: déc.,
  \fi:
}
% 
\cxset{locale~french/.style = {
  locale/french/captions/refname        = Références,
  locale/french/captions/abstractname   = Résumé,
  locale/french/captions/bibname        = Bibliographie,
  locale/french/captions/prefacename    = Préface,
  locale/french/captions/chaptername    = Chapitre,
  locale/french/captions/appendixname   = Annexe,
  locale/french/captions/contentsname   = Table~des~matières,
  locale/french/captions/listfigurename = Table~des~figures,
  locale/french/captions/listtablename  = Liste~des~tableaux,
  locale/french/captions/indexname      = Index ,
  locale/french/captions/figurename     = \textsc{Fig.},
  locale/french/captions/tablename      = \textsc{Tab.} ,
  locale/french/captions/partname       = ,
  locale/french/captions/pagename       = page,
  locale/french/captions/seename        = \emph{voir},
  locale/french/captions/alsoname       = \emph{voir~aussi} ,
  locale/french/captions/enclname       = P.~J.,
  locale/french/captions/ccname         = Copie~à ,
  locale/french/captions/headtoname     = {} ,
  locale/french/captions/proofname      = Démonstration,
  locale/french/captions/glossaryname   = ,
  locale/french/date = 
      {
        \ifx\ier\undefined\def\ier{er}\fi
        \ifnum\day=1\relax~1\ier%
        \else \number\day\fi
        \space 
        \l_phd_months_wide_french {\month}
        \space 
        \number\year
      },
}} 

\cxset{locale~French/.alias = locale~french}
%    \end{macrocode}
% \CaptionsList{French}
% \section{Italian}
%    \begin{macrocode}
\cs_set:Npn \l_phd_months_wide_italian #1 {
  \if_case:w #1
     \or: gennaio
     \or: febbraio
     \or: marzo
     \or: aprile
     \or: maggio
     \or: giugno
     \or: luglio
     \or: agosto
     \or: settembre
     \or: ottobre
     \or: novembre
     \or: dicembre
  \fi:
}
\cxset{locale~italian/.style = {
  locale/italian/captions/refname = Riferimenti bibliografici,
  locale/italian/captions/abstractname   = Sommario ,
  locale/italian/captions/bibname        = Bibliografia ,
  locale/italian/captions/prefacename    = Prefazione ,
  locale/italian/captions/chaptername    = Capitolo,
  locale/italian/captions/appendixname   = Appendice ,
  locale/italian/captions/contentsname   = Indice ,
  locale/italian/captions/listfigurename = Elenco~delle~figure,
  locale/italian/captions/listtablename  = Elenco~delle~tabelle,
  locale/italian/captions/indexname      = Indice~analitico,
  locale/italian/captions/figurename     = Figura,
  locale/italian/captions/tablename      = Tabella,
  locale/italian/captions/partname       = Parte ,
  locale/italian/captions/pagename       = Pag. , %in Italian abbreviation is preferred
  locale/italian/captions/seename        = vedi ,
  locale/italian/captions/alsoname       = vedi~anche,
  locale/italian/captions/enclname       = Allegati,
  locale/italian/captions/ccname         = e~p.~c. ,
  locale/italian/captions/headtoname     = Per,
  locale/italian/captions/proofname      = Dimostrazione,
  locale/italian/captions/glossaryname   = Glossario,
  locale/italian/date = {\number\day\space
    \l_phd_months_wide_italian {\month}
   \space\number\year},
}} 
\cxset{locale~Italian/.alias = locale~italian}
%    \end{macrocode}
% \CaptionsList{Italian}
% \section{Albanian}
%    \begin{macrocode}
\cs_set:Npn \l_phd_months_wide_albanian #1 {
  \if_case:w #1
    \or: Janar
    \or: Shkurt
    \or: Mars
    \or: Prill
    \or: Maj
    \or: Qershor
    \or: Korrik
    \or: Gusht
    \or: Shtator
    \or: Tetor
    \or: Nëntor
    \or: Dhjetor  
  \fi:
}
\cxset{locale~albanian/.style = {
  locale/albanian/captions/refname        = Referencat,
  locale/albanian/captions/abstractname   = Përmbledhja ,
  locale/albanian/captions/bibname        = Bibliografia ,
  locale/albanian/captions/prefacename    = Parathenia,
  locale/albanian/captions/chaptername    = Kapitulli,
  locale/albanian/captions/appendixname   = Shtesa,
  locale/albanian/captions/contentsname   = Përmbajta,
  locale/albanian/captions/listfigurename = Figurat,
  locale/albanian/captions/listtablename  = Tabelat,
  locale/albanian/captions/indexname      = Indeksi,
  locale/albanian/captions/figurename     = Figura,
  locale/albanian/captions/tablename      = Tabela,
  locale/albanian/captions/partname       = Pjesa,
  locale/albanian/captions/pagename       = Faqe,
  locale/albanian/captions/seename        = shiko,
  locale/albanian/captions/alsoname       = shiko~dhe,
  locale/albanian/captions/enclname       = ,
  locale/albanian/captions/ccname         = ,
  locale/albanian/captions/headtoname     = ,
  locale/albanian/captions/proofname      = Vërtetim,
  locale/albanian/captions/glossaryname   = Përhasja~e~Fjalëve ,
  locale/albanian/date=\number\day\space
  \l_phd_months_wide_albanian {\month}
   \space \number\year,
}} 
%    \end{macrocode}
% \section{Malayalam}
%    \begin{macrocode}
\cxset{locale~malayalam/.style = {
  locale/malayalam/captions/refname = ,
  locale/malayalam/captions/abstractname   = സാരാംശം,
  locale/malayalam/captions/bibname        = ,
  locale/malayalam/captions/prefacename    = ,
  locale/malayalam/captions/chaptername    = \panunicode അദ്ധ്യായം,
  locale/malayalam/captions/appendixname   = ശിഷ്ടം,
  locale/malayalam/captions/contentsname   = \panunicode ഉള്ളടക്കം,
  locale/malayalam/captions/listfigurename = ചിത്രസൂചിക,
  locale/malayalam/captions/listtablename  = പട്ടികകളുടെ~സൂചിക,
  locale/malayalam/captions/indexname      = സൂചിക,
  locale/malayalam/captions/figurename     = ചിത്രം,
  locale/malayalam/captions/tablename      = പട്ടിക,
  locale/malayalam/captions/partname       = ,
  locale/malayalam/captions/pagename       = , 
  locale/malayalam/captions/seename        = കാണുക,
  locale/malayalam/captions/alsoname       = ഇതും~കാണുക,
  locale/malayalam/captions/enclname       = ,
  locale/malayalam/captions/ccname         = ,
  locale/malayalam/captions/headtoname     = ,
  locale/malayalam/captions/proofname      = ,
  locale/malayalam/captions/glossaryname   = ,
  locale/malayalam/date                    = {
  \panunicode\number\year\space\ifcase\month\or
     ജനുവരി\or
     ഫിബ്രുവരി\or
     മാർച്ച്\or
     ഏപ്രിൽ\or
     മെയ്\or
     ജൂൺ\or
     ജൂലായ്\or
     ആഗസ്ത്\or
     സെപ്തംബർ\or
     ഒക്ടോബർ\or
     നവംബർ\or
     ഡിസംബർ\fi
     \space\number\day},
}} 
%    \end{macrocode}
% \CaptionsList{malayalam}
% \section{Russian}
%    \begin{macrocode}
\cs_set:Npn \l_phd_months_wide_russian #1 {
  \if_case:w #1
     \or: января
     \or: февраля
     \or: марта
     \or: апреля
     \or: мая
     \or: июня
     \or: июля
     \or: августа
     \or: сентября
     \or: октября
     \or: ноября
     \or: декабря
  \fi:
}
\cxset{locale~russian/.style = {
  locale/russian/captions/refname        = Список~литературы,
  locale/russian/captions/abstractname   = Аннотация,
  locale/russian/captions/bibname        = Литература,
  locale/russian/captions/prefacename    = Предисловие,
  locale/russian/captions/chaptername    = Глава,
  locale/russian/captions/appendixname  
  
   = Приложение,
  locale/russian/captions/contentsname   = Оглавление,
  locale/russian/captions/listfigurename = Список~иллюстраций,
  locale/russian/captions/listtablename  = Список~таблиц,
  locale/russian/captions/indexname      = Предметный~указатель,
  locale/russian/captions/figurename     = Рис.,
  locale/russian/captions/tablename      = Таблица,
  locale/russian/captions/partname       = Часть,
  locale/russian/captions/pagename       = с., 
  locale/russian/captions/seename        = см.,
  locale/russian/captions/alsoname       = см.~также,
  locale/russian/captions/enclname       = вкл.,
  locale/russian/captions/ccname         = исх.,
  locale/russian/captions/headtoname     = вх.,
  locale/russian/captions/proofname      = Доказательство,
  locale/russian/captions/glossaryname = ,
  locale/russian/date = 
    {\number\day%
      \space
       \l_phd_months_wide_russian {\month}
      \space \number\year\space г.},
  }
} 
%    \end{macrocode}
% \CaptionsList{russian}
% \section{Basque}
% The Basque language definitions are in line with those of Babel and Polyglossia. Sadly we do not
% have any miller's numbers yet. See Languages Monograph for more details.
%    \begin{macrocode}
\cs_set:Npn \l_phd_months_wide_basque #1 {
  \if_case:w #1
    \or: urtarrilaren
    \or: otsailaren
    \or: martxoaren
    \or: apirilaren
    \or: maiatzaren
    \or: ekainaren
    \or: uztailaren
    \or: abuztuaren
    \or: irailaren
    \or: urriaren
    \or: azaroaren
    \or: abenduaren
  \fi:
}
\cxset{locale~basque/.style = {
  locale/basque/captions/refname        = Erreferentziak,
  locale/basque/captions/abstractname   = Laburpena,
  locale/basque/captions/bibname        = Bibliografia,
  locale/basque/captions/prefacename    = Hitzaurrea,
  locale/basque/captions/chaptername    = Kapitulua,
  locale/basque/captions/appendixname   = Eranskina,
  locale/basque/captions/contentsname   = Gaien~Aurkibidea,
  locale/basque/captions/listfigurename = Irudien~Zerrenda,
  locale/basque/captions/listtablename  = Taulen Zerrenda,
  locale/basque/captions/indexname      = Kontzeptuen Aurkibidea,
  locale/basque/captions/figurename     = Irudia,
  locale/basque/captions/tablename      = Taula,
  locale/basque/captions/partname       = Atala,
  locale/basque/captions/pagename       = Orria , 
  locale/basque/captions/seename        = Ikusi,
  locale/basque/captions/alsoname       = {Ikusi,~halaber}, % has a comma!
  locale/basque/captions/enclname       = Erantsia,
  locale/basque/captions/ccname         = Kopia,
  locale/basque/captions/headtoname     = Nori,
  locale/basque/captions/proofname      = Frogapena,
  locale/basque/captions/glossaryname   = Glosarioa,
  locale/basque/date = {\number\year.eko\space
  \l_phd_months_wide_basque{\month}
  \space\number\day},
}} 
%    \end{macrocode}
% \CaptionsList{basque}
% \section{Brazil}
%    \begin{macrocode}
\cs_set:Npn \l_phd_months_wide_brazil #1 
  {
    \if_case:w #1
      \or: janeiro
      \or: fevereiro
      \or: março
      \or: abril
      \or: maio
      \or: junho
      \or: julho
      \or: agosto
      \or: setembro
      \or: outubro
      \or: novembro
      \or: dezembro 
    \fi:
  }
\cxset{locale~brazil/.style = {
  locale/brazil/captions/refname        = Referências,
  locale/brazil/captions/abstractname   = Resumo,
  locale/brazil/captions/bibname        = Referências~Bibliográficas,
  locale/brazil/captions/prefacename    = Prefácio,
  locale/brazil/captions/chaptername    = Capitulo,
  locale/brazil/captions/appendixname   = Apêndice,
  locale/brazil/captions/contentsname   = Sumário,
  locale/brazil/captions/listfigurename = Lista~de~Figuras,
  locale/brazil/captions/listtablename  = Lista~de~Tabelas,
  locale/brazil/captions/indexname      = Índice~Remissivo,
  locale/brazil/captions/figurename     = Figura,
  locale/brazil/captions/tablename      = Tabela,
  locale/brazil/captions/partname       = Parte,
  locale/brazil/captions/pagename       = Página, 
  locale/brazil/captions/seename        = veja,
  locale/brazil/captions/alsoname       = veja~também,
  locale/brazil/captions/enclname       = Anexo,
  locale/brazil/captions/ccname         = Cópia~para,
  locale/brazil/captions/headtoname     = Para,
  locale/brazil/captions/proofname      = Demonstração,
  locale/brazil/captions/glossaryname   = Glossário,
  locale/brazil/date = {\number\day\space de\space\ifcase\month\or
      janeiro\or fevereiro\or março\or abril\or maio\or junho\or
      julho\or agosto\or setembro\or outubro\or novembro\or dezembro%
      \fi\space de\space\number\year},
}} 
%    \end{macrocode}
% \CaptionsList{brazil}
% \section{Breton}
%    \begin{macrocode}
\cs_set:Npn \l_phd_months_wide_breton #1 
  {
    \if_case:w #1
     \or: Genver
     \or: C'hwevrer
     \or: Meurzh
     \or: Ebrel
     \or: Mae
     \or: Mezheven
     \or: Gouere
     \or: Eost
     \or: Gwengolo
     \or: Here
     \or: Du
     \or: Kerzu
    \fi:
  }
\cxset{locale~breton/.style = {
  locale/breton/captions/refname        = Daveennoù,
  locale/breton/captions/abstractname   = Dvierrañ,
  locale/breton/captions/bibname        = Lennadurezh,
  locale/breton/captions/prefacename    = Rakskrid,
  locale/breton/captions/chaptername    = Pennad,
  locale/breton/captions/appendixname   = Stagadenn,
  locale/breton/captions/contentsname   = Taolenn,
  locale/breton/captions/listfigurename = Listenn~ar~Figurennoù,
  locale/breton/captions/listtablename  = Listenn~an~taolennoù,
  locale/breton/captions/indexname      = Meneger,
  locale/breton/captions/figurename     = Figurenn,
  locale/breton/captions/tablename      = Taolenn,
  locale/breton/captions/partname       = Lodenn,
  locale/breton/captions/pagename       = Pajenn, 
  locale/breton/captions/seename        = Gwelout,
  locale/breton/captions/alsoname       = Gwelout~ivez,
  locale/breton/captions/enclname       = Dielloù~kevret,
  locale/breton/captions/ccname         = Eilskrid~da,
  locale/breton/captions/headtoname     = evit,
  locale/breton/captions/proofname      = Proof,
  locale/breton/captions/glossaryname   = Glossary,
  locale/breton/date = 
    {\ifnum\day=1\relax 1\/\textsuperscript{añ}\else
    \number\day\fi \space a\space viz\space
    \l_phd_months_wide_breton {\month}
    \space\number\year},
  }
} 
%    \end{macrocode}
% \section{Bulgarian}
%    \begin{macrocode}
\cs_set:Npn \l_phd_months_wide_bulgarian #1 
  {
    \if_case:w #1
      \or: януари
      \or: февруари
      \or: март
      \or: април
      \or: май
      \or: юни
      \or: юли
      \or: август
      \or: септември
      \or: октомври
      \or: ноември
      \or: декември
    \fi:
  }
\cxset{locale~bulgarian/.style = {
  locale/bulgarian/captions/refname        = Литература,
  locale/bulgarian/captions/abstractname   = Абстракт,
  locale/bulgarian/captions/bibname        = Библиография,
  locale/bulgarian/captions/prefacename    = Предговор,
  locale/bulgarian/captions/chaptername    = Глава,
  locale/bulgarian/captions/appendixname   = Приложение,
  locale/bulgarian/captions/contentsname   = Съдържание,
  locale/bulgarian/captions/listfigurename = Списък~на~фигурите,
  locale/bulgarian/captions/listtablename  = Списък~на~таблиците,
  locale/bulgarian/captions/indexname      = Азбучен~указател,
  locale/bulgarian/captions/figurename     = Фигура,
  locale/bulgarian/captions/tablename      = Таблица,
  locale/bulgarian/captions/partname       = ,
  locale/bulgarian/captions/pagename       = Стр., 
  locale/bulgarian/captions/seename        = вж.,
  locale/bulgarian/captions/alsoname       = вж.\~също~и,
  locale/bulgarian/captions/enclname       = Приложения,
  locale/bulgarian/captions/ccname         = копия,
  locale/bulgarian/captions/headtoname     = {},
  locale/bulgarian/captions/proofname      = Proof,
  locale/bulgarian/captions/glossaryname   = Glossary,
  locale/bulgarian/date = {\number\day\space
       \l_phd_months_wide_bulgarian {\month}
       \space\number\year\space г.},
}} 
%    \end{macrocode}
% \CaptionsList{bulgarian}
% \section{Catalan}
%    \begin{macrocode}
\cs_set:Npn \l_phd_months_wide_catalan #1 
  {
    \if_case:w #1
     \or: de~gener
     \or: de~febrer
     \or: de~març
     \or: d'abril
     \or: de~maig
     \or: de~juny
     \or: de~juliol
     \or: d'agost
     \or: de setembre
     \or: d'octubre
     \or: de~novembre
     \or: de~desembre 
    \fi:
  }
\cxset{locale~catalan/.style = {
  locale/catalan/captions/refname        = Referències,
  locale/catalan/captions/abstractname   = Resum,
  locale/catalan/captions/bibname        = Bibliografia,
  locale/catalan/captions/prefacename    = Pròleg,
  locale/catalan/captions/chaptername    = Capítol,
  locale/catalan/captions/appendixname   = Apèndix,
  locale/catalan/captions/contentsname   = Índex,
  locale/catalan/captions/listfigurename = Índex~de~figures,
  locale/catalan/captions/listtablename  = Índex~de~taules,
  locale/catalan/captions/indexname      = Índex~alfabètic,
  locale/catalan/captions/figurename     = Figura,
  locale/catalan/captions/tablename      = Taula,
  locale/catalan/captions/partname       = Part,
  locale/catalan/captions/pagename       = Pàgina , 
  locale/catalan/captions/seename        = Vegeu,
  locale/catalan/captions/alsoname       = Vegeu~també ,
  locale/catalan/captions/enclname       = Adjunt,
  locale/catalan/captions/ccname         = Còpies~a,
  locale/catalan/captions/headtoname     = A,
  locale/catalan/captions/proofname      = Demostració,
  locale/catalan/captions/glossaryname   = Glossari,
  locale/catalan/date ={\number\day\space
    \l_phd_months_wide_catalan {\month} 
    \space de~\number\year} ,
}} 
\cxset{locale~Catalan/.alias = locale~catalan}
%    \end{macrocode}
%
% \CaptionsList{Catalan}
%
% \section{Croatian}
%    \begin{macrocode}
\cs_set:Npn \l_phd_months_wide_croatian #1 
  {
    \if_case:w #1
     \or: siječnja
     \or: veljače
     \or: ožujka
     \or travnja
     \or svibnja
     \or lipnja
     \or srpnja
     \or kolovoza
     \or rujna
     \or listopada
     \or studenoga
     \or prosinca 
    \fi:
  }
\cxset{locale~croatian/.style = {
  locale/croatian/captions/refname        = Literatura,
  locale/croatian/captions/abstractname   = Sažetak,
  locale/croatian/captions/bibname        = Bibliografija,
  locale/croatian/captions/prefacename    = Predgovor,
  locale/croatian/captions/chaptername    = Poglavlje,
  locale/croatian/captions/appendixname   = Dodatak,
  locale/croatian/captions/contentsname   = Sadržaj,
  locale/croatian/captions/listfigurename = Popis~slika,
  locale/croatian/captions/listtablename  = Popis~tablica,
  locale/croatian/captions/indexname      = Kazalo,
  locale/croatian/captions/figurename     = Slika,
  locale/croatian/captions/tablename      = Tablica,
  locale/croatian/captions/partname       = Dio,
  locale/croatian/captions/pagename       = Stranica, 
  locale/croatian/captions/seename        = Vidjeti,
  locale/croatian/captions/alsoname       = Također~vidjeti,
  locale/croatian/captions/enclname       = Prilozi,
  locale/croatian/captions/ccname         = Kopija,
  locale/croatian/captions/headtoname     = Prima,
  locale/croatian/captions/proofname      = Dokaz,
  locale/croatian/captions/glossaryname   = Pojmovnik,
  locale/croatian/date ={\number\day.\space
  \l_phd_months_wide_croatian {\month} 
  \space \number\year.},
}} 
%    \end{macrocode}
% \section{Czech}
%    \begin{macrocode}
\cs_set:Npn \l_phd_months_wide_czech #1{
  \if_case:w #1
    \or: ledna
    \or: února
    \or: března
    \or: dubna
    \or: května
    \or: června
    \or: července
    \or: srpna
    \or: září
    \or: října
    \or: listopadu
    \or: prosince 
  \fi:
}  
\cxset{locale~czech/.style = {
  locale/czech/captions/refname        = Referències,
  locale/czech/captions/abstractname   = Resum,
  locale/czech/captions/bibname        = Bibliografia,
  locale/czech/captions/prefacename    = Pròleg,
  locale/czech/captions/chaptername    = Kapitola,
  locale/czech/captions/appendixname   = Apèndix,
  locale/czech/captions/contentsname   = Obsah,
  locale/czech/captions/listfigurename = Índex~de~figures,
  locale/czech/captions/listtablename  = Índex~de~taules,
  locale/czech/captions/indexname      = Índex~alfabètic,
  locale/czech/captions/figurename     = Figura,
  locale/czech/captions/tablename      = Taula,
  locale/czech/captions/partname       = Part,
  locale/czech/captions/pagename       = Pàgina, 
  locale/czech/captions/seename        = Vegeu,
  locale/czech/captions/alsoname       = Vegeu~també,
  locale/czech/captions/enclname       = Adjunt,
  locale/czech/captions/ccname         = Còpies~a,
  locale/czech/captions/headtoname     = A,
  locale/czech/captions/proofname      = Demostració,
  locale/czech/captions/glossaryname   = Glossari,
  locale/czech/date = 
    {\number\day.\space
     \l_phd_months_wide_czech {\month}
    \space \number\year},
    }
} 
%    \end{macrocode}
% \section{Danish}
%    \begin{macrocode}
\cs_set:Npn \l_phd_months_wide_danish #1{
  \if_case:w #1
    \or: januar
    \or: februar
    \or: marts
    \or: april
    \or: maj
    \or: juni
    \or: juli
    \or: august
    \or: september
    \or: oktober
    \or: november
    \or: december
  \fi:
}  
\cxset{locale~danish/.style = {
  locale/danish/captions/refname        = Litteratur,
  locale/danish/captions/abstractname   = Resumé,
  locale/danish/captions/bibname        = Litteratur,
  locale/danish/captions/prefacename    = Forord,
  locale/danish/captions/chaptername    = Kapitel,
  locale/danish/captions/appendixname   = Bilag,
  locale/danish/captions/contentsname   = Indhold,
  locale/danish/captions/listfigurename = Figurer,
  locale/danish/captions/listtablename  = Tabeller,
  locale/danish/captions/indexname      = Indeks,
  locale/danish/captions/figurename     = Figur,
  locale/danish/captions/tablename      = Tabel,
  locale/danish/captions/partname       = Del,
  locale/danish/captions/pagename       = Side, 
  locale/danish/captions/seename        = Se,
  locale/danish/captions/alsoname       = Se~også,
  locale/danish/captions/enclname       = Vedlagt,
  locale/danish/captions/ccname         = Kopi~til,
  locale/danish/captions/headtoname     = Til,
  locale/danish/captions/proofname      = Bevis,
  locale/danish/captions/glossaryname   = Gloseliste ,
  locale/danish/date ={\number\day.\space
    \l_phd_months_wide_danish {\month}
    \space\number\year},
}} 
%    \end{macrocode}
% \section{Estonian}
%   See \href{http://www.eki.ee/itstandard/2000/FDCC.shtml.en#c4}{estonian} local page for 
%    standards. 
%    \begin{macrocode}
\cs_set:Npn \l_phd_months_wide_estonian #1{
  \if_case:w #1
    \or: jaanuar
    \or: veebruar
    \or: märts
    \or: aprill
    \or: mai
    \or: juuni
    \or: juuli
    \or: august
    \or: september
    \or: oktoober
    \or: november
    \or: detsember
  \fi:
}  
\cxset{locale~estonian/.style = {
  locale/estonian/captions/refname = Viited,
  locale/estonian/captions/abstractname   = Kokkuvõte,
  locale/estonian/captions/bibname        = Kirjandus,
  locale/estonian/captions/prefacename    = Sissejuhatus,
  locale/estonian/captions/chaptername    = Peatükk,
  locale/estonian/captions/appendixname   = Lisa,
  locale/estonian/captions/contentsname   = Sisukord,
  locale/estonian/captions/listfigurename = Joonised,
  locale/estonian/captions/listtablename  = Tabelid,
  locale/estonian/captions/indexname      = Indeks,
  locale/estonian/captions/figurename     = Joonis ,
  locale/estonian/captions/tablename      = Tabel,
  locale/estonian/captions/partname       = Osa,
  locale/estonian/captions/pagename       = Lk., 
  locale/estonian/captions/seename        = vt.,
  locale/estonian/captions/alsoname       = vt.~ka,
  locale/estonian/captions/enclname       = Lisa(d),
  locale/estonian/captions/ccname         = Koopia(d),
  locale/estonian/captions/headtoname     = ,
  locale/estonian/captions/proofname      = Korrektuur,
  locale/estonian/captions/glossaryname   = , %unknown
  locale/estonian/date =
   {\number\day.\space
      \l_phd_months_wide_estonian{\month}
      \space\number\year.\space a.
   },
}} 
\cxset{locale~esti/.alias=locale~estonian,
       esti/.alias=locale~estonian,
       locale~Greek/.alias=locale~greek}
       
%    \end{macrocode}
% \section{Finnish}
% The data for Finnish has been collected from Polyglossia, Babel, Translator,
% CLDR files and IBM. It is missing at this stage . 
%    \begin{macrocode}   
\cs_set:Npn \l_phd_months_abbreviated_finnish#1{
  \if_case:w #1
    \or: tammi
    \or: helmi
    \or: maalis
    \or: huhti
    \or: touko
    \or: kesä
    \or: heinä
    \or: eloa
    \or: syys
    \or: loka
    \or: marras
    \or: joulu
  \fi:
} 
%    \end{macrocode}
% I am not too sure if the below are correct. They have been copied from the relevant
% CLDR file, but as we can see these abbreviations will result in ambiguous date strings.
% The example at IBM \footnote{See \href{https://www.ibm.com/support/knowledgecenter/en/SSS28S_8.2.0/XFDL_Specification/i_xfdl_r_formats_fi_FI.html}{IBM Knowledge Center}} uses two character abbreviations, but does not list all of them.
%    \begin{macrocode}
\cs_set:Npn \l_phd_months_narrow_finnish#1{
  \if_case:w #1
    \or: T
    \or: H
    \or: M
    \or: H
    \or: T
    \or: K
    \or: H
    \or: E
    \or: S
    \or: L
    \or: M
    \or: J
  \fi:
} 
%    \end{macrocode}
% The wide months are used as the default.
%    \begin{macrocode}
\cs_set:Npn \l_phd_months_wide_finnish#1{
  \if_case:w #1
    \or: tammikuuta
    \or: helmikuuta
    \or: maaliskuuta
    \or: huhtikuuta
    \or: toukokuuta
    \or: kesäkuuta
    \or: heinäkuuta
    \or: elokuuta
    \or: syyskuuta
    \or: lokakuuta
    \or: marraskuuta
    \or: joulukuuta
  \fi:
}  
%    \end{macrocode}
% \subsection{Lists}
% The Finnish alphabet has three extra letters Å, Ä and Ö. The \enquote{Swedish o} is 
% redundant in Finnish, but is used for writing Finland-Swedish proper names.
%
%    \begin{macrocode}
\cs_new:Npn \int_to_Alph_finnish:n #1
 {
 \int_to_symbols:nnn {#1} { 29 }
 {
  { 1 } { A }
  { 2 } { B }
  { 3 } { C }
  { 4 } { D }
  { 5 } { E }
  { 6 } { F }
  { 7 } { G }
  { 8 } { H }
  { 9 } { I }
  { 10 } { J }
  { 11 } { K }
  { 12 } { L }
  { 13 } { M }
  { 14 } { N }
  { 15 } { O }
  { 16 } { P }
  { 17}  { Q }
  { 18 } { R }
  { 19 } { S }
  { 20 } { T }
  { 21 } { U }
  { 22 } { V }
  { 23 } { W }
  { 24 } { X }
  { 25 } { Y }
  { 26 } { Z }
  { 27 } { Å }
  { 28 } { Ä }
  { 29 } { Ö }
 }
}
\cs_new:Npn \int_to_alph_finnish:n #1
 {
 \int_to_symbols:nnn {#1} { 29 }
 {
  { 1 } { a }
  { 2 } { b }
  { 3 } { c }
  { 4 } { d }
  { 5 } { e }
  { 6 } { f }
  { 7 } { g }
  { 8 } { h }
  { 9 } { i }
  { 10 } { j }
  { 11 } { k }
  { 12 } { l }
  { 13 } { m }
  { 14 } { n }
  { 15 } { o }
  { 16 } { p }
  { 17}  { q }
  { 18 } { r }
  { 19 } { s }
  { 20 } { t }
  { 21 } { u }
  { 22 } { v }
  { 23 } { w }
  { 24 } { x }
  { 25 } { y }
  { 26 } { z }
  { 27 } { å }
  { 28 } { ä }
  { 29 } { ö }
 }
}
%    \end{macrocode}
% We will avoid getting into collations orders and leave this for external sorting
% libraries.
%    \begin{macrocode}

\cxset{locale~finnish/.style = {
  locale/finnish/captions/refname        = Viitteet,
  locale/finnish/captions/abstractname   = Tiivistelmä,
  locale/finnish/captions/bibname        = Kirjallisuutta,
  locale/finnish/captions/prefacename    = Esipuhe,
  locale/finnish/captions/chaptername    = Luku,
  locale/finnish/captions/appendixname   = Liite,
  locale/finnish/captions/contentsname   = Sisältö,
  locale/finnish/captions/listfigurename = Kuvat,
  locale/finnish/captions/listtablename  = Taulukot,
  locale/finnish/captions/indexname      = Hakemisto,
  locale/finnish/captions/figurename     = Kuva,
  locale/finnish/captions/tablename      = Taulukko,
  locale/finnish/captions/partname       = Osa,
  locale/finnish/captions/pagename       = Sivu, 
  locale/finnish/captions/seename        = katso,
  locale/finnish/captions/alsoname       = katso~myös,
  locale/finnish/captions/enclname       = Liitteet,
  locale/finnish/captions/ccname         = Jakelu,
  locale/finnish/captions/headtoname     = Vastaanottaja,
  locale/finnish/captions/proofname      = Todistus,
  locale/finnish/captions/glossaryname   = Sanasto,
  locale/finnish/date ={
    \number\day.\space
    \l_phd_months_wide_finnish{\month}
    \space\number\year},
}} 
\cxset{locale~Finnish/.alias = locale~finnish}  
%    \end{macrocode} 
%
% \CaptionsList{finnish}  
% \section{Friulan}   
% See \href{http://www.siencis-par-furlan.net/wp-content/uploads/cil.e.tiere_.2014.1.pdf}{Friulian Journal of Science}.
%    \begin{macrocode}  
\cs_set:Npn \l_phd_months_wide_friulian#1 {
  \if_case:w #1
   \or: Genâr
   \or: Fevrâr
   \or: Març
   \or: Avril
   \or: Mai
   \or: Jugn
   \or: Lui
   \or: Avost
   \or: Setembar
   \or: Otobar
   \or: Novembar
   \or: Dicembar
  \fi:
}       
\cxset{locale~friulan/.style = {
  locale/friulan/captions/refname = ,
  locale/friulan/captions/abstractname   = Somari,
  locale/friulan/captions/bibname        = Bibliografie,
  locale/friulan/captions/prefacename    = Prefazion,
  locale/friulan/captions/chaptername    = Cjapitul,
  locale/friulan/captions/appendixname   = Zonte,
  locale/friulan/captions/contentsname   = Tabele~gjenerâl,
  locale/friulan/captions/listfigurename = Liste~des~figuris,
  locale/friulan/captions/listtablename  = Liste~des~tabelis,
  locale/friulan/captions/indexname      = Tabele~analitiche,
  locale/friulan/captions/figurename  	 = Figure,
  locale/friulan/captions/tablename      = Tabele,
  locale/friulan/captions/partname       = Part,
  locale/friulan/captions/pagename       = Pagjine, 
  locale/friulan/captions/seename        = cjale,
  locale/friulan/captions/alsoname       = cjale~ancje,
  locale/friulan/captions/enclname       = Zonte(is),
  locale/friulan/captions/ccname         = Cun~copie~a,
  locale/friulan/captions/headtoname     = Par,
  locale/friulan/captions/proofname      = Dimostrazion,
  locale/friulan/captions/glossaryname   = Glossari,
  locale/friulan/date ={\number\day\space di\space
  \l_phd_months_wide_friulian {\month}
  \space dal\space\number\year},
}}  
%    \end{macrocode}
% \section{Galician}
%    \begin{macrocode}
\cs_set:Npn \l_phd_months_wide_galician #1 {
  \if_case:w #1
   \or: Genâr
   \or: Fevrâr
   \or: Març
   \or: Avril
   \or: Mai
   \or: Jugn
   \or: Lui
   \or: Avost
   \or: Setembar
   \or: Otobar
   \or: Novembar
   \or: Dicembar
  \fi:
}  
%    \end{macrocode}
% \section{Galician}
%    \begin{macrocode}
\cxset{locale~galician/.style = {
  locale/galician/captions/refname        = Referencias ,
  locale/galician/captions/abstractname   = Resumo,
  locale/galician/captions/bibname        = Bibliografía,
  locale/galician/captions/prefacename    = Prefacio,
  locale/galician/captions/chaptername    = Capítulo,
  locale/galician/captions/appendixname   = Apéndice,
  locale/galician/captions/contentsname   = Índice~Xeral,
  locale/galician/captions/listfigurename = Índice~de~Figuras,
  locale/galician/captions/listtablename  = Índice~de~Táboas,
  locale/galician/captions/indexname      = Índice~de~Materias,
  locale/galician/captions/figurename     = Figura,
  locale/galician/captions/tablename      = Táboa,
  locale/galician/captions/partname       = Parte,
  locale/galician/captions/pagename       = Páxina, 
  locale/galician/captions/seename        = véxase,
  locale/galician/captions/alsoname       = véxase~tamén,
  locale/galician/captions/enclname       = Adxunto,
  locale/galician/captions/ccname         = Copia~a,
  locale/galician/captions/headtoname     = A,
  locale/galician/captions/proofname      = Demostración,
  locale/galician/captions/glossaryname   = Glosario,
  locale/galician/date ={\number\day~de\space\ifcase\month\or
    xaneiro\or febreiro\or marzo\or abril\or maio\or xuño\or
    xullo\or agosto\or setembro\or outubro\or novembro\or decembro\fi
    \space de~\number\year},
}}   
%    \end{macrocode}
% \CaptionsList{galician}
% \section{Icelandic}
%    \begin{macrocode}
\cxset{locale~icelandic/.style = {
  locale/icelandic/captions/refname        = Heimildir,
  locale/icelandic/captions/abstractname   = Útdráttur,
  locale/icelandic/captions/bibname        = Heimildir,
  locale/icelandic/captions/prefacename    = Formáli,
  locale/icelandic/captions/chaptername    = Kafli,
  locale/icelandic/captions/appendixname   = Viðauki,
  locale/icelandic/captions/contentsname   = Efnisyfirlit ,
  locale/icelandic/captions/listfigurename = Myndaskrá,
  locale/icelandic/captions/listtablename  =Töfluskrá,
  locale/icelandic/captions/indexname      = Atriðisorðaskrá,
  locale/icelandic/captions/figurename     = Mynd,
  locale/icelandic/captions/tablename      = Tafla,
  locale/icelandic/captions/partname       = Hluti,
  locale/icelandic/captions/pagename       = Blaðsíða, 
  locale/icelandic/captions/seename        = Sjá,
  locale/icelandic/captions/alsoname       = Sjá einnig,
  locale/icelandic/captions/enclname       = Hjálagt,
  locale/icelandic/captions/ccname         = Samrit,
  locale/icelandic/captions/headtoname     = Til:,
  locale/icelandic/captions/proofname      = Sönnun,
  locale/icelandic/captions/glossaryname   = Orðalisti,
  locale/icelandic/date ={\number\day.\space\ifcase\month\or
    janúar\or febrúar\or mars\or apríl\or maí\or
    júní\or júlí\or ágúst\or september\or
    október\or nóvember\or desember\fi
    \space\number\year},
}}     
%    \end{macrocode}
% \CaptionsList{icelandic}
% \section{Irish}
%    \begin{macrocode}
\cxset{locale~irish/.style = {
  locale/irish/captions/refname        = Tagairtí,
  locale/irish/captions/abstractname   = Achoimre,
  locale/irish/captions/bibname        = Leabharliosta,
  locale/irish/captions/prefacename    = Réamhrá,
  locale/irish/captions/chaptername    = Tagairtí,
  locale/irish/captions/appendixname   = Aguisín ,
  locale/irish/captions/contentsname   = Clár~Ábhair,
  locale/irish/captions/listfigurename = Léaráidí ,
  locale/irish/captions/listtablename  = Táblaí ,
  locale/irish/captions/indexname      = Innéacs,
  locale/irish/captions/figurename     = Léaráid,
  locale/irish/captions/tablename      = Tábla,
  locale/irish/captions/partname       = Cuid,
  locale/irish/captions/pagename       = Leathanach, 
  locale/irish/captions/seename        = féach,
  locale/irish/captions/alsoname       = féach~freisin,
  locale/irish/captions/enclname       = faoi~iamh,
  locale/irish/captions/ccname         = cc,
  locale/irish/captions/headtoname     = Go,
  locale/irish/captions/proofname      = Cruthúnas,
  locale/irish/captions/glossaryname   = Glossary,
  locale/irish/date ={\number\day\space \ifcase\month\or
    Eanáir\or Feabhra\or Márta\or Aibreán\or
    Bealtaine\or Meitheamh\or Iúil\or Lúnasa\or
    Meán Fómhair\or Deireadh Fómhair\or
    Mí na Samhna\or Mí na Nollag\fi
    \space \number\year},
}}   
%    \end{macrocode}
% \CaptionsList{irish}
% \section{Latin}
%    \begin{macrocode}
\cxset{locale~latin/.style = {
  locale/latin/captions/refname = Conspectus librorum,
  locale/latin/captions/abstractname = Summarium,
  locale/latin/captions/bibname       = Conspectus librorum,
  locale/latin/captions/prefacename   = Praefatio, % change for medieval
  locale/latin/captions/chaptername   = Caput,
  locale/latin/captions/appendixname  = Additamentum,
  locale/latin/captions/contentsname = Index,
  locale/latin/captions/listfigurename = Conspectus~descriptionum,
  locale/latin/captions/listtablename = Conspectus~tabularum,
  locale/latin/captions/indexname = Index~rerum~notabilium,
  locale/latin/captions/figurename  = Descriptio,
  locale/latin/captions/tablename = Tabula,
  locale/latin/captions/partname = Pars,
  locale/latin/captions/pagename = charta, 
  locale/latin/captions/seename  = cfr.,
  locale/latin/captions/alsoname = cfr.,
  locale/latin/captions/enclname = Additur,
  locale/latin/captions/ccname = Exemplar,
  locale/latin/captions/headtoname =,
  locale/latin/captions/proofname = Demonstratio,
  locale/latin/captions/glossaryname = Glossarium,
  locale/latin/date ={\uppercase\expandafter{\romannumeral\day}%
      \space \ifcase\month%
      \or Januarii\or Februarii\or Martii\or Aprilis\or Maji\or
      Junii\or Julii\or Augusti\or Septembris\or Octobris\or
         Novembris
      \or Decembris\fi%
      \space \uppercase\expandafter{\romannumeral\year}},
}} 
%    \end{macrocode}
% \section{Latvian}
%    \begin{macrocode}
\cxset{locale~latvian/.style = {
  locale/latvian/captions/refname        = Literatūras~saraksts,
  locale/latvian/captions/abstractname   = Anotācija,
  locale/latvian/captions/bibname        = Literatūra,
  locale/latvian/captions/prefacename    = Priekšvārds,
  locale/latvian/captions/chaptername    = Nodaļa,
  locale/latvian/captions/appendixname   = Pielikums,
  locale/latvian/captions/contentsname   = Saturs,
  locale/latvian/captions/listfigurename = Attēlu~saraksts,
  locale/latvian/captions/listtablename  = Tabulu~saraksts,
  locale/latvian/captions/indexname      = Index,
  locale/latvian/captions/figurename     = Att.,
  locale/latvian/captions/tablename      = Tabula,
  locale/latvian/captions/partname       = Daļa,
  locale/latvian/captions/pagename       = lpp., 
  locale/latvian/captions/seename        = sk.,
  locale/latvian/captions/alsoname       = sk. arī,
  locale/latvian/captions/enclname       = encl,
  locale/latvian/captions/ccname         = cc,
  locale/latvian/captions/headtoname     = To,
  locale/latvian/captions/proofname      = Pierādījums,
  locale/latvian/captions/glossaryname   =,
  locale/latvian/date ={\number\year.\thinspace gada%
      \space\number\day.\thinspace%
      \ifcase\month\or%
      janvārī\or februārī\or martā\or%
      aprīlī\or maijā\or jūnijā\or%
      jūlijā\or augustā\or septembrī\or%
      oktobrī\or novembrī\or decembrī\fi},
}}    
%    \end{macrocode}
% \section{Lithuanian}
%    \begin{macrocode}
\cxset{locale~lithuanian/.style = {
  locale/lithuanian/captions/refname        = Literatūra,
  locale/lithuanian/captions/abstractname   = Santrauka,
  locale/lithuanian/captions/bibname        = Literatūra,
  locale/lithuanian/captions/prefacename    = Pratarmė,
  locale/lithuanian/captions/chaptername    = Skyrius,
  locale/lithuanian/captions/appendixname   = Priedas,
  locale/lithuanian/captions/contentsname   = Turinys,
  locale/lithuanian/captions/listfigurename = Iliustracijų~sąrašas,
  locale/lithuanian/captions/listtablename  = Lentelių~sąrašas,
  locale/lithuanian/captions/indexname      = Rodyklė,
  locale/lithuanian/captions/figurename     = pav.,
  locale/lithuanian/captions/tablename      = lentelė,
  locale/lithuanian/captions/partname       = Dalis,
  locale/lithuanian/captions/pagename       = puslapis, 
  locale/lithuanian/captions/seename        = žiūrėk,
  locale/lithuanian/captions/alsoname       = taip~pat,
  locale/lithuanian/captions/enclname       = Įdėta,
  locale/lithuanian/captions/ccname         = Kopijos,
  locale/lithuanian/captions/headtoname     = Kam,
  locale/lithuanian/captions/proofname      = Įrodymas,
  locale/lithuanian/captions/glossaryname   = Terminų~žodynas,
  locale/lithuanian/date ={\number\year\space m.\space\ifcase\month\or
      sausio\or
      vasario\or
      kovo\or
      balandžio\or
      gegužės\or
      birželio\or
      liepos\or
      rugpjūčio\or
      rugsėjo\or
      spalio\or
      lapkričio\or
      gruodžio\fi
      \space\number\day\space d.
      },
}}    
%    \end{macrocode}
% \section{Lsorbian}
% See also usorbian for Upper Sorbian.
%    \begin{macrocode}
\cxset{locale~lsorbian/.style = {
  locale/lsorbian/captions/refname        = Referency,
  locale/lsorbian/captions/abstractname   = Abstrakt,
  locale/lsorbian/captions/bibname        = Literatura,
  locale/lsorbian/captions/prefacename    = Zawod,
  locale/lsorbian/captions/chaptername    = Kapitl,
  locale/lsorbian/captions/appendixname   = Dodawki,
  locale/lsorbian/captions/contentsname   = Wopśimjeśe,
  locale/lsorbian/captions/listfigurename = Zapis~wobrazow,
  locale/lsorbian/captions/listtablename  = Zapis~tabulkow,
  locale/lsorbian/captions/indexname      = Indeks,
  locale/lsorbian/captions/figurename     = Wobraz,
  locale/lsorbian/captions/tablename      = Tabulka,
  locale/lsorbian/captions/partname       = Źěl,
  locale/lsorbian/captions/pagename       = Strona , 
  locale/lsorbian/captions/seename        = gl.,
  locale/lsorbian/captions/alsoname       = gl.~teke,
  locale/lsorbian/captions/enclname       = Pśiłoga,
  locale/lsorbian/captions/ccname         = CC,
  locale/lsorbian/captions/headtoname     = Komu,
  locale/lsorbian/captions/proofname      = Proof,
  locale/lsorbian/captions/glossaryname   = Glossary,
  locale/lsorbian/date ={
    \number\day.~\ifcase\month\or
    januara\or februara\or měrca\or apryla\or maja\or
    junija\or julija\or awgusta\or septembra\or oktobra\or
    nowembra\or decembra\fi
    \space \number\year},
}}    
%    \end{macrocode}
%
% \section{Magyar (Hungarian)}
%    \begin{macrocode}
\cxset{locale~magyar/.style = {
  locale/magyar/captions/refname        = Hivatkozások ,
  locale/magyar/captions/abstractname   = Kivonat,
  locale/magyar/captions/bibname        = Irodalomjegyzék,
  locale/magyar/captions/prefacename    = Előszó,
  locale/magyar/captions/chaptername    = fejezet,
  locale/magyar/captions/appendixname   = Függelék,
  locale/magyar/captions/contentsname   = Tartalomjegyzék,
  locale/magyar/captions/listfigurename = Ábrák~jegyzéke,
  locale/magyar/captions/listtablename  = Táblázatok~jegyzéke,
  locale/magyar/captions/indexname      = Tárgymutató,
  locale/magyar/captions/figurename     = ábra,
  locale/magyar/captions/tablename      = táblázat,
  locale/magyar/captions/partname       = rész,
  locale/magyar/captions/pagename       = oldal, 
  locale/magyar/captions/seename        = lásd,
  locale/magyar/captions/alsoname       = lásd~még,
  locale/magyar/captions/enclname       = Melléklet,
  locale/magyar/captions/ccname         = Körlevél–címzettek,
  locale/magyar/captions/headtoname     = Címzett,
  locale/magyar/captions/proofname      = Bizonyítás,
  locale/magyar/captions/glossaryname   = Szójegyzék,
  locale/magyar/date = 
    {
      \number\year.\nobreakspace\ifcase\month\or
      január\or február\or március\or
      április\or május\or június\or
      július\or augusztus\or szeptember\or
      október\or november\or december\fi
    \space\number\day
    },
}}   
\cxset{locale~Magyar/.alias = locale~magyar,
       locale~Hungarian/.alias = locale~magyar} 
%    \end{macrocode}
% \section{Marathi}
%  
%    \begin{macrocode}
\cxset{locale~marathi/.style = {
  locale/marathi/captions/refname        = संदर्भ ,
  locale/marathi/captions/abstractname   = सारांश,
  locale/marathi/captions/bibname        = संदर्भ~ग्रंथांची~यादी,
  locale/marathi/captions/prefacename    = प्रस्तावना,
  locale/marathi/captions/chaptername    = प्रकरण,
  locale/marathi/captions/appendixname   = परिशिष्ट,
  locale/marathi/captions/contentsname   = अनुक्रमणिका,
  locale/marathi/captions/listfigurename = आकृत्यांची~यादी ,
  locale/marathi/captions/listtablename  = कॊष्टकांची~यादी,
  locale/marathi/captions/indexname      = सूची,
  locale/marathi/captions/figurename     = आक्रुती ,
  locale/marathi/captions/tablename      = कोष्टक,
  locale/marathi/captions/partname       = भाग,
  locale/marathi/captions/pagename       = पान, 
  locale/marathi/captions/seename        = पहा,
  locale/marathi/captions/alsoname       = हे~सुध्दा~पहा ,
  locale/marathi/captions/enclname       = समाविष्ट,
  locale/marathi/captions/ccname         = सि.सि.,
  locale/marathi/captions/headtoname     = प्रति,
  locale/marathi/captions/proofname      = सिद्धता,
  locale/marathi/captions/glossaryname   = स्पष्टीकरणांची~यादी,
  locale/marathi/date ={{\panunicode
     \number\day\space
      \ifcase\month\or
       जानेवारी\or
       फेब्रुवारी\or
       मार्च\or
       एप्रिल\or
       मे\or
       जून\or
       जुलै\or
       ऑगस्ट\or
       सप्टेंबर\or
       ऑक्टोबर\or
       नोव्हेंबर\or
       डिसेंबर\fi
     \space\number\year
    }},
}}  
\cxset{locale~Marathi/.alias=locale~marathi} 
%    \end{macrocode}
% \newfontfamily\marathifont{NotoSansDevanagariUI-Regular.ttf}
% \begingroup
% \let\panunicode\marathifont
% \CaptionsList{Marathi}
% \endgroup
% \section{Nko}
% The script is RTL, it needs to be fixed for LuaLaTeX
%    \begin{macrocode}
\RequirePackage{nkonumbers}
\cxset{locale~nko/.style = {
  locale/nko/captions/refname        = संदर्भ,
  locale/nko/captions/abstractname   = सारांश,
  locale/nko/captions/bibname        = संदर्भ~ग्रंथांची~यादी,
  locale/nko/captions/prefacename    = प्रस्तावना,
  locale/nko/captions/chaptername    = {{\textdir TRT ߛߌ߰ߘߊ }},
  locale/nko/captions/appendixname   = परिशिष्ट,
  locale/nko/captions/contentsname   = {{ \textdir TRT ߞߣߐߘߐ }},
  locale/nko/captions/listfigurename = आकृत्यांची~यादी,
  locale/nko/captions/listtablename  = कॊष्टकांची~यादी,
  locale/nko/captions/indexname      = सूची,
  locale/nko/captions/figurename     = आक्रुती,
  locale/nko/captions/tablename      = कोष्टक,
  locale/nko/captions/partname       = भाग,
  locale/nko/captions/pagename       = पान, 
  locale/nko/captions/seename        = पहा,
  locale/nko/captions/alsoname       = हे~सुध्दा~पहा,
  locale/nko/captions/enclname       = समाविष्ट,
  locale/nko/captions/ccname         = सि.सि.,
  locale/nko/captions/headtoname     = प्रति ,
  locale/nko/captions/proofname      = सिद्धता,
  locale/nko/captions/glossaryname   = स्पष्टीकरणांची~यादी,
  locale/nko/date ={{\panunicode\textdir TRT 
  %FIX FOR NKO NUMBERS
    \nkonumber{\year}\space
    \ifcase\month
    \or ߓߌ߲ߠߊߥߎߟߋ߲%
    \or ߞߏ߲ߞߏߜߍ%
    \or ߕߙߊߓߊ%
    \or ߞߏ߲ߞߏߘߌ߬ߓߌ%
    \or ߘߓߊ߬ߕߊ%
    \or ߥߊ߬ߛߌߥߊ߬ߙߊ%
    \or ߞߊ߬ߙߌߝߐ߭%
    \or ߘߓߊ߬ߓߌߟߊ%
    \or ߕߎߟߊߝߌ߲%
    \or ߞߏ߲ߓߌߕߌ߮%
    \or ߣߍߣߍߓߊ%
    \or ߞߏ߬ߟߌ߲߬ߞߏߟߌ߲\fi
    \space ߕߟߋ߬
    \space\nkonumber{\day}
    }},
}}  
%    \end{macrocode}
% \section{Norwegian (Norsk)}
%    \begin{macrocode}
\cxset{locale~norsk/.style = {
  locale/norsk/captions/refname        = Referanser,
  locale/norsk/captions/abstractname   = Sammendrag,
  locale/norsk/captions/bibname        = Bibliografi,
  locale/norsk/captions/prefacename    = Forord,
  locale/norsk/captions/chaptername    = Kapittel,
  locale/norsk/captions/appendixname   = Tillegg,
  locale/norsk/captions/contentsname   = Innhold,
  locale/norsk/captions/listfigurename = Figurer,
  locale/norsk/captions/listtablename  = Tabeller,
  locale/norsk/captions/indexname      = Register,
  locale/norsk/captions/figurename     = Figur,
  locale/norsk/captions/tablename      = Tabell,
  locale/norsk/captions/partname       = Del,
  locale/norsk/captions/pagename       = Side, 
  locale/norsk/captions/seename        = Se,
  locale/norsk/captions/alsoname       = Se~også,
  locale/norsk/captions/enclname       = Vedlegg,
  locale/norsk/captions/ccname         = Kopi~sendt,
  locale/norsk/captions/headtoname     = Til,
  locale/norsk/captions/proofname      = Bevis,
  locale/norsk/captions/glossaryname   = Ordliste,
  locale/norsk/date                    =
    {
      \number\day.~
      \ifcase\month
        \or januar
        \or februar
        \or mars
        \or april
        \or mai
        \or juni
        \or juli
        \or august
        \or september
        \or oktober
        \or november
        \or desember
     \fi
     \space\number\year
    },
}} 
\cxset{locale~Norsk/.alias = locale~norsk,
       locale~Norwegian/.alias = locale~norsk} 
%    \end{macrocode}
%
% \CaptionsList{Norsk}
% \section{Piedmontese}
%    \begin{macrocode}
\cxset{locale~piedmontese/.style = {
  locale/piedmontese/captions/refname        = Riferiment,
  locale/piedmontese/captions/abstractname   = Somari,
  locale/piedmontese/captions/bibname        = Bibliografìa,
  locale/piedmontese/captions/prefacename    = Prefassion,
  locale/piedmontese/captions/chaptername    = Kapittel,
  locale/piedmontese/captions/appendixname   = Gionta,
  locale/piedmontese/captions/contentsname   = Innhald,
  locale/piedmontese/captions/listfigurename = Lista~dle~figure ,
  locale/piedmontese/captions/listtablename  = Lista~dle~tabele,
  locale/piedmontese/captions/indexname      = Tàula~analìtica,
  locale/piedmontese/captions/figurename     = Figura,
  locale/piedmontese/captions/tablename      = Tabela,
  locale/piedmontese/captions/partname       = Part,
  locale/piedmontese/captions/pagename       = Pàgina, 
  locale/piedmontese/captions/seename        = vëd,
  locale/piedmontese/captions/alsoname       = vëd~anche,
  locale/piedmontese/captions/enclname       = Gionta/e,
  locale/piedmontese/captions/ccname         = Con~còpia~a,
  locale/piedmontese/captions/headtoname     = Për,
  locale/piedmontese/captions/proofname      = Dimostrassion,
  locale/piedmontese/captions/glossaryname   = Glossari,
  locale/piedmontese/date ={
  	\number\day\space\ifcase\month\or
      ëd~gené\or ëd~fevré\or ëd~mars\or d'avril\or ëd~maj\or ëd~giugn\or
      ëd~luj\or d'agost\or dë~stèmber\or d'otóber\or ëd~novèmber\or dë~dzèmber%
      \fi\space dal\space\number\year
  },
}}  
%    \end{macrocode}
% \CaptionsList{piedmontese}
% \section{Occitan}
%    \begin{macrocode}
\cxset{locale~occitan/.style = {
  locale/occitan/captions/refname        = Referéncias,
  locale/occitan/captions/abstractname   = Resumit,
  locale/occitan/captions/bibname        = Bibliografia,
  locale/occitan/captions/prefacename    = Prefaci,
  locale/occitan/captions/chaptername    = Capítol,
  locale/occitan/captions/appendixname   = Annèx,
  locale/occitan/captions/contentsname   = Ensenhador,
  locale/occitan/captions/listfigurename = Taula~de~las~figuras,
  locale/occitan/captions/listtablename  = Taula~dels~tablèus,
  locale/occitan/captions/indexname      = Indèx,
  locale/occitan/captions/figurename     = Figura,
  locale/occitan/captions/tablename      = Tablèu,
  locale/occitan/captions/partname       = Partida,
  locale/occitan/captions/pagename       = Pagina, 
  locale/occitan/captions/seename        = vejatz,
  locale/occitan/captions/alsoname       = vejatz~tanben,
  locale/occitan/captions/enclname       = Pèça~junta,
  locale/occitan/captions/ccname         = còpia~a,
  locale/occitan/captions/headtoname     = A,
  locale/occitan/captions/proofname      = Demostracion,
  locale/occitan/captions/glossaryname   = Glossari,
  locale/occitan/date ={
  {\def\occitanmonth{\ifcase\month\or
      de~genièr\or
      de~febrièr\or
      de~març\or
      d'abril\or
      de~mai\or
      de~junh\or
      de~julhet\or
      d'agost\or
      de~setembre\or
      d'octobre\or
      de~novembre\or
      de~decembre\fi
   }
   \def\occitanday{\ifcase\day\or
      1èr\else% primièr
      \number\day\fi% all other numbers
   }
   \occitanday\space \occitanmonth\space de~\number\year
   }},
}} 
\cxset{locale~Occitan/.alias = locale~occitan}
%    \end{macrocode}
% \CaptionsList{Occitan}

% \section{Polish}
%    \begin{macrocode}
\cxset{locale~polish/.style = {
  locale/polish/captions/refname = Literatura,
  locale/polish/captions/abstractname   =Streszczenie ,
  locale/polish/captions/bibname        = Bibliografia,
  locale/polish/captions/prefacename    = Przedmowa,
  locale/polish/captions/chaptername    = Rozdział,
  locale/polish/captions/appendixname   = Dodatek,
  locale/polish/captions/contentsname   = Spis treści,
  locale/polish/captions/listfigurename = Spis rysunków,
  locale/polish/captions/listtablename  = Spis tabel,
  locale/polish/captions/indexname      = Indeks,
  locale/polish/captions/figurename     = Rysunek,
  locale/polish/captions/tablename      = Tabela,
  locale/polish/captions/partname       = Część,
  locale/polish/captions/pagename       = Strona, 
  locale/polish/captions/seename        = Zobacz,
  locale/polish/captions/alsoname       = Zobacz też,
  locale/polish/captions/enclname       = Załącznik,
  locale/polish/captions/ccname         = Kopie:,
  locale/polish/captions/headtoname     = Do,
  locale/polish/captions/proofname      = Dowód ,
  locale/polish/captions/glossaryname   = Glossary, % no ranslation
  locale/polish/date ={\number\day\space\ifcase\month\or
      stycznia\or lutego\or marca\or kwietnia\or maja\or czerwca\or
      lipca\or sierpnia\or września\or października\or
      listopada\or grudnia\fi\space
      \number\year},
}}  
%    \end{macrocode}
% \section{Portuges}
%    \begin{macrocode}
\cxset{locale~portuges/.style = {
  locale/portuges/captions/refname        = Referências,
  locale/portuges/captions/abstractname   = Resumo,
  locale/portuges/captions/bibname        = Bibliografia,
  locale/portuges/captions/prefacename    = Prefácio,
  locale/portuges/captions/chaptername    = Capítulo,
  locale/portuges/captions/appendixname   = Apêndice,
  locale/portuges/captions/contentsname   = Conteúdo,
  locale/portuges/captions/listfigurename = Lista~de~Figuras,
  locale/portuges/captions/listtablename  = Lista~de~Tabelas,
  locale/portuges/captions/indexname      = Índice,
  locale/portuges/captions/figurename     = Figura,
  locale/portuges/captions/tablename      = Tabela,
  locale/portuges/captions/partname       = Parte,
  locale/portuges/captions/pagename       = Página, 
  locale/portuges/captions/seename        = ver,
  locale/portuges/captions/alsoname       = ver~também,
  locale/portuges/captions/enclname       = Anexo,
  locale/portuges/captions/ccname         = Com~cópia~a,
  locale/portuges/captions/headtoname     = Para,
  locale/portuges/captions/proofname      = Dowód,
  locale/portuges/captions/glossaryname   = Glossário,
  locale/portuges/date = 
    {
      \number\day\space de\space\ifcase\month\or
      Janeiro\or Fevereiro\or Março\or Abril\or Maio\or Junho\or
      Julho\or Agosto\or Setembro\or Outubro\or Novembro\or Dezembro\fi
      \space de\space\number\year
    },
  }
}  
%    \end{macrocode}
% \section{Romanian}
%    \begin{macrocode}
\cxset{locale~romanian/.style = {
  locale/romanian/captions/refname        = Bibliografie,
  locale/romanian/captions/abstractname   = Rezumat,
  locale/romanian/captions/bibname        = Bibliografie,
  locale/romanian/captions/prefacename    = Prefață,
  locale/romanian/captions/chaptername    = Capitolul,
  locale/romanian/captions/appendixname   = Anexa,
  locale/romanian/captions/contentsname   = Cuprins,
  locale/romanian/captions/listfigurename = Listă~de~figuri,
  locale/romanian/captions/listtablename  = Listă~de~tabele,
  locale/romanian/captions/indexname      = Glosar,
  locale/romanian/captions/figurename     = Figura,
  locale/romanian/captions/tablename      = Tabela,
  locale/romanian/captions/partname       = Partea,
  locale/romanian/captions/pagename       = Pagina, 
  locale/romanian/captions/seename        = Vezi,
  locale/romanian/captions/alsoname       = Vezi~de~asemenea,
  locale/romanian/captions/enclname       = Anexă,
  locale/romanian/captions/ccname         = Copie,
  locale/romanian/captions/headtoname     = Pentru,
  locale/romanian/captions/proofname      = Demonstrație ,
  locale/romanian/captions/glossaryname   = Glosar,
  locale/romanian/date ={\number\day\space\ifcase\month\or
    ianuarie\or februarie\or martie\or aprilie\or mai\or
    iunie\or iulie\or august\or septembrie\or octombrie\or
    noiembrie\or decembrie\fi
    \space \number\year},
}}  
%    \end{macrocode}
% \section{Romansh}
%    \begin{macrocode}
\cxset{locale~romansh/.style = {
  locale/romansh/captions/refname        = Bibliografia,
  locale/romansh/captions/abstractname   = Recapitulaziun,
  locale/romansh/captions/bibname        = Index~bibliografic,
  locale/romansh/captions/prefacename    = Prefaziun,
  locale/romansh/captions/chaptername    = Chapitel,
  locale/romansh/captions/appendixname   = Appendix,
  locale/romansh/captions/contentsname   = Tavla~dal~cuntegn,
  locale/romansh/captions/listfigurename = Tavla~da~las~figuras,
  locale/romansh/captions/listtablename  = Tavla~da~las~tabellas,
  locale/romansh/captions/indexname      = Register~da~materias,
  locale/romansh/captions/figurename     = Figura,
  locale/romansh/captions/tablename      = Tabella,
  locale/romansh/captions/partname       = Part,
  locale/romansh/captions/pagename       = pagina, 
  locale/romansh/captions/seename        = vesair,
  locale/romansh/captions/alsoname       = vesair~era,
  locale/romansh/captions/enclname       = Agiunta(s),
  locale/romansh/captions/ccname         = Copia~a,
  locale/romansh/captions/headtoname     = A,
  locale/romansh/captions/proofname      = Demonstraziun,
  locale/romansh/captions/glossaryname   = Glossari,
  locale/romansh/date ={{\ifcase\day\or1.\else ils\space\number\day\fi\space da\space 
    \ifcase\month\or
    schaner\or favrer\or mars\or avrigl\or matg\or zercladur\or
    fanadur\or avust\or settember\or october\or november\or
    december\fi\space \number\year}},
}} 
%    \end{macrocode}
% \section{Sami Languages}
% Sami languages (/ˈsɑːmi/[5]) is a group of Uralic languages spoken by the Sami people in Northern Europe (in parts of northern Finland, Norway, Sweden and extreme northwestern Russia). There are, depending on the nature and terms of division, ten or more Sami languages. Several names are used for the Sami languages: Saami, Sámi, Saame, Samic, Saamic, as well as the exonyms Lappish and Lappic. The last two, along with the term Lapp, are now often considered pejorative
%    \begin{macrocode}
\cxset{locale~samin/.style = {
  locale/samin/captions/refname        = Čujuhusat,
  locale/samin/captions/abstractname   = Čoahkkáigeassu,
  locale/samin/captions/bibname        = Girjjálašvuohta,
  locale/samin/captions/prefacename    = Ovdasátni,
  locale/samin/captions/chaptername    = Kapihttal,
  locale/samin/captions/appendixname   = Čuovus,
  locale/samin/captions/contentsname   = Sisdoallu,
  locale/samin/captions/listfigurename = Govvosat,
  locale/samin/captions/listtablename  = Tabeallat ,
  locale/samin/captions/indexname      = Registtar,
  locale/samin/captions/figurename     = Govus,
  locale/samin/captions/tablename      = Tabealla,
  locale/samin/captions/partname       = Oassi,
  locale/samin/captions/pagename       = Siidu, 
  locale/samin/captions/seename        = geahča,
  locale/samin/captions/alsoname       = geahčageahča~maiddái ,
  locale/samin/captions/enclname       = Mielddus,
  locale/samin/captions/ccname         = Kopia sáddejuvvon,   
  locale/samin/captions/headtoname     = Vuostáiváldi,
  locale/samin/captions/proofname      = Dowód ,
  locale/samin/captions/glossaryname   = Sátnelistu,
  locale/samin/date ={\ifcase\month\or
    ođđajagemánu\or
    guovvamánu\or
    njukčamánu\or
    cuoŋománu\or
    miessemánu\or
    geassemánu\or
    suoidnemánu\or
    borgemánu\or
    čakčamánu\or
    golggotmánu\or
    skábmamánu\or
    juovlamánu\fi
    \space\number\day.\space b.\space\number\year},
}} 
%    \end{macrocode}
%
% \section{Scottish}
%    \begin{macrocode}
\cxset{locale~scottish/.style = {
  locale/scottish/captions/refname = Iomraidh,
  locale/scottish/captions/abstractname   = Brìgh ,
  locale/scottish/captions/bibname        = Leabhraichean,
  locale/scottish/captions/prefacename    = Preface, % needs translation
  locale/scottish/captions/chaptername    = Caibideil,
  locale/scottish/captions/appendixname   = Ath-sgr`ıobhadh,
  locale/scottish/captions/contentsname   = Clàr-obrach,
  locale/scottish/captions/listfigurename = Liosta~Dhealbh,
  locale/scottish/captions/listtablename  = Liosta~Chlàr,
  locale/scottish/captions/indexname      = Clàr-innse,
  locale/scottish/captions/figurename     = Dealbh ,
  locale/scottish/captions/tablename      = Clàr ,
  locale/scottish/captions/partname       = Cuid,
  locale/scottish/captions/pagename       = t.d., 
  locale/scottish/captions/seename        = see, % needs translation
  locale/scottish/captions/alsoname       = ,
  locale/scottish/captions/enclname       = a-staigh,
  locale/scottish/captions/ccname         = lethbhreac~gu,
  locale/scottish/captions/headtoname     = gu,
  locale/scottish/captions/proofname      = Proof,
  locale/scottish/captions/glossaryname   = Glossary,
  locale/scottish/date ={\number\day\space \ifcase\month\or
    am~Faoilteach\or an~Gearran\or am~Màrt\or an~Giblean\or
    an~Cèitean\or an~t-Òg mhios\or an~t-Iuchar\or
    Lùnasdal\or an~Sultuine\or an~Dàmhar\or
    an~t-Samhainn\or an~Dubhlachd\fi
    \space \number\year},
}} 
%    \end{macrocode}
% \section{Serbian}
% Standard Serbian language uses both Cyrillic (ћирилица, ćirilica) and Latin script (latinica, латиница). Serbian is a rare example of synchronic digraphia, a situation where all literate members of a society have two interchangeable writing systems available to them. Media and publishers typically select one alphabet or another.
% The default is for latin script.
%    \begin{macrocode}
\cxset{locale~serbian/.style = {
  locale/serbian/captions/refname        = Bibliografija,
  locale/serbian/captions/abstractname   = Sažetak,
  locale/serbian/captions/bibname        = Literatura,
  locale/serbian/captions/prefacename    = Predgovor,
  locale/serbian/captions/chaptername    = Glava,
  locale/serbian/captions/appendixname   = Dodatak,
  locale/serbian/captions/contentsname   = Sadržaj,
  locale/serbian/captions/listfigurename = Spisak slika,
  locale/serbian/captions/listtablename  = Spisak tabela,
  locale/serbian/captions/indexname      = Registar,
  locale/serbian/captions/figurename     = Slika,
  locale/serbian/captions/tablename      = Tabela ,
  locale/serbian/captions/partname       = Deo,
  locale/serbian/captions/pagename       = Strana , 
  locale/serbian/captions/seename        = Vidi,
  locale/serbian/captions/alsoname       = Vidi takođe ,
  locale/serbian/captions/enclname       = Prilozi,
  locale/serbian/captions/ccname         = Kopije,
  locale/serbian/captions/headtoname     = Prima,
  locale/serbian/captions/proofname      = Dokaz,
  locale/serbian/captions/glossaryname   = Rečnik~nepoznatih~reči,
  locale/serbian/date ={\number\day .\space\ifcase\month\or
    januar\or februar\or mart\or april\or maj\or
    jun\or jul\or avgust\or septembar\or oktobar\or novembar\or
    decembar\fi \space \number\year.},
}} 
\cxset{locale~serbian~cyrillic/.style = {
  locale/serbian~cyrillic/captions/refname        = Библиографија,
  locale/serbian~cyrillic/captions/abstractname   = Сажетак,
  locale/serbian~cyrillic/captions/bibname        = Литература,
  locale/serbian~cyrillic/captions/prefacename    = Предговор,
  locale/serbian~cyrillic/captions/chaptername    = Глава,
  locale/serbian~cyrillic/captions/appendixname   = Додатак,
  locale/serbian~cyrillic/captions/contentsname   = Садржај,
  locale/serbian~cyrillic/captions/listfigurename = Списак~слика,
  locale/serbian~cyrillic/captions/listtablename  = Списак табела,
  locale/serbian~cyrillic/captions/indexname      = Регистар,
  locale/serbian~cyrillic/captions/figurename     = Слика,
  locale/serbian~cyrillic/captions/tablename      = Табела,
  locale/serbian~cyrillic/captions/partname       = Део,
  locale/serbian~cyrillic/captions/pagename       = Страна, 
  locale/serbian~cyrillic/captions/seename        = Види,
  locale/serbian~cyrillic/captions/alsoname       = Види такође,
  locale/serbian~cyrillic/captions/enclname       = Прилози,
  locale/serbian~cyrillic/captions/ccname         = Копије,
  locale/serbian~cyrillic/captions/headtoname     = Прима,
  locale/serbian~cyrillic/captions/proofname      = Доказ,
  locale/serbian~cyrillic/captions/glossaryname   = Речник непознатих речи,
  locale/serbian~cyrillic/date ={\number\day .\space\ifcase\month\or
    јануар\or фебруар\or март\or април\or мај\or
    јун\or јул\or август\or септембар\or октобар\or новембар\or
    децембар\fi \space \number\year.},
}} 
%    \end{macrocode}
% \section{Slovak}
%    \begin{macrocode}
\cxset{locale~slovak/.style = {
  locale/slovak/captions/refname        = Referencie,
  locale/slovak/captions/abstractname   = Abstrakt,
  locale/slovak/captions/bibname        = Literatúra,
  locale/slovak/captions/prefacename    = Úvod,
  locale/slovak/captions/chaptername    = Kapitola,
  locale/slovak/captions/appendixname   = Dodatok,
  locale/slovak/captions/contentsname   = Obsah,
  locale/slovak/captions/listfigurename = Zoznam~obrázkov ,
  locale/slovak/captions/listtablename  = Zoznam~tabuliek,
  locale/slovak/captions/indexname      = Index,
  locale/slovak/captions/figurename     = Obrázok,
  locale/slovak/captions/tablename      = Tabuľka,
  locale/slovak/captions/partname       = Časť,
  locale/slovak/captions/pagename       = Strana, 
  locale/slovak/captions/seename        = viď,
  locale/slovak/captions/alsoname       = viď~tiež,
  locale/slovak/captions/enclname       = Prílohy,
  locale/slovak/captions/ccname         = cc.,
  locale/slovak/captions/headtoname     = Pre,
  locale/slovak/captions/proofname      = Dôkaz,
  locale/slovak/captions/glossaryname   = Slovník,
  locale/slovak/date ={\number\day.~\ifcase\month\or
    januára\or februára\or marca\or apríla\or mája\or
    júna\or júla\or augusta\or septembra\or októbra\or
    novembra\or decembra\fi
    \space \number\year},
}} 
%    \end{macrocode}
% \section{Slovenian}
%    \begin{macrocode}
\cxset{locale~slovenian/.style = {
  locale/slovenian/captions/refname        = Literatura,
  locale/slovenian/captions/abstractname   = Povzetek,
  locale/slovenian/captions/bibname        = Literatura,
  locale/slovenian/captions/prefacename    = Predgovor,
  locale/slovenian/captions/chaptername    = Poglavje,
  locale/slovenian/captions/appendixname   = Dodatek,
  locale/slovenian/captions/contentsname   = Kazalo,
  locale/slovenian/captions/listfigurename = Slike,
  locale/slovenian/captions/listtablename  = Tabele,
  locale/slovenian/captions/indexname      = Stvarno~kazalo,
  locale/slovenian/captions/figurename     = Slika,
  locale/slovenian/captions/tablename      = Tabela,
  locale/slovenian/captions/partname       = Del,
  locale/slovenian/captions/pagename       = Stran, 
  locale/slovenian/captions/seename        = glej,
  locale/slovenian/captions/alsoname       = glej~tudi,
  locale/slovenian/captions/enclname       = Priloge,
  locale/slovenian/captions/ccname         = Kopije,
  locale/slovenian/captions/headtoname     = Prejme,
  locale/slovenian/captions/proofname      = Dokaz,
  locale/slovenian/captions/glossaryname   = Slovar,
  locale/slovenian/date ={\number\day.~\ifcase\month\or
    januar\or februar\or marec\or april\or maj\or junij\or
    julij\or avgust\or september\or oktober\or november\or december\fi
    \space \number\year},
}} 
% Alphabet consists of 25 lower and 25 upper letters
\cs_new:Npn \int_to_Alph_slovenian:n #1
 {
 \int_to_symbols:nnn {#1} { 25 }
 {
  { 1 } { A }
  { 2 } { B }
  { 3 } { C }
  { 4 } { Č }
  { 5 } { D }
  { 6 } { E }
  { 7 } { F }
  { 8 } { G }
  { 9 } { H }
  { 10 } { I }
  { 11 } { J }
  { 12 } { K }
  { 13 } { L }
  { 14 } { M }
  { 15 } { N }
  { 16 } { O }
  { 17 } { P }
  { 18 } { R }
  { 19 } { S }
  { 20 } { Š }
  { 21 } { T }
  { 22 } { U }
  { 23 } { V }
  { 24 } { Z }
  { 25 } { Ž }
 }
}

\cs_new:Npn \int_to_alph_slovenian:n #1
 {
 \int_to_symbols:nnn {#1} { 25 }
 {
  { 1 } { a }
  { 2 } { b }
  { 3 } { c }
  { 4 } { č }
  { 5 } { d }
  { 6 } { e }
  { 7 } { f }
  { 8 } { g }
  { 9 } { h }
  { 10 } { i }
  { 11 } { j }
  { 12 } { k }
  { 13 } { l }
  { 14 } { m }
  { 15 } { n }
  { 16 } { o }
  { 17 } { p }
  { 18 } { r }
  { 19 } { s }
  { 20 } { š }
  { 21 } { t }
  { 22 } { u }
  { 23 } { v }
  { 24 } { z }
  { 25 } { ž }
 }
}
%    \end{macrocode}
% \section{Spanish}
%
%    \begin{macrocode}
\cxset{locale~spanish/.style = {
  locale/spanish/captions/refname        = Referencias,
  locale/spanish/captions/abstractname   = Resumen,
  locale/spanish/captions/bibname        = Bibliografía,
  locale/spanish/captions/prefacename    = Prefacio,
  locale/spanish/captions/chaptername    = Capítulo,
  locale/spanish/captions/appendixname   = Apéndice,
  locale/spanish/captions/contentsname   = Índice~general,
  locale/spanish/captions/listfigurename = Índice~de~figuras,
  locale/spanish/captions/listtablename  = Índice~de~cuadros,
  locale/spanish/captions/indexname      = Índice~alfabético,
  locale/spanish/captions/figurename     = Figura,
  locale/spanish/captions/tablename      = Cuadro,
  locale/spanish/captions/partname       = Parte,
  locale/spanish/captions/pagename       = Página, 
  locale/spanish/captions/seename        = véase,
  locale/spanish/captions/alsoname       = véase~también,
  locale/spanish/captions/enclname       = Adjunto(s),
  locale/spanish/captions/ccname         = Copia~a,
  locale/spanish/captions/headtoname     = A,
  locale/spanish/captions/proofname      = Prueba,
  locale/spanish/captions/glossaryname   = Glosario,
  locale/spanish/date ={\number\day~de~\ifcase\month\or
    enero\or febrero\or marzo\or abril\or mayo\or junio\or
    julio\or agosto\or septiembre\or octubre\or noviembre\or
    diciembre\fi\space de~\number\year},
}} 
%    \end{macrocode}
% \section{Swedish}
%    \begin{macrocode}
\cxset{locale~swedish/.style = {
  locale/swedish/captions/refname        = Referenser,
  locale/swedish/captions/abstractname   = Sammanfattning,
  locale/swedish/captions/bibname        = Litteraturförteckning,
  locale/swedish/captions/prefacename    = Förord,
  locale/swedish/captions/chaptername    = Kapitel,
  locale/swedish/captions/appendixname   = Bilaga,
  locale/swedish/captions/contentsname   = Innehåll,
  locale/swedish/captions/listfigurename = Figurer,
  locale/swedish/captions/listtablename  = Tabeller,
  locale/swedish/captions/indexname      = Sakregister,
  locale/swedish/captions/figurename     = Figur,
  locale/swedish/captions/tablename      = Tabell,
  locale/swedish/captions/partname       = Del,
  locale/swedish/captions/pagename       = Sida, 
  locale/swedish/captions/seename        = se,
  locale/swedish/captions/alsoname       = se~även,
  locale/swedish/captions/enclname       = Bil.,
  locale/swedish/captions/ccname         = Kopia~för~kännedom,
  locale/swedish/captions/headtoname     = Till,
  locale/swedish/captions/proofname      = Bevis,
  locale/swedish/captions/glossaryname   = Ordlista,
  locale/swedish/date ={\number\day\space\ifcase\month\or
    januari\or februari\or mars\or april\or maj\or juni\or
    juli\or augusti\or september\or oktober\or november\or
    december\fi
    \space\number\year},
}} 
%    \end{macrocode}
% \CaptionsList{swedish}
%
% \section{Tamil}
% 
%    \begin{macrocode}
\cxset{locale~tamil/.style = {
  locale/tamil/captions/refname = சாராம்சம்,
  locale/tamil/captions/abstractname = பிற்சேர்க்கை,
  locale/tamil/captions/bibname       = ,
  locale/tamil/captions/prefacename   = ,
  locale/tamil/captions/chaptername   = அத்தியாயம்,
  locale/tamil/captions/appendixname  = ,
  locale/tamil/captions/contentsname  = உள்ளே,
  locale/tamil/captions/listfigurename = படங்களின்~பட்டியல்,
  locale/tamil/captions/listtablename = அட்டவணை~பட்டியல்,
  locale/tamil/captions/indexname = சுட்டி,
  locale/tamil/captions/figurename  = படம்,
  locale/tamil/captions/tablename = அட்டவணை,
  locale/tamil/captions/partname = ,
  locale/tamil/captions/pagename = , 
  locale/tamil/captions/seename  = பார்க்க,
  locale/tamil/captions/alsoname = ,
  locale/tamil/captions/enclname = {},
  locale/tamil/captions/ccname = {},
  locale/tamil/captions/headtoname ={} ,
  locale/tamil/captions/proofname = ,
  locale/tamil/captions/glossaryname = ,
  locale/tamil/date ={\number\year\space\ifcase\month\or
     ஜனவரி\or
     பிப்ரவரி\or
     மார்ச்\or
    ஏப்ரல்\or
     மே\or
     ஜூன்\or
     ஜூலை\or
    ஆகஸ்ட்\or
     செப்டம்பர்\or
     அக்டோபர்\or
     நவம்பர்\or
     டிசம்பர்\fi
     \space\number\day},
}} 
%    \end{macrocode}
% \CaptionsList{tamil}
% \section{Telugu}
%    \begin{macrocode}
\cxset{locale~telugu/.style = {
  locale/telugu/captions/refname = ఆధారాలు,
  locale/telugu/captions/abstractname = సారాంశం,
  locale/telugu/captions/bibname       = గ్రంథాల జాబితా,
  locale/telugu/captions/prefacename   = ముందుమాట,
  locale/telugu/captions/chaptername   = అధ్యాయము,
  locale/telugu/captions/appendixname  = అదనంగా,
  locale/telugu/captions/contentsname  = విషయాలు,
  locale/telugu/captions/listfigurename =ఆకృతుల జాబితా ,
  locale/telugu/captions/listtablename = పట్టికల జాబితా,
  locale/telugu/captions/indexname = విషయ సూచిక,
  locale/telugu/captions/figurename  = ఆకృతి,
  locale/telugu/captions/tablename = పట్టిక,
  locale/telugu/captions/partname = భాగం,
  locale/telugu/captions/pagename = పేజి, 
  locale/telugu/captions/seename  = చూడండి,
  locale/telugu/captions/alsoname = కూడా చూడండి,
  locale/telugu/captions/enclname = ఎంక్లోజర్*,
  locale/telugu/captions/ccname = సిసి,
  locale/telugu/captions/headtoname = కి,
  locale/telugu/captions/proofname = రుజువు,
  locale/telugu/captions/glossaryname = నిఘంటువు,
  locale/telugu/date ={{\panunicode 
  \def\telugu@month{
    \ifcase\month\or
         జనవరి\or
         ఫిబ్రవరి\or
         మార్చ్\or
         ఏప్రిల్\or
         మే\or
         జూన్\or
         జూలై\or
         ఆగస్ట్\or
         సెప్టెంబర్\or
         అక్తోబెర్\or
         నవంబర్\or
         డిసంబర్\fi}
   \telugu@month\space\number\day,\space\number\year}},
}}
%    \end{macrocode}
% \CaptionsList{telugu}
% \section{Turkish}
%    \begin{macrocode}
\cxset{locale~turkish/.style = {
  locale/turkish/captions/refname        = Kaynaklar,
  locale/turkish/captions/abstractname   = Özet,
  locale/turkish/captions/bibname        = Kaynakça,
  locale/turkish/captions/prefacename    = Önsöz,
  locale/turkish/captions/chaptername    = Bölüm,
  locale/turkish/captions/appendixname   = Ek,
  locale/turkish/captions/contentsname   = İçindekiler,
  locale/turkish/captions/listfigurename = Şekil Listesi,
  locale/turkish/captions/listtablename  = Tablo Listesi,
  locale/turkish/captions/indexname      = Dizin,
  locale/turkish/captions/figurename     = Şekil,
  locale/turkish/captions/tablename      = Tablo,
  locale/turkish/captions/partname       = Kısım,
  locale/turkish/captions/pagename       = Sayfa, 
  locale/turkish/captions/seename        = bkz.,
  locale/turkish/captions/alsoname       = ayrıca~bkz.,
  locale/turkish/captions/enclname       = İlişik,
  locale/turkish/captions/ccname         = Diğer~Alıcılar,
  locale/turkish/captions/headtoname     = Alıcı,
  locale/turkish/captions/proofname      = Kanıt,
  locale/turkish/captions/glossaryname   = Lügatçe,
  locale/turkish/date ={\number\day\space\ifcase\month\or
    Ocak\or Şubat\or Mart\or Nisan\or Mayıs\or Haziran\or
    Temmuz\or Ağustos\or Eylül\or Ekim\or Kasım\or
    Aralık\fi
    \space\number\year},
}}
%    \end{macrocode}
% \section{Turkmen}
%    \begin{macrocode}
\cxset{locale~turkmen/.style = {
  locale/turkmen/captions/refname        = Çeşmeler,
  locale/turkmen/captions/abstractname   = Gysgaça~manysy,
  locale/turkmen/captions/bibname        = Çeşmeler,
  locale/turkmen/captions/prefacename    = Sözbaşy,
  locale/turkmen/captions/chaptername    = Bap,
  locale/turkmen/captions/appendixname   = Goşmaça,
  locale/turkmen/captions/contentsname   = Mazmuny,
  locale/turkmen/captions/listfigurename = Suratlaryň~sanawy,
  locale/turkmen/captions/listtablename  = Tablisalaryň~sanawy,
  locale/turkmen/captions/indexname      = Indeks,
  locale/turkmen/captions/figurename     = Surat,
  locale/turkmen/captions/tablename      = Tablisa,
  locale/turkmen/captions/partname       = Bölüm,
  locale/turkmen/captions/pagename       = Sahypa, 
  locale/turkmen/captions/seename        = ser.,
  locale/turkmen/captions/alsoname       = şuňa-da~ser.,
  locale/turkmen/captions/enclname       = Goşmaça,
  locale/turkmen/captions/ccname         = Iberilenler,
  locale/turkmen/captions/headtoname     = Kime,
  locale/turkmen/captions/proofname      = Delil,
  locale/turkmen/captions/glossaryname   = Sözlük,
  locale/turkmen/date = {\number\day\space\ifcase\month\or
    Ýanwar\or Fewral\or Mart\or Aprel\or Maý\or Iýun\or
    Iýul\or Awgust\or Sentýabr\or Oktýabr\or Noýabr\or
    Dekabr\fi
    \space\number\year},
}}
%    \end{macrocode}
% \CaptionsList{turkmen}
% \section{Ukrainian}
% Ukrainian	uk	uk	1058	422	1251
%    \begin{macrocode}
\cxset{locale~ukrainian/.style = {
  locale/ukrainian/captions/refname        = Література,
  locale/ukrainian/captions/abstractname   = Анотація,
  locale/ukrainian/captions/bibname        = Бібліоґрафія,
  locale/ukrainian/captions/prefacename    = Вступ,
  locale/ukrainian/captions/chaptername    = Розділ,
  locale/ukrainian/captions/appendixname   = Додаток,
  locale/ukrainian/captions/contentsname   = Зміст,
  locale/ukrainian/captions/listfigurename = Перелік~ілюстрацій,
  locale/ukrainian/captions/listtablename  = Перелік~таблиць,
  locale/ukrainian/captions/indexname      = Покажчик,
  locale/ukrainian/captions/figurename     = Рис.,
  locale/ukrainian/captions/tablename      = Табл.,
  locale/ukrainian/captions/partname       = Частина,
  locale/ukrainian/captions/pagename       = с., 
  locale/ukrainian/captions/seename        = див.,
  locale/ukrainian/captions/alsoname       = див.\ ~також,
  locale/ukrainian/captions/enclname       = вкладка,
  locale/ukrainian/captions/ccname         = копія,
  locale/ukrainian/captions/headtoname     = До,
  locale/ukrainian/captions/proofname      = Доведення,
  locale/ukrainian/captions/glossaryname   = Словник~термінів,
  locale/ukrainian/date ={\number\day\space\ifcase\month\or
    січня\or
    лютого\or
    березня\or
    квітня\or
    травня\or
    червня\or
    липня\or
    серпня\or
    вересня\or
    жовтня\or
    листопада\or
    грудня\fi
    \space\number\year\space р.},
}}
%    \end{macrocode}
%
% \section{Usorbian}
% \label{sec:usorbian}
%
% The Sorbian languages (Upper Sorbian: Serbska rěč, Lower Sorbian: Serbska rěc) are two closely related, but only partially mutually intelligible, West Slavic languages spoken by the Sorbs, a West Slavic minority in the Lusatia region of eastern Germany. They are classified under the West Slavic branch of the Indo-European languages and are therefore closely related to the other two West Slavic subgroups: Lechitic and Czech–Slovak.\footnote{Online article by Hermut Feska in:\href{https://web.archive.org/web/20120212191519/http://www.uni-leipzig.de/~sorb/seiten/eng/03/language.html}{About Sorbian Language}.}
%Historically the languages have also been known as Wendish (named after the Wends, earliest Slavic people in modern Poland and Germany) or Lusatian. Their collective ISO 639-2 code is |wen|.
%
%There are two literary languages: Upper Sorbian (hornjoserbsce), spoken by about 40,000 people in Saxony, and Lower Sorbian (dolnoserbski) spoken by about 10,000 people in Brandenburg. The area where the two languages are spoken is known as Lusatia (Łužica in Upper Sorbian, Łužyca in Lower Sorbian, or Lausitz in German).
%
%
%    \begin{macrocode}
\cxset{locale~usorbian/.style = {
  locale/usorbian/captions/refname        = Referency ,
  locale/usorbian/captions/abstractname   = Abstrakt,
  locale/usorbian/captions/bibname        = Literatura ,
  locale/usorbian/captions/prefacename    = Zawod,
  locale/usorbian/captions/chaptername    = Kapitl,
  locale/usorbian/captions/appendixname   = Dodawki,
  locale/usorbian/captions/contentsname   = Wobsah,
  locale/usorbian/captions/listfigurename = Zapis~wobrazow,
  locale/usorbian/captions/listtablename  = Zapis~tabulkow,
  locale/usorbian/captions/indexname      = Indeks,
  locale/usorbian/captions/figurename     = Wobraz,
  locale/usorbian/captions/tablename      = Tabulka,
  locale/usorbian/captions/partname       = Dźěl,
  locale/usorbian/captions/pagename       = Strona, 
  locale/usorbian/captions/seename        = hl.,
  locale/usorbian/captions/alsoname       = hl.~tež,
  locale/usorbian/captions/enclname       = Přłoha,
  locale/usorbian/captions/ccname         = CC,
  locale/usorbian/captions/headtoname     = Komu ,
  locale/usorbian/captions/proofname      = Proof,
  locale/usorbian/captions/glossaryname   = Glossary,
  locale/usorbian/date                    = 
    {
      \number\day.\space
      \ifcase\month
        \or januara
        \or februara
        \or měrca
        \or apryla
        \or meje
        \or junija
        \or julija
        \or awgusta
        \or septembra
        \or oktobra
        \or nowembra
        \or decembra
      \fi
      \space\number\year
    },
}}
\cxset{locale~Usorbian/.alias = locale~usorbian}
%    \end{macrocode}
% \CaptionsList{Usorbian}
%
% \section{hangul}
%    \begin{macrocode}

\cxset{locale~hangul/.style = {
  locale/hangul/captions/refname        = Tài~liệu,
  locale/hangul/captions/abstractname   = Tóm~tắt~nội~dung,
  locale/hangul/captions/bibname        = Tài~liệu~tham~khảo,
  locale/hangul/captions/prefacename    = Lời~nói~đầu,
  locale/hangul/captions/chaptername    = Chương,
  locale/hangul/captions/appendixname   = Phụ~lục,
  locale/hangul/captions/contentsname   = Mục~lục,
  locale/hangul/captions/listfigurename = Danh~sách~hình~vẽ ,
  locale/hangul/captions/listtablename  = Danh~sách~bảng,
  locale/hangul/captions/indexname      = Chỉ~mục,
  locale/hangul/captions/figurename     = Hình,
  locale/hangul/captions/tablename      = Bảng,
  locale/hangul/captions/partname       = Phần ,
  locale/hangul/captions/pagename       = Trang, 
  locale/hangul/captions/seename        = Xem ,
  locale/hangul/captions/alsoname       = Xem~thêm,
  locale/hangul/captions/enclname       = Kèm~theo,
  locale/hangul/captions/ccname         = Cùng~gửi,
  locale/hangul/captions/headtoname     = Gửi,
  locale/hangul/captions/proofname      = Chứng~minh,
  locale/hangul/captions/glossaryname   = Từ~điển~chú~giải,
  locale/hangul/date ={Ngày\space\number\day\space
    tháng\space\number\month\space
    năm\space\number\year},
}}
\cxset{locale~hangul/.alias = locale~hangul}
%    \end{macrocode}
%
% \CaptionsList{hangul}
%
% \section{Welsh}
% for dates and times see \href{http://beta.swansea.gov.uk/media/8594/Welsh-language-service-handy-guide---Welsh-dates-and-times/pdf/Welsh_Language_Service_Handy_Guide_-_Welsh_dates_and_times.pdf}{swansea}
%    \begin{macrocode}
\newif\ifwelsh@first

\def\welsh@article#1{
  \welsh@firsttrue 
  y
  \expandafter\welsh@article@do#1
 }

\def\welsh@article@do#1{
  \ifwelsh@first\welsh@isvowel#1
  \ifwelsh@vowel 
     r\space
     \welsh@vowelfalse
   \else
   \space
   \fi#1
   \welsh@firstfalse
   \fi
 }

\newif\ifwelsh@vowel
\def\welsh@isvowel#1{
   \ifx#1a\welsh@voweltrue
   \else
   \ifx#1u
    \welsh@voweltrue
    \else
    \ifx#1w\welsh@voweltrue
    \fi\fi\fi}% FIXME Add the other vowels, just for good measure

\def\welsh@ordinal@long#1{%
  \if_case:w #1
    \or: cyntaf
    \or: ail
    \or: trydydd
    \or: pedwerydd
    \or: pumed
    \or: chweched
    \or: seithfed
    \or: wythfed
    \or: nawfed
    \or: degfed
    \or: unfed~ar~ddeg
    \or: deuddegfed
    \or: trydydd~ar~ddeg
    \or: pedwerydd~ar~ddeg
    \or: pymthegfed
    \or: unfed~ar~bymtheg
    \or: ail~ar~bymtheg
    \or: deunawfed
    \or: pedwerydd~ar~bymtheg
    \or: ugeinfed
    \else:
     \exp_after:wN \welsh@ordinalplusxx@long#1
    \fi:
  }

\def\welsh@ordinalplusxx@long#1{%
  \let\dday=#1\advance\dday~by~-20\relax\welsh@ordinal@long\dday\space ar~hugain%
}

\cxset{welsh~date~format/.is~choice,
       welsh~date~format/standard/.code = 
       \def\today{\expandafter\welsh@article\welsh@ordinal@long\day\space o\space
       \if_case:w \month
       \or: Ionawr
       \or: Chwefror
       \or: Fawrth
       \or: Ebrill
       \or: Fai
       \or: Fehefin
       \or: Orffenaf
       \or: Awst
       \or: Fedi
       \or: Hydref
       \or: Dachwedd
       \or: Ragfyr
      \fi:
      \space\number\year},
     welsh~date~format/formal/.code    = \def\today{\ifcase\day\or 1af\or 2ail\or 3ydd\or 4ydd\or 5ed\or 6ed%
    \or 7fed\or 8fed\or 9fed\or 10fed\or 11eg\or 12fed\or 13eg\or
    14eg\or 15fed\or 16eg\or 17eg\or 18fed\or 19eg\or
    20fed\else\number\day ain\fi\space\ifcase\month\or
    Ionawr\or Chwefror\or Mawrth\or Ebrill\or
    Mai\or Mehefin\or Gorffennaf\or Awst\or
    Medi\or Hydref\or Tachwedd\or Rhagfyr\fi%
    \space\number\year},
    welsh~date~format=formal,
  }
\cxset{locale~welsh/.style = {
  locale/welsh/captions/refname        = Cyfeiriadau,
  locale/welsh/captions/abstractname   = Crynodeb,
  locale/welsh/captions/bibname        = Llyfryddiaeth,
  locale/welsh/captions/prefacename    = Rhagair,
  locale/welsh/captions/chaptername    = Pennod,
  locale/welsh/captions/appendixname   = Atodiad,
  locale/welsh/captions/contentsname   = Cynnwys,
  locale/welsh/captions/listfigurename = Rhestr~Ddarluniau,
  locale/welsh/captions/listtablename  = Rhestr~Dablau,
  locale/welsh/captions/indexname      = Mynegai,
  locale/welsh/captions/figurename     = Darlun,
  locale/welsh/captions/tablename      = Taflen,
  locale/welsh/captions/partname       = Rhan,
  locale/welsh/captions/pagename       = tudalen, 
  locale/welsh/captions/seename        = gweler,
  locale/welsh/captions/alsoname       = gweler~hefyd,
  locale/welsh/captions/enclname       = amgaeëdig,
  locale/welsh/captions/ccname         = copïau,
  locale/welsh/captions/headtoname     = At,
  locale/welsh/captions/proofname      = Prawf,
  locale/welsh/captions/glossaryname   = Rhestr~termau,
  welsh~date~format=standard,
}}
\cxset{locale~Welsh/.alias=locale~welsh}
%    \end{macrocode}
% \section{Hangul}
%    \begin{macrocode}
\cxset{locale~hangul/.style = {
  locale/hangul/captions/refname        = ,
  locale/hangul/captions/abstractname   = ,
  locale/hangul/captions/bibname        = ,
  locale/hangul/captions/prefacename    = ,
  locale/hangul/captions/chaptername    = ,
  locale/hangul/captions/appendixname   = ,
  locale/hangul/captions/contentsname   = 차~례,
  locale/hangul/captions/listfigurename = 그림~차례,
  locale/hangul/captions/listtablename  = 표~차례,
  locale/hangul/captions/indexname      = 찾아보기,
  locale/hangul/captions/figurename     = ,
  locale/hangul/captions/tablename      = 표,
  locale/hangul/captions/partname       = ,
  locale/hangul/captions/pagename       = , 
  locale/hangul/captions/seename        = ,
  locale/hangul/captions/alsoname       = ,
  locale/hangul/captions/enclname       = ,
  locale/hangul/captions/ccname         = ,
  locale/hangul/captions/headtoname     = ,
  locale/hangul/captions/proofname      = ,
  locale/hangul/captions/glossaryname   = ,
  locale/hangul/date ={Ngày\space\number\day\space
    tháng\space\number\month\space
    năm\space\number\year},
}}
%    \end{macrocode}
%  \CaptionsList{hangul}    
%
% \CaptionsList{Welsh}
% The Captions List is helpful for presenting the captions in
% various languages. Inspired from MonTeX!!!
%
%    \begin{macrocode}    
\newcommand{\CaptionsList}[1]{%
   \bgroup
   \cxset{locale~#1}
   %\ttfamily
   \panunicode%
	\begin{longtable}[]{lll}
	\caption{Captions~in~#1\label{#1captions}}\\
	\toprule
   Command               & English	       & #1 \\
	 \midrule
	 \cs{prefacename}	     & Preface	       & \prefacename\\
	 \cs{refname}		       & References	     & \refname\\
	 \cs{abstractname}	   & Abstract    	   & \abstractname\\
	 \cs{bibname}		       & Bibliography	   & \bibname\\
	 \cs{chaptername}	     & Chapter	       & \chaptername\\
	 \cs{appendixname}	   & Appendix	       & \appendixname\\
	 \cs{contentsname}	   & Contents	       & \contentsname\\
	 \cs{listfigurename}   & List~of~Figures & \listfigurename\\
	 \cs{listtablename}	   & List~of~Tables  & \listtablename\\
	 \cs{indexname}	       & Index           & \indexname\\
	 \cs{figurename}	     & Figure	         & \figurename\\
	 \cs{tablename}	       & Table		       & \tablename\\
	 \cs{partname}	       & Part		         & \partname\\
	 \cs{enclname}	       & encl		         & \enclname\\
	 \cs{ccname}		       & cc		           & \ccname\\
	 \cs{headtoname}	     & To		           & \headtoname\\
	 \cs{pagename}	       & Page		         & \pagename\\
	 \cs{seename}		       & see		         & \seename\\
   \cs{alsoname}	       & see~also	       & \alsoname\\
	 \cs{glossaryname}     & Glossary        & \glossaryname\\
	 \cs{today}            &                 & \today \\
	 %\cs{dateitalian}     &                 & \dateitalian\\
	 \bottomrule
	 \end{longtable}
	%\captionsenglish
	\egroup
}

\clist_set:cn {en_clist} {, January, February, March, April, May, 
   June, July, August, September, October, November, December}

\ExplSyntaxOff
%    \end{macrocode}
% 
% \ExplSyntaxOn
% \def\englishmonth#1#2{\clist_item:cn {#1_clist} {#2}}
% \englishmonth{en}{\month}
% \ExplSyntaxOff
%\iffalse
%</package>
%\fi
%
%
% \printindex
% \Finale
\endinput



