\documentclass[book]{phddoc}
\usepackage{expl3}
\usepackage{fontspec}
\newfontfamily\panunicode{code2000.ttf}
\usepackage{fancyvrb,xfancyvrb,marginnote,xcolor}

\begin{document}
%\makeatletter
%\color{green700}\fbox{aaaa}\current@color tt
%\makeatother

\meaning\normalcolor


\ExplSyntaxOn
\tl_new:N \g_store_tl

\cs_new:Npn \make_at_letter:
 {
  \char_set_catcode_letter:N @
 }
  
\cs_new:Npn \make_at_other:
 {
  \char_set_catcode_letter:N @
 }  
 
\cs_new:Npn \make_other:n #1 
  {
 		\char_set_catcode_other:N #1 \relax
  } 
  
\make_at_letter:
\begingroup
\char_set_catcode_escape:N | 
\char_set_catcode_group_begin:N [
\char_set_catcode_group_end:N ] 
\char_set_catcode_other:N\{
\char_set_catcode_other:N\}
\char_set_catcode_other:N\\

|cs_gset:Npn|phd_xverbatim:n#1\end{zverbatim}
            [|process_argument[#1]|end[zverbatim]]
|endgroup


\cs_set:Npn \zverbatim_aux: 
   {
     \bfseries
     \let\do\make_other:n\dospecials
     \obeylines \verbatimfont \@noligs
     \hyphenchar\font\m@ne
   }

% just set some preliminaries with
% |\@zverbatim|
\cs_gset:Npn \zverbatim
   {
     \zverbatim_aux: 
     \phd_xverbatim:n
   }

% Cannot do anything more at this time
\cs_gset:Npn \endzverbatim {}

% Process the argument
\regex_const:Nn \c_braces_regexs { (\{|\})+ }
\regex_const:Nn \c_comments_regexs 
  {  
    (\%\ *\%*[ a-zA-Z\ \\\*\_\:\;\.\!\"\`\'\#\&\-\^\(\)\|\d\=\+\,\@\}\{\[\]\/ ]*) }
  

\def\colorbraces#1{\fbox{\colorbr{#1}}} 
 \edef\colorbr{\noexpand\color{green800}}
 
\gdef\process_argument #1 {
     \makeatletter
    \tl_set:Nn\g_store_tl{#1}
     % replace braces and put them in a box

	
    \regex_replace_all:NnN \c_braces_regexs
     { \c{colorbraces}\cB{\1\cE}\c{color}\cB{.\cE}  }\g_store_tl   
    
    \meaning\g_store_tl\\
    
    \def\colorgreen{\color{green800}}
    
    \regex_replace_all:NnN \c_comments_regexs 
      {\c{colorgreen}\1\c{normalcolor}}
   \g_store_tl
   
    \meaning\g_store_tl\\
   
  
  
       
%     \meaning\Bg_store_tl\\  
   
  

   
      
%   \meaning\g_store_tl
%    
%   \regex_replace_all:nnN {\c{color}\cB{blue\cE}} 
%      {\c{fbox}\cB{\cE}}\g_store_tl
%   

   
  
   \tl_use:N \g_store_tl
}  

\make_at_other:

\ExplSyntaxOff


% Output to test that everything is # working
\begin{zverbatim}
% Comments only \def\test{test}%
 \def{\our} 
\end{zverbatim}

\end{document}

\makeatletter

The |\dospecials| macro is conventionally used by verbatim macros to alter the catcodes of the currently active
characters. This was originally defined by Knuth in plainTeX and carried over to LaTeX by Lamport in
ltxplain.dtx in source2e.


\begin{Verbatim}
% 11 chars
\def\dospecials{\do\ \do\\\do\{\do\}\do\$\do\&\do\#\do\^\do\_\do\%\do\~}
\end{Verbatim}

\MakeShortVerb{\*}

\MakeShortVerb{\"}

\meaning\dospecials\\

\add@special{\;}

\ExplSyntaxOn
\seq_set_split:Nnn {\do}\dospecials
\meaning\dospecials

Two sequences for dealing with special characters. The first is characters which may be
active, the second longer list is for “special” characters more generally. Both lists are
escaped so that for example bulk code assignments can be carried out. In both cases, the
order is by ascii character code (as is done in for example |\ExplSyntaxOn|).

Used to track which tokens will require special handling when working with verbatimlike
material at the document level as they are not of categories hletteri (catcode 11) or
hotheri (catcode 12). Each entry in the sequence consists of a single escaped token, for
example |\\| for the backslash or |\{| for an opening brace. Escaped tokens should be added
to the sequence when they are defined for general document use.

\begin{Verbatim}
 \seq_new:N \l_char_special_seq
 
 \seq_set_split:Nnn \l_char_special_seq { }
   { \ \" \# \$ \% \& \\ \^ \_ \{ \} \~ }

 \seq_new:N \l_char_active_seq
   \seq_set_split:Nnn \l_char_active_seq { }
   { \" \$ \& \^ \_ \~ }
\end{Verbatim}

\ExplSyntaxOn
\seq_map_inline:Nn \l_char_special_seq{\string #1}\\

\seq_pop_right:NN \l_char_special_seq\l_tempa_seq

\seq_map_inline:Nn \l_char_special_seq{\string #1}


\end{document}
\def\exmpl{\hrule\vspace*{10pt}}
\lorem

\begin{Verbatim}[curlyquotes=false,highlightlines={1,3-4},numbers=left,mathescape=true,fontfamily=panunicode,formatcom={\exmpl}]
   "quoted text"
   first line
   second line
   $a=b^2+3$
   third line
   fourth line
\end{Verbatim}

\end{document}

%\ExplSyntaxOn
%\makeatletter
%\ttfamily
%
%\the\fontdimen2\font
%\advance\@tempdimb-\fv_verb_tab_size_tl~sp 
%
%\fv_verb_tab_size_tl\\
%
%
%\FancyVerbSpace
%
%\ExplSyntaxOff

\fboxsep=15pt

\begin{Verbatim*}[frame=lines,gobble=1,fontseries=b,fontfamily=tt,label={[Start of code]End of code},labelposition=all,showspaces=true,firstline=3,lastline=3,tabsize=3,vspace=30pt]
 Line 1 \test test
 Line 2 \test
 Line 3 \test	\test	test		test		test
 Line 4 \test 	\test
 Line 5 \test
 Line 6	\test
\end{Verbatim*}

\begin{Verbatim*}
\test \test
\end{Verbatim*}


\def\b{something}

\ExplSyntaxOn

\cs_set:Npn \testmacro #1{
\cs_set:Npx \a{something}
  \if_meaning:w \a\b
    TRUE
  \else:
    FALSE
  \fi:  
  
}

\testmacro{something}


\begin{minipage}{0.3\textwidth}
\cs_set:Npn \marginverb#1
  {
    \SaveVerb{danger}=Real #1=
    \footnote{\UseVerb*{danger}}
  }
\marginverb{test}
\end{minipage}

\ExplSyntaxOff

%% Check baseline stretch for auto
\newenvironment{myexample}
{\VerbatimEnvironment\begin{VerbatimOut}{test.vrb}} 
{\end{VerbatimOut}

\VerbatimInput[numbers=left,firstnumber=2,firstline=2,baselinestretch=1,showspaces,frame=single,label=\fbox{Example},labelposition=all,numbersep=2pt]{test.vrb}
 function test(N)
  return N
end
tex.print(test(5))

} 

\begin{myexample}
first line
second line
third line
fourth line
\end{myexample}

\NewVerbatimEnvironment{ExVerbatim}{Verbatim}{numbers=none}

\makeatletter
\begin{myexample}
foo\newline\mbox{%
\put(7,2){%
\circle*{\strip@pt\normalbaselineskip}}}%
\newline
bar
\end{myexample}
\makeatother

\newcommand\Example{%
\VerbatimEnvironment
\begin{VerbatimOut}{\jobname.vrb}}

\def\endExample{%
\end{VerbatimOut}
{\centering \input{\jobname.vrb}}
\VerbatimInput{\jobname.vrb}
\input{\jobname.vrb}
}




%%%%%%%%%%%%%%%%%%%%%%%%%%%%%%
   Define Active
%%%%%%%%%%%%%%%%%%%%%%%%%%%%%%


\begin{Verbatim}[frame=single,numbers=left,numbersep=3pt]
! A small "hello" program

program hello
  print *, "Hello World"
\end{Verbatim}

\fvset{frame=single,numbers=left,numbersep=3pt,curlyquotes}
\begin{Verbatim}
! A small "hello" program

program hello
  print *, `Hello World'
\end{Verbatim}

%
%\def\ExclamationPoint{\char33}
% \catcode`!=\active
% 
%\begin{Verbatim}[defineactive=\def!{\color{cyan}\itshape\ExclamationPoint}]
%! A small "hello" program
%
%program hello
%  print *, "Hello World"
%\end{Verbatim}

\begin{BVerbatim}[baseline=b,boxwidth=100pt,fontfamily=courier]
! A small "hello" program

Α α, Β β, Γ γ, Δ δ, Ε ε, Ζ ζ,
\end{BVerbatim}
\hfill
\begin{enumerate}
\item test
\begin{LVerbatim}[baseline=b,boxwidth=100pt,fontfamily=panunicode]
Α α, Β β, Γ γ, Δ δ, Ε ε, Ζ ζ, Η η, Θ θ,Α α, Β β, 
Γ γ, Δ δ, Ε ε, Ζ ζ, Η η, Θ θ, 
\end{LVerbatim}
\end{enumerate}
\end{document}