\chapter{The Knowledge Base}

The knowledge base part of the application provides an extensive write-up and catalogues for
different countries. In a way is an organized wiki, a blog and a CMS, an application on its own.

\section{Organization of the Knowledge Base}

Articles are organized by Country or Issuing Authority. There are approximately 500 countries, including countries that do not exist anymore such as the Cape of Good Hope, Natal, Orange Free State, The Confederate States etc.

\section{Articles}

Articles use a custom mark-up language, which is half-way between markdown and LaTeX. They can be collated and printed into books using LaTeX (from within the Browser) and they can also be rendered in HTML. Before being displayed we apply a filter (see filterClass)  to provide these transforamtions. This mark-up language can be extended via plugins (for example to use wiki-type markup language).

\begin{verbatim}
article fields
title:
subtitle:
category:
country:
menus:[]
\end{verbatim}





