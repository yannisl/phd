\documentclass{article}
\usepackage{expl3, xtemplate, xparse, csquotes, xcolor, fontspec, xspace}
\newfontfamily\arial{Arial}

\begin{document}
\ExplSyntaxOn
\DeclareObjectType{inlineobj}{1}
\DeclareTemplateInterface{inlineobj}{span}{1}
{
  font-face: tokenlist,
  font-shape: choice {italic, slanted, normal}=normal,
  font-weight: choice {bold, normal}=normal,
  font-color: tokenlist=black,
  quote: choice {none, enquote}=none,
}

\DeclareTemplateCode{inlineobj}{span}{1}
{
  font-face         =  \l_font_tl,
  font-shape = {
     italic       = \cs_set_nopar:Nn \afontshape: {\itshape},
     slanted   = \cs_set_nopar:Nn \afontshape: {\itshape},
     normal   = \cs_set_nopar:Nn \afontshape: {\upshape}
  },
  font-weight = {
     bold    = \cs_set_nopar:Nn \afontseries: {\bfseries},
     normal =\cs_set_nopar:Nn \afontseries: {\mdseries}
   },
  font-color = \l_tmpa_tl,  
  quote = {
     none = \cs_set_nopar:Npn \quotemacro:n #1 {\detokenize{#1}},  
     enquote = \cs_set_nopar:Npn \quotemacro:n #1 {\enquote{\detokenize{#1}}},
     unknown = \cs_set_nopar:Npn \quotemacro:n #1 {\detokenize{#1}} 
  },
}
{
% the implementation part
  \AssignTemplateKeys
  
  \group_begin:
  \color\l_tmpa_tl
   \cs:w \l_font_tl \cs_end: 
   \afontshape:
   \afontseries:\selectfont 
       \quotemacro:n{#1} 
   \group_end:
 }
 
\DeclareInstance {inlineobj}{docFunction}{span}
    {
        font-face=arial,
        font-shape=italic, 
        font-weight=bold,
        font-color=green!40!black,
        quote=enquote
     }

\DeclareDocumentCommand\docFunction { m } {
   \IfInstanceExistTF {inlineobj}{docFunction} 
     {\UseInstance{inlineobj}{docFunction}{#1}}
     {ERROR}
}

\DeclareInstance {inlineobj}{tn}{span}
    {
        font-face=arial,
        font-shape=normal, 
        font-weight=normal,
        font-color=red!80!black,
        quote=none,
     }

\DeclareDocumentCommand\tn{ m }{%
   \IfInstanceExistTF {inlineobj}{tn}
    {\UseInstance{inlineobj}{tn}{#1}}
    {ERROR}
} 
\ExplSyntaxOff
   
The function \docFunction {get_string ( )} is used throughout to get a string in LuaTeX, where macros in text paragraphs are shown as \docFunction{\my_macro} in green. Typewrite text is obtained by using \tn{tn}. 
\bigskip

\ExplSyntaxOn
\parindent0pt
\centering
\cs_set:Npn \make_table:nnn #1#2 #3 #4{
    \cs_set:Npn \cell_func:n ##1 {\framebox[2em][c]{\strut \int_eval:n{##1+3} }\thinspace}
    \cs_set:Npn \head_func:n ##1 {\framebox[2em][c]{\strut c##1 }\thinspace}
    \int_step_function:nnnN {#1} {#2} {#3} {\head_func:n}\vskip1pt
    \cs_set:Npn \make_rows:n ##1 { \int_step_function:nnnN {#1} {#2} {#3} { \cell_func:n}\vskip1pt   }
    \int_step_function:nnnN {#1} {#2} {#4} {\make_rows:n}\par
}    
\make_table:nnn {1}{1}{10}{12}
\ExplSyntaxOff
 
