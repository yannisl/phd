\documentclass{article}


\usepackage{datetime}
\usepackage{datenumber}



\begin{document}

\setdate{2002}{1}{1}%

\thedatenumber

Result: 73780

\setdatetoday

\addtocounter{datenumber}{10}%
\setdatebynumber{\thedatenumber}%

In 10 days is \datedate
Result: In 10 days is August 18, 2001

\newcounter{dateone}\newcounter{datetwo}%

\newcommand{\daydifftoday}[3]{%
  \setmydatenumber{dateone}{\the\year}{\the\month}{\the\day}%
  \setmydatenumber{datetwo}{#1}{#2}{#3}%
  \addtocounter{datetwo}{-\thedateone}%
  \thedatetwo
}

The time is  \oclock

\newcommand{\sd}{%
\ifcase\thedatedayname \or
    Mon\or Tue\or Wed\or Thu\or
    Fri\or Sat\or Sun\fi
}%

\newcommand{\pnext}{%
\thedateyear/%
\ifnum\value{datemonth}<10 0\fi
\thedatemonth/%
\ifnum\value{dateday}<10 0\fi
\thedateday%
\nextdate
}
Result: \setdate{2001}{9}{29}%

\[\begin{tabular}{lll}
\sd & \pnext & Abc\\
\sd & \pnext & Def\\
\sd & \pnext & Ghi\\
\sd & \pnext & Jkl\\
\end{tabular}\]


\newthought{Get your age calculated}

\begin{teXXX}
\documentclass{article}
\usepackage{datenumber,fp}
\begin{document}
\newcounter{dateone}%
\newcounter{datetwo}%
\setmydatenumber{dateone}{1989}{08}{01}%
\setmydatenumber{datetwo}{\the\year}{\the\month}{\the\day}%
\FPsub\result{\thedatetwo}{\thedateone}
\FPdiv\myage{\result}{365.25} 
\FPround\myage{\myage}{0}\myage\ years old
\end{document}
\end{teXXX}


\subsection{Other}
Because of the Protestant Reformation, however, many Western European countries did not initially follow the Gregorian reform, and maintained their old-style systems. Eventually other countries followed the reform for the sake of consistency, but by the time the last adherents of the Julian calendar in Eastern Europe (Russia and Greece) changed to the Gregorian system in the 20th century, they had to drop 13 days from their calendars, due to the additional accumulated difference between the two calendars since 1582.

The leapyear \index{dates!leapyear} can be tested using
\pmac{datenumber}{leapyear} and the date can be checked for validity using
\pmac{datenumber}{ifvaliddate}. the examples below show such tests

\begin{itemize}
\item leap year test
\begin{quote}
\begin{verbatim}
The year 2012 is
\ifleapyear{2012} a \else no \fi leap year.
\end{verbatim}
Result: The year |2012| is \ifleapyear{2012} a \else no \fi leap year.
\end{quote}
\item date test
\begin{quote}
\begin{verbatim}
The 29.2.1900 is
\ifvaliddate{1900}{2}{29} a \else no \fi valid date.
\end{verbatim}


Result: The 29.2.1900 is \ifvaliddate{1900}{2}{29} a \else no \fi valid date.%
\end{quote}
\end{itemize}

\section*{Calculating the week number}
\begin{marginfigure}%
  \centering
  \includegraphics[width=1.1\linewidth]{./graphics/babylonianmaps.jpg}
  \caption[Babylonian Imago Mundi]{\protect\footnotesize \protect\raggedright The Babylonian Imago Mundi, dated to the 6th century BC (Neo-Babylonian Empire). The map shows Babylon on the Euphrates, surrounded by a circular landmass showing Assyria, Armenia and several cities, in turn surrounded by a `bitter river' (Oceanus), with seven islands arranged around it so as to form a seven-pointed star.}
  \label{fig:eleven days}
\end{marginfigure}

I an attempt to produce gantt charts (see Section \ref{ganttcharts}) that follow Tufte's ideas of simplicity, I came across the need to define a week number. The ISO week date system is a leap week calendar system that is part of the ISO 8601 date and time standard. The system is used (mainly) in government and business for fiscal years, as well as in timekeeping.

The system uses the same cycle of 7 weekdays as the Gregorian calendar. Weeks start with Monday. ISO week-numbering years have a year numbering which is approximately the same as the Gregorian years, but not exactly (see below). An ISO week-numbering year has 52 or 53 full weeks (364 or 371 days). The extra week is here called a leap week (ISO 8601 does not use the term).



A date is specified by the ISO week-numbering year in the format YYYY, a week number in the format ww prefixed by the letter W, and the weekday number, a digit d from 1 through 7, beginning with Monday and ending with Sunday. For example, |2006-W52-7| (or in compact form |2006W527|) is the Sunday of the 52nd week of 2006. In the Gregorian system this day is called 31 December 2006.

The system has a 400-year cycle of 146 097 days (20 871 weeks), with an average year length of exactly 365.2425 days, just like the Gregorian calendar. In every 400 years there are 71 years with 53 weeks.

\textsc{The first week of a year is the week that contains the first Thursday of the year.}

Based on this a calculation can be made using routines available from the above packages. However, how many weeks are included in a typical month it is still a problem.


\section{Summary}

This rather long chapter discussed the various options and packages available to deal with dates. The best way so far, for pdfLaTeX and pdfTeX users is to use the \cmd{pdfcreation} to access system time. Once the information made available by this command is parsed the rest of the routines can be developed. And now we have dates. Next we are going to try and develop some scheduling routines for gantt charts.

\end{document}