\begin{enumerate}
\item Staged approach. First studied the code and changed all commands to new commands.
      Kept compiling to look for errors. Search and replace is dangerous better a mix approach
      some by hand some by search and replace.
      
      Aim for consistency.
      
      |@|->|  |@@| to aux |@@@| auxi, |\z@| |@ne| etc.
      
		 Some highly TeX idioms hard to change and prone to errors. |@@nil|, |@nil|

\item Once over the first couple of hurdles, I tried to use \enquote{idiomatic} \latex3 code.
      What is my definition of this? Use as much as possible the new functions and variables,
      avoid hard to use |\def| and |\newcommand| etc. In the package I tried to avoid using
      xparse as well.
     
      It is difficult to define what is idiomatic. I looked for inspiration mainly at Joseph's
      and Wills packages.
      
\item Common errors were many. edef nomally nopar      

\item  Keys were all defined using |keyval| and to make matters worse were also developed with 
       custom commands. Mapped them early to l3 equivalents. 

\item LaTeX counters changed to integers using |int| mapped |c@| to these integers for latex allocations
      in reality unecessary.

\item I took my revenge with a vengeance on all the extraneous spaces and the hundreds of \% I had to write
      previously by removing the hundreds of comment characters.
      
\item The keys I want to revisit. Not possible to remove all the adds 

      \begin{Verbatim}      
         \def\KV@FV@firstline@default{%
  			  \cs_set_eq:NN \FancyVerbStartNum\z@
          \cs_set_eq:NN \FancyVerbStartString\scan_stop:}
      \end{Verbatim}
      
      Prefixes satyed as well as equivalents.

\item User API
\item If you a coffee drinker I have good news. Compilation on larger documents is much slower allowing you
      to have more coffee breaks. A lot of this is initial loading time, as well as the colorization of
      the code, font loading etc.
\item I had a probelm between local and global variable translations 
\item Private macros I left them last as @@ was interfering with my understanding. 

\item Handling errors, was done through smaller test documents (MWE)  

\item Sammler tips include that index is your friend.

\item |_lineno|         or |line_no|

\item If one is consistent, rules can be build to parse the code in a faster language and 
      provide a pretty printer and a faster colorizer. Also it opens the possibility of
      a translator from TeX/LateX2e to l3, at least a partial translator.
\item Choice keys

\item fonts Since I wanted the package to be used in modern engines and especially LuaLaTeX, the original
      font setup was problematic and I have moved it to using fontspec.

\item Weird items |\hbox to\FancyVerbTabSize\fontdimen2\font{\hss\FV@TabChar}| changed to more readbale
      code.

		 \dim_set:Nn \l_tmpa_dim {\fv_verb_tab_size_tl\fontdimen2\font}
      \hbox_to_wd:nn\l_tmpa_dim {\hss\fv_tab_char}

\define@key{FV}{defineactive*}{%
  \expandafter\def\expandafter\FancyVerbDefineActive\expandafter{%
    \FancyVerbDefineActive#1\relax}}
    
    Watch also begin and end groups, espcially for open hboxes and vboxes
    


\end{enumerate}