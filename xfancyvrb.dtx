%%
%% This is file `fancyvrb.sty',
%%
%% COPYRIGHT 1992-1999, by Timothy Van Zandt <tvz@zandtwerk.kellogg.nwu.edu>
%%           2010-2018, by Herbert Voss <hvoss@tug.org>
%%
%% This package may be distributed under the terms of the LaTeX Project Public
%% License, as described in lppl.txt in the base LaTeX distribution.
%% Either version 1.3 or, at your option, any later version.
%%
%% DESCRIPTION:
%%   fancyvrb.sty is a LaTeX style option, containing flexible
%%   verbatim environments and commands and extensive documentation.
%%
% \iffalse meta-comment
%<*internal>
\iffalse
%</internal>
%<*readme>
----------------------------------------------------------------
phd-documentation - version 1.0 (2018-10-26)
E-mail: yannislaz@gmail.com
Released under the LaTeX Project Public License v1.3c or later
See http://www.latex-project.org/lppl.txt

This work has the LPPL maintenance status `author-maintained'

This work consists of all files listed in README.md
%</readme>
%<*readmemd>
###The `phd-documentation` LaTeX2e package

The `phd` latex package and the class with the same name provide
convenient methods to create new styles for books, reports
and articles. It also loads the most commonly used packages 
and resolves conflicts.

This work consists of the file  `phd-documentation.dtx`,
and the derived files   `phd-documentation.ins`,  `phd-documentation.pdf`, 
and `phd-documentation.sty`.

###Installation

run
           phd-lua.bat on windows
           pdflatex phd.dtx
           makeindex -s gind.ist -g phd 

If you have any difficulties with the package come and join us at
http://tex.stackexchange.com and post a new question or
add a comment at http://tex.stackexchange.com/a/45023/963.
or send me a message at  yannislaz at gmail.com

### Documentation

The package was written using the `doc` and `docscript` packages,
so that it is self documented in a literary programming style. 
The .pdf is a fat document, providing over fifty book styles (the
equivalent of classes) plus there is a lot of write-up on the inner
workings of TeX and LaTeX2e. However, you don't need to know much
to use it.

      \usepackage{phd}
      %%%%%%%%%%%%%%%%%%%%%%%%%%%%%%%%%%%%%%%%%%%
%%%%%%  STYLE 13
%%%%%%%%%%%%%%%%%%%%%%%%%%%%%%%%%%%%%%%%%%%

\cxset{style13/.style={
 name={Chapter},
 numbering=arabic,
 number font-size=\HUGE,
 number font-family=\sffamily,
 number font-weight=\bfseries,
 number color=\color{gray!50},
 number before=\par\vspace*{5pt}\hfill\hfill,
 number dot=,
 number after={\hspace*{7pt}\par},
 number position=rightname,
 chapter font-family=\sffamily,
 chapter font-weight=\normalfont,
 chapter font-size=\LARGE,
 chapter before={\thickrule\vspace*{20pt}\par\hfill\hfill},
 chapter after={\vskip0pt\par},
 chapter color={black!50},
 title beforeskip={\vspace*{10pt}},
 title afterskip={\vspace*{50pt}\par},
 title before={\hfill\hfill\raggedleft},
 title after={},
 title font-family=\sffamily,
 title font-color=\color{thered},
 title font-weight=\bfseries,
 title font-size=\huge,
 section indent=-1em,
 section align=\raggedright,
 section numbering=arabic,
 section indent=0pt,
 section beforeskip=0pt,
 section afterskip=\baselineskip,
 subsection align=\raggedright,
 subsection font-family=\sffamily,
 subsection font-weight=\bfseries,
 subsection font-size=\large,
 subsection font-shape=\itshape,
 subparagraph number after=\space,
}
}

\def\setstyle#1{\cxset{style#1}%
 \renewsection\renewsubsection\renewsubsubsection%
 \renewparagraph\renewsubparagraph}

\setstyle{13}


\chapter{Introduction to Chapter\\ Style Thirteen}

\section{A Brief History of Biomedical\\ Fluid Mechanics}
\lorem
\medskip
\begin{figure}[ht]
\centering
\includegraphics[width=0.45\textwidth]{./chapters/chapter14}
\includegraphics[width=0.45\textwidth]{./chapters/chapter14a}
\end{figure}
\lorem


All choices, are made via an extended key-value interface. 
Although not a compliment, it resembles CSS and the keys are a bit verbose but
attributes are easy to change and have a consistent and easy to remember interface.

To set or add a key we only use one command:

      \cxset{chapter name font-size = Huge,
             chapter number font-size = HUGE} 

### Future Development

This is still an experimental version, but I will retain the
interface in future releases. There is a large amount of
work still to be carried out to improve the template styles
provided, to test it more thoroughly and to add a number of
improvements in the special designs. At present I estimate
that I have completed about 70% of the work that needs
to be done.

__The package as it stands is not production stable.__ 


%</readmemd>
%
%<*TODO>
1.  Split package into three diffferent parts. One for listings settings. Use def, docCommands and
    indexing commands. Indexing commands remove symbols defs into sybpackage.
2.  Finish symbol management, both text and math. Math already 80% incorporated.
%</TODO>
%<*internal>
\fi
\def\nameofplainTeX{plain}
\if_meaning:w \fmtname\nameofplainTeX\else
  \expandafter \begingroup
\fi
%</internal>
%<*install>
\input l3docstrip.tex
\keepsilent
\askforoverwritefalse
\preamble
----------------------------------------------------------------
phd --- A package to beautify documents.
E-mail: yannislaz@gmail.com
Released under the LaTeX Project Public License v1.3c or later
See http://www.latex-project.org/lppl.txt
----------------------------------------------------------------
\endpreamble
%\BaseDirectory{C:/users/admin/my documents/github/phd}
%\usedir{MWE}
\generate{\file{\jobname.sty}{%
  \from{\jobname.dtx}{package}%
   }%
  }%
%\nopreamble\nopostamble
%</install>
%<install>\endbatchfile
%<*internal>
%\usedir{tex/latex/phd}
\generate{
  \file{\jobname.ins}{\from{\jobname.dtx}{install}}
}
\nopreamble\nopostamble
\generate{
	\file{README.txt}{\from{\jobname.dtx}{readme}}
  }
\generate{
  \file{\jobname.md}{\from{\jobname.dtx}{readmemd}}
}
\generate{
  \file{\jobname-todo.md}{\from{\jobname.dtx}{TODO}}
}
\if_meaning:w \fmtname\nameofplainTeX
  \expandafter \endbatchfile
\else
  \expandafter \endgroup 
\fi
%</internal>
%<*driver>
%\listfiles
\NeedsTeXFormat{LaTeX2e}[2017/04/15]
\documentclass[book,oneside,10pt,a4paper,
               microtype=off]{phddoc}
\let\textls\textit               
%\usepackage[left=3cm,bottom=2cm]{geometry}
%\savegeometry{std}
% \usepackage[style=mla]{biblatex}
\usepackage{phd-scriptsmanager}
\usepackage{phd-lowersections}
%\usepackage{phd-toc}
\sethyperref
% add bib resource
\addbibresource{phd1.bib}% Syntax f
%\usepackage{phd-toc}
\makeindex
\PageIndex
\EnableCrossrefs
\urlstyle{rm}

\usepackage[cache=false]{minted} 
\usemintedstyle[latex]{borland}  
\setminted[html]{fontsize=\footnotesize,style=friendly}
%%
%% This is file `phd-documentation-defaults.def',
%% generated with the docstrip utility.
%%
%% The original source files were:
%%
%% phd-fontmanager.dtx  (with options: `DFLT')
%% phd-colorpalette.dtx  (with options: `DFLT')
%% phd-lowersections.dtx  (with options: `DFLT')
%% phd-toc.dtx  (with options: `DFLT')
%% phd-documentation.dtx  (with options: `DFLT')
%% ----------------------------------------------------------------
%% phd --- A package to beautify documents.
%% E-mail: yannislaz@gmail.com
%% Released under the LaTeX Project Public License v1.3c or later
%% See http://www.latex-project.org/lppl.txt
%% ----------------------------------------------------------------
\cxset{
   % settings for document fonts.
    main font-size                 = 10pt,
    main font-face                 = Georgia,
    main sans font-face            = Georgia, %Arial,
    main mono font-face            = Consolas, %Apl385, %Consolas,%Source Code Pro,B-612,%
    chapter label font-face        = Georgia,
    chapter number font-face       = Arial,
    chapter title font-face        = Times New Roman,
    section label font-face        = Arial,
    section number font-face       = Arial,
    section title font-face        = Arial,
    subsection label font-face     = Arial,
    subsection number font-face    = Arial,
    subsection title font-face     = Arial,
    subsubsection label font-face  = Arial,
    subsubsection number font-face = Arial,
    subsubsection title font-face  = Arial,
    paragraph label font-face      = Arial,
    paragraph number font-face     = Arial,
    paragraph title font-face      = Arial,
    subparagraph label font-face   = Arial,
    subparagraph number font-face  = Arial,
    subparagraph title font-face   = Arial,
    % default palette
    palette orange sakura,
    part format                       = traditional,
    part afterindent                  = off,
    chapter title margin-top-width    =  0cm,
    chapter title margin-right-width  =  1cm,
    chapter title margin-bottom-width = 10pt,
    chapter title margin-left-width   = 0pt,
    chapter align                     = left,
    chapter title align               = left, %checked
    chapter name                      = chapter,
    chapter format                    = fashion,
    chapter font-size                 = Huge,
    chapter font-weight               = bold,
    chapter font-family               = sffamily,
    chapter font-shape                = upshape,
    chapter color                     = black,
    chapter number prefix             = ,
    chapter number suffix             = ,
    chapter numbering                 = arabic,
    chapter indent                    = 0pt,
    chapter beforeskip                = -3cm,
    chapter afterskip                 = 30pt,
    chapter afterindent               = off,
    chapter number after              = ,
    chapter arc                       = 0mm,
    chapter background-color          = white,
    % takes indentation of first line off
    chapter afterindent               = on,
    chapter grow left                 = 0mm,
    chapter grow right                = 0mm,
    chapter rounded corners           = northeast,
    chapter shadow                    = fuzzy halo,
    chapter border-left-width         = 0pt,
    chapter border-right-width        = 0pt,
    chapter border-top-width          = 0pt,
    chapter border-bottom-width       = 0pt,
    chapter padding-left-width        = 0pt,
    chapter padding-right-width       = 10pt,
    chapter padding-top-width         = 10pt,
    chapter padding-bottom-width      = 10pt,
    chapter number color              = white,
    chapter label color               = black,
    chapter number font-size        = huge,
    chapter number font-weight      = bfseries,
    chapter number font-family      = sffamily,
    chapter number font-shape       = upshape,
    chapter number align            = Centering,
    chapter title font-size        = Huge,
    chapter title font-weight      = bold,
    chapter title font-family      = sffamily,
    chapter title font-shape       = upshape,
    chapter title color            = black,
    section name                   = Section,
    section format                 = hang,
    section align                  = Centering,
    section title align            = Centering, %checked
    section font-size              = Large,
    section font-weight            = bfseries,
    section font-family            = serif,
    section font-shape             = upshape,
    section number font-size       = Large,
    section number font-weight     = bfseries,
    section number font-family     = serif,
    section number font-shape      = upshape,
    section number color           = thesectionnumbercolor,
    section title font-size        = Large,
    section title font-weight      = bfseries,
    section title font-family      = serif,
    section title font-shape       = upshape,
    section color                  = black,
    section number prefix          = \thechapter.,
    section number suffix          =,
    section numbering              = arabic,
    section indent                 = -10pt,
    section beforeskip             = 3ex,
    section afterskip              = 1.5ex plus .1ex,
    section afterindent            = on,
    section number after           = \quad,
    section arc                    = 3pt,
    section background-color       = white,
    section grow left              = 0mm,
    section grow right             = 0mm,
    section rounded corners        = northeast,
    section border-left-width      = 0pt,
    section border-right-width     = 0pt,
    section border-top-width       = 2pt,
    section border-bottom-width    = 2pt,
    section padding-left-width     = 0pt,
    section padding-right-width    = 10pt,
    section padding-top-width      = 2pt,
    section padding-bottom-width   = 2pt,
    section title margin-top-width = 2pt,
    section title color            = thesectiontitlecolor,
    section shadow                 = no shadow,
%% sybsection
    subsection name                   = Subsection,
    subsection format                 = hang,
    subsection font-size              = large,
    subsection font-weight            = bfseries,
    subsection font-family            = rmfamily,
    subsection font-shape             = upshape,
    subsection number font-size       = large,
    subsection number font-weight     = bfseries,
    subsection number font-family     = rmfamily,
    subsection number font-shape      = upshape,
    subsection title font-size        = Large,
    subsection title font-weight      = bfseries,
    subsection title font-family      = sffamily,
    subsection title font-shape       = upshape,
    subsection title color            = bgsexy,
    subsection color                  = bgsexy,
    subsection numbering              = arabic,
    subsection align                  = Centering, %checked
    subsection title align            = Centering, %checked
    subsection beforeskip             = -3.25explus -1ex minus -.2ex,
    subsection afterskip              = 1.5ex plus .2ex,
    subsection number prefix          = \thesection.,
    subsection indent                 = 0pt,
    subsection number after           = 0pt,
    subsection background-color       = white,
    subsection border-left-width      = 0pt,
    subsection border-right-width     = 0pt,
    subsection border-top-width       = 5pt,
    subsection border-bottom-width    = 5pt,
    subsection padding-left-width     = 0pt,
    subsection padding-right-width    = 0pt,
    subsection padding-top-width      = 20pt,
    subsection padding-bottom-width   = 20pt,
    subsection shadow                 = drop shadow,
    subsection afterindent            = on,
    subsubsection name                    = Subsubsection,
    subsubsection format                  = hang,
    subsubsection background-color        = white, %checked
    subsubsection afterindent             = on,
    subsubsection font-family             = rmfamily,
    subsubsection font-size               = large,
    subsubsection font-weight             = bfseries,
    subsubsection font-family             = tiresias,
    subsubsection font-shape              = upshape,
    subsubsection font-family             = sffamily,
    subsubsection font-size               = large,
    subsubsection font-weight             = bfseries,
    subsubsection font-family             = tiresias,
    subsubsection font-shape              = upshape,
    subsubsection color                   = black,
    subsubsection number prefix           = \thesubsection,
    subsubsection number suffix           = ,
    subsubsection numbering               = arabic,
    subsubsection indent                  = 0pt,
    subsubsection beforeskip              = -3.25explus -1ex minus -.2ex,
    subsubsection afterskip               = 1.5ex plus .2ex,
    subsubsection align                   = center,
    subsubsection title align             = center,
    subsubsection number after     =,
    subsubsection border-left-width       = 0pt,
    subsubsection border-right-width      = 0pt,
    subsubsection border-top-width        = 2pt,
    subsubsection border-bottom-width     = 0pt,
    subsubsection padding-left-width      = 0pt,
    subsubsection padding-right-width     = 0pt,
    subsubsection padding-top-width       = 20pt,
    subsubsection padding-bottom-width    = 20pt,
    subsubsection shadow                  = no shadow,
    subsubsection title font-size         = large,
    subsubsection title font-weight       = bfseries,
    subsubsection title font-family       = serif,
    subsubsection title font-shape        = itshape,
    subsubsection title color             = thesubsectiontitlecolor,
    paragraph name                = paragraph,
    paragraph format              = inline,
    paragraph name                = paragraph,
    paragraph font-size           = small,
    paragraph font-weight         = bfseries,
    paragraph font-family         = sffamily,
    paragraph font-shape          = upshape,
    paragraph numbering           = alpha,
    paragraph number prefix       = \thesubsubsection,
    paragraph align               = flushleft,
    paragraph beforeskip          = 3.25ex plus1ex minus.2ex,
    paragraph afterskip           = -1em,
    paragraph indent              = 0pt,
    paragraph number after        = \quad,
    paragraph color               = bgsexy,
    paragraph background-color    = white,
    paragraph shadow              = no shadow,
    paragraph afterindent         = off
    subparagraph name             = subparagraph,
    subparagraph format           = inline,
    subparagraph name             = subparagraph,
    subparagraph font-size        = small,
    subparagraph font-weight      = mdseries,
    subparagraph font-family      = sffamily,
    subparagraph font-shape       = itshape,
    subparagraph color            = bgsexy,
    subparagraph background-color = bgsexy,
    subparagraph numbering        = none,
    subparagraph align            = flushleft,
    subparagraph beforeskip       = 3.25ex plus1ex minus .2ex,
    subparagraph afterskip        = -1em,
    subparagraph indent           = 0pt,
    subparagraph number after     = ,
    subparagraph afterindent      = off,
    %subparagraph shadow           = off,
%% toc contents element settings
    toc name               = Table of Contents,
    toc  before            =,
    toc  after             =,
    toc  numwidth          = 0pt,
    toc  color             = thetocname,
    toc  background-color  = bgsexy!20,
    toc  frame-color       = red,
    toc  shadow            = none,
    toc  font-weight       = normal,
    toc  font-family       = rmfamily,
    toc  font-shape        = upshape,
    toc  font-size         = Huge,
    toc  afterskip         = 30pt,
    toc  after             = ,
    toc  align             = left,
    toc  indent            = 0pt,
    toc case               = none,
    toc  page after        = A,
    toc  pagestyle         = headings,
    toc  rmarg             = 4em,
%% TOC part keys
    toc part font-size    = LARGE,
    toc part color        =  black,
    toc part beforeskip   =  1em,
    toc part page before  =,
    toc part indent       =  0pt,
    toc part numwidth     = 1.5em,
    % table of contents defaults
    % toc chapter keys
    toc chapter font-size   = Large,
    toc chapter font-family = rmfamily,
    toc chapter font-weight = normal,
    % the toc chapter color thetocchapter
    % is fetched from the palette define
    % your own color in the palette rather than
    % change this here
    toc chapter color       = thetocchapter,
    toc chapter beforeskip  =1em,
    toc chapter afterskip   = 12pt plus0.2pt minus .2pt,
    toc chapter case        = upper,
    toc chapter numwidth    = 1.5em,
    %  TOC chapter page formatting
    toc chapter page font-size        = Large,
    toc chapter page font-shape       = upshape,
    toc chapter page font-weight      = normal,
    toc chapter page font-family      = rmfamily,
    toc chapter page color            = black,
    toc chapter page background-color = white,
    toc chapter page before           =,
    toc chapter page after            =,
      %TOC section
      % indentation
       toc section indent=1.5em,
       toc section numwidth= 2.3em,
       toc section beforeskip=0pt,
       toc section afterskip=0pt,
      % page number fonts
       toc  section page font-size          = large,
       toc  section page font-shape         = upshape,
       toc  section page font-weight        = normal,
       toc  section page font-family        = rmfamily,
       % page number colors
       toc  section page color              = bgsexy,
       toc  section page background-color   = white, %theblue!10,
       % page number before after elements
       toc  section page before             =,
       toc  section page after              =,
       toc section page after = ,
       toc section page before =,
%%
%% subsection defaults
    toc subsection indent        = 3.8em,
    toc subsection numwidth      = 3.2em,
    toc subsection page before   = {},
    toc subsection page after    = {},
%%
    % List of Figures
    lof name              = List of Figures,
    lof before            =,
    lof after             =,
    lof numwidth          = 0pt,
    lof color             = thelofname,
    lof background-color  = white,
    lof frame-color       = white,
    lof shadow            = none,
    lof font-weight       = normal,
    lof font-family       = rmfamily,
    lof font-shape        = upshape,
    lof font-size         = Huge,
    lof afterskip         = 40pt,
    lof after             = ,
    lof align             = left,
    lof indent            = 0pt,
    lof case              = none,
    lof page after        = ,
   color command     = themacrocolor,
   color environment = theenvironment,
   color key         = thekey,
   color value       = thevalue,
   color color       = black,%leaks to index
   color option      = theoption,
   color meta        = themeta,
   color frame       = theframe,
   % indexing
   index actual  = {@},
   index quote   = {!},
   index level   = {>},
   index doc settings,
  docexample/.style={colframe=ExampleFrame,colback=ExampleBack,fontlower=\footnotesize},
  documentation minted style=,
  documentation minted options={tabsize=2,fontsize=\small},
  english language/.code={\cxset{doclang/.cd,
    color=color,colors=Colors,
    environment content=environment content,
    environment=environment,environments=Environments,
    key=key,keys=Keys,
    index=Index,
    pageshort={P.},
    value=value,values=Values}},
 }
\endinput
%%
%% End of file `phd-documentation-defaults.def'.

\cxset{palette oprah,
       subsection afterindent=off}
%\usepackage{hypdoc}
\let\solution\undefined
%\usepackage{fancyvrb-ex}
\usepackage{tasks}
\usepackage{exsheets}
\usepackage{exsheets-listings}
\usepackage{xfancyvrb}
\usepackage{xcoffins}
\begin{document}
\DEBUGOFF
\overfullrule0pt
\parindent1em
\coverpage{monkey}{Book Design Monographs}{Camel Press}{LaTeX}{Verbatims} 
\pagestyle{empty}

\secondpage
\pagestyle{empty}
\clearpage

\tableofcontents

\pagestyle{empty}
\setcounter{secnumdepth}{6}
\parskip0pt plus.1ex minus.1ex
\mainmatter
\pagenumbering{arabic}
\pagestyle{headings} 
% \chapter{Characters}


\normalsize

\tex\ works internally by translating characters into character codes. The way characters are encoded in a computer
may differ from system to system.\index{characters>encoding}


There are 256 characters that \tex\  might encounter at
each step, in a file or in a line of text typed directly on your terminal. These
256 characters are classified into 16 categories numbered 0 to 15:\index{characters>catcodes}\index{catcodes}


\arial
\begin{table}[htbp]
\centering
\begin{tabular}{rll}
\toprule
Code & Description & Representation\\
\midrule
0 & Escape character & (\textbackslash in this book)\\
1 & Beginning of group & (|{| in this book)\\
2 & End of group & (|\}| in this book )\\
3 & Math shift & (|\$| in this book)\\
4 & Alignment tab & (|\&| in this book)\\
5 & End of line &(return in this book)\\
6 & Parameter &(|\#| in this book\\
7 & Supescript &(|\^| in this book)\\
8 & Subscript &(|\_| in this bookl)\\
9 & Ignored character &(null in this manual)\\
10 & Space &(\textvisiblespace in this book)\\
11 &Letter &(A,\ldots,Z and a,\ldots z)\\
12 &Other character &(none of the above or below)\\
13 &Active character &(|\~| in this manual)\\
14 &Comment character &(|\%| in this book)\\
15 &Invalid character &(delete in this book)\\
\bottomrule
\end{tabular}
^^A\captionof{table}{Character Codes}
\end{table}
\medskip

When \tex reads a line of text from a file, or a line of text that you entered
directly on your keyboard, it converts that text into a list of \cmd{\tokens}. A
token is either (a) a single character with an attached category code, or (b) a control
sequence. For example, if the normal conventions of plain \tex  are in force, the text
\verb*+ `{\hskip 36 pt}'+  is converted into a list of \textit{eight} tokens:
\medskip

$ \{_{1}$ hskip $3_{12}~~6_{12}~~\_{10}~~p_{11}~~t_{11}~~\}_2 $

\medskip
The subscripts here are the category codes, as listed earlier:
\begin{itemize}
\item[1] for beginning of group,
\item[12] for |other| character," and so on. The |hskip| doesn't get a subscript, because it
represents a control sequence token instead of a character token. Notice that the space
after \cmd{hskip} does not get into the token list, because it follows a control word.
\end{itemize}

Knuth in the \texbook notes that:

\begin{quotation}

It is important to understand the idea of token lists, if you want to gain a
thorough understanding of \tex, and it is convenient to learn the concept by
thinking of \tex as if it were a living organism. The individual lines of input in your
files are seen only by \tex's \textit{eyes} and \textit{mouth}; but after that text has been gobbled
up, it is sent to \tex's \textit{stomach} in the form of a token list, and the digestive processes
that do the actual typesetting are based entirely on tokens. As far as the stomach is
concerned, the input 
flows in as a stream of tokens, somewhat as if your \tex manuscript
had been typed all on one extremely long line.
\end{quotation}

\section{Control sequences for characters}

\DescribeMacro{\char}
There are a number of ways in which a control sequence can denote a character. The \cmd{\char} command
specifies a character to be typeset; the \cmd{\let} command introduces a synonym for a character
token, that is, the combination of character code and category code.

\section{Denoting characters to be typeset: \texttt{char}}

\index{\string\char}
Characters can be denoted numerically by, for example, \verb+ \char98 +. This command tells \tex to add
character number 98 of the current font to the horizontal list currently under construction.

Instead of decimal notation, it is often more convenient to use octal or hexadecimal notation. For
octal the single quote is used: \verb+ \char’142+; hexadecimal uses the double quote: \verb+ \char"62+. Note that

\begin{texexample}{Characters}{ex:chars}
\bgroup
\ttfamily

\char65

\char`b

\char`\b

\char"70

\egroup
\end{texexample}

\verb+ \char`'62+  is incorrect; the process that replaces two quotes by a double quote works at a later
stage of processing (the visual processor) than number scanning (the execution processor).

Because of the explicit conversion to character codes by the back quote character it is also possible
to get a ‘b’ – provided that you are using a font organized a bit like the ASCII table – with \verb+ \char‘b+
or \verb+ \char‘\b+.

The \cmd{\char} command looks superficially a bit like the \verb+  ^^+ substitution mechanism.

Both mechanisms access characters without directly denoting them. However, the \verb+ ^^+ mechanism
operates in a very early stage of processing (in the input processor of \tex, but before category
code assignment); the \cmd{\char} command, on the other hand, comes in the final stages of processing.
In effect it says ‘typeset character number so-and-so’.

\CMDI{\Uchar} The LuaTeX expandable command \cmd{\Uchar} reads a number between 0 and 1,114,111 and expands to the
associated Unicode character. 

\DescribeMacro{\chardef}
There is a construction to let a control sequence stand for some character code: the \cmd{\chardef}
command. The syntax of this is\\
\cs{chardef}\meta{control sequence}=\meta{number},\\
where the number can be an explicit representation or a counter value, but it can also be a character
code obtained using the left quote command (see above; the full definition of hnumberi is
given in Chapter 7). In the plain format the latter possibility is used in definitions such as

\verb+ \chardef\%=‘\%+

or 

\verb+ \chardef\%=37   +

command to typeset character 37 (usually the per cent character).\index{characters!percent character}

A control sequence that has been defined with a \cmd{\chardef} command can also be used as a hnumberi.
This fact is used in allocation commands such as \verb+ \char{newbox}+ (see Chapters 7 and 31). Tokens defined
with \verb+ \char{mathchardef}+ can also be used this way.


But \tex\ actually provides another kind of number that makes it unnecessary
for you to know texttt{ASCII} at all! The token `12 (left quote), when followed by
any character token or by any control sequence token whose name is a single character,
stands for \tex's internal code for the character in question. For example, \verb+\char`b+ and
\verb+ \char`\b+ are also equivalent to \verb+ \char98+. If you look in Appendix B to see how \verb+ \%+ is
defined, you'll notice that the definition is

\verb+\def\%{\char`\%}+

instead of \verb+ \char37+  as claimed above.

\section{Special notation for invisible characters}

\tex has a standard way to refer to the invisible characters of |ASCII|: 

Code 0 can be typed as the sequence of three characters \verb+ ^^@+, code 1 can be typed
\verb+ ^^A+, and so on up to code 31, which is \verb+ ^^_  +(see Appendix C). If the character following
\verb+ ^^+ has an internal code between 64 and 127, TEX subtracts 64 from the code; if the
code is between 0 and 63, \tex adds 64. 

Hence code 127 can be typed \verb+^^?+, and
the dangerous bend sign can be obtained by saying \verb+{\manual^^?}+. However, you must
change the category code of character 127 before using it, since this character ordinarily
has category 15 (invalid); say, e.g., 

\verb+ \catcode`\^^?=12 +

The \verb+ ^^+ notation is different from
\cmd{\char}, because \verb+ ^^+ combinations are like single characters; for example, it would not
be permissible to say \verb+ \catcode`\char127+, but \verb+^^+ symbols can even be used as letters within control words.

\begin{texexample}{Special notation}{ex:texbook1}
\def\^^zz{test}
\^^zz
\end{texexample}


Most of the \verb+ ^^+ codes are unimportant except in unusual applications. But
\verb+ ^^M+ is particularly noteworthy because it is code 13, the |ASCII| return that
\tex normally places at the right end of every line of your input file. By changing the
category of \verb+ ^^M+  you can obtain useful special effects, as we shall see later.

\section{Upper and Lowercase characters}

\verb*+\lccode +the character code for the lowercase form of a letter (p. 103)

\DescribeMacro{\lowercase}
\DescribeMacro{\uppercase}
The twin operations \cmd{\uppercase}\marg{token list} and \cmd{\lowercase}\marg{token listi}
go through a given token list and convert all of the character tokens to their
\cmd{\uppercase}  or \cmd{lowercase} equivalents.

\begin{texexample}{Uppercase and Lowercase}{ex:lowercase} 
\uppercase{abcdefgh} 
\lowercase{ABCDEFGH}
\end{texexample}

Here's how: Each of the 256 possible characters
has two associated values called the \cmd{\uccode} and the \cmd{lccode}; these values are
changeable just as a \cmd{\catcode} is. Conversion to uppercase means that a character
is replaced by its \cmd{\uccode} value, unless the \cmd{\uccode} value is zero (when no change
is made). Conversion to lowercase is similar, using the
\verb+  \lccode+. The category codes
aren't changed. 

When INITEX begins, all \verb+ \uccode+ and \verb+ \lccode+ values are zero except
that the letters a to z and A to Z have \verb+\uccode+ values A to Z and \verb+\lccode+ values a to z.

These tow control sequences are used to build a hash table, mapping all the capital and lowercase letters to their respective character codes.
(see pg 41 TeXbook)

\section{Some Practical Examples}

If you are typesetting anything that has to do with \tex\ or \latex\ you are bound to have to typeset a lot of commands. This short code below will change the category code of the \texttt{"} (double quote) to be the active command. This way anything between double quotes will be  typed out verbatim and in a Maroon color. By mainipulating the \cmd{catcode} of characters we can achieve this.

\begin{teX}
%% Code to catch commands
\def\Meaningless#1>{}
\catcode`\"=\active
\def\startV{\leavevmode\begingroup
  \ifdim 0pt=\lastskip\penalty200 \fi
  \catcode`\{11 \catcode`\}11 \catcode`\%11
  \moreV}
\long\def\moreV#1"{%
  \def\LtxCode{#1}%
  \ignorespaces
      \expandafter\Meaningless\meaning\LtxCode
      \unskip%
  \endgroup}
\let"\startV

\bgroup
\catcode`\<=\active
\def<#1>{\ensuremath{\langle\mbox{\textsl{#1}}\rangle}}
\end{teX}

\begin{comment}
\bgroup
\def\Meaningless#1>{}
\catcode`\"=\active
\def\startV{\leavevmode\begingroup
  \ifdim 0pt=\lastskip\penalty200 \fi
  \catcode`\{11 \catcode`\}11 \catcode`\%11
  \moreV}
\long\def\moreV#1"{%
  \def\LtxCode{#1}%
  \ignorespaces
      \expandafter\Meaningless\meaning\LtxCode
      \unskip%
  \endgroup}
\let"\startV

\catcode`\<=\active
\def<#1>{\ensuremath{\langle\mbox{\textsl{#1}}\rangle}}

\noindent Testing it out with a few commands we get 
"\catcode", "\char" ,"\def" etc. We will revert back to this short example later on in our book, when you have learned a bit more about macros and programming \tex\. Note that this also affects "quotes".

\egroup
\end{comment}

A more complex example is the \pkg{shortvrb} package code.

\begin{teX}
%% Copyright (C) 1989-1999 Frank Mittelbach, all rights reserved.
\def\MakeShortVerb{%
  \@ifstar
    {\def\@shortvrbdef{\verb*}\@MakeShortVerb}%
    {\def\@shortvrbdef{\verb}\@MakeShortVerb}}

\def\@MakeShortVerb#1{%
  \expandafter\ifx\csname cc\string#1\endcsname\relax
    \@shortvrbinfo{Made }{#1}\@shortvrbdef
    \add@special{#1}%
    \expandafter
    \xdef\csname cc\string#1\endcsname{\the\catcode`#1}%
    \begingroup
      \catcode`\~\active  \lccode`\~`#1%
      \lowercase{%
      \global\expandafter\let
         \csname ac\string#1\endcsname~%
      \expandafter\gdef\expandafter~\expandafter{\@shortvrbdef~}}%
    \endgroup
    \global\catcode`#1\active
  \else
    \@shortvrbinfo\@empty{#1 already}{\@empty\verb(*)}%
  \fi}
\def\DeleteShortVerb#1{%
  \expandafter\ifx\csname cc\string#1\endcsname\relax
    \@shortvrbinfo\@empty{#1 not}{\@empty\verb(*)}%
  \else
    \@shortvrbinfo{Deleted }{#1 as}{\@empty\verb(*)}%
    \rem@special{#1}%
    \global\catcode`#1\csname cc\string#1\endcsname
    \global \expandafter\let \csname cc\string#1\endcsname \relax
    \ifnum\catcode`#1=\active
      \begingroup
        \catcode`\~\active   \lccode`\~`#1%
        \lowercase{%
          \global\expandafter\let\expandafter~%
          \csname ac\string#1\endcsname}%
      \endgroup \fi \fi}
\def\@shortvrbinfo#1#2#3{%
  \PackageInfo{shortvrb}{%
     #1\expandafter\@gobble\string#2 a short reference
                                          for \expandafter\string#3}}
\def\add@special#1{%
  \rem@special{#1}%
  \expandafter\gdef\expandafter\dospecials\expandafter
    {\dospecials \do #1}%
  \expandafter\gdef\expandafter\@sanitize\expandafter
    {\@sanitize \@makeother #1}}
\def\rem@special#1{%
  \def\do##1{%
    \ifnum`#1=`##1 \else \noexpand\do\noexpand##1\fi}%
  \xdef\dospecials{\dospecials}%
  \begingroup
    \def\@makeother##1{%
      \ifnum`#1=`##1 \else \noexpand\@makeother\noexpand##1\fi}%
    \xdef\@sanitize{\@sanitize}%
  \endgroup}
\endinput
%%
%% End of file `shortvrb.sty'.
\end{teX}

We will spent the rest of the book in trying to understand and write code like this. My ultimate aim is  to be able to produce \tex\ code like any other program. 

\section{Example}

In this example we wish to redefine some of the active codes to act as text only:

\begin{teX}
\newenvironment{plaintext}{%
        \catcode`\$12
        \def\&{&}%
        \catcode`\&12
        \def\_{_}%
        \catcode`\_12
        \def\^{^}%
        \catcode`\^12
        \catcode`\#12
        \catcode`\%12
        \let\~~%
        \catcode`\~12
}{}
\end{teX}

\newenvironment{plaintext}{%
        \catcode`\$12
        \def\&{&}%
        \catcode`\&12
        \def\_{_}%
        \catcode`\_12
        \def\^{^}%
        \catcode`\^12
        \catcode`\#12
        \catcode`\%12
        \let\~~%
        \catcode`\~12
}{}
Use it like

\begin{plaintext}
Here is some test text % ^ & _ $ # &.

How about some math \(x\_y\^z\). You're still out of luck with braces
though.
\end{plaintext}

\begingroup
\catcode`\{=11 
\catcode`\}=11
\catcode`\[=1
\catcode`\]=2

{This is a test}

\endgroup


\section{Checking to see the meaning of a control sequence
}
Finding out just what a control sequence has been defined to be with |\let| can be done using

%\meaning: the sequence

\begin{teXXX}
\let\x=3 \meaning\x
\end{teXXX}
\graybox{
gives 'the character 3'.}




% \chapter{The LaTeX3 l3token package}
\label{ch:l3token}

The \tex concept of tokens is central to its operation. In earlier chapters we discussed extensively the use of category codes and other important aspects of \tex’s tokens. Rememeber a \tex token is either a single character or a control sequence such as a the control sequence |\test|.

A review of all possible tokens is appropriate at this stage, before we examine the module in more detail. We distinguish the meaning of a token which controls the expansion of the token and its effect on \tex’s state,
and its shape, which is used when comparing token lists such as for delimited arguments.
Two tokens of the same shape must have the same meaning, but the converse does not
hold.

A token has one of the following shapes:

\begin{enumerate}
\item A control sequence, characterized by the sequence of characters that constitute its
name: for instance, |\use:n| is a five-letter control sequence.
\end{enumerate}

Now is perhaps a good time to mention some subtleties relating to tokens with
category code 10 (space). Any input character with this category code (normally, space
and tab characters) becomes a normal space, with character code 32 and category code 10.

When a macro takes an undelimited argument, explicit space characters (with character
code 32 and category code 10) are ignored. If the following token is an explicit
character token with category code 1 (begin-group) and an arbitrary character code,
then TEX scans ahead to obtain an equal number of explicit character tokens with category
code 1 (begin-group) and 2 (end-group), and the resulting list of tokens (with outer
braces removed) becomes the argument. Otherwise, a single token is taken as the argument
for the macro: we call such single tokens \enquote{N-type}, as they are suitable to be used
as an argument for a function with the signature :N.



\begin{texexample}{Space tokens}{ex:sptoken}
\ExplSyntaxOn  
 \cs_set:Npn \my_space_token { }
 \token_to_meaning:N \my_space_token\\
 \token_to_meaning:N \c_space_token
 
 % Note that the ~ active character in an ExplSyntaxOn
 % environment has a more complicated definition.
 \token_to_meaning:N ~
\ExplSyntaxOff  
\end{texexample}

The actual definition from the kernel code for the \cs{c_space_token}
\begin{teX}
\use:n { \tex_global:D \tex_let:D \c_space_token = ~ } ~
\end{teX}

\begin{texexample}{makeatletter}{}
\ExplSyntaxOn
\group_begin:
\char_set_catcode_letter:N @
\char_set_catcode_letter:N 1
\def\@store1a{AAAA}
\@store1a\\
\token_to_meaning:N @\\
\token_to_meaning:N 1\\
\char_set_catcode_other:N @
\char_set_catcode_other:N 1
\token_to_meaning:N @\\
\token_to_meaning:N 1\\
\group_end:
\ExplSyntaxOff
\end{texexample}

There are sixteen different commands to set the catcode to any of the predefined groups used by \tex. If you cannot remember the catcode number for a character, try and remember its normal name!

\begin{verbatim}
 \char_set_catcode_escape:N 
 \char_set_catcode_group_begin:N
 \char_set_catcode_group_end:N
 \char_set_catcode_math_toggle:N
 \char_set_catcode_alignment:N
 \char_set_catcode_end_line:N
 \char_set_catcode_parameter:N
 \char_set_catcode_math_superscript:N
 \char_set_catcode_math_subscript:N
 \char_set_catcode_ignore:N
 \char_set_catcode_space:N
 \char_set_catcode_letter:N
 \char_set_catcode_other:N
 \char_set_catcode_active:N
 \char_set_catcode_comment:N
\char_set_catcode_invalid:N
\end{verbatim}

\section{Token predicate functions}

\begin{docCommand}{token_if_macro:NTF} { \meta{token} \marg{true code} \marg{false code}}
tests if the \meta{token} is a \tex macro.
\end{docCommand}

\begin{texexample}{Test if is a macro}{ex:assertt}
\ExplSyntaxOn

% This is a common problem in LaTeXe. 
% \sometest is let tto |\relax| in a csname
 \csname my_sometest\endcsname
 
 % traditional definition using a csname and 
 % \expandafter
 \expandafter\def\csname my_sometesti\endcsname{}
 
 % All tests must pass
 \token_if_macro:NTF \par           { \FAIL } { \PASS } 
 \token_if_macro:NTF \minipage      { \PASS } { \FAIL } 
 \token_if_macro:NTF \my_sometest   { \FAIL } { \PASS }

 \token_if_macro:NTF \my_sometesti  { \PASS } { \FAIL }
 \token_if_macro:NTF Z              { \FAIL } { \PASS } 

 % True was set to relax
 \token_if_eq_meaning:NNTF \my_sometest\relax { \PASS } { \FAIL }
 
 \ExplSyntaxOff
\end{texexample}

Notice the unusual syntax for \cs{ifx} which is named \cs{token_if_eq_meaning:NN}. Also note that Example~\ref{ex:assertt}, uses an assertion style where all tests must return true (\mbox{\PASS}). If you have a lot
of tests in a test file, it is easier to spot what is failing. See below where I redefined the token \enquote{Z} as an active
character and then defined a macro with it. Our test file will then clearly show the test failing. Just a small reminder to turn a character into a macro, you need to set it first to |\active| and then define it. Here is the test file again. 

\begin{texexample}{Test if is a macro}{ex:assertt1}
\ExplSyntaxOn

% \define Z
\group_begin:
\catcode `\Z = \active
\cs_set:Npn  Z {hello~}
% This is a common problem in LaTeXe. 
% \sometest is let tto |\relax| in a csname
 \csname my_sometest\endcsname
 
 % traditional definition using a csname and 
 % \expandafter
 \expandafter\def\csname my_sometesti\endcsname{}
 
 % All tests must pass
 \token_if_macro:NTF \par           { \FAIL } { \PASS } 
 \token_if_macro:NTF \minipage      { \PASS } { \FAIL } 
 \token_if_macro:NTF \my_sometest   { \FAIL } { \PASS }

 \token_if_macro:NTF \my_sometesti  { \PASS } { \FAIL }
 \token_if_macro:NTF Z              { \FAIL } { \PASS } 

 % True was set to relax
 \token_if_eq_meaning:NNTF \my_sometest\relax { \PASS } { \FAIL }
 \group_end:
 \ExplSyntaxOff
\end{texexample}


\bigskip

\begin{question}
It is recommended that you code these exercises as MWEs and try and not refer to the source3 manual,
during your first attempt. Namespace any macros you have to develop as part of the tasks below
with the prefix |yourname|.
\begin{tasks}
\task Define four macros and using \cs{token_if_macro:NTF} typeset a word.
\task Test the meaning of the four macros.
\end{tasks}
\end{question}

\subsection{Test if a control sequence is primitive}

One of the advantages of \latex3 is that it provides new names for all the primitives. This enables
one to check if a primitive has been redefined and to provide suitable tests and replacements.

 If it is a primitive we can find out, using yet another boolean construction \docAuxCommand*{token_if_primitive:NTF}  We can also check its meaning. It is interesting to note that \docAuxCommand*{par} is not a macro. Interestingly we can view what \tex does when we say |\csname somecs\endcsname|. It justs sets it equal to |\relax|. 
 
 Again this is important in parsing and in automating the generation of commands. For example  in the |phd| package, we allow for a key value to be entered either as a control sequence for example, |\Large| or simply as a |large|. A test could be provided before further processing such type of input.

\begin{texexample}{Test if is a macro}{ex:ifprimitive1}
\ExplSyntaxOn
\makeatletter
\token_to_meaning:N \par\\
\token_to_meaning:N \toks
\token_if_primitive:NTF \par       { \PASS } { \FAIL }\\
\token_if_primitive:NTF \@@par     { \PASS } { \FAIL }\\
\token_if_primitive:NTF \tex_par:D { \PASS } { \FAIL }
\makeatother
\ExplSyntaxOff

\end{texexample}

Example~\ref{ex:ifprimitive1} can be used to test if a primitive has been redefined (this can be important for your code and to restore its meaning if necessary or issue an error message.  Another test which is available is to check if a token is a macro. 


\begin{texexample}{Test if a cs is primitive}{ex:primitive}
\ExplSyntaxOn
\group_begin:
\makeatletter
% LaTeX2e normally defines this as @par. 
% Use \par in a group to test.
\def\par{\let\par\@@par\par}
\token_if_primitive:NTF \@par { \PASS } { \FAIL }
\makeatother
\group_end:
\ExplSyntaxOff
\end{texexample}

\subsection{Test for category codes}

The next set of available commands are helper functions equivalent to the output of |\ifcat| 

\begin{docCommand} {token_if_group_begin:NTF} {\meta{token} \marg{true code} \marg{false code}}
Tests if \meta{token} has the category code of a begin group token (\{) when normal TEX
category codes are in force). Note that an explicit begin group token cannot be tested in
this way, as it is not a valid N-type argument. To test it you have to use |\c_group_begin_token|. This is mostly
used in conjuction with |futurelet| type constructions and or parsing.
\end{docCommand}


\begin{texexample} {Test if group begin} {ex:ifgroubbegin}
\ExplSyntaxOn
 \token_if_group_begin:NTF \c_group_begin_token { \PASS } { \FAIL }
 \token_if_group_end:NTF   \c_group_end_token   { \PASS } { \FAIL }\par
 \the\catcode`{
\ExplSyntaxOff
\end{texexample}

Behind the scenes |expl3| uses the |\ifcat| primitive to test the token against the catcode values. Constructions for all categories are available and summarized in the test below rather than described.
\begin{texexample} {Test if group begin} {ex:ifgroubbegin}
\ExplSyntaxOn
 \token_if_group_begin:NTF \c_group_begin_token { \PASS } { \FAIL }
 \token_if_group_end:NTF   \c_group_end_token   { \PASS } { \FAIL }\par
 \token_if_alignment:NTF   \c_alignment_token   { \PASS } { \FAIL }\par
 \token_if_parameter:NTF   \c_parameter_token   { \PASS } { \FAIL }\par
\ExplSyntaxOff
\end{texexample}

Use the constant form of these tokens to avoid errors and to make the code more readable.

The module is feature rich, with too many functions to remember easily. If your code needs to deal
with too many changes of catcodes, lccodes and the like, you will have to study it carefully.



\section{LaTeX3 Futurelet type functions}

In Chapter Futurelet, we spend considerable effort to understand how \tex’s futurelet macro works. There is often a need to look ahead at the next token in the input stream while leaving
it in place. This is handled using the “peek” functions. The generic \docAuxCommand*{peek_after:Nw} is
provided along with a family of predefined tests for common cases. As peeking ahead does
not skip spaces the predefined tests include both a space-respecting and space-skipping
version.

\begin{texexample}{Peek ahead ignoring spaces} {}
\ExplSyntaxOn
\peek_catcode_remove_ignore_spaces:NTF =  
    { 
      \PASS  
      \token_if_letter:NTF
          {l_peek_token ~= ~\token_to_meaning:N \l_peek_token \\  } 
          {   }
    } 
    { \FAIL }  
 = abcde \\
\ExplSyntaxOff
\end{texexample}

Most applications would require to recursively pick up tokens from the input stream and only terminated once a special token is found. This is the most powerful method to parse input strings and create really powerful functions. 

You will understand better if we hide the code in a function.

\begin{texexample}{Peek ahead ignoring spaces} {ex}
\ExplSyntaxOn
\cs_new:Npn \checkletter #1 {
\peek_catcode_remove_ignore_spaces:NTF #1  
    { 
      \PASS  
      \token_if_letter:NTF
          {l_peek_token ~= ~\token_to_meaning:N \l_peek_token \\  } 
          {   }
    } 
    { \FAIL } }

\checkletter {=} =abcde \par
\checkletter {A} Abcde \par
\ExplSyntaxOff
\end{texexample}

\begin{texexample}{Peek ahead ignoring spaces} {}
\ExplSyntaxOn
\cs_set:Npn \check_letter_and_removeall #1 {
\peek_catcode_remove_ignore_spaces:NTF #1  
    { 
      \PASS  
      \removeallaux:w  
    } 
   { \FAIL } 
 }

\cs_set:Npn \removeallaux:w #1; { removed~#1~ }

\check_letter_and_removeall {W}  W 12pt; \par
\ExplSyntaxOff
\end{texexample}

In the next example we will try and remove from the input stream recursively any |;|.
Tests if the next non-space token in the input stream has the same character code as
the test token (as defined by the test \cs{token_if_eq_charcode:NNTF}). Explicit and
implicit space tokens (with character code 32 and category code 10) are ignored and
removed by the test and thehtokeni is removed from the input stream if the test is true.
The function then places either the htrue codei or hfalse codei in the input stream (as
appropriate to the result of the test).
\begin{texexample}{ex:recursivefl}  { }                            
\ExplSyntaxOn

\cs_set:Npn \remove_colon: #1 {
   \peek_charcode_remove_ignore_spaces:NTF#1 
   {
    \TRUE
    \meaning #1 \par
    \peek_charcode_remove_ignore_spaces:NTF#1
   } 
   {
    \meaning#1
    \FALSE
     
   }
}

\remove_colon:;;;;;;;!
\ExplSyntaxOff
\end{texexample}

\subsection{Using higher functions}

\cs{peek_catcode_collect_inline:Nn}\Arg{test token}\Arg{inline code}. Collects and removes tokens from the input stream until finding a token that does not match the \meta{test token}. The colected tokens are passed to the \meta{inline code} as |#1|.   

In the example we collect tokens until we reach the comma (\ExplSyntaxOn\char_value_catcode:n{`\,}\ExplSyntaxOff) character which does not have the same category code as Z ({\ExplSyntaxOn\char_value_catcode:n{`\Z}\ExplSyntaxOff)}. We store the results in the |\grubber|.



\begin{texexample}{Collect tokens}{}
\ExplSyntaxOn

\cs_set:Npn \decorate_and_remove {
    {\space\bfseries \tl_use:N \g_tmpa_tl}
   }

\cs_set:Npn \collect_letters {
  \peek_catcode_collect_inline:Nn Z {\tl_put_right:Nn \g_tmpa_tl {##1}}
}

\cs_set:Npn \collect_others  {
  \peek_catcode_collect_inline:Nn ; {\tl_put_right:Nn \g_tmpb_tl {##1}}
}

\cs_set:Npn \maybe_first_is_surname:w #1   
  {  
    % Clear any contents from the token list 
    \tl_clear:N \g_tmpa_tl
    
    % Collect any letters until catcode is other
    \cs_set:Npn \result {\peek_catcode_collect_inline:Nn Z {\tl_put_right:Nn \g_tmpa_tl {####1}}#1}
    
    \def\removecomma##1##2;;{
      #2
    }
    \removecomma\result;;
    \tl_if_empty:NTF \g_tmpa_tl {\TRUE}
        {}
  }
  
\maybe_first_is_surname:w {Lazarides, Yiannis} 
{\bfseries \tl_use:N \g_tmpa_tl}
    
{\color{red}\tl_use:N \g_tmpb_tl}

%\maybe_first_is_surname:w {Lazarides Yiannis} ;

%\maybe_first_is_surname:n { Lazarides, Yiannis\par }
%
%\maybe_first_is_surname:n { Yiannis;Lazarides   }\par

\ExplSyntaxOff
\end{texexample}






















%
% 
\chapter{GROUPING AND SCOPING RULES}
\index{Grouping}
\label{ch:grouping}

Like most computer languages \tex\ has a scoping mechanism that is able to confine most changes to a particular locality. This chapter explains what sort of actions can be local, and how groups are formed.
\medskip

\begin{docCommand}{bgroup}{}
Implicit beginning of group character.
\end{docCommand}

\begin{docCommand}{egroup}{}
 Implicit end of group character.
 \end{docCommand}

\begin{docCommand}{begingroup}{}
 Open a group that must be closed with |\endgroup|.
\end{docCommand}

\begin{docCommand}{endgroup}{} 
Close a group that was opened with |\begingroup|.
\end{docCommand}

\begin{docCommand}{aftergroup}{} 
Save the next token for insertion after the current group ends.
\end{docCommand}

\begin{docCommand}{global}{}
 Make assignments, macro definitions, and arithmetic global.
\end{docCommand} 

\begin{docCommand}{globaldefs}{}
 Parameter for overriding |\global| prefixes. IniTEX default: 0.
\end{docCommand}



The grouping mechanism can be thought of a bit like scope in other programming languages, with the
exception that in \tex the mechanism is much more Pascal-like. Most assignments made inside a group are local to that group
unless explicitly indicated otherwise, and outside the group old values are restored (pretty much like in Pascal). 

The most common way to group a portion of your program is to use braces. If we type the following  example:

\begin{texexample}{}{}
\def\i{42} 

{
  \def\i{43}
  \def\b{2}
}

The value of the \textbackslash i is now \i

\def\x{a}
\let\y\x
\bgroup
  \def\x{b}
  Within group \x\par
\egroup
  Outside group \x
\end{texexample}
We get   \texttt{The value of the \textbackslash i is now 42}. Due to the way \tex scoping rules work, the old program state
will be restored \textit{completely} after returning from the local group. Neither the change to |\i| nor the definition of |\b| will survive. This is also true for register changes or other assignments.



\section{Local and global assignments}

An assignment or macro definition is usually made global by prefixing it with \cs{global}, but nonzero
values of the integer parameter |globaldefs| override |doccmd{global}|
is positive every assignment is implicitly prefixed with \docAuxCommand{global}, and if |\globaldefs| is negative,
|\global| is ignored. Ordinarily this parameter is zero. It has very
limited use and even in the \latex\ kernel we can only find 3-4 uses when defining math fonts.\footnote{In file \texttt{ltfssbas.dtx}.}


Some assignment are always global: the \marg{global} assignments are:

\begin{description}
\item[font assignment] assignments involving \cs{fontdimen}, \cs{hyphenchar}, and \cs{skewchar}.

\item[hyphenation] assignment \cs{hyphenation} and \cs{patterns} commands.

\item[hbox size assignment] altering box dimensions with \cs{ht}, \cs{dp}, and \cs{wd} 

\item[interaction mode assignment] run modes for a \tex job.

\item[intimate assignment] assignments to a special integer or special dimen
\end{description}

\section{Braces}

The most common way to group is to use braces. They are used for two purposes:

\begin{enumerate}
\item to indicate the start and end of a group. For example |{\small here is some text}|.

\item to indicate that a string of tokens should be treated as one unit. For example in |\def\abc{...}| the braces are used
to delimit the argument.
\end{enumerate}

It is important to note that the characters `\{', `\}' are not hardwired in \tex. Any tokens with catcodes 1 and 2 can be used.
The plain format starts [343] by defining:

\begin{teX}
\catcode`\{ =1
\catcode `} = 2
\end{teX}

Tokens with catcodes 1 and 2 are called \emph{explicit braces}. An \emph{implicit} brace is a control sequence whose replacement text is an explicit brace. Thus the two |plain| control sequences 
|\bgroup| and |\egroup| are implicit braces. 

There is also a low-level \tex operator pair for creating groups. It works
just as the braces. A group is started with \cs{begingroup} and ended with
\cs{endgroup}. These operators may be freely mixed with braces but pairs
should be properly matched. So |{ \begingroup \endgroup }| is allowed
but |{ \begingroup } \endgroup| is not.

\begin{teX}
\let\bgroup={
let\egroup=}
\end{teX}

They can be used where unbalanced braces are needed.

Salomon gives an example to typeset a number of paragraphs with a negative indentation\footnote{This style can sometimes be found in old books.}:

\begin{teX}
\def\negIndent{\brgoup\parindent=-20pt}
\def\endIndent{\par\egroup}

\negIndent
  \small\lipsum[1]
\endIndent
\end{teX}

This will typeset:

\def\beginindent{\bgroup\parindent=-20pt}
\def\endindent{\par\egroup}

\beginindent
  \small\lipsum[1-3]
\endindent

\section{Forming Groups Using \textbackslash begingroup and \textbackslash endgroup} 

The other two primitives \docAuxCommand{begingroup} and \docAuxCommand{endgroup} can also be used to define a group. However a group that starts with a |\begingroup| must end with an |\endgroup|. This provides a mechanism for error checking, which \tex's parsing routines can easily catch.

Note that |\begingroup| and |\endgroup| can only be used to define a group, not to delimit a string. You can say:

\begin{teX}
\begingroup
  \it abc
\endgroup
\end{teX}

but the following will get \tex to complain about missing braces

\begin{teX}
\hbox\begingroup\it abc\endgroup
\end{teX}

It should be pointed out that |\begingroup| and |\endgroup| do not really
add any new grouping functionality that could not be provided by curly braces
or |\bgroup| and |\egroup|. On the other hand, these two instructions are very
useful in nested groups of complicated structures, where one wants to make sure
that a certain "begin group instruction" is matched by a certain "end group
instruction." For this pair of grouping instructions, and this pair only, use |\begingroup|
and |\endgroup|. In case a |\begingroup| is not matched by a |\endgroup|,
an error is generated by \tex.\footcite{bechto1993} 

The case when not to use |begingroup| is clear. However, if one should use it for cases where
|\bgroup| is possible, is a subject with different opinions.\footnote{See \url{https://tex.stackexchange.com/questions/1930/when-should-one-use-begingroup-instead-of-bgroup/1932\#1932}.} Unless you are using |mathmode| or have deeply nested structures, |bgroup| is fine to use. In all
other cases it is preferable to use |\begingroup|.

\section*{Examples}
From the TexBook Exercise 7.4

Suppose that the commands
\begin{texexample}{}{}
{\catcode`\<=1 \catcode`\>=2
 \bfseries test
>
 test
\end{texexample}

appear near the beginning of a group that begins with |{| these specifications instruct
TEX to treat |<| and |>| as group delimiters. According to \tex's rules of locality, the
characters |<| and |>| will revert to their previous categories when the group ends. But
should the group end with |}| or with |>| ?

It ends with either |>| or |}| or any character of category 2; then the effects of all
\cs{catcode} definitions within the group are wiped out, except those that were global.
\tex  doesn't have any built-in knowledge about how to pair up particular kinds of
grouping characters. New category codes take effect as soon as a |\catcode| assignment
has been digested. For example,

\begin{teX}
{\catcode`\>=2 >
\end{teX}

is a complete group. But without the space after |2|  it would not be complete, since TEX
would have read the |>|  and converted it to a token before knowing what category code
was being specified; \tex always reads the token following a constant before evaluating
that constant.

\topline

\textbf{Example}: \textsc{Adjusting the spacing of a font} An interesting example that illustrates some of the concepts that were discussed so far is to try and change the \textit{inter word spacing} of text using the \cs{fontdimen2} parameter. The interesting aspect of this example is that
we want to change the spacing, but since the font changes are global, we want to revert back to the original font at the end of the group. Although there are many other ways of achieving this we will use the \cs{aftergroup}.

\begin{teX}
\font \roman=cmr10
\font\specroman=cmr10
%% Next, the special registers
\newdimen\savedvalue
\savedvalue=\fontdimen2\roman
\newdimen\specialvalue
\specialvalue=13.0pt
%% Finally, definitions.
\def \rm{%
  \fontdimen2\roman=\savedvalue }
\def\specrm{%
  \aftergroup\restoredimen
  \fontdimen2\specroman=\specialvalue
  \specroman  }
\def\restoredimen{%
\fontdimen2\roman=\savedvalue }
\end{teX}
{
%% First, fonts.
\font \roman=cmr10
\font\specroman=cmr10
%% Next, the special registers
\newdimen\savedvalue
\savedvalue=\fontdimen2\roman
\newdimen\specialvalue
\specialvalue=13.0pt
%% Finally, definitions.
\def \rm{%
  \fontdimen2\roman=\savedvalue }
\def\specrm{%
  \aftergroup\restoredimen
  \fontdimen2\specroman=\specialvalue
  \specroman  }
\def\restoredimen{%
\fontdimen2\roman=\savedvalue }


{\bf Spaced Out Text} 
\medskip
{\specrm \lorem} dimension2 the interword   value \the\fontdimen2\font


{\bf  Back to Normal}
\medskip

\rm
\lorem

}

\section{\textbackslash aftergroup}

The \cs{aftergroup} control sequence saves a token for insertion after the current group. Several
tokens can be set aside by this command, and they are inserted in the left-to-right order in which
they were stated.

\begin{texexample}{}{}
\def\x#1;{#1}
\def\y{15}
{\globaldefs1
\bgroup
   \def\y{0}
   \aftergroup\x\aftergroup\y\aftergroup;
   \aftergroup}
\egroup
\y


\globaldefs0

\def\z{1}
{\def\z{0}
\z
}

\z

\end{texexample}

\begin{texexample}{}{}
{ \def\z{1}
  {\def\z{0}\globaldefs1
     \z
    {
	\z
    }
   \z
  }
 \z
}
\end{texexample}
\section{afterassignment}

An interesting primitive is \docAuxCommand{afterassignment}. The primitive saves the token immediately following it without
expansion. Nothing happens until after the next assignment; immediately after the next assignment the saved token is expanded.

\begin{texexample}{Aftergroup}{ex:aftergroup}
\def\yy{%
  \afterassignment\yyb
  \let\yyDiscard = 
}

\def\yyb{%
 ``%
 \bgroup
 \itshape
 \aftergroup\yyc
}
\def\yyc{%
  ''%
}

\yy{This is a test}  
\end{texexample}

The above example is not a very common or idiomatic way of writing macros. So what is |\afterassignment| good for? Its main use is to write macros with \enquote{arguments} similar to the way \tex assigns registers. Afterassignment allow you to define macros which avoid curly braces to enclose arguments.

The most common use of |\afterassignment| is in a macro whose parameter is glue or dimen. Consider the definition of a macro such as:
\begin{quote}
 |\def\myglue#1{\leftskip=#1 \rightskip=#1}|
\end{quote}

Such a macro can be called as |\myglue{3pt plus5pt minus3pt}|, but if we want to keep the same conventions as \tex we might prefer to have the ability to call it as |\myglue 3pt plus5pt minus3pt|. To achieve this we can do:

\begin{texexample}{Afterassignment}{ex:afterassignment}
\bgroup
\font\larger=cmr10 scaled\magstep1
\larger
\newskip\tempskip
\def\myglue{\afterassignment\myglueaux \tempskip}
\def\myglueaux{\leftskip=\tempskip \rightskip=\tempskip}
\myglue=30pt plus1pt minus1pt
\lorem\par
\egroup
\lorem
\end{texexample}



\section{Scoping Rules for boxes}

The scoping rules for boxes work similarly to those for other command sequences, since they are just macros defined by \latex or |plain|. In the example below, we define a box |\mybox| and we save a sentence both in global scope as well as local scope.

\begin{teX}
\documentclass{article}
\begin{document}
  \newsavebox{\mybox}
  \savebox{\mybox}{Outside scope}
  \usebox\mybox
  \begin{minipage}{5cm}
    \sbox{\mybox}{from first minipage}(*@ \label{global} @*)
    \usebox\mybox
  \end{minipage}
  \usebox{\mybox}
\end{document}
\end{teX}


This will typeset:
\medskip

\newsavebox{\myboxi}
\savebox{\myboxi}{\tt > Outside scope}

\noindent\usebox\myboxi

\noindent\begin{minipage}{5cm}
\sbox{\myboxi}{\tt > from first minipage}
\noindent\usebox\myboxi
\end{minipage}

\noindent\usebox{\myboxi}


\medskip 
Changing line [\ref{global}] to |\global\sbox| will make the definition of |\mybox| within the minipage environment global and would change the output to:
\medskip


To save memory space, box registers become empty by using them: \tex assumes
that after you have inserted a box by calling |\boxnn| in some mode, you do not need the contents of that register any more and empties it. In case you do need the contents of a box register more
than once, you can |\copy| it. Calling |\copynn| is equivalent to |\boxnn| in all respects except that the register is not cleared.


There are 256 box registers, numbered 0–255. Either a box register is empty (‘void’), or it contains
a horizontal or vertical box. This section discusses specifically box registers; the sizes of boxes,
and the way material is arranged inside them, is treated below.




\newbox\MyBox

\setbox\MyBox=\hbox{\hfil Test\hfill}

\unhbox\MyBox


\noindent\unhbox\MyBox

\noindent{\hfill Test \hfill}



\framebox{\parbox{\linewidth}{\color{theblue}
\textbf{\textcolor{purple}{\textsf{CAUTION}}}
\begin{enumerate}
\itemsep-5pt
\item \latex will not empty a box as it uses the \cs{copy} command in the definition of the \cs{newsavebox}.
\item It is better to use \LaTeX\ commands rather than \tex primitives, when defining boxes, as \latex tests for duplication of names - which is very important if a user uses a lot of different packages.
\item Give always preferences to local definitions rather than global. Globals always create maintenance problems in programming.
\end{enumerate}
}}


\section{Implicit Grouping}

There are  instances where grouping is \textit{implicit}. What this means is that \text starts and ends a group automatically and without any action by the user. There are two major cases where this happens:

\begin{enumerate}
\item The text inside a box such as |\hbox|, |\vbox|, |\vtop|, |\vcenter| etc. is automatically treated by \tex as a group.  For example |\hbox{\bf My Heading}|, will print  \hbox{\bf My Heading}  and it will not continue with the bold font once outside the group. All these commands have curly brackets and these curly brackets form implicit groups.
\item In five cases \tex forms implicit groups. In some of these cases not even curly braces are involved.
\end{enumerate}

\begin{enumerate}
\item The text inside math mode is treated as a group. This is true both for inline math as well as display math.
\item Matching |\left| and |\right| primitives treat the formula in between them as a group.
\item Fractions are treated as a group.
\item The execution of an ouput routine is implicitly enclosed in a group.
\item Columns in |\halign| based tables are local.
\end{enumerate} 

\subsection{\texttt{afterssignment and grouping}}

\begin{macro}{\afterassignment}
The primitive |\afterasignment| does not follow grouping in that it does not save the definition of a token when |\afterassignment| is executed. Consider the following example:
\end{macro}

Define the two macros |\xx| and |\yy|.

\begin{texexample}{afterassignment}{}
\def\xx{\string\xx\ executed\par }

\def\yy{\string\yy\ executed\par }

\afterassignment\xx
\end{texexample}

We start a group, where we have two definitions of |\xx| and |\yy|

\begin{texexample}{afterassignment}{}
\def\yy{42}
{
  \def\xx{\string\xx executed inside a group\par}

  \def\yy{\string\yy executed inside a group\par}

The second afterassignment is execute

  \afterassignment\yy

The group is ended

}
\end{texexample}

Note \cs{afterassignment} saves the token following \cs{afterassignment} without expanding it. Nothing happens until after the next assignment; immediately after the next assignment the saved token is expanded. This is a bit of a tricky part and you can go over it to make sure you understand it well.
\footnote{\url{http://tug.org/TUGboat/tb32-2/tb101grunewald.pdf}}
\footnote{\url{http://tex.stackexchange.com/questions/65462/plain-tex-theory-afterassignment}}


\begin{texexample}{Combining bgroup and begingroup}{}
\begingroup
\newbox\savedparbox

\def\saveparbox{\par\begingroup
  \def\par{\egroup\endgroup}
  \global\setbox\savedparbox\vbox\bgroup}

Ordinary paragraph.
\saveparbox
This paragraph will be saved in \string\box\string\savedparbox.
If you wish, you can unpack the box and do all kinds of processing on it.
In this demo, I won't do any processing.
Look in the log file to examine the box contents.

Another ordinary paragraph.
\endgroup
\end{texexample}


































%\chapter{Verbatim}
\label{ch:verbatim}
\epigraph{Please do not attempt to simplify this code.\\
Keep the space shuttle flying.}{Note form the Kubernettes code at github.com}

\section{Introduction}

The verbatim environment uses the monospaced |\ttfamily| font, turns blanks into spaces, starts a new line  for each carriage return and interprets \emph{every}  character literally. What this means is that all special characters are catcoded to other.

The command \cs{verb} produces in-line verbatim text, where the argument is delimited by any pair of characters.

The star variants of the commands are the same, except that spaces print as in the \tex{}book's space character instead of as blank spaces such as \verb* +test this+. This is font dependent. Some of the fonts may not contain the character.

 \LaTeX's \texttt{verbatim} and \texttt{verbatim*} environments
 have a few features that may give rise to problems. These are:
 \begin{itemize}
   \item
     Due to the method used to detect the closing |\end{verbatim}|
     (i.e.\ macro parameter delimiting) you cannot leave spaces
     between the |\end| token and |{verbatim}|.
   \item
     Since \TeX{} has to read all the text between the
     |\begin{verbatim}| and the |\end{verbatim}| before it can output
     anything, long verbatim listings may overflow \TeX's memory.
 \end{itemize}
 Whereas the first     of these points can be considered
 only a minor nuisance the other one is a real limitation, for the older engines. For the newer engines this is not a real issue now.
 
The package \pkg{verbatim} eliminates the above limitatiosn by searching for the |\end{verbatim}| string rather than absorbing the full argument of the pseudo environment. The \latexe verbatim code just absorbs the whole argument.

The package \pkg{Fancyvrb} provides a more advanced interface, but also has its limitations.

Full parsers such as listings are discussed in other sections of this Monograph.




\section{LaTeX's verbatim environment}



The environment and the command are provided in source2e under the \docFile{ltmiscen.dtx}.

A number of other packages were also developed over the years. These will be reviewed here as well as add examples as to how to handle the programming of such matters.

\section{How verbatim code works?}

\begin{texexample}{vobeyspaces}{ex:obeyspaces}
\makeatletter
\begingroup
\obeylines
\@xobeysp

\ttfamily
         This is some text
           Try any shape of line       test

           
\endgroup


\makeatother
\end{texexample}
\subsection{Verbatim}


  The verbatim environment uses the fixed-width |\ttfamily| font, turns
  blanks into spaces, starts a new line for each carriage return (or
  sequence of consecutive carriage returns), and interprets
  \emph{every} character literally.
  I.e., all special characters |\, {, $|, etc.
   are |\catcode|'d to 'other'.

 The command |\verb| produces in-line verbatim text, where the argument
 is delimited by any pair of characters.  E.g., |\verb #...#| takes
  `|...|' as its argument, and sets it verbatim in |\ttfamily| font.

The *-variants of these commands are the same, except that spaces
print as the \TeX{}book's space character instead of as blank spaces.

\begin{macro}{\@vobeyspaces}
\begin{teX}
{\catcode`\ =\active%
\gdef\@vobeyspaces{\catcode`\ \active\let \@xobeysp}}
\end{teX}
\end{macro}
%
\begin{macro}{\@xobeysp}
Moved to ltspace.dtx
% \changes{v1.0z}{1995/07/13}{Use \cs{nobreak}}
% \changes{v1.1f}{1996/09/28}{Moved to ltspace.dtx}
\end{macro}
%
The next two macros, which follow conventions for star and unstarred versions
use a common trickery to define environments that capture their contents
in a pseudo-environment.

Just to refresh a delimited macro can be delimited essentially with almost anything except 
braces, \# and some other esoteric characters that can confuse \tex's parser. In the 
example below we will create a delimited macro \cs{test}, which is delimited by
\cs{endtest}. 

\begin{texexample}{Delimited Refresher}{ex:delim1}
\def\test#1\endtest{#1}
\test
  100
\endtest
\end{texexample}

If the braces are changed to the catcode for other so that they can be used
to delimit the macro and by adding a matching end{} we can trick LaTeX that everything
is in order.

Our exampel will be a bit more complicated than our normal examples. We will develop a rudimentary environment named \docAuxEnvironment{zverbatim} that will capture the user input, typeset it verbatim and colorize comment strings.

We will mostly use |expl3| syntax to continue our study of the language and also to make the code a bit more readable. The description of the functions are listed below:



\begin{texexample}{}{}
\ExplSyntaxOn
\tl_new:N \g_store_tl

\cs_new:Npn \make_at_letter:
 {
  \char_set_catcode_letter:N @
 }
  
\cs_new:Npn \make_at_other:
 {
  \char_set_catcode_letter:N @
 }  
 
\cs_new:Npn \make_other:n #1 
  {
 		\char_set_catcode_other:N #1 \relax
  } 
  
\make_at_letter:
\begingroup
\char_set_catcode_escape:N | 
\char_set_catcode_group_begin:N [
\char_set_catcode_group_end:N ] 
\char_set_catcode_other:N\{
\char_set_catcode_other:N\}
\char_set_catcode_other:N\\


|cs_gset:Npn|phd_xverbatim:n#1\end{zverbatim}
            [|process_argument[#1]|end[zverbatim]]
|endgroup


\cs_set:Npn \zverbatim_aux: 
   {
     \bfseries
     \let\do\make_other:n\dospecials
     \obeylines \verbatimfont \@noligs
     \hyphenchar\font\m@ne
   }

% just set some preliminaries with
% |\@zverbatim|
\cs_gset:Npn \zverbatim
   {
     \zverbatim_aux: 
     \phd_xverbatim:n
   }

% Cannot do anything more at this time
\cs_gset:Npn \endzverbatim {}

% Process the argument
\gdef\process_argument#1 {
   
   \tl_set:Nn\g_store_tl{#1}
  
   \regex_replace_all:nnN 
   {\%[a-zA-Z\ \#\d]+}
   { \cB{\c{color}\cB{green800\cE}\0\cE} }
   \g_store_tl
   \tl_use:N \g_store_tl
}
\make_at_other:

\ExplSyntaxOff


% Output to test that everything is # working
\begin{zverbatim}
% Test to see a comment #1  
    \testing \our 
      \macrotest{\someother}
\end{zverbatim}


\end{texexample}
 

Now that we have more or less understood how verbatim pseudo-environments are
created, let us examine the \latexe code a bit more carefully and then we will
revisit our example in a bit more detail to add a few more features.


Catcodes are set in a group and the macros |\@xverbatim| and |\@sverbatim|
are defined exactly in a similar fashion to that of our example.

\begin{macro}{\@xverbatim,\@sxverbatim}
\begin{teX}
\begingroup \catcode `|=0 \catcode `[= 1
\catcode`]=2 \catcode `\{=12 \catcode `\}=12
\catcode`\\=12 |gdef|@xverbatim#1\end{verbatim}[#1|end[verbatim]]
|gdef|@sxverbatim#1\end{verbatim*}[#1|end[verbatim*]]
|endgroup
\end{teX}
\end{macro}


The next macro is the starting macro that is call by |\verbatim| at the beginning
of the environment. It is responsible for setting decorative parameters. This is
done by the use of a |trivlst|. 

\begin{macro}{\@verbatim}
\begin{teX}
\def\@verbatim{\trivlist \item\relax
  \if@minipage\else\vskip\parskip\fi
  \leftskip\@totalleftmargin\rightskip\z@skip
  \parindent\z@\parfillskip\@flushglue\parskip\z@skip
\end{teX}
% \changes{LaTeX2.09}{1991/08/26}{\cs{@@par} added}
    Added |\@@par| to clear possible |\parshape| definition
    from a surrounding list (the verbatim guru says).
% \changes{v0.9p}{1994/01/18}
%         {Only add \cs{penalty} if in hmode}
\begin{teX}
  \@@par
  \@tempswafalse
  \def\par{%
    \if@tempswa
\end{teX}
    A |\leavevmode| added: needed if, for example, a blank verbatim
    line is the first thing in a list item (wow!).
% \changes{v1.0f}{1994/04/29}{\cs{leavevmode} added}
\begin{teX}
      \leavevmode \null \@@par\penalty\interlinepenalty
    \else
      \@tempswatrue
      \ifhmode\@@par\penalty\interlinepenalty\fi
    \fi}%
\end{teX}
    To allow customization we hide the font used in a separate macro.
  \changes{v0.9a}{1993/11/21}{use \cs{verbatim@font} instead of \cs{tt}}
  \changes{v0.9h}{1993/12/13}{Readded \cs{@noligs}}
  \changes{v1.1d}{1996/06/03}{Exchanged the following two code lines
           so that \cs{dospecials} cannot reset the category code
           of characters handled by \cs{@noligs}.}
  \changes{v1.1h}{2000/01/07}{Disable hyphenation even if the font allows it.}
\begin{teX}
  \let\do\@makeother \dospecials
  \obeylines \verbatim@font \@noligs
  \hyphenchar\font\m@ne
\end{teX}
To avoid a breakpoint after the labels box, we remove the penalty
put there by the list macros: another use of |\unpenalty|!
% \changes{v1.0f}{1994/04/29}{Change to \cs{everypar} added}
\begin{teX}
  \everypar \expandafter{\the\everypar \unpenalty}%
}
\end{teX}
\end{macro}
%

\begin{docCommand}{verbatim}{}
\begin{docCommand}{endverbatim}{}
%    (RmS 93/09/19) Protected against `missing item' error message
%               triggered by empty verbatim environment.
\begin{teX}
\def\verbatim{\@verbatim \frenchspacing\@vobeyspaces \@xverbatim}
\def\endverbatim{\if@newlist \leavevmode\fi\endtrivlist}
\end{teX}
\end{docCommand}
\end{docCommand}
%
\begin{macro}{\verbatim@font}
    Macro to select the font  used for verbatim typesetting.
    It also does other work if necessary for the font used.
\begin{teX}
\def\verbatim@font{\normalfont\ttfamily}
\end{teX}
\end{macro}


\begin{environment}{verbatim*}
\begin{teX}
\@namedef{verbatim*}{\@verbatim\@sxverbatim}
\expandafter\let\csname endverbatim*\endcsname =\endverbatim
\end{teX}
\end{environment}

The following code needs no explanation, in our example we collected it at the beginning
of our code with the possible intention to make a small package with utilities.
\begin{macro}{\@makeother}
\begin{teX}
\def\@makeother#1{\catcode`#112\relax}
\end{teX}
\end{macro}

\begin{question}
\begin{tasks}(1)
 \task What is the main disantvantages of the method used here to define
       the |verbatim| environment?
 \task Using our example, try and insert verbatim code into a list. Is this satisfactory.
\end{tasks}
\end{question}


\begin{texexample}{}{}
\begin{itemize}
\item First verbatim
      \begin{verbatim}
      \lorem
      \end{verbatim}
\item Second verbatim
      \begin{verbatim}
      Second verbatim
       \end{verbatim}
\end{itemize}

\end{texexample}

\subsection{Definition of inline verbatim \textbackslash verb}

\begin{macro}{\verb@balance@group}
% \changes{LaTeX2.09}{1993/09/07}
%     {(RmS) Changed definition of \cs{verb} so that it detects a
%              missing second delimiter.}
\begin{teX}
\let\verb@balance@group\@empty
\end{teX}
\end{macro}
%
\begin{macro}{\verb@egroup}
\begin{teX}
\def\verb@egroup{\global\let\verb@balance@group\@empty\egroup}
\end{teX}
\end{macro}
%
\begin{macro}{\verb@eol@error}
\begin{teX}
\begingroup
  \obeylines%
  \gdef\verb@eol@error{\obeylines%
    \def^^M{\verb@egroup\@latex@error{%
            \noexpand\verb ended by end of line}\@ehc}}%
\endgroup
\end{teX}
\end{macro}
%
\begin{macro}{\verb}
% \changes{LaTeX2.09}{1992/08/24}
%         {Changed \cs{verb} and \cs{@sverb} to work correctly
%            in math mode}
% \changes{v0.9a}{1993/11/21}{Use \cs{verbatim@font} instead of
%                             \cs{tt}.}
% \changes{v1.1a}{1995/09/19}{Put \cs{@noligs} after
%                    \cs{verbatim@font} where it belongs.}
%    Typesetting a small piece verbatim.
%  \changes{v1.1d}{1996/06/03}{Put setting of verbatim font after
%           \cs{dospecials}
%           so that \cs{dospecials} cannot reset the category code
%           of characters handled by \cs{@noligs}.}
\begin{teX}
\def\verb{\relax\ifmmode\hbox\else\leavevmode\null\fi
  \bgroup
    \verb@eol@error \let\do\@makeother \dospecials
    \verbatim@font\@noligs
    % common latex method for defining star and unstarred commands
    \@ifstar\@sverb\@verb}
\end{teX}
\end{macro}


\begin{macro}{\@sverb}
\begin{teX}
\def\@sverb#1{%
  \catcode`#1\active
  \lccode`\~`#1%
  \gdef\verb@balance@group{\verb@egroup
     \@latex@error{\noexpand\verb illegal in command argument}\@ehc}%
  \aftergroup\verb@balance@group
  \lowercase{\let~\verb@egroup}}%
\end{teX}
\end{macro}

%
\begin{macro}{\@verb}
\begin{teX}
\def\@verb{\@vobeyspaces \frenchspacing \@sverb}
\end{teX}
\end{macro}
%
\begin{macro}{\verbatim@nolig@list}

\begin{teX}
\def\verbatim@nolig@list{\do\`\do\<\do\>\do\,\do\'\do\-}
\end{teX}
\end{macro}
%
\begin{macro}{\do@noligs}
\begin{teX}
\def\do@noligs#1{%
  \catcode`#1\active
  \begingroup
     \lccode`\~`#1\relax
     \lowercase{\endgroup\def~{\leavevmode\kern\z@\char`#1}}}
\end{teX}
\end{macro}
%
\begin{macro}{\@noligs}
To stay compatible with packages that use |\@noligs| we keep it.
\begin{teX}
\def\@noligs{\let\do\do@noligs \verbatim@nolig@list}
\end{teX}
\end{macro}



\section{The fancyvrb package}


The code works by scanning a line at a time from an environment or a file. This allows it to pre-process each line, rejecting it, removing spaces, numbering it, etc., before going on to execute the body of the line with the appropriate catcodes set.

According to Poore\footcite{poore} the first public release of the package was in January 1998, but from the copyright note its first version must have been in 1994; irrespective the package has remained almost  unchanged except for a few bug fixes. \pkg{fancyvrb} has become one of the primary \latex packages for working with verbatim text.

Poore extended the package with \pkg{fvextra}. All commands defined by |fancyverb| can still be used, but additional
features and keys are made available. The package is used by minted and pythontex which are the only alternatives to
\pkg{listings}.


\begin{texexample}{Using fancyvrb}{ex:fancyvrb}
\begin{Verbatim}
  First verbatim line.
  Second verbatim line.
\end{Verbatim}
\end{texexample}

 \subsection{Fonts}

All four axis of the NFSS scheme can be set using a key value interface.

\begin{texexample}{Using fancyvrb}{ex:fancyvrblarge}
\begin{Verbatim}[commentchar=!,gobble=0,fontfamily=tt]
 A comment
 Verbatim line.
! A comment that you will not see
\end{Verbatim}

\begin{Verbatim}[commentchar=!,gobble=0,fontfamily=tt]
 A comment
 Verbatim line.
! A comment that you will not see
\end{Verbatim}
\end{texexample}



\begin{texexample}{Using fancyvrb}{ex:fancyvrb1}
\begin{Verbatim}[commentchar=!,gobble=0]
 A comment
 Verbatim line.
! A comment that you will not see
\end{Verbatim}
\end{texexample}


\begin{texexample}{Using fancyvrb}{ex:fancyvrb2}
 \fvset{gobble=0}
 \begin{Verbatim}[frame=single,
 label=My text]
 First verbatim line.
 Second verbatim line.
 \end{Verbatim}
\end{texexample}

\begin{texexample}{Using fancyvrb}{ex:fancyvrb3}
  \begin{Verbatim*}
   Verbatim line.
  \end{Verbatim*}
\end{texexample}

\begin{docKey}{defineactive}{ = \meta{code}}{default=[]}
 
\end{docKey}

\begin{texexample}{Adding an active character and a command}{ex:defineactive}
\begin{Verbatim}[frame=single,numbers=left,numbersep=3pt]
! A small "hello" program

program hello
  print *, "Hello World"
\end{Verbatim}

\fvset{frame=single,numbers=left,numbersep=3pt}
\begin{Verbatim}
! A small "hello" program

program hello
  print *, "Hello World"
\end{Verbatim}


\def\ExclamationPoint{\char33}
 \catcode`!=\active
 
\begin{Verbatim}[defineactive=\def!{\color{cyan}\itshape\ExclamationPoint}]
! A small "hello" program

program hello
  print *, "Hello World"
\end{Verbatim}
\end{texexample}

\subsection{Writing and reading verbatim to files.}

To write data verbatim to a file the environment \docAuxEnvironment{VerbatimOut} is available.
It takes one mandatory argument: the file name into which to write the data. If you try and use the
environment directly inside your own environment, the moment we start the |VerbatimOut| environment everythingis absorbed without processing and so the end of your own environment is not recognized.
As a solution the package offers the command |\VerbatimEnvironment|, which is executed within the |\begin|
code of your environment, ensures that the end tag of your environment will be recognized in verbatim mode and the corresponding code executed. To read the file back the command |\VerbatimInput| is used.

\begin{texexample}{Write and read verbatim to file}{ex:vrbout}
\newenvironment{myexample} 
{\VerbatimEnvironment\begin{VerbatimOut}{test.vrb}} 
{\end{VerbatimOut}
\VerbatimInput[numbers=left,firstnumber=2,firstline=2,fontshape=it,fontseries=b]{test.vrb}}% 

\begin{myexample}
first line
second line
third line
fourth line
\end{myexample}
\end{texexample}

Since we can input and type our verbatim text, it opens the oportunity to have self-running examples, similar to the example boxes I have been using for this book.

\begin{texexample}{Self-running examples}{ex:srexamples}
\newcommand\Example{%
\VerbatimEnvironment
\begin{VerbatimOut}{\jobname.vrb}}

\def\endExample{%
\end{VerbatimOut}
{\centering \input{\jobname.vrb}}
\VerbatimInput{\jobname.vrb}
\input{\jobname.vrb}
}

\begin{Example}
    \def\test{Some test}
    \test
\end{Example}

\end{texexample}



\subsection{Saving and re-using verbatim text}

The package can be used to save and re-use verbatim texts in a document. This enables us to place verbatim in normally inaccessible areas such as the argument of sectioning commands (not recommended) or |\marginpar|.

\DescribeMacro{\SaveVerb}{\oarg{key/val list}=data=}{}
The syntax of the command is unusual in that we need to write it the same way as we write
a |\verb| command. It takes one mandatory argument, a \textit{label} denoting the storage
bin in which to save the parsed data. 

\DescribeMacro{\UseVerb*}{\oarg{key/val-list}\Arg{label}}{}

Not recommended for general use.

\begin{texexample}{Saving Verbatim Text}{ex:saveverb}
\begin{minipage}{0.5\linewidth}
 \SaveVerb{danger}=\test \something=
A verbatim \footnote{\UseVerb*{danger}} test in a minpage.
 \end{minipage}
\end{texexample}

In the LaTeX Companion there is an example using the commands to define a macro |\vitem| that can be used
to replace |\item| 


\section{The alltt package}

This package defines the eponymous \docAuxEnvironment{alltt} environment, which is like the \docAuxEnvironment{verbatim} environment except that |\|, |{|, and |}| have their usual meanings.
Thus, other commands and environments can appear within an |alltt| environment.



\begin{texexample}{Using the alltt package}{ex:alltt} 
\begin{alltt} 
\rmfamily
One can have font changes, 
\emph{emphasized text}. 
Some special characters: # & $ ^ _
\lorem 
\end{alltt} 
\end{texexample}


Should you need to typeset mathematical material you will have to use the \latexe constructs |\[|\ldots|\]| as the math toggle |$| is disabled in an |alltt| environment. For subscripts or superscripts you  will need to use the little known \latexe commands \docAuxCommand{sb} or \docAuxCommand{sp} that refer to subscript or superscript respectively.\tcbdocmarginnote{Added Dec 2018}

\begin{texexample}{Using the alltt package}{ex:alltt1} 
\begin{alltt} 
\rmfamily
One can have font changes, 
\emph{emphasized text}. 
Some special characters: # & $ ^ _
\lorem 

\[ a\sb{2} + b\sb{5}  \]

\end{alltt} 
\end{texexample}




\DocInput{\jobname.dtx}
%\EnableImplementation
%\DocInputAgain{\jobname.dtx}
%\IndexInput{phddoc.sty}
%\nocite{*}
%\printbibliography

%\PrintIndex 
\end{document}
%</driver>
%\fi
%
% \DoNotIndex{\@,\@@par,\@beginparpenalty,\@empty}
% \DoNotIndex{\@flushglue,\@gobble,\@input}
% \DoNotIndex{\@makefnmark,\@makeother,\@maketitle}
% \DoNotIndex{\@namedef,\@ne,\@spaces,\@tempa}
% \DoNotIndex{\@tempb,\@tempswafalse,\@tempswatrue}
% \DoNotIndex{\@thanks,\@thefnmark,\@topnum}
% \DoNotIndex{\@@,\@elt,\@forloop,\@fortmp,\@gtempa,\@totalleftmargin}
% \DoNotIndex{\",\/,\@ifundefined,\@nil,\@verbatim,\@vobeyspaces}
% \DoNotIndex{\|,\~,\ ,\active,\advance,\aftergroup,\bgroup,\bgroup}
% \DoNotIndex{\mathcal,\csname,\def,\documentstyle,\dospecials,\edef}
% \DoNotIndex{\egroup}
% \DoNotIndex{\else,\endcsname,\egroup ,\endinput,\endtrivlist}
% \DoNotIndex{\exp_after:wN ,\fi,\fnsymbol,\futurelet,\cs_gset:Npn ,\global}
% \DoNotIndex{\hbox,\hss,\if,\if@inlabel,\if@tempswa,\if@twocolumn}
% \DoNotIndex{\ifcase}
% \DoNotIndex{\ifcat,\iffalse,\if_meaning:w ,\ignorespaces,\index,\input,\item}
% \DoNotIndex{\jobname,\kern,\leavevmode,\leftskip,\cs_set_eq:NN ,\llap,\lower}
% \DoNotIndex{\m@ne,\next,\newpage,\nobreak,\noexpand,\nonfrenchspacing}
% \DoNotIndex{\obeylines,\or,\protect,\raggedleft,\rightskip,\rm,\sc}
% \DoNotIndex{\setbox,\setcounter,\small,\space,\string,\strut}
% \DoNotIndex{\strutbox}
% \DoNotIndex{\thefootnote,\thispagestyle,\topmargin,\trivlist,\tt}
% \DoNotIndex{\twocolumn,\typeout,\vss,\vtop,\xdef,\z@}
% \DoNotIndex{\,,\@bsphack,\@esphack,\@noligs,\@vobeyspaces,\@xverbatim}
% \DoNotIndex{\`,\catcode,\end,\escapechar,\frenchspacing,\glossary}
% \DoNotIndex{\hangindent,\hfil,\hfill,\hskip,\hspace,\ht,\it,\langle}
% \DoNotIndex{\leaders,\long,\makelabel,\marginpar,\markboth,\mathcode}
% \DoNotIndex{\mathsurround,\mbox,\newcount,\newdimen,\newskip}
% \DoNotIndex{\nopagebreak}
% \DoNotIndex{\parfillskip,\parindent,\parskip,\penalty,\raise,\rangle}
% \DoNotIndex{\section,\setlength,\TeX,\topsep,\underline,\unskip,\verb}
% \DoNotIndex{\vskip,\vspace,\widetilde,\\,\%,\@date,\@defpar}
% \DoNotIndex{\[,\{,\},\]}
% \DoNotIndex{\count@,\if_int_compare:w,\loop,\today,\uppercase,\uccode}
% \DoNotIndex{\baselineskip,\begin,\tw@}
% \DoNotIndex{\a,\b,\c,\d,\e,\f,\g,\h,\i,\j,\k,\l,\m,\n,\o,\p,\q}
% \DoNotIndex{\r,\s,\t,\u,\v,\w,\x,\y,\z,\A,\B,\C,\D,\E,\F,\G,\H}
% \DoNotIndex{\I,\J,\K,\L,\M,\N,\O,\P,\Q,\R,\S,\T,\U,\V,\W,\X,\Y,\Z}
% \DoNotIndex{\1,\2,\3,\4,\5,\6,\7,\8,\9,\0}
% \DoNotIndex{\!,\#,\$,\&,\',\(,\),\+,\.,\:,\;,\<,\=,\>,\?,\_}
% \DoNotIndex{\discretionary,\immediate,\makeatletter,\makeatother}
% \DoNotIndex{\meaning,\newenvironment,\par,\scan_stop:,\renewenvironment}
% \DoNotIndex{\repeat,\scriptsize,\selectfont,\the,\undefined}
% \DoNotIndex{\arabic,\do,\makeindex,\null,\number,\show,\write,\@ehc}
% \DoNotIndex{\@author,\@ehc,\@ifstar,\@sanitize,\@title,\everypar}
% \DoNotIndex{\if@minipage,\if@restonecol,\ifeof,\ifmmode}
% \DoNotIndex{\lccode,\newtoks,\onecolumn,\openin,\p@,\SelfDocumenting}
% \DoNotIndex{\settowidth,\@resetonecoltrue,\@resetonecolfalse,\bf}
% \DoNotIndex{\clearpage,\closein,\lowercase,\@inlabelfalse}
% \DoNotIndex{\selectfont,\mathcode,\newmathalphabet,\rmdefault}
% \DoNotIndex{\bfdefault}
%
%  \CheckSum{0}
%  \CharacterTable
%  {Upper-case    \A\B\C\D\E\F\G\H\I\J\K\L\M\N\O\P\Q\R\S\T\U\V\W\X\Y\Z
%   Lower-case    \a\b\c\d\e\f\g\h\i\j\k\l\m\n\o\p\q\r\s\t\u\v\w\x\y\z
%   Digits        \0\1\2\3\4\5\6\7\8\9
%   Exclamation   \!     Double quote  \"     Hash (number) \#
%   Dollar        \$     Percent       \%     Ampersand     \&
%   Acute accent  \'     Left paren    \(     Right paren   \)
%   Asterisk      \*     Plus          \+     Comma         \,
%   Minus         \-     Point         \.     Solidus       \/
%   Colon         \:     Semicolon     \;     Less than     \<
%   Equals        \=     Greater than  \>     Question mark \?
%   Commercial at \@     Left bracket  \[     Backslash     \\
%   Right bracket \]     Circumflex    \^     Underscore    \_
%   Grave accent  \`     Left brace    \{     Vertical bar  \|
%   Right brace   \}     Tilde         \~}
%
%
%
% \changes{1.0}{2013/01/26}{Converted to DTX file}
%
% \DoNotIndex{\newcommand,\newenvironment}
% \GetFileInfo{phd.dtx}
% 
%  \def\fileversion{v1.0}          
%  \def \filedate{2012/03/06}
% \title{The \textsf{phd} package.
% \thanks{This
%        file (\texttt{phd.dtx}) has version number \fileversion, last revised
%        \filedate.}
% }
% \author{Dr. Yiannis Lazarides \\ \url{yannislaz@gmail.com}}
% \date{\filedate}
%
%
% 
% ^^A\maketitle
% 
% ^^A\frontmatter
%  ^^A\coverpage{./images/hine02.jpg}{Book Design }{Camel Press}{}{}
%  \newpage
% ^^A\secondpage
% \pagestyle{empty}
%
%
% 
%
%
% \pagestyle{headings}
% \raggedbottom
%  \OnlyDescription
%  
% ^^A\StopEventually{\PrintIndex}
%
% \CodelineNumbered
% \pagestyle{headings}
% 
% \chapter{Package Development}
%
% Timothy Van Zandt's package has been developed in 1992 and is still widely used, especially in
% defining verbatim commands and environments. I could not find a |.dtx| file on \ctan and as I 
% was looking to understand how verbatim code works I decided to translate it to expl3, who at
% first mind sound as madness. The method in the madness though was to at the same time test
% how well two other packages of mine worked by using this code as a test file. The other
% two packages colorize latex3 packages and classes in a different way than other packages such
% as listings work.
%
% The other benefit of this developemnt was to reinforce my understanding of \tex especially in
% regards to the use of active characters, catcoding and the like.
%
% ^^A\part{IMPLEMENTATION AND FRIENDS}
% 
% 
% \iffalse
%<*package>
% \fi
% \chapter{Code Implementation Objectives and Strategy}

% \section{Preliminaries}
% 		We first handle the package definitions. Although the package essentially is an
% 		|expl3| package we define it as a \latex2e as it contains some \latex2 materials.
%    \begin{macrocode}
%<@@=fv>
\NeedsTeXFormat{LaTeX2e}
\def\fileversion{3.1a}
\def\filedate{2018/11/20}
\ProvidesPackage{xfancyvrb}[\filedate]
\message{Style option: `xfancyvrb' v\fileversion \space  <\filedate> (YL)}
%    \end{macrocode}
%  \TestFiles{fvtest.tst}
% standard test not to load the package again
%    \begin{macrocode}
\csname xfancyvrb@loaded\endcsname
\let\xfancyvrb@loaded\endinput
%\ProcessOptions
\@ifpackageloaded{xcolor}{}{\RequirePackage{xcolor}}
%    \end{macrocode}
% \begin{macro}[int]{fv_define_key:nnnn}
%   Define compatibility commands to keep the style of the code consistent with l3.
%    \begin{macrocode}
\ExplSyntaxOn 
\let\fv_define_key:nnnn\define@key      
\ExplSyntaxOff
%    \end{macrocode}
% \end{macro}

% \begin{macro}{FV@error,FV@eha}
%    \begin{macrocode}
\ExplSyntaxOn
 
\cs_set:Npn \FV@Error#1#2{%
  \edef\@tempc{#2}\exp_after:wN \errhelp\exp_after:wN {\@tempc}%
  \errmessage{FancyVerb Error:^^J\space\space #1^^J}}

\def\FV@eha{Your command was ignored. Type <return> to continue.}
\ExplSyntaxOff
%    \end{macrocode}
% \end{macro}
%
% Shorthands for messagesinspired from |fontspec|. The original |fancyvrb| code
% is rich in error trapping. \latexe errors have been mapped to l3 messages.
%    \begin{macrocode}
\ExplSyntaxOn
\cs_new:Npn \@@_error:n     { \msg_error:nn     {xfancyvrb} }
\cs_new:Npn \@@_error:nn    { \msg_error:nnn    {xfancyvrb} }
\cs_new:Npn \@@_error:nx    { \msg_error:nnx    {xfancyvrb} }
\cs_new:Npn \@@_warning:n   { \msg_warning:nn   {xfancyvrb} }
\cs_new:Npn \@@_warning:nx  { \msg_warning:nnx  {xfancyvrb} }
\cs_new:Npn \@@_warning:nxx { \msg_warning:nnxx {xfancyvrb} }
\cs_new:Npn \@@_info:n      { \msg_info:nn      {xfancyvrb} }
\cs_new:Npn \@@_info:nx     { \msg_info:nnx     {xfancyvrb} }
\cs_new:Npn \@@_info:nxx    { \msg_info:nnxx    {xfancyvrb} }
\cs_new:Npn \@@_trace:n     { \msg_trace:nn     {xfancyvrb} }
\ExplSyntaxOff
%    \end{macrocode}
% Allow messages to be written with spaces acting as normal:
%    \begin{macrocode}
\ExplSyntaxOn
\cs_generate_variant:Nn \msg_new:nnn  {nnx}
\cs_generate_variant:Nn \msg_new:nnnn {nnxx}
\cs_new:Nn \@@_msg_new:nnn
  { \msg_new:nnx {#1} {#2} { \tl_trim_spaces:n {#3} } }
\cs_new:Nn \@@_msg_new:nnnn
  { 
    \msg_new:nnxx {#1} {#2} { \tl_trim_spaces:n {#3} } 
                            { \tl_trim_spaces:n {#4} } 
  }
\char_set_catcode_space:n {32}

\@@_msg_new:nnn {xverbatim}{not-vervbatim-command}
  {
    Command \string#1 is not an xverbatim command.
  }
  
\ExplSyntaxOff
  
%    \end{macrocode}
%% DG/SR modification begin - Jan. 21, 1998
%% Suggested by Bernard Gaulle to solve a compatibility problem with `french'
%% (it introduce the restriction to put \VerbatimFootnotes AFTER the preambule)
%% \def\VerbatimFootnotes{\cs_set_eq:NN \@footnotetext\V@footnotetext}
%    \begin{macrocode}
\ExplSyntaxOn
\cs_set_eq:NN \V@footnote\footnote
\def\VerbatimFootnotes
  {
    \cs_set_eq:NN \@footnotetext\V@footnotetext
    \cs_set_eq:NN \footnote\V@footnote
  }

%% DG/SR modification end
\cs_set:Npn \V@footnotetext
  {
    \afterassignment\v_footnotetext_aux
    \cs_set_eq:NN \@tempa
  }
  
\def\v_footnotetext_aux{%
  \insert\footins\bgroup
  \csname reset@font\endcsname
  \footnotesize
  \interlinepenalty\interfootnotelinepenalty
  \splittopskip\footnotesep
  \splitmaxdepth\dp\strutbox
  \floatingpenalty \@MM
  \hsize\columnwidth
  \@parboxrestore
  \edef\@currentlabel{\csname p@footnote\endcsname\@thefnmark}%
  \@makefntext{}%
  \rule{\z@}{\footnotesep}%
  \bgroup
  \aftergroup\fv_footnotetext_auxiii
  \ignorespaces}
  
\cs_set:Npn \fv_footnotetext_auxiii {\strut\egroup}
\ExplSyntaxOff
%    \end{macrocode}
%
%    \begin{macrocode}
\RequirePackage{keyval}
%    \end{macrocode}
%   
% \begin{variable}{\l_@@_tmp_int}
% \begin{variable}{\l_@@_tmp_prop}
% \begin{variable}{\l_@@_tmp_tl}
%   Scratch space.
%   
%    \begin{macrocode}
\ExplSyntaxOn
\int_new:N  \l_@@_tmp_int
\prop_new:N \l_@@_tmp_prop
\tl_new:N   \l_@@_tmp_tl
\cs_set:Npn   \l_@@_tmpa: {}
\ExplSyntaxOff
%    \end{macrocode}
% \end{variable}
% \end{variable}
% \end{variable}
%
% \begin{macro}{\fv_define_boolean_key:nnTF}{\Arg{family}\Arg{keyname}\Arg{true code}\Arg{false code}}
% Timothy Van Zandt used keyval, but provided a couple of modifications to it.
% Prefix |KV| is from the keyval package and has to be used. 
% Suffixes |default| 
%    \begin{macrocode}
\ExplSyntaxOn
\cs_gset:Npx \g_@@_prefix_tl{KV}

\cs_set:Npn \fv_space_tl{\c_space_token}

\cs_set:Npn \define@booleankey#1#2#3#4
  {
    \cs_set:cpn {\g_@@_prefix_tl @#1@#2@default}{#3}%
    \cs_set:cpn {KV@#1@#2@false}{#4}%
    \cs_set:cpn {KV#1@#2}##1{\KV@booleankey{##1}{#1}{#2}}  
  }
  
\cs_set:Npn\KV@booleankey#1#2#3{%
  \edef\@tempa{#1}
  \exp_after:wN \KV_booleankey_aux:nnnn \@tempa\scan_stop:\@nil{#2}{#3}
  }
  
\cs_set:Npn \KV_booleankey_aux:nnnn #1#2\@nil#3#4{%
  \use:c {KV@#3@#4@\if t#1default\else\if T#1default\else false\fi\fi}}
%    \end{macrocode} 
% Map the |\define@booleankey| to l3 style for consistency. 
%    \begin{macrocode}
\cs_gset_eq:NN \fv_define_boolean_key:nnTF \define@booleankey  
\ExplSyntaxOff  
%    \end{macrocode}
% \end{macro}
%
% Define some of the default key settings.
% \begin{macro}{\fv_none_tl, \fv_auto_tl, \fvset}
%  Use token lists to hold values for keys, rather than definitions. Use
%  |tl_new:N| to avoid errors in re-definitions.
%    \begin{macrocode}  
\ExplSyntaxOn
\tl_new:N \fv_none_tl
\tl_new:N \fv_auto_tl
\tl_set:Nn \fv_none_tl {none}
\tl_set:Nn \fv_auto_tl {auto}

% the |\setkeys| is from the keyval package. If we have xkeyval
% probably it is redefined.
\cs_set_nopar:Npn\fvset#1{\setkeys{FV}{#1}}
\ExplSyntaxOff
%    \end{macrocode}
% \end{macro}

% \begin{macro}{\fv_command:nn, \fv_command_auxi}
%  Handles star commands and optional commands. We can also replace with 
%  xparse if necessary.
%    \begin{macrocode}
\ExplSyntaxOn
%\def\FV@Command#1#2{%
%  \@ifstar
%    {\def\FV@KeyValues{#1,showspaces}\FV@@Command{#2}}%
%    {\def\FV@KeyValues{#1}\FV@@Command{#2}}}
\tl_new:N \fv_key_values_tl 

\cs_new:Npn \fv_command:nn#1#2
  {
    \@ifstar
      {
      \tl_gput_left:Nn \fv_key_values_tl{#1,showspaces}\fv_command_auxi{#2}}%
      {\tl_gput_left:Nn \fv_key_values_tl{#1}\fv_command_auxi{#2}}
  }
%    \end{macrocode}
% \end{macro}
%
% Used for SaveVerbs etc |FVC| prefix
%    \begin{macrocode}    
\cs_set:Npn \fv_command_auxi#1{%
  \@ifnextchar[%
    {\fv_get_keyvalues:nn {\use:c {FVC@#1}}}
    {\use:c {FVC@#1}}}
    
   
\cs_set:Npn \fv_get_keyvalues:nn #1[#2]
  {
    \tl_gput_left:Nn\fv_key_values_tl {#2}
    \tl_use:N #1
  }
\ExplSyntaxOff  
%    \end{macrocode}  
%
% \begin{macro}{\fv_custom_verbatim_command:nnnn}
% 		Auxiliary macro to check if command has not been defined.
%    \begin{macrocode}  
\ExplSyntaxOn
\cs_set:Npn \fv_custom_verbatim_command:nnnn  #1 #2 #3 #4
  {
    \bgroup
      \fvset{#4}
    \egroup   
    \cs_if_free:cTF {FVC@#3}
      {\@@_warning:nx {verbatim-not-defined} {#1}}
      {#1{#2}{\fv_command:nn{#4}{#3}}}
  }
\ExplSyntaxOff  
%    \end{macrocode}
% \end{macro}
%
% \begin{macro}{\CustomVerbatimCommand, \RecustomVerbatimCommand}
%    \begin{macrocode}
\ExplSyntaxOn
\cs_set:Npn \CustomVerbatimCommand
  {
    \fv_custom_verbatim_command:nnnn \newcommand
  }

\cs_set:Npn \RecustomVerbatimCommand
  {
    \fv_custom_verbatim_command:nnnn \renewcommand
  }
\ExplSyntaxOff
%    \end{macrocode}
% \end{macro}
%
% \begin{macro}{\fv_environment}{\Arg{verbatim name}\Arg{options}}
% |FVB@Verbatim| prefix FVB used for begin verbatim etc.
%    \begin{macrocode}
\ExplSyntaxOn
\cs_set:Npn \fv_environment:nn #1 #2
{
  \tl_gput_left:Nn \fv_key_values_tl{#1}
  \char_set_catcode_active:N \^^M
   \@ifnextchar[
    {
     \char_set_catcode_end_line:N \^^M
     \fv_get_keyvalues:nn { \use:c {FVB@#2} }
    }
    {
      \char_set_catcode_end_line:N \^^M
      \use:c {FVB@#2}
    }
}
\ExplSyntaxOff
%    \end{macrocode}   
% \end{macro} 

% \begin{macro}{\NewVerbatimEnvironment,\RenewVerbatimEnvironment,\fv_make_verbatim_env:nnnn}
% I have changed the original names from Custom/Recustom to New/Renew to make it more
% understandable to users.
%    \begin{macrocode}
\ExplSyntaxOn
\cs_set:Npn \NewVerbatimEnvironment
  {
    \fv_make_verbatim_env:nnnn \newenvironment
  }

\cs_set:Npn \RenewVerbatimEnvironment
  {
    \fv_make_verbatim_env:nnnn \renewenvironment
  }
  
\cs_set_eq:NN \CustomVerbatimEnvironment\NewVerbatimEnvironment
\cs_set_eq:NN \RecustomVerbatimEnvironment\RenewVerbatimEnvironment
%    \end{macrocode}
%
% |\fvset| is set in a group so if there are errors, they are easier to locate.
%    \begin{macrocode}
\cs_set:Npn \fv_make_verbatim_env:nnnn #1#2#3#4
 {
    \bgroup
      \fvset{#4}
    \egroup  
    \cs_if_free:cTF {FVB@#3}%
      {\FV@Error{`#3' is not an xfancyvrb environment.}\@eha}
      {
         #1{#2}{\fv_environment:nn{#4}{#3}}{\use:c{FVE@#3}}
         #1{#2*}{\fv_environment:nn{#4,showspaces}{#3}}{\use:c {FVE@#3}}
      }
  }
\ExplSyntaxOff     
%    \end{macrocode}  
% \end{macro}

% \begin{macro}{\DefineVerbatimEnvironment}\Arg{environment name} \Arg{Verbatim} \Arg{options} 
% Defines a new verbatim environment based on an existing. It creates two macros
% |\my_verbatim| and |\my_verbatim*|
%    \begin{macrocode}  
\ExplSyntaxOn
\cs_set:Npn \DefineVerbatimEnvironment #1 #2 #3
  {
    \cs_set:cpn {#1}     { \fv_environment:nn{#3}{#2} }
    \cs_set:cpn {end#1}  { \use:c {FVE@#2} }
    \cs_set:cpn {#1*}    { \fv_environment:nn{#3,showspaces}{#2} }
    \cs_set:cpn {end#1*} { \use:c {FVE@#2} }
  }
\ExplSyntaxOff  
%    \end{macrocode}
% \end{macro}
% 
% \begin{macro}{\fv_use_values:} 
% To use the values we need to iterate through the comma list of values. I have kept the
% original definition here which uses the \pkg{keyval} |\KV@do| to iterate. One night
% I will need to come back to this and completely divorce |keyval|, but this daughter 
% of David Carlisle is such a sweet girl\ldots.
%
% 
%    \begin{macrocode}
\ExplSyntaxOn
\cs_set:Npn \fv_use_values:
  {
    \cs_if_eq:NNTF \fv_key_values_tl \c_empty_tl {}
    {
     
      \cs_set:Npn \KV@prefix{KV@FV@}
      \exp_after:wN \KV@do\fv_key_values_tl,\scan_stop:,
      
      % we cannot just use this, as keyval handles defaults etc.
      %\clist_map_inline:Nn \fv_key_values_tl {\cs:w \KV@prefix##1\cs_end:w}
      \tl_gclear:N \fv_key_values_tl
    }
  }
\ExplSyntaxOff  
%    \end{macrocode}  
% \end{macro}
%
% \begin{macro}{\fv_catcodes:}
% Set up the preliminary work by setting all the catcodes.
%    \begin{macrocode} 
\ExplSyntaxOn
\cs_set:Npn \fv_catcodes:
  {
    \cs_set_eq:NN \do\@makeother\dospecials  % The usual stuff.
    \fv_active_white_space_tl                % See below.
    \FV@FontScanPrep                         % See below.
    \fv_catcodes_hook_tl                     % A style hook.
    \fv_verb_codes_tl                        % A user-defined hook.
  }
\ExplSyntaxOff  
%    \end{macrocode}  
% \end{macro}
%
% \begin{macro}{\FV@ActiveWhiteSpace}
% Set the catcodes for white spaces to be active.
%    \begin{macrocode}  
\ExplSyntaxOn
\cs_set:Npn \fv_active_white_space_tl
  {
    \char_set_catcode_active:N \^^M   % End of line
    \catcode`\ =\active               % Space
    \catcode`\^^I=\active             % Tab
  }  
  
\cs_set_eq:NN \FV@ActiveWhiteSpace \fv_active_white_space_tl       
\ExplSyntaxOff  
%    \end{macrocode}
% \end{macro}
 
%
% \begin{macro}{\fv_catcodes_hook_tl, \fv_add_to_hook:nn}
%    \begin{macrocode} 
\ExplSyntaxOn
\tl_new:N \fv_catcodes_hook_tl

\cs_set:Npn \fv_add_to_hook:nn #1 #2
  {
    %\exp_after:wN \def\exp_after:wN #1\exp_after:wN {#1#2\scan_stop:}
    \tl_put_right:Nn #1{#2}
  }
\ExplSyntaxOff  
%    \end{macrocode} 
% \end{macro}


% \begin{macro}{codes, codes*}
%    \begin{macrocode}
\ExplSyntaxOn
\fv_define_key:nnnn{FV}{codes}[]
  {
    \cs_set:Npn \fv_verb_codes_tl {#1\scan_stop:}
  }
 
% change this to proper tl 
\fv_define_key:nnnn{FV}{codes*}
  {
    \exp_after:wN 
      \def\exp_after:wN\fv_verb_codes_tl \exp_after:wN {\fv_verb_codes_tl #1\scan_stop:}
  }
\fvset{codes}
%    \end{macrocode}
% \end{macro}
%
% \begin{macro}{vspace}
%  Key for adding space at the top of the list. 
%    \begin{macrocode}
\fv_define_key:nnnn{FV}{vspace}[\topsep]
  {
    \def\FancyVerbVspace {#1}
  }
\fvset{vspace}
\ExplSyntaxOff
%    \end{macrocode}
% \end{macro}
%
%    \begin{macrocode}
\ExplSyntaxOn
\fv_define_key:nnnn{FV}{commandchars}[\\\{\}]
  {\cs_set_nopar:Npx \@tempa{#1}%
    \if_meaning:w \@tempa\fv_none_tl
      \cs_set_eq:NN \fv_command_chars:nnn\scan_stop:
    \else
      \fv_define_command_chars:nnn #1 \scan_stop:\scan_stop:\scan_stop:
    \fi}
\ExplSyntaxOff    
%    \end{macrocode} 
%   
% \begin{macro}{\fv_define_command_chars:nnn}\Arg{command char}\Arg{begin group char}\Arg{end group char}
%    Characters which define the character which starts a macro 
%    and marks the beginning and end of a group; thus letting us 
%    introduce \emph{escape} sequences in verbatim code. By
%    default it is empty.
%
%    \begin{macrocode}
\ExplSyntaxOn
\cs_set:Npn \fv_define_command_chars:nnn #1 #2 #3
  {
    \cs_set:Npn \fv_command_chars
      {
        \catcode`#1=0\scan_stop:
        \catcode`#2=1\scan_stop:
        \catcode`#3=2\scan_stop:
      }
  }
\ExplSyntaxOff
%    \end{macrocode}
% \end{macro}
%
% Add the command chars to the catcode hook 
%    \begin{macrocode}
\ExplSyntaxOn
\fv_add_to_hook:nn \fv_catcodes_hook_tl \fv_command_chars:nnn
\ExplSyntaxOff
%    \end{macrocode}
%

% \begin{macro}{commentchar}
%  Key to set a character as a comment character i.e, |\catcode `!=14|.
%    \begin{macrocode}
\ExplSyntaxOn
\fv_define_key:nnnn{FV}{commentchar}[\!]{
    \cs_set_nopar:Npn \@tempa{#1}
    \if_meaning:w \@tempa\fv_none_tl
      \cs_set_eq:NN \fv_comment_char\scan_stop:
    \else
      \def\fv_comment_char{\catcode`#1=14}
    \fi
  }

\fv_add_to_hook:nn\fv_catcodes_hook_tl\fv_comment_char


\fvset{commandchars=none,commentchar=none}
%    \end{macrocode}
% \end{macro}
%
% \begin{macro}{firstline, \fv_parse_start:n}
% \begin{macrocode}
\fv_define_key:nnnn{FV}{firstline}
  {
    \afterassignment
      \fv_parse_start:n \@tempcnta=0#1\scan_stop:\@nil{#1}
  }
  
  
\def\fv_parse_start:n #1\scan_stop:\@nil#2
  {
  \if_meaning:w \@nil#1\@nil
    \edef\fv_start_num_int{\the\@tempcnta}%
    \cs_set_eq:NN \FancyVerbStartString\scan_stop:
  \else
    \edef\FancyVerbStartString{#2}%
  \fi}
%    \end{macrocode}
% \end{macro}
%
%
% \begin{macro}{\fv_define_default_key:nn}
%    \begin{macrocode}
\cs_new:Npn \fv_define_default_key:nn #1#2
  {
    \cs_set:cpn {KV@FV@#1@default} {#2}
  }  
  
%\def\KV@FV@firstline@default{%
%  \cs_set_eq:NN \fv_start_num_int\z@
%  \cs_set_eq:NN \FancyVerbStartString\scan_stop:}
  
\fv_define_default_key:nn {firstline}
  {
    \cs_set_eq:NN \fv_start_num_int\c_zero_int
    \cs_set_eq:NN \FancyVerbStartString\scan_stop:
  }  
%    \end{macrocode}
% \end{macro}
%
% \begin{macro}{lastline}
% 		Define the |lastline| key. Take an integer to denote the lastline.
%    \begin{macrocode}
\fv_define_key:nnnn{FV}{lastline}{%
  \afterassignment\fv_parse_stop:nn \@tempcnta=0#1\scan_stop:\@nil{#1}}
\ExplSyntaxOff  
%    \end{macrocode} 
% \end{macro}
%
% \begin{macro}{\fv_parse_stop:nn, \KV@FV@lastline@default}
%    Use a delimited macro to capture the stop number. Set the stop string to relax. We can
%    only select by numbers or strings. 
%    \begin{macrocode}
\ExplSyntaxOn
\cs_set:Npn \fv_parse_stop:nn #1\scan_stop:\@nil#2
  {
    \if_meaning:w \@nil#1\@nil
      \cs_set_nopar:Npx \fv_verb_stop_num:n{\the\@tempcnta}
      \cs_set_eq:NN \fv_verb_stop_string_tl\scan_stop:
    \else
      \cs_set_nopar:Npx \fv_verb_stop_string_tl{#2}
    \fi
  }
  
\def\fv_define_key_default:nn {lastline} 
  {
  	  \cs_set_eq:NN \fv_verb_stop_num:n \c_zero_int
    \cs_set_eq:NN \fv_verb_stop_string_tl\scan_stop:
  }
\ExplSyntaxOff  
\fvset{firstline,lastline}
%    \end{macrocode}
% \end{macro}
%
% \begin{macro}{\fv_codeline_int}
%  	Define a counter to take care of the numbering of lines.
%    \begin{macrocode}
\ExplSyntaxOn
\int_zero_new:N \fv_codeline_int
\ExplSyntaxOff
%    \end{macrocode}
% \end{macro}
%

% \begin{macro}{\fv_preprocess_line:}
%  Start by incrementing the counter and then find StartStop
%    \begin{macrocode}
\ExplSyntaxOn
\def\fv_preprocess_line:
  {
    % increase counter by one 
    \int_gincr:N \fv_codeline_int
    \FV@FindStartStop
  }
\ExplSyntaxOff  
%    \end{macrocode}
% \end{macro}
%
% \begin{macro}{\fv_preprocessline_auxii}
% \label{lin:preprocess}
%    \begin{macrocode}  
\ExplSyntaxOn
\cs_set:Npn \fv_preprocessline_auxii
  {
    \FV@StepLineNo
    \fv_gobble:
    \exp_after:wN \fv_process_line:n\exp_after:wN {\fv_line}
  }
\ExplSyntaxOff  
%    \end{macrocode}  
% \end{macro}
%
%
%    \begin{macrocode}
\ExplSyntaxOn
\cs_set:Npn \FV@FindStartStop{\FV@DefineFindStart\FV@FindStartStop}
\ExplSyntaxOff
%    \end{macrocode}

%% \def\FV@DefinePreProcessLine{%
%%   \setcounter{FancyVerbLine}{0}%
%%   \FV@DefineFindStart}
%
% \begin{macro}{\c@fv_verbline_int, \DefineFindStart}
%    \begin{macrocode}
\ExplSyntaxOn
 \int_new:N \c@fv_verbline_int
 \cs_set:Npn \FV@DefineFindStart
  {
    \if_meaning:w  \FancyVerbStartString\scan_stop:
      \if_int_compare:w \fv_start_num_int < \c_two
        \FV@DefineFindStop
      \else: 
        \cs_set_eq:NN \FV@FindStartStop\FV@FindStartNum
      \fi:
    \else:
      \cs_set_eq:NN \FV@FindStartStop\FV@FindStartString
    \fi:
  }
\ExplSyntaxOff  
%    \end{macrocode}  
% \end{macro} 
%

% \begin{macro}{\FV@FindStratNum}
%    \begin{macrocode} 
\ExplSyntaxOn 
\cs_set:Npn \FV@FindStartNum
  {
    \if_int_compare:w \fv_start_num_int > \fv_codeline_int
    \else:
      \FV@DefineFindStop
      \exp_after:wN \fv_preprocessline_auxii
    \fi:
  }
\ExplSyntaxOff  
%    \end{macrocode}
% \end{macro}
%
% \begin{macro}{\FV@FindStartString}
%  Find the string marked as START. We do by comparing the meaning.
%    \begin{macrocode}
\ExplSyntaxOn
\cs_set:Npn \FV@FindStartString
  {  
    \exp_after:wN \fv_find_startstring_aux:
    {\meaning\fv_line}
    {\meaning\FancyVerbStartString}
  }
\ExplSyntaxOff
%    \end{macrocode}
% \end{macro}

% \begin{macro}{\fv_find_startstring_aux:}{\Arg{1}\Arg{2}}
%    \begin{macrocode}
\ExplSyntaxOn
\cs_set:Npn \fv_find_startstring_aux: #1 #2
  {  
     \edef\@fooA{#1} \edef\@fooB{#2}
     %%%%%%%%%%%%%%%%%%%%%%%%%%%%%%%%%%%%%%%%%%%%%%%%%%%
     \typeout{\meaning\@fooA \space\space \meaning\@fooB}
     \if_meaning:w  \@fooA\@fooB
       \typeout{Call stop} 
       \FV@DefineFindStop
     \fi:
  }
\ExplSyntaxOff
%    \end{macrocode}
% \end{macro}
%
% 
% \begin{macro}{\FV@DefineFindStop, \FV@FindStopNum}
%  Find start and stop number.
%    \begin{macrocode}
\ExplSyntaxOn
\cs_set:Npn \FV@DefineFindStop 
  {
    \if_meaning:w \fv_verb_stop_string_tl\scan_stop:
      \if_int_compare:w \FancyVerbStopNum < \@ne
        \cs_set_eq:NN \FV@FindStartStop\fv_preprocessline_auxii
      \else:
        \cs_set_eq:NN \FV@FindStartStop\FV@FindStopNum
      \fi:
    \else:
      \cs_set_eq:NN \FV@FindStartStop\FV@FindStopString
    \fi:
  }
  
\cs_set:Npn \FV@FindStopNum
  {
    \if_int_compare:w \FancyVerbStopNum>\fv_codeline_int
    \else:
      \cs_set_eq:NN \FV@FindStartStop\scan_stop:
      \ifeof\fv_infile\else
        \immediate\closein\fv_infile
      \fi:
    \fi:
    \if_int_compare:w\FancyVerbStopNum<\fv_codeline_int
    \else:
      \fv_preprocessline_auxii
    \fi:
  }
\ExplSyntaxOff  
%    \end{macrocode}
% \end{macro}

% \begin{macro}{FV@FindStopString}  
% Compares the contents of a line to that of a STOP string
% if the meaning is the same then we found the marker.
% Here we can use a string, but I have kept the original definition
% this way we maybe allowed to use a macro as a stop string
%    \begin{macrocode}
\ExplSyntaxOn
\cs_set:Npn \FV@FindStopString
  {
    \exp_after:wN \fv_FindStopString_aux:nn
    {\meaning\fv_line}
    {\meaning\fv_verb_stop_string_tl}
  }

% The auxiliary function to compare
\cs_set:Npn \fv_FindStopString_aux:nn #1#2
  { 
    \edef\@fooA{#1} \edef\@fooB{#2}%
    \if_meaning:w \@fooA\@fooB
      \cs_set_eq:NN \FV@FindStartStop\scan_stop:
      \ifeof\fv_infile
      \else
        \immediate\closein\fv_infile
      \fi
    \else
      \exp_after:wN \fv_preprocessline_auxii
    \fi
  }
\ExplSyntaxOff  
%    \end{macrocode}  
% \end{macro}
%  

% \begin{macro}{\fv_gobble:,\fv_gobbleii,\fv_gobbleiii, \fv_gobbleiiii}
%    This is where \tex got its notorious repudation that programming using \tex iss black art!
%    The purpose of the gobble key is to set how many characters
%    to gobble. It is set to gobble anything less than 9 parameters.
%    Gobbling starts with |\fv_gobble:| during the line pre-processing.
%    see \ref{lin:preprocess}. If gobbling is valid then this is let
%    to auxiliary 2 |\fv_gobbleii| defined below, which starts the chain
%    with |auxiii| and |iiii|.
%    \begin{macrocode}
\ExplSyntaxOn
\cs_set:Npn \fv_gobbleii
  {
  \exp_after:wN \exp_after:wN \exp_after:wN \fv_gobbleiii
  \exp_after:wN \fv_gobbleiiii\fv_line
    \@nil\@nil\@nil\@nil\@nil\@nil\@nil\@nil\@nil\@nil\@@nil
  }
%    \end{macrocode}

%    \begin{macrocode}    
\cs_set:Npn \fv_gobbleiii#1\@nil#2\@@nil{\cs_set:Npn \fv_line{#1}}
%    \end{macrocode}
%
%    \begin{macrocode}
\fv_define_key:nnnn{FV}{gobble}{%
  \@tempcnta=#1\scan_stop:
  \if_int_compare:w\@tempcnta<\@ne
    \cs_set_eq:NN \fv_gobble:\scan_stop:
  \else
    \if_int_compare:w\@tempcnta>9
      \FV@Error{gobble parameter must be less than 10}\FV@eha
    \else
      % why is teh counter here? 
      \renewcommand{\fv_gobbleiiii}[\@tempcnta]{}
      \cs_set_eq:NN \fv_gobble: \fv_gobbleii
    \fi
  \fi}
%    \end{macrocode} 
% Define the auxiliary to empty. 
%    \begin{macrocode}
\cs_set:Npn \fv_gobbleiiii {}
%    \end{macrocode}
%
% Define the default |gobble| key to be equal to |\scan_stop:| and set it.
%    \begin{macrocode}
\def\KV@FV@gobble@default
  {
    \cs_set_eq:NN \fv_gobble:\relax
  }
\fvset{gobble}
\ExplSyntaxOff
%    \end{macrocode}
% \end{macro}
%
% \begin{macrocode}
\ExplSyntaxOn
\cs_set:Npn \FV@Scan
  {
    \fv_catcodes:
    \VerbatimEnvironment
    \FV@DefineCheckEnd
    \FV@BeginScanning
  }
\ExplSyntaxOff    
%    \end{macrocode} 

%
% \begin{macro}{\VerbatimEnvironment}  
%  Sets the local environment name to the current environment.
%  remember xdef.  
%    \begin{macrocode}
\ExplSyntaxOn
\cs_set:Npn \VerbatimEnvironment 
  {
     \if_meaning:w \FV@EnvironName \scan_stop:
       \cs_gset_nopar:Npx \FV@EnvironName{\@currenvir}
     \fi:
  }
  
\cs_set_eq:NN \FV@EnvironName\scan_stop:
\ExplSyntaxOff
%    \end{macrocode}
% \end{macro}
%

%    \begin{macrocode}
\ExplSyntaxOn
\bgroup
\catcode`\!=0
\catcode`\[=1
\catcode`\]=2
!gdef!FV@CheckEnd@i#1[!FV@@CheckEnd#1\end{}!@nil]

!gdef!FV@@CheckEnd@i#1\end#2#3!@nil[!def!@tempa[#2]!def!@tempb[#3]]

!gdef!fv_checkendiii_i[\end{}]
\catcode`!\=12

!gdef!FV@CheckEnd@ii#1[!FV@@CheckEnd#1\end{}!@nil]
!gdef!FV@@CheckEnd@ii#1\end#2#3!@nil[!def!@tempa[#2]!def!@tempb[#3]]
!gdef!fv_checkendiii_ii[\end{}]
!catcode`!{=12
!catcode`!}=12

!gdef!FV@CheckEnd@iii#1[!FV@@CheckEnd#1\end{}!@nil]
!gdef!FV@@CheckEnd@iii#1\end{#2}#3!@nil[!def!@tempa[#2]!def!@tempb[#3]]
!gdef!fv_checkendiii_iii[\end{}]
!catcode`!\=0

!gdef!FV@CheckEnd@iv#1[!FV@@CheckEnd#1\end{}!@nil]
!gdef!FV@@CheckEnd@iv#1\end{#2}#3!@nil[!def!@tempa[#2]!def!@tempb[#3]]
!gdef!fv_checkendiii_iv[\end{}]
\egroup 
\ExplSyntaxOff
%    \end{macrocode}
%
%    \begin{macrocode}
\ExplSyntaxOn
\def\FV@BadCodes#1{%
  \FV@Error
    {\string\catcode\space of \exp_after:wN \@gobble\string#1 is wrong:
    \the\catcode`#1}%
    {Only the following catcode values are allowed:
    ^^J\@spaces \exp_after:wN \@gobble\string\\ \space\space --> 0 or 12.
    ^^J\@spaces \string{ \string} --> 1 and 2, resp., or both 12.
    ^^JTo get this error, either you are a hacker or you got bad advice.}%
  \def\FV@CheckEnd##1{\iftrue}}
  
\def\FV@DefineCheckEnd{%
  \if_int_compare:w\catcode`\\=\z@
    \if_int_compare:w\catcode`\{=\@ne               
      \cs_set_eq:NN \FV@CheckEnd\FV@CheckEnd@i
      \cs_set_eq:NN \FV@@CheckEnd\FV@@CheckEnd@i
      \cs_set_eq:NN \FV@@@CheckEnd\fv_checkendiii_i
    \else
      \if_int_compare:w\catcode`\{=12
        \cs_set_eq:NN \FV@CheckEnd\FV@CheckEnd@iv
        \cs_set_eq:NN \FV@@CheckEnd\FV@@CheckEnd@iv
        \cs_set_eq:NN \FV@@@CheckEnd\fv_checkendiii_iv
      \else
        \FV@BadCodes\{%
      \fi
    \fi
  \else
    \if_int_compare:w\catcode`\\=12
      \if_int_compare:w\catcode`\{=\@ne
        \cs_set_eq:NN \FV@CheckEnd\FV@CheckEnd@ii
        \cs_set_eq:NN \FV@@CheckEnd\FV@@CheckEnd@ii
        \cs_set_eq:NN \FV@@@CheckEnd\fv_checkendiii_ii
      \else
        \if_int_compare:w\catcode`\{=12
          \cs_set_eq:NN \FV@CheckEnd\FV@CheckEnd@iii
          \cs_set_eq:NN \FV@@CheckEnd\FV@@CheckEnd@iii
          \cs_set_eq:NN \FV@@@CheckEnd\fv_checkendiii_iii
        \else
          \FV@BadCodes%check this
        \fi
      \fi
    \else
      \FV@BadCodes%
    \fi
  \fi}
  
\bgroup
  \char_set_catcode_active:N \^^M
  \cs_gset:Npn \FV@BeginScanning #1^^M 
    {
      \def\@tempa{#1}\if_meaning:w \@tempa\@empty\else\@@_bad_begin_error:\fi%
      \FV@GetLine
    }
\egroup 

\cs_set:Npn \@@_bad_begin_error: #1
 {
  \exp_after:wN \@temptokena\exp_after:wN {\@tempa}%
  \FV@Error
    {Extraneous input `\the\@temptokena' between
      \string\begin{\FV@EnvironName}[<key=value>] and line end}%
   {This input will be discarded. Hit <return> to continue.}
 }
\ExplSyntaxOff
%    \end{macrocode}
%
% \begin{macro}{\FV@GetLine}
% The |\@noligs| is from the kernel and breaks ligatures. Get the
% line using |\FancyVerbGetLine| and pass it onto |\fv_check_scan:n |
%    \begin{macrocode}
\ExplSyntaxOn
\cs_set:Npn \FV@GetLine
  { \@noligs\exp_after:wN \fv_check_scan:n \FancyVerbGetLine }
\ExplSyntaxOff
%    \end{macrocode}
% \end{macro}
%
% \begin{macro}{\FancyVerbGetLine}
% Reads and process one line at a time. The line is delimited by |^^M|.
%    \begin{macrocode}
\ExplSyntaxOn
\bgroup
  \char_set_catcode_active:N \^^M
  \cs_gset:Npn \FancyVerbGetLine#1^^M
    {
      \@nil%
      % check if line contains |\end|
      \FV@CheckEnd{#1}%
      % check if we are within the environment
      \if_meaning:w \@tempa\FV@EnvironName            
        \if_meaning:w \@tempb\FV@@@CheckEnd
        \else
          \FV@BadEndError
        \fi
        \cs_set_eq:NN \next\fv_end_scanning: 
      \else:
        \cs_set:Npn \fv_line{#1}%
        \cs_set:Npn \next {\fv_preprocess_line:\FV@GetLine}%
      \fi:
    \next
    }
\egroup
\ExplSyntaxOff
%    \end{macrocode}
% \end{macro}
%

% \begin{macro}{\fv_end_scanning: }
%  
%    \begin{macrocode}
\ExplSyntaxOn
\cs_set:Npn \fv_end_scanning: 
  {
    \cs_set_nopar:Npx \next{\exp_not:N \end{\FV@EnvironName}}
    % tex_global:Dly set the marker for the comparisons
    \cs_gset_eq:NN \FV@EnvironName\scan_stop:
    % iterate
    \next
  }
\ExplSyntaxOff  
%    \end{macrocode}
%
% \end{macro}
%    \begin{macrocode}
\def\FV@BadEndError{%
  \exp_after:wN \@temptokena\exp_after:wN {\@tempb}%
  \FV@Error
      {Extraneous input `\the\@temptokena' between
        \string\end{\FV@EnvironName} and line end}%
      {This input will be discarded. Type <return> to continue.}}

\ExplSyntaxOn  
\cs_if_free:cTF {@currenvline}
   { \cs_set_eq:NN \@currenvline\@empty } {}

\def\fv_check_scan:n #1{\@ifnextchar\@nil{\@gobble}{\FV@EOF}}


\msg_new:nnn {xfancyvrb}{no-end} {I~could~not~find~
    `\string\end{\FV@EnvironName}'~to~end~
    a~verbatim~environment~\@currenvline. 
    Probably~ you~mistyped~the environment~name or 
    included an extraneous
    space, or are using an improperly defined 
    verbatim environment.
    Hit return and I will try to terminate this job.}
    
\cs_set:Npn \fv_check_scan:n  #1
  {
    \if_meaning:w \@nil#1\@empty
    \else:
    \exp_after:wN
    \msg_fatal:nn {xfancyvrb}{no-end}
    \fi:
  }



\ExplSyntaxOff
%    \end{macrocode}
%
%    \begin{macrocode}
\ExplSyntaxOn
\def\FV@EOF{%
  \FV@Error{Couldn't find `\string\end{\FV@EnvironName}' to end
    a verbatim environment\@currenvline.}%
    {Probably you mistyped the environment name or included an extraneous
    ^^Jspace, or are using an improperly defined verbatim environment.
    ^^JHit return and I will try to terminate this job.}%
  \fv_end_scanning: 
  \end{document}}
\ExplSyntaxOff  
%    \end{macrocode}  


% \subsection{Verbatim input}
%
% \begin{macro}{\fv_infile,\fv_input:}
% Allocate a stream and read the contents line by line. Send to preprocess
% which takes care of what to do. \Verb*|a b c|
%    \begin{macrocode}
\ExplSyntaxOn
\ior_new:N \fv_infile

\cs_set:Npn \fv_input: #1
  {
    \ior_open:Nn \fv_infile{#1}
    \ior_if_eof:NTF \fv_infile
      {
        \FV@Error{No verbatim file #1}\FV@eha
        \ior_close:N \fv_infile
      }  
      {
        \fv_catcodes:
        \fv_input_aux:
      }
  }
  
\cs_set:Npn \fv_input_aux:
  {
    \cs_set:Npn \fv_line{}
    \fv_readline:
    \ior_if_eof:NTF\fv_infile
      {
        \if_meaning:w \fv_line\@empty
        \else
          \fv_preprocess_line:
        \fi
        \ior_close:N \fv_infile
      }
      {
        \fv_preprocess_line:
        \fv_input_aux:
      }
  }
\ExplSyntaxOff  
%    \end{macrocode}
% \end{macro}

% \begin{macro}{\fv_readline:}
% This can be replaced |ior_str_get?|
% Reads one line from a stream.
%    \begin{macrocode} 
\ExplSyntaxOn
\bgroup
\char_set_catcode_active:N \^^M 
\cs_gset:Npn \fv_readline:
  {
    \ifeof\fv_infile
    \else
      %\immediate\read\fv_infile to\@tempa%
      \ior_get:NN \fv_infile\@tempa
      \exp_after:wN 
        \fv_readLine_aux:nnn \@tempa^^M\scan_stop:^^M\@nil
    \fi
  }
\cs_gset:Npn \fv_readLine_aux:nnn #1^^M#2^^M#3\@nil
  {
    \exp_after:wN \def\exp_after:wN\fv_line\exp_after:wN{%
      \fv_line#1}%
   \if_meaning:w \scan_stop:#2\@empty\exp_after:wN \fv_readline:\fi
  }
\egroup 
\ExplSyntaxOff
%    \end{macrocode}
% \end{macro}
%
%    \begin{macrocode}
\ExplSyntaxOn
\cs_set:Npn \fv_formatting_prep:
  {
    \tex_global:D\fv_codeline_int\z@
    \frenchspacing                   % Cancels special punctuation spacing.
    \fv_setup_font:                  % See below.
    \fv_define_whitespace:           % See below.
    \tl_use:N \fv_define_active_tl   % Use any active user definitions
    \FancyVerbFormatCom
    
  }       % A user-defined hook (formatcom parameter).
\ExplSyntaxOff  
%    \end{macrocode}  

% \begin{macro}{\fv_setup_font:}
%     Setup the fonts based on a test for existence of |\selectfont| which should not be necessary. This
%     part I am not happy at all. Original fancyvrb code more all less hard-coded font families, did
%     not allow for coloring or text direction.
%    \begin{macrocode}
\ExplSyntaxOn
\exp_after:wN 
  \if_meaning:w \csname selectfont\endcsname\scan_stop:
    \cs_set:Npn \fv_setup_font:
      {
        \FV@BaseLineStretch
        \if_meaning:w \@currsize\small
          \normalsize
        \else
          \small
        \fi
        \@currsize
        \fv_fontsize:
        \FV@FontFamily
      }
  \else:
    \cs_set:Npn \fv_setup_font:
     {
       
       \FV@BaseLineStretch
       \bfseries\color{blue}
       \macro@font\normalsize
       \fv_fontsize:
       \FV@FontFamily
%      \FV@FontSeries
%      \FV@FontShape
       \selectfont
       \@noligs
     }
  \fi:
\ExplSyntaxOff
%    \end{macrocode}
% \end{macro}

%
% \subsection{Font parameter keys}
% Define all four axes of NFSS fonts as keys.
%
% \begin{macro}{fontsize}
% Define a key for font sizing.
%    \begin{macrocode}
\ExplSyntaxOn
\fv_define_key:nnnn{FV}{fontsize}
  {  
     %\show\fv_auto_tl
     \cs_set_nopar:Npx \@tempa{#1}
     %\show\@tempa
     \cs_if_eq:NNTF\@tempa\fv_auto_tl
    {
      \cs_set_eq:NN \fv_fontsize:\scan_stop:
    }  
    {
      \cs_set:Npn \fv_fontsize:{#1}
    
    }
  }

\cs_set:Npn \KV@FV@fontsize@default
  {
    \cs_set_eq:NN \fv_fontsize:\scan_stop:
  }  
  
\ExplSyntaxOff  
%    \end{macrocode} 
% \end{macro}
% 
%    \begin{macrocode}
\ExplSyntaxOn
\fv_define_key:nnnn{FV}{baselinestretch}[auto]{%
  \cs_set_nopar:Npx \@tempa{#1}%
  \if_meaning:w \@tempa\fv_auto_tl
    \cs_set_eq:NN \FV@BaseLineStretch\scan_stop:
  \else:
    \cs_set:Npn\FV@BaseLineStretch{\cs_set:Npn \baselinestretch{#1}}%
  \fi:
}

\def\KV@FV@baselinestretch@default{\cs_set_eq:NN \FV@BaseLineStretch\scan_stop:}

% set without a |\| i.e., fontfamily=tt
\fv_define_key:nnnn{FV}{fontfamily}{%
  \cs_if_free:cTF {FV@fontfamily@#1}%
    {\cs_set:Npn \FV@FontScanPrep{}\cs_set:Npn \FV@FontFamily{\fontfamily{#1}}}
    {\use:c {FV@fontfamily@#1}}
}   
\ExplSyntaxOff
%    \end{macrocode}
%    \begin{macrocode} 
\ExplSyntaxOn 
\fv_define_key:nnnn{FV}{fontseries}{%
  \cs_set_nopar:Npx \@tempa{#1}%
  \if_meaning:w \@tempa\fv_auto_tl
    \cs_set_eq:NN \FV@FontSeries\scan_stop:
  \else:
    \cs_set:Npn \FV@FontSeries{\fontseries{#1}}%
  \fi:
}
\ExplSyntaxOff
%    \end{macrocode}
% \begin{macro}{fontshape}
%    \begin{macrocode}
\ExplSyntaxOn  
\fv_define_key:nnnn{FV}{fontshape}{%
  \cs_set_nopar:Npx \@tempa{#1}%
  \if_meaning:w \@tempa\fv_auto_tl
    \cs_set_eq:NN \FV@FontShape\scan_stop:
  \else
    \def\FV@FontShape{\fontshape{#1}}%
  \fi}
\ExplSyntaxOff
%    \end{macrocode}
% \end{macro}
%
% \begin{macro}{FV@MakeActive, \FV@MakeUnActive}
% In an expl3 catcode regime |~| is category 10 for space. We set it back to other
% As it interfered in unknow ways!
%
% However, as short so far explanation is as follows:
% It has to do with ligatures the |@noligs| and the backquote character.
% It is all unecessary if we use a more modern font such as lmodern as the font.
%    \begin{macrocode}
\ExplSyntaxOn
\char_set_catcode_other:N \~
\def\FV@MakeActive#1{%
  \catcode`#1=\active
  
  % this can be simplified?
  \def\next##1{\expandafter\def\expandafter\FV@MakeUnActive\expandafter{%
      \FV@MakeUnActive\def##1{\string##1}}}%
    
    % use lccode to keep original catcode of token  
    \bgroup
      \lccode`~=`#1\relax\expandafter\next\expandafter~
    \egroup
  }

\cs_set:Npn \FV@MakeUnActive{}
%    \end{macrocode}
% \end{macro}
% 
% \begin{macro}{\FV@ScanPrep}
%    \begin{macrocode}
\bgroup
% Set the backquote to active
\char_set_catcode_active:N \`

\cs_gset:Npn \FV@fontfamily@tt%
  {%
    \def\FV@FontScanPrep{\FV@MakeActive\`}%
    \def\FV@FontFamily{\ttfamily\edef`{{\string`}}}%
  }
  
  
\cs_gset:Npn \FV@fontfamily@cmtt
  {
    \def\FV@FontScanPrep{\FV@MakeActive\`}
    \def\FV@FontFamily
      {
        \edef`{{\string`}}\fontfamily{cmtt}
      }
  }
\egroup
\ExplSyntaxOff
%    \end{macrocode}
% \end{macro}
% 
% \begin{Verbatim}[fontfamily=cmtt,gobble=1]
%   This is some 'text' with `quotes` and fi typset in cmtt
% \end{Verbatim}
% \begin{Verbatim}[fontfamily=tt,gobble=1]
%   This is some 'text' with `quotes` and fi typset in tt
% \end{Verbatim}
% \begin{Verbatim}[fontfamily=helvetica,gobble=1]
%   This is some 'text' with `quotes` and fi typeset in courier
% \end{Verbatim}
%
% \begin{macro}{tt, cmtt,cmtt-spanish, courier,helvetica}
% Define some key values for the typewriter font. This is a very awkward way to define
% them. I will simplify later, as well as make it more user friendly.
%    \begin{macrocode}
\ExplSyntaxOn
\cs_set:cpn {FV@fontfamily@cmtt-spanish}
  {
    \def\FV@FontScanPrep{}
    \def\FV@FontFamily{\fontfamily{cmtt}}
  }
\cs_set:cpn {FV@fontfamily@courier}
  {
    \cs_set:Npn \FV@FontScanPrep{}%
    \cs_set:Npn \FV@FontFamily{\fontfamily{pcr}}
  }
\cs_set:cpn {FV@fontfamily@helvetica}
  {
    \def\FV@FontScanPrep{}
    \let\FV@FontFamily\sffamily
  }
\cs_set:cpn {FV@fontfamily@helvetica}
  {
    \def\FV@FontScanPrep{}
    \cs_set_eq:NN \FV@FontFamily\sffamily
  }
\cs_set:cpn {FV@fontfamily@panunicode}
  {\def\FV@FontScanPrep{}\let\FV@FontFamily\panunicode
  }
\ExplSyntaxOff
\fvset{fontfamily=pcr}
%    \end{macrocode}
% \end{macro}

% \begin{macro}{\myFont}
%    \begin{macrocode}

\ExplSyntaxOn
\def\myFont{}
\use:c {FV@fontfamily@myFont}{\def\FV@FontScanPrep{}\def\FV@FontFamily{\myFont}}
\fvset{fontfamily=tt,fontsize=auto,fontshape=auto,%
       baselinestretch=1}
\ExplSyntaxOff       
%    \end{macrocode}
% \end{macro}

%
% \begin{macro}{\FVDefineWhiteSpace} 
% The space and tab characters are defined and then set to appropriate
% internal macros.
%    \begin{macrocode}
\ExplSyntaxOn
\bgroup
  \catcode`\ =\active
  \catcode`\^^I=\active
 
  \gdef\fv_define_whitespace:{\def {\fv_space_tl}\def^^I{\fv_tab:}}%
\egroup
\ExplSyntaxOff
%    \end{macrocode}
% \end{macro}

% \begin{macro}{\fv_tab}
% Puts a tab in an hbox and typesets.
%    \begin{macrocode}
\ExplSyntaxOn
\cs_set:Npn \fv_tab:
  {
    \dim_set:Nn \l_tmpa_dim {\fv_verb_tab_size_tl\fontdimen2\font}
    \hbox_to_wd:nn\l_tmpa_dim {\hss\fv_tab_char}
  }
\ExplSyntaxOff
%    \end{macrocode}
% \end{macro}
%
% \begin{macro}{defineactive}
%    \begin{macrocode}
\ExplSyntaxOn
\tl_new:N \fv_define_active_tl
\fv_define_key:nnnn{FV}{defineactive}[]
  {
    \cs_set:Npn \fv_define_active_tl{#1\scan_stop:}
  }

\fv_define_key:nnnn{FV}{defineactive*}
  {
    \tl_put_right:Nn \fv_define_active_tl {#1}
  }
\fvset{defineactive}
\ExplSyntaxOff
%    \end{macrocode}
% \end{macro}
%
% \begin{macro}{showspaces}
% The default definition of visible spaces (|showspaces=true|) could allow font commands to escape under some circumstances, depending on how it is used:

% When redefining the tab, you should include the font family, font shape, and text color in the definition.  Otherwise these may be inherited from the surrounding text.  This is particularly important when using the tab with syntax highlighting, such as with the \pkg{minted} or \pkg{pythontex} packages. \pkg{fvextra} patches \pkg{fancyvrb} tab expansion so that variable-width symbols such as |\rightarrowfill| may be used as tabs. 
% 
% I have changed the below to use |\textvisiblespace| instead of |{\tt }|, thus making it incompatible with
% people using \tex.
%
%    \begin{macrocode}
\ExplSyntaxOn
\fv_define_boolean_key:nnTF{FV}{showspaces}
  {\def\fv_space_tl{{\FancyVerbSpace}}}
  {\def\fv_space_tl{\ }}
  {\catcode`\~=12 \cs_gset:Npn \FancyVerbSpace{\textvisiblespace}}
  
\fvset{showspaces=false}
\ExplSyntaxOff
%    \end{macrocode}
% \end{macro}
%
% \begin{macro}{tabsize, \fv_tab_size_tl}
%  Define the |tabsize| key, which has to be less than 100. Store it in |\fv_verb_tab_size_tl|.
%
%    \begin{macrocode}
\ExplSyntaxOn
\fv_define_key:nnnn{FV}{tabsize}
  {
    \int_set:Nn \l_tmpa_int{#1}
    \if_int_compare:w \l_tmpa_int > 100
      \FV@Error{Tab size too large: `\int_use:N\l_tmpa_int'. (Max size = 100)}\FV@eha
    \else:
    \cs_set_nopar:Npx \fv_verb_tab_size_tl { \int_use:N\l_tmpa_int }
    \fi:
  }
\ExplSyntaxOff  
%    \end{macrocode}
% \end{macro}
%
% \begin{macro}{showtabs}
%    \begin{macrocode}
\ExplSyntaxOn  
\fv_define_boolean_key:nnTF{FV}{showtabs}
  {\cs_set:Npn \fv_tab_char{\fancy_verb_tab}}
  {\cs_set_eq:NN \fv_tab_char\scan_stop:}
  
\fvset{tabsize=2,showtabs=true}
\ExplSyntaxOff
%    \end{macrocode}
% \end{macro}
%
% \begin{macro}{\fancy_verb_tab \FancyVerbTab}
%  Draws the tab symbol |\fancy_verb_tab| \FancyVerbTab
%
%    \begin{macrocode}
\ExplSyntaxOn
\cs_set:Npn \fancy_verb_tab
  { 
    \tex_valign:D
      {
        \vfil##\vfil\tex_cr:D
        \hbox:n {$\scriptscriptstyle-$} \tex_cr:D
        \hbox_to_wd:nn {0pt}
          { 
            \hss$\scriptscriptstyle\rangle\mskip -.8mu$ 
          }\tex_cr:D
        \hbox:n 
          {
            $\scriptstyle\mskip -3mu\mid\mskip -1.4mu$
          }\cr
    }
  }
\let\FancyVerbTab \fancy_verb_tab  
\ExplSyntaxOff      
%    \end{macrocode}
% \end{macro}
%
% \begin{macro}{\fv_tab_box}
% Holds the contents of a tab.
%    \begin{macrocode}  
\ExplSyntaxOn 

\box_new:N \fv_tab_box
%    \end{macrocode}
% Initialize auxiliary. 
%    \begin{macrocode}
\cs_set:Npn \fv_obey_tabs_init_aux:
  {
    \dim_set:Nn\l_tmpb_dim {\fv_verb_tab_size_tl\fontdimen\tw@\font}
    
    \edef\fv_obey_tab_size: {\number\l_tmpb_dim}
    
    \dim_add:Nn\@tempdimb{\fontdimen\tw@\font}
    \advance\@tempdimb-\fv_verb_tab_size_tl~sp  
    
    \cs_set:Npx \FV@@ObeyTabSize{\number\@tempdimb}
    \cs_set_eq:NN \fv_obey_tabs:\@@_obey_tabs:
    \cs_set_eq:NN \fv_tab:\FV@TrueTab
  }

\cs_set:Npn \@@_obey_tabs: #1
  {
    \hbox_set:Nn \fv_tab_box {#1}
    \box_use:N \fv_tab_box
  }

\cs_set_eq:NN \fv_obey_tabs:\scan_stop:

\cs_set:Npn \FV@TrueTab
  {
    \egroup
    \@tempdima=\fv_obey_tab_size: ~sp\scan_stop:
    \@tempcnta=\wd\fv_tab_box
    \advance\@tempcnta\FV@@ObeyTabSize\scan_stop:
    \divide\@tempcnta\@tempdima
    \multiply\@tempdima\@tempcnta
    \advance\@tempdima-\wd\fv_tab_box
    \setbox\fv_tab_box\hbox\bgroup
    \unhbox\fv_tab_box\kern\@tempdima\hbox_to_wd:nn \z@ {\hss\fv_tab_char}
  }
\ExplSyntaxOff  
%    \end{macrocode}
% \end{macro}
%
% \begin{macro}{obeytabs}
%    \begin{macrocode} 
\ExplSyntaxOn   
\fv_define_boolean_key:nnTF{FV}{obeytabs}%
  {\cs_set_eq:NN \fv_obey_tabs_init: \fv_obey_tabs_init_aux:}%
  {\cs_set_eq:NN \fv_obey_tabs_init: \scan_stop:}
\fvset{obeytabs=false}
\ExplSyntaxOff
%    \end{macrocode}
% \end{macro}
%
% \begin{macro}{formatcom}
%    \begin{macrocode}
\ExplSyntaxOn
\fv_define_key:nnnn{FV}{formatcom}[]{\def\FancyVerbFormatCom{#1\scan_stop:}}

\fv_define_key:nnnn{FV}{formatcom*}{%
  \exp_after:wN \def\exp_after:wN \FancyVerbFormatCom\exp_after:wN {%
    \FancyVerbFormatCom#1\scan_stop:}}
    
\fvset{formatcom}
\ExplSyntaxOff
%    \end{macrocode}
% \end{macro}
%
% \begin{macro}{\FV@XLeftMargin,\FV@XRightMargin}
% Keys for setting |xleftmargin| and |xrightmargin|
%    \begin{macrocode}
\ExplSyntaxOn
\cs_set:Npn \FancyVerbFormatLine #1
  {
    \fv_obey_tabs:{#1}
  }
  
\fv_define_key:nnnn{FV}{xleftmargin}{\def\FV@XLeftMargin{#1}}
\cs_set_eq:NN \FV@XLeftMargin\z@

\fv_define_key:nnnn{FV}{xrightmargin}{\def\FV@XRightMargin{#1}}
\cs_set_eq:NN \FV@XRightMargin\z@

\ExplSyntaxOff
%    \end{macrocode}
% \end{macro}
%
%    \begin{macrocode}
\ExplSyntaxOn
\fv_define_boolean_key:nnTF{FV}{resetmargins}%
  {\cs_set_eq:NN \if@FV@ResetMargins\iftrue}
  {\cs_set_eq:NN \if@FV@ResetMargins\iffalse}
\fvset{resetmargins=false}
\ExplSyntaxOff
%    \end{macrocode}
% 
%    \begin{macrocode}
\ExplSyntaxOn
\fv_define_key:nnnn{FV}{listparameters}{\def\FV@ListParameterHook{#1}}
\def\FV@ListParameterHook{}
\ExplSyntaxOff
%    \end{macrocode}
% \begin{docKey}[FV]{hfuzz}{ = \Arg{dim}}{default:2pt}
%  Sets \cs{hfuzz}.
% The original code was:
% \begin{Verbatim}[fontfamily=tt,gobble=1]
% \@tempdima=#1\relax
% \edef\fv_hfuzz{\number\@tempdima sp}}
% \end{Verbatim}
% \end{docKey}
%    \begin{macrocode}
\ExplSyntaxOn
\fv_define_key:nnnn{FV}{hfuzz}
  {
    \cs_set_nopar:Npx \fv_hfuzz {\dim_to_decimal_in_sp:n { #1 }sp}
  } 
\fvset{hfuzz=2pt}
\ExplSyntaxOff
%    \end{macrocode}
% 
%    \begin{macrocode}
\ExplSyntaxOn
\fv_define_boolean_key:nnTF{FV}{samepage}%
  {\cs_set:Npn \fv_interline_penalty{\interlinepenalty\@M}}%
  {\cs_set_eq:NN \fv_interline_penalty\scan_stop:}
\fvset{samepage=false}
\ExplSyntaxOff
%    \end{macrocode}
% 
% \begin{macro}{\FV@List}
%    \begin{macrocode}
\ExplSyntaxOn
\cs_set:Npn \FV@List#1
  {
    \bgroup
    \fv_use_values:
    \fv_leave_vmode:
    \if@inlabel
    \else
       \setbox\@labels=\box\voidb@x
   \fi
   \fv_list_nesting:n {#1}%
   \FV@ListParameterHook
   \FV@ListVSpace
   \FV@SetLineWidth
   \fv_interline_penalty
   \cs_set_eq:NN \fv_process_line:n\fv_list_process_line_i:n
   \fv_catcodes:
   \fv_formatting_prep:
   \fv_obey_tabs_init: 
   \FV@BeginListFrame
  }
\ExplSyntaxOff  
%    \end{macrocode}
% \end{macro}

% \begin{macro}{\fv_leave_vmode:}
% |\leavevmode| will start a paragraph if necessary unboxing a void box.
% It has no effect in horizontal or math mode.
% l3 has its own version which does not use a a void box. in vertical mode it switches to horizontal mode, and inserts a box of width |\parindent|, followed by the |\everypar| token list.
%    \begin{macrocode}  
\ExplSyntaxOn
\cs_set:Npn \fv_leave_vmode:
  {
    \if@noskipsec
      \mode_leave_vertical:
    \else
      \if@FV@ResetMargins
        \if@inlabel
          \mode_leave_vertical:
        \fi
      \fi
    \fi
    \if_mode_vertical:
      \@noparlisttrue
    \else:
      \@noparlistfalse\unskip\par
    \fi:
  }
\ExplSyntaxOff  
%    \end{macrocode}
% \end{macro}

% \begin{macro}{\fv_nesting:n}
%    \begin{macrocode} 
\ExplSyntaxOn 
\cs_set:Npn\fv_list_nesting:n  #1
  {
    \if@FV@ResetMargins
      \@listdepth=\z@
    \else
      \if_int_compare:w\@listdepth>5\scan_stop:
        \@toodeep
      \else
        \advance\@listdepth\@ne
      \fi
    \fi
    \rightmargin\z@
    \cs:w @list\romannumeral\the\@listdepth\cs_end:
    \if_int_compare:w #1 = \z@
      \rightmargin\z@
      \leftmargin\z@
    \fi
  }
\ExplSyntaxOff  
%    \end{macrocode}
% \end{macro}
%
% \begin{macro}{\FV@ListVSpace}
% Adjusts vertical space to the top of the list.
%    \begin{macrocode} 
\ExplSyntaxOn 
\cs_set:Npn \FV@ListVSpace
  {
%  \@topsepadd\topsep
    \@topsepadd=\FancyVerbVspace
    \if@noparlist
      \advance\@topsepadd\partopsep
    \fi
    \if@inlabel
      \skip_vertical:n \parskip
    \else
      \if@nobreak
        \skip_vertical:n \parskip
        \tex_clubpenalty:D \@M
      \else
        \addpenalty\@beginparpenalty
        \@topsep\@topsepadd
        \advance\@topsep\parskip
        \addvspace\@topsep
      \fi
    \fi
    \tex_global:D\@nobreakfalse
    \tex_global:D\@inlabelfalse
    \tex_global:D\@minipagefalse
    \tex_global:D\@newlistfalse
  }
\ExplSyntaxOff  
%    \end{macrocode}
% \end{macro}

%
% \begin{macro}{\FV@SetLineWidth}
%    \begin{macrocode}  
\ExplSyntaxOn
\cs_set:Npn \FV@SetLineWidth
  {
    \if@FV@ResetMargins
    \else
      \advance\leftmargin\@totalleftmargin
    \fi
    \advance\leftmargin\FV@XLeftMargin\scan_stop:
    \advance\rightmargin\FV@XRightMargin\scan_stop:
    \linewidth\tex_hsize:D
    \advance\linewidth-\leftmargin
    \advance\linewidth-\rightmargin
    \hfuzz\fv_hfuzz\scan_stop:
  }
\ExplSyntaxOff  
%    \end{macrocode}
% \end{macro}
%
% \begin{macro}{\fv_list_process_line:n}
%    \begin{macrocode}  
\ExplSyntaxOn
\cs_set:Npn \fv_list_process_line:n #1
  {
    \hbox_to_wd:nn \tex_hsize:D
      {
        \kern\leftmargin
        \hbox_to_wd:nn \linewidth
          { 
            \FV@LeftListNumber
            \FV@LeftListFrame
            \FancyVerbFormatLine{#1}\hss
            \FV@RightListFrame
            \FV@RightListNumber
          }
        \hss
      }
  }
\ExplSyntaxOff  
%    \end{macrocode}
% \end{macro}

% \begin{macro}{\fv_list_process_line_i:n}
%  First level list processing.
%    \begin{macrocode}  
\ExplSyntaxOn  
\cs_set:Npn \fv_list_process_line_i:n #1
 {
  \hbox:n 
    {
      \if_box_empty:N     %if_void
        \@labels
      \else:
        \hbox_to:nn \z@{\kern\@totalleftmargin\box_use_drop:N \@labels\hss}
      \fi:
      \fv_list_process_line:n {#1}
    }
  \cs_set_eq:NN \fv_process_line:n\fv_list_process_line_ii:n
  }
\ExplSyntaxOff  
%    \end{macrocode}
% \end{macro}
%

% \begin{macro}{\fv_list_process_line_ii:n}
% In second level list nesting. Process the line and set |\fv_process_line:n| to the
% next level.
%    \begin{macrocode}  
\ExplSyntaxOn
\cs_set:Npn \fv_list_process_line_ii:n #1
  {
    \setbox\@tempboxa = \fv_list_process_line:n{#1}%
    \cs_set_eq:NN \fv_process_line:n \fv_list_process_line_iii:n
  }
\ExplSyntaxOff  
%    \end{macrocode}
% \end{macro}
%
% \begin{macro}{\fv_list_process_line_iii:n}
%    \begin{macrocode}  
\ExplSyntaxOn
\cs_set:Npn \fv_list_process_line_iii:n#1{%
  {\advance\interlinepenalty\clubpenalty\penalty\interlinepenalty}%
  \box\@tempboxa
  \setbox\@tempboxa=\fv_list_process_line:n {#1}%
  \cs_set_eq:NN \fv_process_line:n\fv_list_process_line_iv:n}
\ExplSyntaxOff
%    \end{macrocode}
% \end{macro}
%
% \begin{macro}{fv_list_process_line_iv:n}
% Fourth level list nesting processing. This does not call the next level
% but gives control to |\FV@EndList|, which calls the |\fv_process_last_line:|,
% which checks for an empty verbatim environment and then continues.
%
%    \begin{macrocode}  
\ExplSyntaxOn
\cs_set:Npn \fv_list_process_line_iv:n#1{%
  \penalty\interlinepenalty
  \box\@tempboxa
  \setbox\@tempboxa=\fv_list_process_line:n {#1}}%
\ExplSyntaxOff  
%    \end{macrocode}
% \end{macro} 
%
% \begin{macro}{\FV@EndList}
%    \begin{macrocode}  
\ExplSyntaxOn
\cs_set:Npn \FV@EndList
  {
    \fv_list_process_last_line: 
    \FV@EndListFrame
    \@endparenv
    \egroup 
    \@endpetrue
  }
\ExplSyntaxOff  
%    \end{macrocode}
% \end{macro}
%
% \begin{macro}{\fv_list_process_last_line: }
%    \begin{macrocode}  
\ExplSyntaxOn
\cs_set:Npn \fv_list_process_last_line: 
  {
    \if_meaning:w \fv_process_line:n\fv_list_process_line_iv:n
      {\advance\interlinepenalty\widowpenalty\penalty\interlinepenalty}
      \box\@tempboxa
    \else
      \if_meaning:w \fv_process_line:n\fv_list_process_line_iii:n
        {\advance\interlinepenalty\widowpenalty
          \advance\interlinepenalty\clubpenalty
          \penalty\interlinepenalty}
        \box\@tempboxa
      \else
        \if_meaning:w \fv_process_line:n\fv_list_process_line_i:n
          \FV@Error{Empty verbatim environment}{}
          \fv_process_line:n{}
        \fi
      \fi
    \fi
  }
\ExplSyntaxOff  
%    \end{macrocode}
% \end{macro}
%
% \begin{macro}{\FV@VerbatimBegin}
% Starts the list.
%    \begin{macrocode} 
\ExplSyntaxOn
\cs_set:Npn \FV@VerbatimBegin{\FV@List\z@}
\ExplSyntaxOff
%    \end{macrocode}
% \end{macro}
% 
% \begin{macro}{\FV@EndList}
%    \begin{macrocode}
\ExplSyntaxOn
\cs_set:Npn \FV@VerbatimEnd{\FV@EndList}
\ExplSyntaxOff
%    \end{macrocode}
% \end{macro}
%
%    \begin{macrocode}
\ExplSyntaxOn
\cs_set:Npn \FVB@Verbatim{\FV@VerbatimBegin\FV@Scan}
\cs_set:Npn \FVE@Verbatim{\FV@VerbatimEnd}
\ExplSyntaxOff
%    \end{macrocode}
% 
% And finally defne the environment. 
%    \begin{macrocode}
\DefineVerbatimEnvironment{Verbatim}{Verbatim}{}
%    \end{macrocode}
% 
%    \begin{macrocode}
\ExplSyntaxOn
\cs_set:Npn \FV@UseVerbatim #1
  {
    \FV@VerbatimBegin#1\FV@VerbatimEnd
    \@doendpe\tex_global:D\@ignorefalse\ignorespaces
  }
\ExplSyntaxOff  
%    \end{macrocode}
%
% \begin{macro}{\VerbatimIput}
% The command \cs{VerbatimInput} (the variants \cs{BVerbatimInput} and \cs{LVerbatimInput}
% also exist) allows inclusion of the contents of a file with verbatim formatting. Of
% course, the various parameters which we have described for customizing can still be
% used. 
%    \begin{macrocode} 
\ExplSyntaxOn 
\cs_set:Npn \VerbatimInput {\fv_command:nn{}{VerbatimInput}}
\ExplSyntaxOff
%    \end{macrocode}
% \end{macro}
%
%    \begin{macrocode}
\ExplSyntaxOn
\cs_set:Npn \FVC@VerbatimInput#1{\FV@UseVerbatim{\fv_input: {#1}}}
\ExplSyntaxOff
%    \end{macrocode} 
%    \begin{macrocode}
\ExplSyntaxOn
\cs_set:Npn \FV@LVerbatimBegin{\FV@List\@ne}
\cs_set:Npn \FV@LVerbatimEnd{\FV@EndList}
\ExplSyntaxOff
%    \end{macrocode}
% 
% \begin{macro}{LVerbatim, \FV@LVerbatimBegin}
%    \begin{macrocode}
\ExplSyntaxOn
\cs_set:Npn \FVB@LVerbatim{\FV@LVerbatimBegin\FV@Scan}
\cs_set:Npn \FVE@LVerbatim{\FV@LVerbatimEnd}

\DefineVerbatimEnvironment{LVerbatim}{LVerbatim}{}

\cs_set:Npn \FV@LUseVerbatim#1
  {
    \FV@LVerbatimBegin#1\FV@LVerbatimEnd
    \@doendpe\tex_global:D\@ignorefalse\ignorespaces
  }
\ExplSyntaxOff  
%    \end{macrocode}
% \end{macro}
%
%
% \begin{macro}{\LVerbatimInput}
%    \begin{macrocode}
\ExplSyntaxOn
\cs_set:Npn \LVerbatimInput{\fv_command:nn{}{LVerbatimInput}}
\cs_set:Npn \FVC@LVerbatimInput#1{\FV@LUseVerbatim{\fv_input: {#1}}}
\ExplSyntaxOff
%    \end{macrocode}
% \end{macro}
%
% \section{Frames}
%    \begin{macrocode}
\ExplSyntaxOn
\cs_set:Npn  \FV@Frame@none
  {
    \cs_set_eq:NN \FV@BeginListFrame\scan_stop:
    \cs_set_eq:NN \FV@LeftListFrame\scan_stop:
    \cs_set_eq:NN \FV@RightListFrame\scan_stop:
    \cs_set_eq:NN \FV@EndListFrame\scan_stop:
  }
  
\cs_set:Npn  \FV@Frame@single
  {
    \cs_set_eq:NN \FV@BeginListFrame\FV@BeginListFrame@Single
    \cs_set_eq:NN \FV@LeftListFrame\FV@LeftListFrame@Single
    \cs_set_eq:NN \FV@RightListFrame\FV@RightListFrame@Single
    \cs_set_eq:NN \FV@EndListFrame\FV@EndListFrame@Single
  }
  
\cs_set:Npn  \FV@Frame@lines
  {
    \cs_set_eq:NN \FV@BeginListFrame\FV@BeginListFrame@Lines
    \cs_set_eq:NN \FV@LeftListFrame\scan_stop:
    \cs_set_eq:NN \FV@RightListFrame\scan_stop:
    \cs_set_eq:NN \FV@EndListFrame\FV@EndListFrame@Lines
  }
  
\cs_set:Npn  \FV@Frame@topline
  {
    \cs_set_eq:NN \FV@BeginListFrame\FV@BeginListFrame@Lines
    \cs_set_eq:NN \FV@LeftListFrame\scan_stop:
    \cs_set_eq:NN \FV@RightListFrame\scan_stop:
    \cs_set_eq:NN \FV@EndListFrame\scan_stop:
  }
  
\cs_set:Npn  \FV@Frame@bottomline{%
  \cs_set_eq:NN \FV@BeginListFrame\scan_stop:
  \cs_set_eq:NN \FV@LeftListFrame\scan_stop:
  \cs_set_eq:NN \FV@RightListFrame\scan_stop:
  \cs_set_eq:NN \FV@EndListFrame\FV@EndListFrame@Lines}

%% To define a frame with only a left line
\cs_set:Npn  \FV@Frame@leftline{%
  % To define the \FV@FrameFillLine macro (from \FV@BeginListFrame)
  \if_meaning:w \fv_fillcolor_tl\scan_stop:
    \cs_set_eq:NN \FV@FrameFillLine\scan_stop:
  \else
    \@tempdima\FV@FrameRule\scan_stop:
    \multiply\@tempdima-\tw@
    \edef\FV@FrameFillLine{%
      {\noexpand\fv_fillcolor_tl{\vrule\@width\number\@tempdima sp}%
      \kern-\number\@tempdima sp}}%
  \fi
  \cs_set_eq:NN \FV@BeginListFrame\scan_stop:
  \cs_set_eq:NN \FV@LeftListFrame\FV@LeftListFrame@Single
  \cs_set_eq:NN \FV@RightListFrame\scan_stop:
  \cs_set_eq:NN \FV@EndListFrame\scan_stop:}
  
\cs_set:Npn  \FV@BeginListFrame@Single{%
  \lineskip\z@
  \baselineskip\z@
  \if_meaning:w \fv_fillcolor_tl\scan_stop:
    \cs_set_eq:NN \FV@FrameFillLine\scan_stop:
  \else
    \@tempdima\FV@FrameRule\scan_stop:
    \multiply\@tempdima-\tw@
    \edef\FV@FrameFillLine{%
      {\noexpand\fv_fillcolor_tl{\vrule\@width\number\@tempdima sp}%
      \kern-\number\@tempdima sp}}%
  \fi
%% DG/SR modification begin - May. 19, 1998
%%  \fv_single_frameline:n 
  \fv_single_frameline:n {\z@}%
%% DG/SR modification end
  \penalty\@M
  \FV@SingleFrameSep
  \penalty\@M}

%    \end{macrocode}
% \begin{macro}{label}
% Print a label at the top or bottom depending on the |labelposition|. If the labelposition
% is set to |all| accept a parameter similar to a |\newcommand|. This must be braced.
%    \begin{macrocode}
\fv_define_key:nnnn{FV}{label}
  {
    \cs_set_nopar:Npx  \@tempa{#1}
    \if_meaning:w \@tempa\fv_none_tl
      \cs_set_eq:NN \fv_label_begin:n \scan_stop:
      \cs_set_eq:NN \fv_label_end:n \scan_stop:
    \else:
      \fv_labeli:n#1\@nil
    \fi:
  }
\ExplSyntaxOff  
%    \end{macrocode}
% \end{macro}
%
% The auxiliary macros.
%    \begin{macrocode} 
\ExplSyntaxOn 
\cs_set:Npn \fv_labeli:n{\@ifnextchar[{\fv_label_ii}{\fv_label_ii[]}}

\cs_set:Npn \fv_label_ii[#1]#2\@nil
  {
    \cs_set_nopar:Npn \@tempa{#1}%
    \cs_if_eq:NNTF \@tempa\empty
      {\cs_set:Npn \fv_label_begin:n {#2}}
      {
        \cs_set:Npn \fv_label_begin:n {#1}
        \cs_set:Npn \@@_label_position_bottomline:{\@ne}
      }

      \cs_set:Npn \fv_label_end:n {#2}
  }
  
\fvset{label={Test}}
%    \end{macrocode}
% \begin{macro}{labelposition}
% 		Key for positioning of label in framed verbatims. Possible options, |none|,
% 		|topline|, |bottomline|, |all|.
%
%    \begin{macrocode}
\fv_define_key:nnnn{FV}{labelposition}[none]
  {
    \cs_if_free:cTF {FV@LabelPosition@#1}
      { \FV@Error{Label position `#1' not defined.}\FV@eha }
      { \use:c {FV@LabelPosition@#1} }
    }
    
\cs_set:Npn  \FV@LabelPosition@none
  {
    \cs_set_eq:NN   \@@_label_position_topline:\scan_stop:
    \cs_set_eq:NN   \@@_label_position_bottomline:\scan_stop:
  }
  
\cs_set:Npn  \FV@LabelPosition@topline
  {
    \cs_set:Npn   \@@_label_position_topline:{\@ne}
    \cs_set_eq:NN \@@_label_position_bottomline:\scan_stop:
  }
  
\cs_set:Npn  \FV@LabelPosition@bottomline
  {
    \cs_set_eq:NN \@@_label_position_topline:\scan_stop:
    \cs_set:Npn  \@@_label_position_bottomline:{\@ne}
  }
%    \end{macrocode} 
% \end{macro}
%
% The |labelposition=all| can take a value of all. This permits different 
% labels to be positioned at the topline and bottom line but the syntax
% is not very user friendly. It is modelled after |\newcommand|
% |label={[Start of code]End of code}|.
% 
%    \begin{macrocode}
\cs_set:Npn  \@@_label_position_all
  {
  \cs_set:Npn  \@@_label_position_topline: {\c_one_int}
  \cs_set:Npn  \@@_label_position_bottomline: {\c_one_int}
  }
  
\cs_set_eq:NN \FV@LabelPosition@all \@@_label_position_all  

\fvset{labelposition=topline}

%    \end{macrocode}
% \begin{macro}{\fv_single_frameline:n }
% Draws a single line. Based on its value it selects the position to print. Numbers are outside the box. Not very good typography. Everything is drawn in an hbox. Changed hbox to l3 to allow for coloring.
% Persevere on the new code and then refactor.
%    \begin{macrocode}
\cs_set:Npn  \fv_single_frameline:n #1
  {
    \hbox_to_wd:nn \z@ 
      {
        \kern\leftmargin
        \if_int_compare:w#1=\z@
          \cs_set_eq:NN \FV@Label\fv_label_begin:n 
        \else
          \cs_set_eq:NN \FV@Label\fv_label_end:n 
        \fi
        \if_meaning:w \FV@Label\scan_stop:
          \fv_verb_rule_color:{\vrule \@width\linewidth \@height\FV@FrameRule}
        \else
          \if_int_compare:w#1=\z@
            \setbox\z@\hbox:n {\strut\enspace\fv_label_begin:n \enspace\strut}%
          \else
            \setbox\z@\hbox:n {\strut\enspace\fv_label_end:n \enspace\strut}%
          \fi
          \@tempdimb=\dp\z@
          \advance\@tempdimb -.5\ht\z@
          \@tempdimc=\linewidth
          \advance\@tempdimc -\wd\z@
          \divide\@tempdimc\tw@
          
          % topline
          \if_int_compare:w#1=\z@              
            \if_meaning:w \@@_label_position_topline:\scan_stop:
              \fv_verb_rule_color:
                {
                  \vrule \@width\linewidth \@height\FV@FrameRule
                }
            \else
              \fv_frame_line_with_label: 
            \fi
          % bottomline  
          \else                     
            \if_meaning:w \@@_label_position_bottomline:\scan_stop:
            \fv_verb_rule_color:{\vrule \@width\linewidth \@height\FV@FrameRule}%
            \else
              \fv_frame_line_with_label: 
            \fi
          \fi
        \fi
        \hss
      }
  }
\ExplSyntaxOff    
%    \end{macrocode}
% \end{macro}
%
% \begin{macro}{\fv_frame_line_with_label:}
%    \begin{macrocode}  
\ExplSyntaxOn  
\cs_set:Npn \fv_frame_line_with_label: 
  {
    \ht\z@\@tempdimb\dp\z@\@tempdimb%
    \fv_verb_rule_color:
      {
        \vrule \@width\@tempdimc \@height\FV@FrameRule
        \raise\@tempdimb\box\z@
        \vrule \@width\@tempdimc \@height\FV@FrameRule
      }
  }
%    \end{macrocode}
% \end{macro}
%
% \begin{macro}{\FV@BeginListFrame@Lines}
%    \begin{macrocode}
\cs_set:Npn \FV@BeginListFrame@Lines{%
  \bgroup
    \lineskip\z@skip
    \fv_single_frameline:n {\z@}%
    \kern-0.5\baselineskip\scan_stop:
    \baselineskip\z@skip
    \kern\FV@FrameSep\scan_stop:
  \egroup }%
%    \end{macrocode}  
% \end{macro}

%    \begin{macrocode}
\cs_set:Npn    \FV@EndListFrame@Lines{%
  \bgroup
    \baselineskip\z@skip
    \kern\FV@FrameSep\scan_stop:
    \fv_single_frameline:n {\@ne}%
  \egroup }
%    \end{macrocode}  
%    \begin{macrocode}  
\cs_set:Npn    \FV@SingleFrameSep{%
  \hbox_to_wd:nn \z@{%
    \kern\leftmargin
    \hbox_to_wd:nn \linewidth{%
      \fv_verb_rule_color:{%
        \if_meaning:w \fv_fillcolor_tl\scan_stop:
          \vrule\@width 0pt\@height\FV@FrameSep\scan_stop:
        \fi
        \vrule\@width\FV@FrameRule\scan_stop:
        \if_meaning:w \fv_fillcolor_tl\scan_stop:
          \hfil
        \else
          {\fv_fillcolor_tl\leaders\hrule\@height\FV@FrameSep\hfil}%
        \fi
        \if_meaning:w \fv_fillcolor_tl\scan_stop:
          \vrule\@width 0pt\@height\FV@FrameSep\scan_stop:
        \fi
        \vrule\@width\FV@FrameRule\scan_stop:}}%
    \hss}}
%    \end{macrocode}  
%    \begin{macrocode}    
\cs_set:Npn \FV@LeftListFrame@Single{%
  \strut
  {\fv_verb_rule_color:{\vrule \@width\FV@FrameRule}}%
  \FV@FrameFillLine
  \if_meaning:w \fv_fillcolor_tl\scan_stop:
    \kern\FV@FrameSep
  \else
    {\noexpand\leavevmode\fv_fillcolor_tl{\vrule\@width\FV@FrameSep}}%
  \fi}
%    \end{macrocode}  
%    \begin{macrocode}  
\cs_set:Npn \FV@RightListFrame@Single
  {
    \if_meaning:w \fv_fillcolor_tl\scan_stop:
      \kern\FV@FrameSep
    \else
      {\noexpand\leavevmode\fv_fillcolor_tl{\vrule\@width\FV@FrameSep}}%
    \fi
  {
    \noexpand\leavevmode\fv_verb_rule_color:{\vrule\@width\FV@FrameRule}}
  }

\cs_set:Npn \FV@EndListFrame@Single
  {
    \penalty\@M
    \FV@SingleFrameSep
    \penalty\@M
    \fv_single_frameline:n {\@ne}
  }
%    \end{macrocode} 
%
% \begin{macro}{framerule} 
%    \begin{macrocode}  
\fv_define_key:nnnn{FV}{framerule}{%
  \@tempdima=#1\scan_stop:
  \edef\FV@FrameRule{\number\@tempdima sp\scan_stop:}}
  
\cs_set:Npn \KV@FV@framerule@default{\cs_set_eq:NN \FV@FrameRule\fboxrule}
%    \end{macrocode}
% \end{macro}
%
% \begin{macro}{framesep}
% The key value |framesep| is defined in terms of \latex's 
%    \begin{macrocode}
\fv_define_key:nnnn{FV}{framesep}{%
  \@tempdima=#1\scan_stop:
  \edef\FV@FrameSep{\number\@tempdima sp\scan_stop:}}

\def\KV@FV@framesep@default{\cs_set_eq:NN \FV@FrameSep\fboxsep}

\fvset{framerule,framesep}
\ExplSyntaxOff
%    \end{macrocode}
% \end{macro}
%
% \begin{macro}{rulecolor}
%    \begin{macrocode}
\ExplSyntaxOn
\fv_define_key:nnnn{FV}{rulecolor}
  {
    \cs_set_nopar:Npx \@tempa{#1}%
    \if_meaning:w \@tempa\fv_none_tl
      \cs_set_eq:NN \fv_verb_rule_color:\scan_stop:
    \else
      \cs_set_eq:NN \fv_verb_rule_color:\@tempa
    \fi
  }
\ExplSyntaxOff  
%    \end{macrocode}
% \end{macro} 
%
% \begin{macro}{fillcolor, \fv_fillcolor_tl}
%    \begin{macrocode}  
\ExplSyntaxOn
\fv_define_key:nnnn{FV}{fillcolor}
  {
    \cs_set_nopar:Npx \@tempa{#1}
    \if_meaning:w \@tempa\fv_none_tl
      \cs_set_eq:NN \fv_fillcolor_tl\scan_stop:
    \else:
      \cs_set_eq:NN \fv_fillcolor_tl\@tempa
    \fi:
  }
\fvset{rulecolor=none,fillcolor=none}
\ExplSyntaxOff
%    \end{macrocode}
% \end{macro} 
%
% \begin{macro}{frame, \FV@Frame@double} 
% The |double| option key code definition.
%    \begin{macrocode}
\ExplSyntaxOn
\cs_set:Npn \FV@Frame@double
  {
    \cs_set_eq:NN \@@_frame_begin:\FV@FrameBegin@double
    \cs_set_eq:NN \@@_frame_line: \FV@FrameLine@double
    \cs_set_eq:NN \@@_frame_end: \FV@FrameEnd@double
  }
  
\fv_define_key:nnnn{FV}{frame}[none]{%
  \@ifundefined{FV@Frame@#1}%
    {\FV@Error{Frame style `#1' not defined.}\FV@eha}%
    {\use:c {FV@Frame@#1}}}
\fvset{frame=none}
\ExplSyntaxOff
%    \end{macrocode}
% \end{macro}
%
% \begin{macro}{firstnumber, \fv_set_lineno:n}
% Define the |firstnumber| key.  
%    \begin{macrocode}
\ExplSyntaxOn
\fv_define_key:nnnn{FV}{firstnumber}[auto]
{
    \cs_set:Npn \@tempa{#1}\cs_set:Npn \@tempb{auto}
    \if_meaning:w \@tempa\@tempb
      \cs_set:Npn \fv_set_lineno:n
        {
          \int_set:Nn \c@fv_verbline_int\fv_codeline_int
          \int_set:NN \c@fv_verbline_int\m@ne
        }
    \else:
      \def\@tempb{last}
      \if_meaning:w \@tempa\@tempb
        \cs_set_eq:NN \fv_set_lineno:n\scan_stop:
      \else:
        \cs_set:Npn \fv_set_lineno:n
          {
            \int_set:Nn \c@fv_verbline_int {#1}
            \int_add:Nn \c@fv_verbline_int {\m@ne}
          }
      \fi:
   \fi:
}  
\ExplSyntaxOff  
%    \end{macrocode}
% \end{macro}
%
%    \begin{macrocode}
\ExplSyntaxOn  
\fv_define_boolean_key:nnTF{FV}{numberblanklines}
  {\cs_set_eq:NN \if@FV@NumberBlankLines\iftrue}
  {\cs_set_eq:NN \if@FV@NumberBlankLines\iffalse}
\fvset{numberblanklines=true}
\ExplSyntaxOff
%    \end{macrocode}
%
% \begin{macro}{\fv_refstepcounter:n, \FV@StepLineNo}
% We need to be able to step a counter as well as set its reference.
% 
%    \begin{macrocode}
\ExplSyntaxOn
\cs_set:Npn \fv_refstepcounter:n#1
  {
    \int_gincr:c {c@#1}
    \protected@edef\@currentlabel{\csname p@#1\endcsname\arabic{fv_verbline_int}}
  }
%    \end{macrocode}
%
%    \begin{macrocode}
\cs_set:Npn \FV@StepLineNo
  {
    % defined when counter is stepped   
    \fv_set_lineno:n
    \cs_set:Npn \FV@StepLineNo
      {
        \if@FV@NumberBlankLines
          \fv_refstepcounter:n{fv_verbline_int}
        \else
          \if_meaning:w \fv_line\empty
          \else:
            \fv_refstepcounter:n {fv_verbline_int}
          \fi:
        \fi
      }
    \FV@StepLineNo
  }
\ExplSyntaxOff  
%    \end{macrocode}
% \end{macro}
% 
%    \begin{macrocode}
\ExplSyntaxOn
\cs_set:Npn \theFancyVerbLine{\rmfamily\tiny\arabic{fv_verbline_int}}
\ExplSyntaxOff
%    \end{macrocode}
%
% \begin{macro}{\FV@Numbers@none,\FV@Numbers@left,\Fv@Numbers@right}
%    \begin{macrocode}
\ExplSyntaxOn
\fv_define_key:nnnn{FV}{numbers}[none]{%
  \cs_if_free:cTF {FV@Numbers@#1}
    {\FV@Error{Numbers~style~`#1'~not~defined.}\FV@eha}
    {\use:c {FV@Numbers@#1}}}
    
\cs_set:Npn \FV@Numbers@none 
  {
    \cs_set_eq:NN \FV@LeftListNumber\scan_stop:
    \cs_set_eq:NN \FV@RightListNumber\scan_stop:
  }
\ExplSyntaxOff  
%    \end{macrocode}
% \end{macro}
%
% \begin{macro}{stepnumber, \fv_stepnumber_int}
%    \begin{macrocode}
\ExplSyntaxOn
\int_new:N \fv_stepnumber_int
\fv_define_key:nnnn{FV}{stepnumber}{\fv_stepnumber_int#1}
\cs_set:Npn \KV@FV@stepnumber@default{\fv_stepnumber_int\c_one_int}

\fvset{stepnumber=1}
%    \end{macrocode}
% \end{macro}
% \begin{macro}{\FV@Numbers@Left}
%    \begin{macrocode}
\cs_set:Npn \FV@Numbers@left
  {
    \cs_set_eq:NN \FV@RightListNumber\scan_stop:
    
    \cs_set:Npn \FV@LeftListNumber
      {
        \@tempcnta=\fv_codeline_int
        \@tempcntb=\fv_codeline_int
        \divide\@tempcntb\fv_stepnumber_int
        \multiply\@tempcntb\fv_stepnumber_int
        \if_int_compare:w\@tempcnta=\@tempcntb
          \if@FV@NumberBlankLines
            \hbox_to_wd:nn \z@{\hss\theFancyVerbLine\kern\fv_number_sep:n }%
          \else
            \if_meaning:w \fv_line\empty
            \else
              \hbox_to_wd:nn \z@{\hss\theFancyVerbLine\kern\fv_number_sep:n }%
            \fi
          \fi
        \fi
      }
  }
%    \end{macrocode}
% \end{macro}
%
% \begin{macro}{\fv_numbers_right}
%    \begin{macrocode}  
\cs_set:Npn \FV@Numbers@right
  {
  \cs_set_eq:NN \FV@LeftListNumber\scan_stop:
  
  \cs_set:Npn \FV@RightListNumber
    {
      \@tempcnta=\fv_codeline_int
      \@tempcntb=\fv_codeline_int
      \divide\@tempcntb\fv_stepnumber_int
      \multiply\@tempcntb\fv_stepnumber_int
      \if_int_compare:w\@tempcnta=\@tempcntb
        \if@FV@NumberBlankLines
          \hbox_to_wd:nn \z@ {\kern\fv_number_sep:n \theFancyVerbLine\hss}
        \else
          \if_meaning:w \fv_line\empty
          \else
            \hbox_to_wd:nn \z@ {\kern\fv_number_sep:n \theFancyVerbLine\hss}%
          \fi
        \fi
      \fi
    }
  }
\ExplSyntaxOff  
%    \end{macrocode}
% \end{macro}
%
% \begin{macro}{\l_numbersep_dim, \fv_number_sep:n }
%    \begin{macrocode}
\ExplSyntaxOn
\dim_new:N \l_numbersep_dim
\fv_define_key:nnnn{FV}{numbersep}
  {
    \dim_set:Nn \l_numbersep_dim {#1} 
    \cs_set_nopar:Npx \fv_number_sep:n  {\dim_use:N \l_numbersep_dim}
  }
\fvset{numbers=left,numbersep=12pt,firstnumber=1}
\ExplSyntaxOff
%    \end{macrocode}
% \end{macro}
%
%
% \subsection{BVerbatim}
%
%   	This produces boxed verbatims. Most of the previously define key-values can be used, as well 
%   	as to new keys |boxwith| and |baseline| that are used to align boxed verbatims against
%   	others.
% \begin{macro}{\FV@BVerbatimBegin, \FV@BVerbatimEnd}
% 		Starts the process, by opening the box and process the line. 		
%    \begin{macrocode}
\ExplSyntaxOn
\cs_set:Npn \FV@BVerbatimBegin
  {
    \bgroup
      \fv_use_values:
      \FV@BeginVBox
      \cs_set_eq:NN \fv_process_line:n\FV@BProcessLine
      \fv_formatting_prep:
      \fv_obey_tabs_init: 
  }

\cs_set:Npn \FV@BVerbatimEnd{\FV@EndVBox\egroup }
%    \end{macrocode}
% \end{macro}
%
% \begin{macro}{\FV@BeginVBox}   
%  Begin an |hbox| which contains either a |vbox|,|vtop| or |vtop| depending on the
%  value of the |baseline| key. 
%    \begin{macrocode}
\cs_set:Npn \FV@BeginVBox
  {
  \leavevmode
    \hbox\if_meaning:w \FV@boxwidth\scan_stop:\else~to\FV@boxwidth\fi\bgroup
    \ifcase
      \FV@baseline
      \vbox
      \or \vtop
      \or$\vcenter
    \fi\bgroup
  }
  
\cs_set:Npn \FV@EndVBox{\egroup\ifmmode$\fi\hfil\egroup}
%    \end{macrocode}
% \end{macro}
%
% \begin{macro}{boxwidth}
%    \begin{macrocode}
\ExplSyntaxOn
\fv_define_key:nnnn{FV}{boxwidth}{%
  \def\@tempa{#1}\def\@tempb{auto}%
  \if_meaning:w \@tempa\@tempb
    \cs_set_eq:NN \FV@boxwidth\scan_stop:
  \else
    \@tempdima=#1\scan_stop:
    \edef\FV@boxwidth{\number\@tempdima sp}%
  \fi
}
\ExplSyntaxOff 
%    \end{macrocode}
% \end{macro}
%  
% \begin{macro}{\KV@FV@boxwidth@default}
%   	The code for the default value of the key.
%    \begin{macrocode} 
\ExplSyntaxOn
\cs_set:Npn \KV@FV@boxwidth@default
  {
    \cs_set_eq:NN \FV@boxwidth\scan_stop:
  }
\ExplSyntaxOff
%    \end{macrocode}
% \end{macro}
%
% \begin{macro}{baseline}
% The |baselinekey| is used to set the alignment of boxed verbatims.
%    \begin{macrocode}
\ExplSyntaxOn
\fv_define_key:nnnn{FV}{baseline}{
  \ExplSyntaxOn 
  \if t#1\@empty\cs_set_eq:NN \FV@baseline\@ne
  \else
    \if c#1\@empty\cs_set_eq:NN \FV@baseline\tw@
      \else \cs_set_eq:NN \FV@baseline\z@
   \fi
  \fi
  \ExplSyntaxOff
  }
  
\ExplSyntaxOff  
  
\fvset{baseline=b,boxwidth}
\ExplSyntaxOn
\cs_set:Npn \FV@BProcessLine #1
  {
    \hbox{\FancyVerbFormatLine{#1}}
  }
\ExplSyntaxOff
%    \end{macrocode}
% \end{macro} 
% 
%
% \begin{macro}{\FVB@BVerbatim, \FVE@BVerbatim}
%    \begin{macrocode}    
\ExplSyntaxOn
\cs_set:Npn \FVB@BVerbatim {\FV@BVerbatimBegin\FV@Scan}
\cs_set:Npn \FVE@BVerbatim{\FV@BVerbatimEnd}
\ExplSyntaxOff
%    \end{macrocode}
% \end{macro}
%
%
% \begin{macro}{BVerbatim}
%    \begin{macrocode}
\DefineVerbatimEnvironment{BVerbatim}{BVerbatim}{}
%    \end{macrocode}
% \end{macro}
%
% \begin{macro}{\FV@BUseVerbatim, \BVerbatimInput}
%    \begin{macrocode}
\ExplSyntaxOn
\cs_set:Npn \FV@BUseVerbatim#1{\FV@BVerbatimBegin#1\FV@BVerbatimEnd}
\cs_set:Npn \BVerbatimInput{\fv_command:nn{}{BVerbatimInput}}
\cs_set:Npn \FVC@BVerbatimInput#1{\FV@BUseVerbatim{\fv_input: {#1}}}
\cs_set:Npn \SaveVerbatim{\fv_environment:nn{}{SaveVerbatim}}
\ExplSyntaxOff
%    \end{macrocode}
% 
%    \begin{macrocode}
\ExplSyntaxOn
\cs_set:Npn \FVB@SaveVerbatim#1
  {
    \@bsphack
    \bgroup
      \fv_use_values:
      \cs_set:Npn \SaveVerbatim@Name{#1}%
      \cs_gset:Npn \FV@TheVerbatim{}%
      \cs_set:Npn \fv_process_line:n ##1 
        {
          \exp_after:wN \cs_gset:Npn \exp_after:wN \FV@TheVerbatim\exp_after:wN 
            {
              \FV@TheVerbatim\fv_process_line:n {##1}
            }
        }
      \cs_gset:Npn \FV@TheVerbatim{}%
      \FV@Scan
  }
    
    
\cs_set:Npn \FVE@SaveVerbatim
  {
    \exp_after:wN \tex_global:D\exp_after:wN \cs_set_eq:NN 
    \csname FV@SV@\SaveVerbatim@Name\endcsname\FV@TheVerbatim
    \egroup\@esphack
  }
  
\DefineVerbatimEnvironment{SaveVerbatim}{SaveVerbatim}{}


\def\FV@CheckIfSaved#1#2{%
  \@ifundefined{FV@SV@#1}%
  {\FV@Error{No verbatim text has been saved under name `#1'}\FV@eha}%
  {#2{\csname FV@SV@#1\endcsname}}}


\cs_set:Npn \UseVerbatim{\fv_command:nn{}{UseVerbatim}}
\cs_set:Npn \FVC@UseVerbatim#1{\FV@CheckIfSaved{#1}{\FV@UseVerbatim}}
\cs_set:Npn \LUseVerbatim{\fv_command:nn{}{LUseVerbatim}}
\cs_set:Npn \FVC@LUseVerbatim#1{\FV@CheckIfSaved{#1}{\FV@LUseVerbatim}}
\cs_set:Npn \BUseVerbatim{\fv_command:nn{}{BUseVerbatim}}
\cs_set:Npn \FVC@BUseVerbatim#1{\FV@CheckIfSaved{#1}{\FV@BUseVerbatim}}


\newwrite\FV@OutFile
\cs_set:Npn \VerbatimOut{\fv_environment:nn{}{VerbatimOut}}
\cs_set:Npn \FVB@VerbatimOut#1
  {
    \@bsphack
    \bgroup
    \fv_use_values:
    \fv_define_whitespace:
    \cs_set:Npn \fv_space_tl{\c_space_tl}
    \FV@DefineTabOut
    \cs_set:Npn \fv_process_line:n{\immediate
    
    \FV@OutFile}
    \immediate\openout\FV@OutFile #1\scan_stop: 
    \cs_set_eq:NN \FV@FontScanPrep\scan_stop:
    \cs_set_eq:NN \@noligs\scan_stop:
    \FV@Scan
  }
    
\def\FVE@VerbatimOut{\immediate\closeout\FV@OutFile\egroup\@esphack}

\DefineVerbatimEnvironment{VerbatimOut}{VerbatimOut}{}
\cs_set:Npn \FV@DefineTabOut
  {
    \def\fv_tab:{}%
    \@tempcnta=\fv_verb_tab_size_tl\scan_stop:
    \loop\if_int_compare:w\@tempcnta>\z@
      \edef\fv_tab:{\fv_tab:\space}%
      \advance\@tempcnta\m@ne
    \repeat
  }
\ExplSyntaxOff  
%    \end{macrocode}
%
% \begin{macro}{\SaveVerb}
%   \cs{SaveVerb}\Arg{cs name to save to}
%   Saves a a verbatim command |\SaveVerb{danger}= \test \something =|. Can be used to save
%   verbatims and use later using |\UseVerb{danger}|. See usage at example \ref{ex:saveverb}
%   in Chapter~\ref{ch:verbatim}.
%
% There is one additional keyword \docAuxKey[FV]{aftersave}, which takes code to execute immediately 
% after saving the verbatim text into the storage bin. 
%    \begin{macrocode}  
\ExplSyntaxOn

% reminder |\fv_command| will get a macro with |FVC| prefix and indirectly call
% \FVC@SaveVerb
\cs_set:Npn \SaveVerb{\fv_command:nn{}{SaveVerb}}
\ExplSyntaxOff
%    \end{macrocode}
% \end{macro}
%
%
% \begin{macrocode}
\ExplSyntaxOn
\bgroup

\catcode`\^^M=\active%

% \#1 name
% \#2 =
% \something\something=
%
\cs_gset:Npn \FVC@SaveVerb#1#2
  {
    \cs_set:cpn {FV@SV@#1}{}%
    \bgroup
    
      % use any setttings
      \fv_use_values:%
      \fv_catcodes:%
    
      % define the EOL as outer to give an error if the user
      % missed an enclosing token.
      % we could use balance token from verbatim to improve
      % the user interface messages here.  
      \outer\def^^M{\FV@EOL}%
      
      % The key is defined using |\FancyVerbAfterSave|
      % Change original definitions to a tl
      \tl_gset_eq:NN \g_@@_after_save_code\FancyVerbAfterSave
      
      % change the character to other = 
      \char_set_catcode_other:N #2
    
      % This is a strange consruct revisit
      \tl_set:Nn\l_tmpa_tl {\tl_to_str:n{#2}} 
      
      \cs_set:Npn\@tempa{\def\FancyVerbGetVerb####1####2}
      % #2 for example can be an =, change to string 
      \exp_after:wN \@tempa\tl_to_str:n {#2}
      {
        \egroup 
        \cs_set:cpn {FV@SV@#1}{##2}\g_@@_after_save_code
      }
      % if the delimiter is not found continue with error
      \FancyVerbGetVerb\FV@EOL
  }
\egroup 


\cs_set:Npn \FV@EOL{%
  \egroup 
  \FV@Error%
    {Could not find the end delimiter of a short verb command}%
    {You probably just forget the end delimiter of a \string\Verb\space or
      \string\SaveVerb^^J%
      command, or you broke the literal text across input lines.^^J%
      Hit <return> to proceed.}}
      
\fv_define_key:nnnn{FV}{aftersave}{\def\FancyVerbAfterSave{#1}}

\fvset{aftersave=}

\cs_set:Npn \FV@UseVerb#1{\mbox{\fv_use_values:\fv_formatting_prep:#1}}

\ExplSyntaxOff
%    \end{macrocode}
% \end{macro}
%
% \begin{macro}{\UseVerb}
%    \begin{macrocode}
\ExplSyntaxOn
\DeclareRobustCommand\UseVerb{\fv_command:nn{}{UseVerb}}

\cs_set:Npn \FVC@UseVerb #1
  {
    \@ifundefined{FV@SV@#1}%
      {\FV@Error{Short verbatim text never saved to name `#1'}\FV@eha}%
      {\FV@UseVerb{\use:c {FV@SV@#1}}}
  }
\ExplSyntaxOff    
%    \end{macrocode}
% \end{macro}
%
% \section{\textbackslash Verb}
% \begin{macro}{\Verb}
%
% The |\Verb| command acts like verbatims |\verb|, but can use a key value interface. The definition code follow very closely to that of |\SaveVerb| and |\UseVerb|
%    \begin{macrocode}    
\ExplSyntaxOn
\cs_set_nopar:Npn \Verb{\fv_command:nn{}{Verb}}
\ExplSyntaxOff
%    \end{macrocode}
% \end{macro}
%
%    \begin{macrocode}
\ExplSyntaxOn
\char_set_catcode_other:N \~
\bgroup
\catcode`\^^M=\active%
\cs_gset:Npn \FVC@Verb#1
  {
  \bgroup%
    \fv_use_values:
    
    \fv_formatting_prep:

    \fv_catcodes:

    \outer\def^^M{}%

    \catcode`#1=12%

    \def\@tempa{\def\FancyVerbGetVerb####1####2}
    \exp_after:wN \@tempa\string#1{\mbox{##2}\egroup }%

    \FancyVerbGetVerb\FV@EOL
  }
\egroup
%

%\ExplSyntaxOff
%    \end{macrocode}
%
% \section{Short Verbatims}
%
%    Provide a function to make a certain character denote the start and stop of verbatim text 
% 	   without the need to put |\verb| in front. It is possible to use the key value interface as applicable.
% 		
%   ^^A \DefineShortVerb{\"}  
%   ^^A  "\test"

%   ^^A \UndefineShortVerb{\"}
%
% 		This is to test command printed using the short verbatim.
%
% 		Just a reminder we also have the |\MakeShortVerb| from \pkg{doc}
%    Also this can be shortened dramatically using xparse, but then we may loose
%    the keyval interface
%
% \subsection{Handling special characters}
%
%  The \cs{dospecial} macro is conventionally used by verbatim macros to alter the 
%  catcodes of the currently active characters. This was originally defined by 
%  Knuth in plainTeX and carried over to LaTeX by Lamport in |ltxplain.dtx| in |source2e|.
%
% It contains the 11 special characters.
%
% \begin{Verbatim}[gobble=1]
% \def\dospecials{\do\ \do\\\do\{\do\}\do\$\do\&\do\#\do\^
%                                           \do\_\do\%\do\~}
% \end{Verbatim}
%
% As with all Knuth type lists by defining appropriate macros for |do| one
% can manipulate the list items.
%
% With \latex3 there are two sequence lists provided to keep track of these 
% special characters. \docAuxCommand{l_char_special_seq} and
% \docAuxCommand{l_char_active_seq}. These are to designed to be maintained
% at the document level.
%
% Two sequences for dealing with special characters. The first is characters which may be
% active, the second longer list is for “special” characters more generally. Both lists are
% escaped so that for example bulk code assignments can be carried out. In both cases, the
% order is by ascii character code (as is done in for example |\ExplSyntaxOn|).
%
% Used to track which tokens will require special handling when working with verbatimlike
% material at the document level as they are not of categories letter (catcode 11) or
% other (catcode 12). Each entry in the sequence consists of a single escaped token, for
% example |\\| for the backslash or |\{| for an opening brace. Escaped tokens should be added
% to the sequence when they are defined for general document use.
%
% \begin{verbatim}
% \seq_new:N \l_char_special_seq
% 
% \seq_set_split:Nnn \l_char_special_seq { }
%   { \ \" \# \$ \% \& \\ \^ \_ \{ \} \~ }
%
% \seq_new:N \l_char_active_seq
%   \seq_set_split:Nnn \l_char_active_seq { }
%   { \" \$ \& \^ \_ \~ }
%\end{verbatim}
%
%
% Orthogonally to |\dospecials| LaTeX2e has \docAuxCommand{@sanitize} which changes teh catcode of all special characters except for braces to \enquote{other}. It can be used for commands like
% cs{index} taht want to write their arguments verbatim. As mentioned in the kernel this command
% should only be executed within a group, or chaos will ensue. 
%
% When we define a short verb both lists need to be kept uptodate. In our case we will
% update also the l3 lists.
%
% Unlike to the orginal definitions of fancyvrb which had a lot of acrobatics, we will
% simplify and make teh code more understandable using sequences.
%
% \begin{macro}{\fv_add_special:Nn}
%    \begin{macrocode}  
\ExplSyntaxOn

% Define a new sequence for sanitizing all the special chars. 
% Although LaTeX's sanitize can be considered a token list we will 
% define ours as a sequence so we can iterate better over it. 
\seq_new:N \fv_sanitize_seq:N
\seq_gset_split:Nnn \fv_sanitize_seq { } 
  { \ \" \# \$ \% \& \\ \^ \_ \~ }

\cs_set_eq:NN \fv_sanitize_add:Nn \seq_gput_right:Nn
\cs_set_eq:NN \fv_sanitize_pop_right:NN \seq_pop_right:NN  

% if we are in doc there is a command to add to specials we can use it
% or define it
\cs_if_exist:NTF\add@special 
  { 
      \cs_gset_eq:NN \fv_add_special:Nn \add@special 
  }
  { 
    \cs_gset:Npn \fv_add_special:Nn
      {
        \rem@special{#1}
        \exp_after:wN
          \gdef\expandafter
            \dospecials
              \expandafter {\dospecials \do #1} 
      }
    
    \cs_set:Npn \rem@special#1
      {
        \def\do##1
          {
            \ifnum`#1=`##1 \else \noexpand\do\noexpand##1\fi
          }
          
         % update the list globally
         \xdef\dospecials{\dospecials}%
         
         % fixing \@sanitize is the same except that we need to  
         % redefine \@makeother
        \group_begin:
        \cs_set:Npn \@makeother ##1
          {
            % left original definition here
            \ifnum`#1=`##1 \else \noexpand\@makeother\noexpand##1\fi
          }
            \xdef\@sanitize{\@sanitize}
        \group_end:
     }
  }  


\cs_set:Npn \DefineShortVerb{\FV@Command{}{DefineShortVerb}}

\char_set_catcode_other:N \~
\cs_set:Npn\FVC@@DefineShortVerb#1
  {
  
  \group_begin:
    % maybe not needed but to make sure
    \char_gset_catcode_active:N #1 
    
    % remember the `#1 is a number so that the lowercase `\~
    % will behave differently
    \lccode`\~=`#1%
    \lowercase{\gdef\@tempg{\edef~}\global\let\@temph~}
  \group_end:
  
  % Save the lowercase definition? In doc saved as
  \cs_set_eq:cN {FV@AC@\string#1} \@temph
  
  
   
  % Save the catcode of the short verb marker globally. Note
  % doc saves it in cc\string#1
  \cs_gset:cpx {FV@CC@\string#1} {\the\catcode`#1}
  
  
  \cs_set_eql:cN {FV@KV@\string#1} \FV@KeyValues
  
  \@tempg{
    \let\noexpand\FV@KeyValues\expandafter\noexpand
      \csname FV@KV@\string#1\endcsname
      \noexpand\FVC@Verb\expandafter\@gobble\string#1}
    
  % Add to specials the escaped character (say\")
  % Since we have checked already if it is included in the list, we have no worries to
  % add it.
  %\expandafter\def\expandafter\dospecials\expandafter{\dospecials\do#1}%
  \fv_add_special:Nn\dospecials{\do#1}
  % Add to the sanitize list
  %\expandafter\def\expandafter\@sanitize\expandafter{\@sanitize\@makeother#1}
  
  % Make the character active
  \catcode`#1=\active
  }

\ExplSyntaxOff  
%    \end{macrocode}
% \end{macro}  

% \begin{macro}{}
% \begin{macrocode} 
\ExplSyntaxOn
\def\UndefineShortVerb#1
  {
    \@ifundefined{FV@CC@\string#1}%
      {\FV@Error{`\expandafter\@gobble\string#1' is not a short
        verb character}\FV@eha}%
      {\FV@UndefineShortVerb#1}
  }



\def\FV@UndefineShortVerb#1
  {
    \catcode`#1=\csname FV@CC@\string#1\endcsname
%% DG/SR modification begin - Jun. 12, 1998
    \expandafter\let\csname FV@CC@\string#1\endcsname\relax
%% DG/SR modification end
    
%  \begingroup
%    \lccode`\~=`#1%
%    \lowercase{\gdef\@tempg{\let~}}%
%  \endgroup
%  \expandafter\@tempg\csname FV@AC@\string#1\endcsname
%  \def\@tempa##1\do#1##2\@nil##3\@nil##4\@@nil{##3\def\dospecials{##1##2}\fi}%
%  \expandafter\@tempa\dospecials\@nil\iftrue\@nil\do#1\@nil\iffalse\@nil\@@nil
%  \def\@tempa##1\@makeother#1##2\@nil##3\@nil##4\@@nil{%
%    ##3\def\@sanitize{##1##2}\fi}%
%  \expandafter\@tempa\@sanitize\@nil\iftrue\@nil\do#1\@nil\iffalse\@nil\@@nil
%  
}



\def\SaveMVerb{\FV@Command{}{SaveMVerb}}
\begingroup
\catcode`\^^M=\active%
\gdef\FVC@SaveMVerb#1#2{%
  \@ifundefined{FV@SVM@#1}{}%
    {\FV@Error{Moving verbatim name `#1' already used}%
      {I will overwrite the old definition. Hit <return> to continue.}}%
  \global\@namedef{FV@SVM@#1}{}%
  \begingroup%
    \let\FV@SavedKeyValues\FV@KeyValues%
    \FV@UseKeyValues%
    \FV@CatCodes%
    \outer\def^^M{}%
    \global\let\@tempg\FancyVerbAfterSave%
    \catcode`#2=12%
    \def\@tempa{\def\FancyVerbGetVerb####1####2}%
    \expandafter\@tempa\string#2{%
      \if@filesw
        \FV@DefineWhiteSpace%
        \let\FV@Space\space%
        \let\FV@Tab\space%
        \FV@MakeUnActive%
        \let\protect\string
        \immediate\write\@auxout{%
          \noexpand\SaveGVerb[\FV@SavedKeyValues]{#1}\string#2##2\string#2}%
      \fi
      \endgroup%
      \@namedef{FV@SV@#1}{##2}%
      \@tempg}%
    \FancyVerbGetVerb\FV@EOL}%
\endgroup
\def\SaveGVerb{\FV@Command{}{SaveGVerb}}
\begingroup
\catcode`\^^M=\active%
\gdef\FVC@SaveGVerb#1#2{%
  \global\@namedef{FV@SVG@#1}{}%
  \begingroup%
    \FV@UseKeyValues%
    \FV@CatCodes%
    \outer\def^^M{}%
    \catcode`#2=12%
    \def\@tempa{\def\FancyVerbGetVerb####1####2}%
    \expandafter\@tempa\string#2{\endgroup\global\@namedef{FV@SVG@#1}{##2}}%
    \FancyVerbGetVerb\FV@EOL}%
\endgroup
\def\UseMVerb{\protect\pUseMVerb}
\def\pUseMVerb{\FV@Command{}{pUseMVerb}}
\def\FVC@pUseMVerb#1{%
  \expandafter\ifx \csname FV@SVM@#1\endcsname\relax
    \expandafter\ifx \csname FV@SVG@#1\endcsname\relax
      \@warning{Moving verbatim text not defined for name `#1'}\FV@eha
      {\bf ??}%
    \else
      \FV@UseVerb{\@nameuse{FV@SVG@#1}}%
    \fi
  \else
    \FV@UseVerb{\@nameuse{FV@SVM@#1}}%
  \fi}
%    \end{macrocode}


% \end{macro}
%
%    \begin{macrocode}  
\ExplSyntaxOn
\exp_after:wN \if_meaning:w \csname documentclass\endcsname\scan_stop:
  \def\lrbox#1
    {%
      \edef\@tempa
        {
          \egroup 
          \setbox#1\hbox{%
            \bgroup\aftergroup}%
              \def\noexpand\@currenvir{\@currenvir}
        }
      \@tempa
      \@endpefalse
      \bgroup
        \ignorespaces
    }
    
    \def\endlrbox{\unskip\egroup}
\fi
\ExplSyntaxOff
%    \end{macrocode}
%    \begin{macrocode}

\InputIfFileExists{fancyvrb.cfg}{}{}
%    \end{macrocode}
%%
\endinput
%</package>
%%
%% End of file `fancyvrb.sty'.
