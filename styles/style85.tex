\cxset{style85/.style={
 chapter opening=any,
 name=,
 numbering=arabic,
 number font-size=HUGE,
 number font-weight=bfseries,
 number font-family=sffamily,
 number before=,
 number after=\hfill,
 number dot=,
 number position=rightname,
 number color=black!80,
 %chapter 
 chapter color=black!80,
 chapter font-size= Huge,
 chapter font-weight=bfseries,
 chapter font-family=sffamily,
 chapter before=,
 chapter spaceout=none,
 chapter after=,
 chapter margin left=0cm,
 chapter margin top=1sp,
  %chapter title
 title font-family=rmfamily,
 title font-color=black!80,
 title font-weight=bold,
 title font-size=huge,
 title spaceout=none,
 chapter title align=left,
 title margin-left=0cm,
 title margin bottom=0pt,
 title margin top=0pt,
 chapter title align=right,
 chapter title text-align=raggedleft,
 chapter title width=0.7\textwidth,
 %display
 title display=in-line block,
 title before=,
 title after=,
 title beforeskip=,
 title afterskip=,
 author block=true,
 author names=Rebecca Moore,
 section font-size=large,
 section numbering=none,
 section font-family=sffamily,
 section font-weight=bfseries,
 section spaceout=soul,
 section indent=0pt,
 section align=left,
 section color=black,
 epigraph width=\dimexpr(\textwidth-1cm)\relax,
 epigraph align=left,
 header style=empty,
 author block=false}}
 \cxset{style85}
\makeatletter
\DeclareRobustCommand\circlednumber{%
    \tikz \node at (0,0) [shape=ellipse, minimum height=2cm, minimum width=0.8cm, 
                                  fill=black!20] {\@arabic\c@chapter};}
\cxset{chapter numbering custom=\circlednumber,}
\makeatother

\chapter[template 85]{Keeping Count\\ Writing Whole Numbers}


\lettrine{T}{}he problem of how to write numbers efficiently has been with humanity for as long as there have been
sheep to count count or things to trade. The simplest the most primitive way to do this was (and still is) by tallying---making a simple mark, for each thing counted. Thus, \emph{one, two, three, four, five,}\ldots were writen or carved as:\\
\hbox to \textwidth{{\fboxrule0pt\fboxsep0pt\bfseries
\hfill\hfill\hbox to 0.5\textwidth{\fbox{\textbar}\hfill\fbox{\textbar\textbar}\hfill\fbox{\textbar\textbar\textbar}
\hfill\fbox{\textbar\textbar\textbar\textbar} \hfill\cancel{\fbox{\textbar\textbar\textbar\textbar}}}\hfill\hfill
}}\\
and so on. People still use this for scorekeeping in simple games, class elections and the like, sometimes bunching the stroke marks by fives.


\begin{figure}[ht]
\centering
\includegraphics[width=0.9\textwidth]{count-01}
\caption*{Style 85.}
\end{figure}

The simplicity of the tally system is also its greatest weakness. It uses only one simple as very long strings of symbols are required to write even moderately small numbers. As civilization progressed various cultures improved on this
method by choosing a few more symbols and stringing them together until their values added up to the number they wanted. These symbols were pictographic; that is they were small picture drawings of common or not so common things. 
Math Through the Ages: A Gentle History for Teachers and Others
 By William P. Berlinghoff, Fernando Q. Gouvêa

\lipsum[1-3] 

\cxset{chapter opening=anywhere,
          chapter toc=none,}

\chapter{Nothing Becomes a Number\\ The Story of Zero\\ }


%https://books.google.ae/books?id=4ru6F85wGK4C&pg=PA76&dq=history+of+mathematical+notation&hl=en&sa=X&ei=XrgJVeizK6qx7QbK-4GADA&ved=0CD8Q6AEwBg#v=twopage&q&f=false

