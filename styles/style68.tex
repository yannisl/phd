\cxset{style68/.style={
 name=,
 numbering=arabic,
 number font-size=HHUGE,
 number font-weight=mdweight,
 number font-family=sffamily,
 number before=,
 number after=,
 number dot=,
 number position=rightname,
 number color=thelightgray,
 %chapter name
 chapter color=black!80,
 chapter font-size= Large,
 chapter font-weight=normalfont,
 chapter font-family=sffamily,
 chapter before=,
 chapter spaceout=none,
 chapter after=,
 chapter margin-left=0cm,
 chapter margin top=1sp,
  %chapter title
 title font-family=sffamily,
 title font-color=black!80,
 title font-weight=bold,
 title font-size=huge,
 chapter title align=none,
 title margin-left=4cm,
 title margin bottom=2cm,
 title margin top=1sp,
 chapter title align=raggedright,
 chapter title width=11cm,
 title before=,
 title after=,
 title beforeskip=,
 title afterskip=,
 author block=false,
 section font-size=\Large,
 section font-weight=bfseries,
 section indent=0pt,
 epigraph width=\dimexpr(\textwidth-1cm),
 epigraph align=left,
 section font-weight=\normalfont,
 header style=empty}}

\cxset{style68}

\chapter{MEDIEVAL EVOLUTION: HOW THE ASHKENAZI JEWS GOT THEIR SMARTS}

Evolution is happening before our eyes, but we fail to see it. Authors make a compelling argument that the last
10,000 years of civilization has shaped humans in unprecendent ways.

\begin{figure}[ht]
\centering
\includegraphics[width=0.9\textwidth]{ashkenazi}
\caption{Style 68 spread.}
\end{figure}

A lot of the work can be called ``genetic history’’. The authors share the usual facts but use a very different explanatory framework. The historical factors that are shaping natural selection have been mercenries, invading armies, massacres and persecution.  Gregory Cochran and Henry Harpending label themselves as genetic historians. Where an anthropologist looks at how farmers in a certain period and region lived, they are interested to find out how natural selection allowed agriculture to begin with, and how the new pressures of an agricultural life style allowed changes in the populations genetic makeup to take root and spread. There are two large populations that
demonstrate how natural selection works subtly over generations. The authors make compelling arguments regarding the Ashkenazi’s intelligence. Unfortunately they have not included the second population which are the Afrikaaners of South Africa that went through similar but different pathways. 

\section{Template 68}

The prominent feature of the template is the chapter openings and its choice of typography. I am not doing it justice here as I have stayed with the normal default fonts. Contrast is provided by the light gray number and the alignment to the right. I have experimented with the size of the title block a couple of times until I had it right. Alternatively this can be measured and the title width set at an appropriate length.


\cxset{chapter opening=anywhere}

\example
\begin{verbatim}
\cxset{style68,
   chapter opening=anywhere,
  }
\end{verbatim}

\solution
    
\chapter{THE NEANDERTHAL WITHIN}
\lorem


\begin{figure}[ht]
\centering
\includegraphics[width=\textwidth]{ashkenazi-02}
A photo of Jewish women prisoners at the Auschwitz concentration camp in 1944. 
\medskip

\includegraphics[width=\textwidth]{ashkenazi-03}
Ebensee Concentration Camp survivors, 1945.
\end{figure}
% reset settings for other chapters
\cxset{chapter opening=any,}

%http://bloximages.newyork1.vip.townnews.com/stltoday.com/content/tncms/assets/v3/editorial/a/7a/a7a874cc-ce95-11e0-a4d7-0019bb30f31a/4e55690aaef57.preview-1024.jpg?resize=620%2C438  image