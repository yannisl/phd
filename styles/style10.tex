
\cxset{
 name=CHAPTER,
 numbering= Roman,
 number font-size=\huge,
 number font-family=\sffamily,
 number font-weight=\bfseries,
 number before=,
 number dot=,
 number after=\hspace{1em},
 number position=rightname,
 chapter font-family=sffamily,
 chapter font-weight=bfseries,
 chapter font-size=huge,
 chapter before=\vspace*{0.4\textheight}\hfill,
 chapter after=\hfill\hfill\vskip0pt\thinrule\par,
 chapter color=black!90,
 number color=black!90,
 title beforeskip={\vspace*{30pt}},
 title afterskip={\vspace*{30pt}\par},
 title before=\hfill,
 title after=\hfill\hfill,
 title font-family=\sffamily,
 title font-color=black!90,
 title font-weight=\bfseries,
 title font-size=\huge,
 section font-size=\LARGE,
 section font-weight=\normalfont,
 section font-family=\sffamily,
 section align=centering,
 section numbering=none,
 section indent=-1em,
 section beforeskip=20pt,
 section afterskip=10pt,
 section spaceout=soul,
 section font-shape=,
 pagestyle = headings,
}

 %set the sectioning commands


%\renewsection

\chapter{INTRODUCTION TO STYLE TEN}

\section{Basic Description:}
This chapter style has the unique characteristic that the chapter number is spelled out, rather than being in arabic numerals. The setting for this is the option \lstinline{numeric=WORDS}. Use either a capital for uppercase or \lstinline{numeric=words} for lowercase number labels.

\medskip
\begin{figure}[ht]
\centering
\includegraphics[width=0.6\textwidth]{./chapters/chapter10}
\end{figure}

\lipsum[1]
