\restoregeometry
\parskip15pt plus0.5pt minus0.5pt

\newfontfamily\calibri{calibril.ttf}
%\newlength\templength
\newlength\tempheight

\newcommand{\oxfordblockchapter}[2][]{%
  \makeatletter
\bgroup  
\parindent0pt\fboxsep1.5pt\fboxrule0pt
\parskip0pt\lineskip0pt
%\newsavebox\chapterblock
\savebox\chapterblock{%
       \colorbox{black}{%
       \fbox{\color{white}% 
          {\sffamily\bfseries\large
          \space\chaptername\space\thechapter\space}%
       }%
   }%
}
\setlength\templength{\the\wd\chapterblock}
\setlength\tempdepth{\the\dp\chapterblock}
\setlength\tempheight{\the\ht\chapterblock}
% draw the rule

\vrule width\dimexpr(\columnwidth) height0.2pt depth0pt\par
\kern-3.2pt
\usebox\chapterblock%
\par
\vspace*{36pt}
\leavevmode
\bgroup
\setchaptertitlefont
#2\par
\egroup
\vspace*{48pt}
\egroup


\makeatother
}


\cxset{style63/.style={
 name=CHAPTER,
 numbering=arabic,
 number font-size=Large,
 number before=\kern0.5em\relax,
 number after=\hfill\hfill\par\vspace*{0ex},
 number dot=.,
 number position=rightname,
 number font-weight=\normalfont,
 number color=black,
 chapter color=black,
 chapter font-size=Large,
 chapter font-weight=normalfont,
 chapter margin left=0pt,
 title font-family=calibri,
 title font-color=black,
 title font-weight=normalfont,
 title font-size=HUGE,
 title font-shape=upshape,
 chapter title align=left,
 chapter title width=0.8\textwidth,
 chapter afterindent=true,
 title margin top=0pt,
 title margin bottom=30pt,
 section numbering=arabic,
 section indent=-25pt,
 section numbering prefix=\thechapter.,
 section numbering suffix=,
 header style=plain,
 custom=oxfordblockchapter,
 }}

\cxset{style63}

\chapter{The parts of a book}

Like the contents of books, style elements are also recycled in slightly different variations. This template, which is a variation to style 62. An example of such a style can be seen in the \emph{The Oxford Guide to Style} by R.M.Ritter and published by Oxford University Press. The Guide is an indespensable tool for authors and students alike and builds on the earlier work of Horace Hart’s \emph{Rules for Compositors and Readers at the University Press}. Press, Oxford. the course of thirty-nine editions, Hart's Rules has grown to be the standard
work in its field, explaining subject by subject each major aspect of
punctuation, capitalization, italics, hyphenation, abbreviations, foreign
languages, and other publishing matters big and small. The edition I own is in a small format 5.58x8.79 in and many of the ideas and writings in this book are based on the good advice of Hart, which I must admit needs two life times to be absorbed and implemented fully. I am not very fond of the styling of the publication. Oxford could have done better, but I guess with a wide audience in mind the style was appropriate. As the book parallels many of the topics outlined here, we will detail this template more fully to discuss some of the details that we have left out in previous templates. The template will be developed as a \emph{custom} template, despite the fact that it can be also developed by setting appropriate keys to the standard templates.


\section{Template highlights}

The guide was designed by Jane Stevenson and typeset in Swift and Arial, although the soft copy has Helvetica and Times Roman. Stevenson also designed a number of other Oxford University Press books and her designs have a recurring theme of dotted rules that frame the headings one way or another. The sections are numbered to four levels as the text is cross-referenced widely.

Since the template is to be typeset in a wider format a full reproduction is impossible so I have done minor deviations to the style. A painful feature is that paragraphs are not indented, but are marked with an empty line.

\begin{figure}[ht]
\centering
\includegraphics[width=0.8\textwidth]{oxford-chapter}%
\caption{Style 63 Example pages.}
\end{figure}

\section{Preliminary matter}

Preliminary matter is any material that precedes the text. Normally it is
the part of a work providing basic information about the book for
bibliographic and trade purposes, and preparing readers for what
follows. It is usually paginated in lower-case roman numerals rather
than arabic numbers; however, the introduction can begin the arabic
pagination if it acts as the first chapter, rather than falling outside the
body of the main text—as in the case where a book is divided into parts,
for example.

Publishers try to keep prelims to a minimum: paperbacks and children's
books generally have fewer preliminary pages than hardbacks and
monographs; journals and other periodicals have fewer still. No rigid
rules govern the arrangement of preliminary matter, although publishers
routinely develop a house style for its sequence based on the sorts of
publication they produce and the combination of preliminary matter
common to them. Books in a series should have a consistent order, and
those on a single list or subject tend to.

Generally, the more important of the prelim sections start on a new recto
(right-hand page), sometimes ending with a blank verso (left-hand page)
if the text is one page in length or finishes on a recto. Others of lesser
importance start only on a new page, and two or more sections (especially
lists) may be combined to run together on a single page if space
demands and logic allows. The decision is based on what preliminary
matter is to be included in a given work, how long each section is
(often---but not always---equated with how important it is), and the
number of pages available.

In addition to space, a consideration is bleed-through from the other
side of a page: a one-line dedication or epigraph falling on a recto, for
example, often requires a blank verso to avoid the image of the verso's
type showing through on a nearly empty preceding page. (Bleed-through
is for the most part unnoticeable on pages with similar amounts of text.)
Where space permits it is safest to put any dedication or epigraph on a
new recto with a blank verso. But a book much pushed for space---to
accommodate an even working, for example---may actually demand
setting the dedication on the half-title verso.

Assuming sections of standard length and no page restrictions, the
following order of preliminary matter may be recommended:

series title (new recto)
publisher's announcements (verso)
half-title
frontispiece
title page
title page verso
dedication
foreword
preface
acknowledgements
contents
(new recto)
(new verso)
(new recto)
(verso)
(new page)
(new recto)
(new recto)
(new page)
(new recto)
table of cases (new page)
table of statutes (new page)
list of illustrations, figures,
plates, maps, etc. (new page)
list of tables (new page)
list of abbreviations (new page)
list of symbols (new page)
list of (or notes on)
contributors (new page)
epigraph (new page)
introduction (new recto)

\section{Text}

An author's approach to a subject and the formation of the narrative
often moulds the structure into which the text unfolds. Ideally,
this should develop into a form in which each division—volume, part,
chapter, section, and subsection—is of a size more or less the same as
equivalent divisions. While it is rare to find a work that falls effortlessly
into perfectly uniform divisions, and pointless forcing it to do so, a
severely unbalanced structure often indicates a lopsided strategy or
method. Authors and editors should therefore choose a hierarchy of
divisions that most closely mirrors the natural composition of the
work, and strive to rectify any aspects that seem unwieldy or sparse.

As part of marking up the text, the editor will normally label the
hierarchy of headings A, B, C, etc. for those in text, and Χ, Υ, Ζ for
those in the preliminary matter or appendices. Each will later be styled
(often by a design department) to provide an appropriate visual arrangement
of section headings and subheadings.

For the presentation of footnotes \seeref{sections} and \seeref{sections}; for endnotes
\seeref{sections}.




\subsection{Sections and subsections}
\label{sections}

Most books divide chapters into sections and subsections by the use of subheadings (\emph{subheads}). Too many levels of subhead tend to confuse the reader and should be avoided. Most guides recommend that the subheads should be short and clearly indicate the contents of the section.

But should the subheads be numbered or not? The answer depends if the numbering of the subheads will be useful to the reader, and how many cross-references there may be to that level of division. As you will observe in this book, variations in placements (e.g. centered, full-left, indented, marginal, run-in) and in type (e.g. bold, italic, small capital, choice of typeface) is left to teh book designer. The general rule is that larger subdivisions are styled to look more important than smaller divisions.





As we need to have an accurate left alignment for the headings we need to measure the dimensions of the 
typeset chapter block. We define a \cmd{\newsavebox} and the necessary temporary length registers.

\begin{scriptexample}{}{}
\begin{verbatim}
\newlength\tempdepth
\newsavebox\chapterblock

\savebox\chapterblock{%
       \colorbox{black}{%
       \fbox{\color{white}% 
          \sffamily\bfseries\large
          \hspace*{1.5cm}\chaptername\space\thechapter\space%
       }%
   }%
}
\end{verbatim}
\end{scriptexample}

If you notice we used the \cmd{\fbox}  to be able to provide an easy way to set padding in the box. The
\cmd{\colorbox} is used to color the block. 

\section{Putting everything in a macro}

The template mechanism provided by the |phd| package accepts a macro with any name, but with two parameters.



\section{Summary}

We have formatted a block of text to represent the chapter and then set it to point to a template. We can call the template simply by setting:

\begin{verbatim}
\cxset{custom=blockchapter,
          chapter opening=anywhere}
\chapter{First Chapter Special}
\end{verbatim}

The chapter heading is called using the normal \latexe command \cmd{\chapter}. There is no need to provide a starred version of the command, as firstly the template would not look nice with a number and second, if we do not want it to be entered into the ToC, we can simply use the key setting:

\begin{verbatim}
chapter toc=false,
\end{verbatim}

Trying it out we get,

\cxset{custom=oxfordblockchapter,
          chapter opening=any,
          title font-weight=arabicfont}


\chapter[Italic, roman, and other type treatments]{Italic, roman,\\ and other type\\ treatments}

\makeatletter
\@specialfalse
\makeatother

The design outlined in this template is widely used by many book designers. In the next chapter we will vary it to 
give an additional example.







