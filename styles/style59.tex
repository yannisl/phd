
\cxset{style59/.style={
 name=SERMON,
 numbering=Roman,
 number font-size=Large,
 number before=\kern0.5em,
 number after=\hfill\hfill\par\vspace*{1ex},
 number dot=.,
 number position=rightname,
 number font-weight=\normalfont,
 number color=black,
 chapter color=black,
 chapter font-size=Large,
 chapter before=\hfill,
 chapter after=,
 chapter margin left=0pt,
 title font-family=rmfamily,
 title font-color=black,
 title font-weight=bold,
 title font-size=large,
 title font-shape=scshape,
 chapter title align=centering,
 chapter title width=0.8\textwidth,
 title before=\centerline{\rule{3cm}{0.4pt}}\vspace*{10pt},,
 title after=\par\vspace*{0pt}\centerline{\rule{3cm}{0.4pt}}\par,
 title margin top=0pt,
 title margin bottom=10pt,
 header style=empty}}

\cxset{style59}

\chapter{WHEN WE THINK NOT, THE LORD COMETH.}

I am not a religious person, but have read a good deal of religious texts in my younger days, the Bible, the Quaran, The Baghavad Gita lots of books on Buddhism. Typography and religion are inextricably interwined and the birth of typography in the Western World  might not have been what it is if it was not for the Church.

I have picked up \emph{The Parable of the Ten Virgins, Illustrated in Six Sermons}, by James Wood from archive.org\footnote{\url{https://archive.org/stream/parabletenvirgi00woodgoog\#page/n118/mode/1up}} more or less randomly for style 59 as a good example to illustrate the use of labels as chapter names. This is easily accomplished with the \pkgname{phd} package by setting:

\begin{verbatim}
\cxset{chapter name=SERMON}
\end{verbatim}

\begin{figure}[ht]
\centering
\includegraphics[width=0.8\textwidth]{sermon-01}
\caption{Style 53 Page spread.}
\end{figure}

The book was orginally published in 1722 and the edition here was printed in 1830 by J.S \& C. Adams \& Co. in Amherst Massachussets, so it is representative of early American printed works.









