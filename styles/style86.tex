%\makeatletter\@runinheadtrue\makeatother

\cxset{style86/.style={
chapter opening=any,
 name=,
 numbering=none,
 number font-size=large,
 number font-weight=mdweight,
 number font-family= rmfamily,
 number before=,
 number after=,
 number dot=,
 number position=rightname,
 number color=black!80,
 %chapter name
 chapter color=black!80,
 chapter font-size= Large,
 chapter font-weight=bfseries,
 chapter font-family=rmfamily,
 chapter before=,
 chapter spaceout=none,
 chapter after=,
 chapter margin left=0cm,
 chapter margin top=0sp,
  %chapter title
 title font-family=rmfamily,
 title font-color=black!80,
 title font-weight=bfseries,
 title font-size=LARGE,
 chapter title align=none,
 title margin-left=0cm,
 title margin bottom=2cm,
 title margin top=1sp,
 chapter title align=centering,
 chapter title text-align=center,
 chapter title width=\textwidth,
 title before=,
 title after=,
 title beforeskip=,
 title afterskip=,
 author block=false,
 section font-size=\Large,
 section font-weight=bfseries,
 section indent=0pt,
 epigraph width=\dimexpr(\textwidth-2cm)\relax,
 epigraph align=center,
 epigraph text align=center,
 section font-weight=mdseries,
 section align=center,
 epigraph rule width=0pt,
 header style=plain}}

\cxset{style86}

\chapter[Template 86]{UMKHULUMNQANDE}
\thispagestyle{plain}
\pagestyle{headings}

Little remains of the old Swazi religion since the impact of the missionary effort has obliterated most of its traces, but we do know some myths of the gods. The Swazi ancient God and All-father of the Swazi was Umkhulumnqande, one of those formidable words that must have left a hush of awe when pronounced by an elder beore the full assembly of the community. According to the legends he lives in the sky above us, from where
he reigns supreme.

\begin{figure}[ht]
\centering
\includegraphics[width=0.9\textwidth]{swazi}
\caption{Style 86 spread.}
\end{figure}

\def\umkulu{Umkhulumnqande\xspace}  \umkulu created the earth and heaven pretty much like the rest of the gods. His footprints are still pointed out in the rocks where he stepped down before they had solidifed. When he had completed the creation he returned to Heaven and had not been seen here since. If a miracle happen such as a cow giving birth to twins, for instance, the people will say: It must be the work of \umkulu.

\begin{figure}[ht]
\vbox to \textheight{\centering
\includegraphics[width=1.0\textwidth]{swazi-01}
Rural primary school in Swaziland. These are typical of structures in South Africa. They are build cheaply 
from bricks and tin roofs.

\vfill

\includegraphics[width=1.0\textwidth]{incwala}%
\caption{A rural primary school in Swaziland.}%
}
\end{figure}


\section{Template 86}

I visited Swaziland regularly many times in the 1990s on business. At the time Southern Africa was very much the melting pot of politics. The first thing that stikes you when you drive in Manzini is the density of the houses on the hills. The town could be just another South Africa small town. Whites and blacks used to water at the Pepper and Salt club. Those were pre-AIDS times when the chance of a bonus of white business men was high. Me I was visiting the Blue Ribbon bakery. The main street had its normal streak of shops the Barclays Bank, Supermarket. 

Despite the missionaries and despite colonialism Africa is still the free and untamed spirit of centuries ago, transformed with modernity but the cultural DNA is still flowing around. May it stay this way for  ever. To us foreigners to the land it seems uncivilized but I would like to keep it this way. 

There is also a goddess in the Swazi pantheon; her name is Numkubulwane. She too, is feared, for she may send sickness when she has been neglected by the priests. When sickness has erupted people will carry pots of beer and porridge to the mountain where she lives and leave them in an \emph{umnikelo}, an offering. When there was a blight in the corn, the girls would take over the herding of the cattle in the fields, going about stark naked, while the men and boys would stay indoors. After a few days the girls would take some ears of the infested corn to the river and throw the seeds down over a waterfall. By this ceremony the goddess would be propitiated and she would cure the crops. 

\cxset{chapter opening=anywhere}

\example
\begin{verbatim}
\cxset{style86,
   chapter opening=anywhere,
  }
\end{verbatim}


% reset settings for other chapters
\cxset{chapter opening=anywhere,
          chapter toc=none,
          section font-weight=bold,
          section afterskip=1pt plus0.5em minus0.5em}
 
%https://books.google.ae/books?id=aPAUAAAAIAAJ&printsec=frontcover&dq=swaziland&hl=en&sa=X&ei=ZfsKVZ_KEMX4UJ2Jg9AD&ved=0CDsQ6AEwBg#v=onepage&q=swaziland&f=false