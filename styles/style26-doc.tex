\newfontfamily{\neutra}{NeutraText-Bold.otf}
\newfontfamily{\vijaya}{Vijaya-Bold.ttf}
%\newfontfamily{\vijaya}{Caladea-Bold.ttf}
\parindent3em
%%%%%%%%%%%%%%%%%%%%%%%%%%%%%%%%%%%%%%%%%%%
%%%%%%  STYLE 26
%%%%%%%%%%%%%%%%%%%%%%%%%%%%%%%%%%%%%%%%%%%

\cxset{
 name={},
 numbering=arabic,
 number font-size=\huge,
 number font-family=\sffamily,
 number font-weight=\bfseries,
 number before={},
 number position=leftname,
 chapter font-family=\sffamily,
 chapter font-weight=\normalfont,
 chapter font-size=\small,
 number after={},
 chapter before={},
 chapter after={\par\vskip12pt},
 chapter color={black!90},
 number color=\color{black!90},
 title beforeskip={},
 title afterskip={\vspace{30pt}},
 title before=,
 title after={\par},
 title font-family=\sffamily,
 title font-color=\color{black!80},
 title font-weight=\bfseries,
 title font-size=\LARGE}
\chapter{Introduction to style twenty five Dr. Yiannis Lazarides and Athena Lazarides}

The interesting part of this style is that it uses roman numerals to display the counter that is in a different font than that used for the chapter name.
\medskip
\begin{figure}[ht]
\centering
\fbox{\includegraphics[width=0.6\textwidth]{./chapters/chapter26}}
\end{figure}
\lipsum[2-3]

 
\chapter{The Rapid Beauty of Enthusiasm, Style twenty six}
\thispagestyle{plain}


{\huge F}or people wake up every morning and start their day  interesting part of this style is that it uses roman numerals to display the counter that is in a different font than that used for the chapter name.\par
\medskip
\begin{figure}[ht]
\centering
\fbox{\includegraphics[width=0.8\textwidth]{enthusiasm}}
\end{figure}

\lorem 

\section{Getting the Typography Right}

The only way to instill style into a book, which is primarily textual with no images and no mathematics is through the use of suitable fonts at points where a font change is appropriate. These turning points are the headings the emphasizing of words and passages.  The book’s sectional headings it appears they use Vijaya, which according to the Microsoft website is a font primarily meant for use in displaying Tamil text in documents! There are many free clones available on the web for download, if you are not a Windows user. Another interesting aspect of this book and to an extend the series is the use of indentation after the headings. This increases the white space on the page and I guess reduces the amount of words needed to fill the book. Great idea if you are writing a school essay and you need a couple of pages extra. 

I must admit, I don’t have the stomach any more for these self-help gurus that are no longer authors or writers  but merely models so that they look good on television. They talk in sentences full of cockeyed confidence and make utternaces such as ``finding the love behind the mask’’, ``feeding the fire of the spirit in you’’ and so on and so on.
People fall for fast salespersons and do buy snake oil. The worst of the lot would concure a mixture of psudo-science and come up with a couple of buzz-words that they turn into a ``theory’’ and spent the rest of their lives promoting it. I guess it puts food on their table and that is alright with me, but let me have my up and down days and I don’t want to live my life as a grown up happy boy scout patting myself on the back daily. Please also don’t give me 100 Ways to motivate myself and a 100 ways to motivate others. Do I also need 50 ways to create great relationships and is there really something call the joy of selling? I would rather stick to plain logic each and every time, as it works straight out of my head and exercises the brain at the same time.

\subsection{Indentation after the Chapter Head}

There is no right or wrong answer here, but this book requires the first line of the paragraph following a Chapter head to be indented. This is common for many European countries. It also looks better if the book has a lot of dialogues and the first line is a dialogue.

\begin{verbatim}
\cxset{chapter afterindent=true,
          section afterindent=false}
\end{verbatim}

I have set the section identation at false to illustrate that this is possible. Don’t do this in your final draft as the most fundamental rule of typography is consistency. If you indent after a Chapter you should indent right through to all the headings. See also page \pageref{code:chapterafterindent} for details to the changes of the \latexe code. 


