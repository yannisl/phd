\cxset{style90/.style={
 name=CHAPTER,
	 numbering=arabic,
	 number font-size=HHHUGE,
	 number font-weight=bfseries,
	 number font-family=rmfamily,
	 number before=\kern0.5em,
	number after=\vskip5pt
	                           \hrule width\textwidth height0.4pt\relax,
 number dot=,
 number position=rightname,
 number color=black!30,
 number border-width=0pt,
  %chapter 
 chapter color=black!80,
 chapter font-size= LARGE,
 chapter font-weight=bfseries,
 chapter font-family=sffamily,
 chapter before=,
 chapter spaceout=soul,
 chapter after=,
 chapter margin top=50pt,
 chapter margin left=0pt,
 chapter padding-left=0pt,
 chapter padding-right=0pt,
 chapter padding-bottom=25pt,
  chapter border-top-width=0pt,
 chapter border-left-width=0pt,
 chapter border-right-width=0pt,
 chapter border-bottom-width=0pt,
 chapter display=inline,
 number display=inline,
  %chapter title
 title font-family=rmfamily,
 title font-color=black!80,
 title font-weight=bold,
 title font-size=huge,
 title spaceout=none,
 chapter title align=none,
 title margin-left=0pt,
 title margin bottom=30pt,
 title margin top=10pt,
 chapter title align=left,
 chapter title text-align=raggedright,
 chapter title width=0.9\textwidth,
 title before=,
 title after=,
 author block=false,
 section font-size=large,
 section numbering=none,
 section font-family=sffamily,
 section font-weight=bfseries,
 section spaceout=soul,
 section indent=0pt,%check this one out
 section align=left,
 section color=black,
 epigraph width=0.8\textwidth,
 epigraph align=left,
 header style=empty}}



\cxset{style90}

\chapter[template 90]{The Art of English Writing}

The New Oxford Oxford Guide to Writing by Thomas Kane (1988), is  a great guide to get you started in English writing---especially if English is not your mother language. It systematically describes how to analyze topics and how to write well-written sentences. Words shape us; before we are business people or lawyers or Scientists, we are human beings. Our growth as human beings depends on our capacity to understand and to use language. Writing is a way of growing. 

\begin{figure}[ht]
\centering
\includegraphics[width=0.9\textwidth]{guide-writing}
\caption{Style 90 spread.}
\end{figure}

It is a humbling thought that the struggle for existence never gets any easier: however well adapted an animal may become, it still has the same chance of extinction as a newly formed species. Van Valen was reminded of the Red Queen in Alice in Wonderland, who ran fast with Alice only to stand still. 

The application of this theory to the problem of the maintenance of sex is captured by the phrase “genetics arm race”. A typical animal must constantly run the genetic gauntlet   of being able to chase its prey, run away from predators and resist infection from parasites. Parasite infection in particular means that the parasite and its host are locked in an “evolutionary embrace” (Ridley, 1993). Each reproduces sexually in the desperate hope that some combination will gain a tactical advantage in an attack or defence. 

Further support for the parasite exclusion theory comes from the fact that genes that code for the immune response---the majorhistocompatibilty complex (MHC)---are incredibly variable. This is consistent with the idea that variability is needed to keep an advantage over parasites.

Lassa fever virus, Hantavirus, and Ebola virus---all equally lethal infectious agents but members of different viral families---share the ability to cause hemorrhagic fever. Once the person is infected with the viruses, the victim soons suffers profuse breaks in small bloods vessels, causing blood to ooze from the skin, mouth, and rectum. Internally, blood flows into the leural cavity where the lungs are lkocated, into the pericardial cavity surrounding the heart, into
the abdomen, and into organs like the liver, kidney, heart spleen and death. Once it strikes hemorrhagic fever is relentless and devastating.

Another alarming factor is that the number of individuals that are susceptible to these viruses has swelled markedly due to the ever growing population that are on immunosuppressive drugs or are infected by such pathogens susch as HIV, measles virus, malaria and tuburculosis, all of which suppress the immune system. Since 1969, thirty-nine new pathogens have emerged including SARS, HIV, and Ebola. 

\cxset{chapter opening=anywhere}

\chapter{Choosing a Subject}

Most authors will choose a subject. Businessmen and students have their subjects chosen by someone else and worse dictated by a committee. For the latter good writing does not apply. Just give them what they want to hear.
Kurt Vonnegut in \emph{Conversations} hit the nail on the head when he said \emph{Like Shaking Hands with God: A Conversation about Writing} by Kurt Vonnegut, Lee Stringer:

\cxset{quotation font-size=\large}

\begin{quotation}
When I teach---and I’ve taught at the Iowa’s Writers’ Workshop for a couple of years, at City College, Harvard---I am not looking for people who want to be writers. I’m looking for people who are passionate, who care terribly about something. There are people with a hell of a lot on their minds, Lee being a case in point, and if you have a hell of a lot on your mind, the language will arrive, the right words will arrive, the paragraphing will be right. You have the case of Joseph Conrad, for whom English was a third language, and he was passionate in English. The words arrived and formed master-pieces.
\end{quotation}

\chapter{Creative Writing}

As Kurt wrote: Do not use semicolons and don’t capitalize the first letter following a semicolon. “They are transvestite hermaphrodites representing absolutely nothing. All they do is show you’ve been to college.”
Other wise sayings from the same book.

If you really want to hurt your parents, and you do not have the nerve to be gay, the least you can do is go into the arts. I’m not kidding. The arts are not a way to make a living. 

If you writing a thesis in English for a degree, you will have to translate the English to jargon: The following paragraph describes the purpose of writing, which of course is writing:\footnote{\protect
\url{https://books.google.ae/books?id=jsWL_XJt-dMC&printsec=frontcover&dq=university+press+writing&hl=en}}

\begin{quotation}
If visual properties combined with aural attributes of human perception constitute the possibility of writng, then those selfsame visual properties also give writing certain advantages over speech. A simple explanation is that its visual nature makes writing preservative, against the transient nature of speech. 
\end{quotation}

Semelactive my foot. Exercise for the reader reduce the above to a five word sentence.

\chapter{Drafts and Revisions}

The spiral technique for revisions works well for most people. Using \latex makes this somewhat easier as well.
\lipsum[2]


\chapter{When to Finish}

Some writers---especially Academics---might keep on writing for years, until they are happy that the book is completed. This is a trap to be avoided. Think of your book as a conversation with the reader and not as a speech.
No one listened to Buthelezi’s speech and no one probably ever read it.

\begin{quotation}
Literature is the only art that requires our audience to be performers. You have to be able to read and you have to be able to read awfully well. You have to read so well that you get irony! I’ll say one thing meaning another, and you’ll get it.
\end{quotation}






%Reset it problematic in other places
\cxset{title margin-left=0pt}
%https://books.google.ae/books?id=-AL43YC_cZsC&printsec=frontcover&dq=university+press+writing&hl=en&sa=X&ei=iI4NVemELZfYaoGwgMAJ&ved=0CFIQ6AEwCA#v=onepage&q&f=true