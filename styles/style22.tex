

\restoregeometry

\newgeometry{left=6.5cm,right=2.5cm, marginparsep=15pt, marginparwidth=5cm,top=2cm,
reversemarginpar}

\setdefaults
\cxset{style22/.style={
 name={},
 numbering=none,
 number font-size=\Large,
 number font-family=\rmfamily,
 number font-weight=\bfseries,
 number before=,
 number after=,
 number position=rightname,
 chapter font-family=\sffamily,
 chapter font-weight=\normalfont,
 chapter font-size=\large,
 chapter before=\vspace*{-10pt},
 chapter after={},
 chapter color=black!90,
 number color= black!90,
 title beforeskip= \raggedleft, % pushes title to right
 title afterskip={\vspace{70pt}},
 title before=\hspace*{-\marginparwidth},
 title after={},
 title font-family=\sffamily,
 title font-color=black,
 title font-weight=\bfseries,
 title font-size=\Huge,
 chapter title width=1.2\textwidth,
 chapter title align=raggedleft,
 section numbering=none,
 section font-family=\sffamily,
 section font-weight=\bfseries,
 section font-size=huge,
 section color=black,
 section indent= 0pt,
 subsection indent = 0pt,
 header style=verticalrule}}
 


\cxset{style22}
\renewsection\renewsubsection

        \makeatletter
% For style 22 need 

\def\ps@verticalrule{%
    \leftskip\z@\let\@mkboth\@gobbletwo\vfuzz=5\p@
    \def\@oddhead{}%
    \def\@evenhead{}%
     \def\@evenfoot{}%
      \def\@oddfoot{}%
  \def\@oddhead{\verbatimsize
    \vbox to 0pt{\vspace{\the\headsep}%
      \noindent\hbox to \dimexpr\the\textwidth+.75cm\relax{%
            \hfill\mbox{%
                \color{thegray}\rule{1pt}{\dimexpr\the\textheight\relax}\hspace*{1pt}\color{black}\thepage%
                }%
      \makebox[\z@][l]{\@c@pyrightline}%
%     \noindent\hspace*{-9pc}\rule{37pc}{0.25pt}%
    }}%
  }%
 
  \def\sectionmark##1{}%
  \def\subsectionmark##1{}%
 }
\makeatother

 \pagestyle{verticalrule} 
  

\chapter{INTRODUCTION TO THE CHARIOT TEMPLATE}\index{style22}\index{lettrine}\index{drop cap}
\makeatletter

\cxset{%
         section afterskip=5pt,
         section color=black!90,
         subsection color=black!90,
         chapter rule  color=thegray,
         subsection font-size=large,
         subsection afterskip=1pt,
         }  

         
 \def\@seccntformat#1{\protect\makebox[0pt][l]{\csname the#1\endcsname}}
\def\thesection{%
   \raisebox{-6.5pt}{\tikzrule}%
}%
\def\thechapter{\@gobble}   
\makeatother
\section{INTRODUCTION}
 \pagestyle{verticalrule} 
 
\renewcommand{\DefaultLoversize}{0.3}
\renewcommand{\LettrineTextFont}{\fontfamily{Minion Pro}\normalfont\scshape}
\renewcommand{\LettrineFontHook}{%
\fontseries{bx}\fontshape{up}\color{gray}}

\cxset{lettrine lines/.code=\global\setcounter{DefaultLines}{#1}}

\cxset{lettrine lines=4}

\lettrine[lraise=0.0, nindent=0em, slope=-.5em]{T}{is} template follows a style found in many modern books,
that are addressed to both the scholar as well as to the general reader. It has a wide margin hence a narrower type area and bears a large initial drop cap. The Chapter head is in horizontal box and extends into the margin. All sections are underlined and it needs some ingenuity to underline them using the normal LaTeX commands. 


\medskip

\begin{figure}[bt]
\makebox[-10pt][r]{\parbox[b]{4.5cm}{%
   \captionsetup{labelfont=bf, textfont=bf, justification=raggedright, labelsep=period, width={\marginparwidth-\marginparsep},
   belowskip=0pt,aboveskip=0pt}
   \captionof{figure}{This is the caption for a floating image. These images are allowed to float at the bottom or at the top of the page only. Normal keys for the caption setup can be used.}}%
   }\hspace{\the\marginparsep}
\fbox{\includegraphics[width=\textwidth]{chariot-chapter}}\par
\end{figure}


\example The sections and subsections are rendered in different font size and the section has a ruler underneath, extending the full length of the section title.




\begin{figure}[ht]
\centering
\includegraphics[width=\textwidth]{./images/bronze-sections.jpg}
\caption{No number images.}
\end{figure}



\begin{figure}[ht]
\centering
\includegraphics[width=\textwidth]{./images/bronze-figures.jpg}
\caption{No number images.}
\end{figure}

\makeatletter
%\cxset{section numbering custom/.code =
%     \gdef\@seccntformat#1{\protect\makebox[0pt][r]{\csname
%                          the#1\endcsname\quad}}
%}
   \makeatletter
 
\cxset{section numbering=roman}
\cxset{section number after=,
          section align=flushleft,
          section indent=1pt,
          section numbering=arabic}

       
\lorem 


\everypar{}    



\example
Develop the section style for the mycenaean template. Study the sample and
provide the specification.
     
 Leslie Lamport when he developed LaTeX he provided three commands to define lower section commands automatically. It is to an extend problematic to hook into them. However, one of the
aims of the PHD package was to follow up as close as possible, the standard methods provided
by LaTeX and to provide the custom template keys for everything else. So if the modification
to the standard class involves only textual elements then that is what we will do.\footnote{\protect\lorem} 
 \makeatletter

\cxset{%
         section afterskip=5pt,
         section color=black!90,
         subsection color=black!90,
         chapter rule  color=thegray,
         subsection font-size=large,
         subsection afterskip=1pt,
         subsection align=flushleft,
         subsection font-shape=itfamily,
         }  

\let\oldseccntformat\@seccntformat       
 \def\@seccntformat#1{\protect\makebox[0pt][l]{\csname the#1\endcsname}}
\def\thesection{%
   \raisebox{-6.5pt}{\tikzrule}%
}%
\let\oldchapter\thechapter
\def\thechapter{\@gobble}   
\makeatother
               
\section{MYCENAEAN CHARIOT}

As always we start from setting everything from a standard template:

\begin{verbatim}
\cxset{%
         reset,
         section afterskip=5pt,
         section color=black!90,
      }  
\end{verbatim}



\cxset{subsection numbering=none}
\subsection[Numbers]{Numbers}

\lorem

\begin{figure}[t]
\makebox[-10pt][r]{\parbox[b]{4.5cm}{%
   \captionsetup{labelfont=bf, textfont=bf, justification=raggedright, labelsep=period, width={\marginparwidth-\marginparsep},
   belowskip=0pt,aboveskip=0pt}
   \captionof{figure}{This is the caption for a floating image. These images are allowed to float at the bottom or at the top of the page only. Normal keys for the caption setup can be used.}}%
   }\hspace{\the\marginparsep}
\fbox{\includegraphics[width=\textwidth]{./images/chariot-book.jpg}}
\end{figure}

\let\thechapter\oldchapter
\makeatletter
\let\@seccntformat\oldseccntformat 

%
\lipsum[1-5]

\restoregeometry
 \def\@seccntformat#1{\protect\makebox[0pt][l]{\csname the#1\endcsname}\quad}
\cxset{style13}
\makeatother
\pagestyle{headings}









