\cxset{style81/.style={
 name=,
 numbering=WORDS,
 number font-size=LARGE,
 number font-weight=bfseries,
 number font-family=sffamily,
 number before=\kern0.5em,
 number after=\kern.3cm\raisebox{0pt}{\hbox{\color{thelightgray}\LARGE\textbullet}}\hfill\hfill\par\vskip30pt,
 number dot=,
 number position=rightname,
 number color=thelightgray,
 %chapter name
 chapter color=black!80,
 chapter font-size= Large,
 chapter font-weight=mdseries,
 chapter font-family=sffamily,
 chapter before=\hfill\raisebox{0pt}{\hbox{\color{thelightgray}\LARGE\textbullet}}\kern.3cm,
 chapter spaceout=none,
 chapter after=,
 chapter margin left=0cm,
 chapter margin top=1sp,
  %chapter title
 title font-family=gyre,
 title font-color=black!80,
 title font-weight=bold,
 title font-size=Huge,
 chapter title align=none,
 title margin left=0cm,
 title margin bottom=2cm,
 title margin top=40pt,
 chapter title align=center,
 chapter title width=\textwidth,
 title before=,
 title after=,
 title beforeskip=,
 title afterskip=,
 author block=false,
 section font-size=\LARGE,
 section font-weight=bfseries,
 section indent=0pt,
 epigraph width=\dimexpr(\textwidth-1cm)\relax,
 epigraph align=left,
 section font-weight=\normalfont,
 header style=empty}}

\cxset{style81}

\chapter[template 81]{Cyprus}
\rmfamily

The search for durable peace in lands torn by ethno-national conflict is among the most urgent issues of international politics. Looking closely at five flashpoints of regional crisis, Sumantra Bose asks the question upon which our global future may depend: how can peace be made, and kept, between warring groups with seemingly incompatible claims? Global in scope and implications but local in focus and method, Contested Lands critically examines the recent or current peace processes in Israel–Palestine, Kashmir, Bosnia, Cyprus, and Sri Lanka for an answer.

Since the line has opened I visited Cyprus numerous times and cannot bear as yet to visit the North. I would like
to keep my memories of the North as it was before the invasion. You can visit a graveyard but do you really want to open the graves and view the bodies?

\section{Template 81}

This is very similar to style69, which you probably did not believe any one seriously considered to use it. Here is another one, with just a small variation in numbering fonts and spacing. 

\begin{figure}[ht]
\centering
\includegraphics[width=0.45\textwidth]{cyprus-01}
\caption{Style 81 no frills style from the University of Harvard Press.}
\end{figure}

Israelis and Palestinians, Turkish and Greek Cypriots, Bosnia’s Muslims, Serbs, and Croats, Sinhalese and Tamil Sri Lankans, and pro-independence, pro-Pakistan, and pro-India Kashmiris share homelands scarred by clashing aspirations and war. Bose explains why these lands became zones of zero-sum conflict and boldly tackles the question of how durable peace can be achieved. The cases yield important general insights about the benefits of territorial self-rule, cross-border linkages, regional cooperation, and third-party involvement, and the risks of a deliberately gradual (“incremental”) strategy of peace-building.

Rich in narrative and incisive in analysis, this book takes us deep into the heartlands of conflict---Jerusalem, Kashmir’s Line of Control, the divided cities of Mostar in Bosnia and Nicosia in Cyprus, Sri Lanka’s Jaffna peninsula. Contested Lands illuminates how chronic confrontation can yield to compromise and coexistence in the world’s most troubled regions—and what the United States can do to help.


%https://books.google.ae/books?id=lFVVyJr_xbwC&printsec=frontcover&dq=eoka+:pdf&hl=en&sa=X&ei=UFIEVbShLcvsUt3lgLAO&ved=0CE0Q6AEwCTgU#v=onepage&q&f=false