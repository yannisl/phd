
\cxset{style59/.style={
 name=SECTION,
 numbering=Roman,
 number font-size=Large,
 number before=\kern0.5em\relax,
 number after=\hfill\hfill\par\vspace*{0ex},
 number dot=.,
 number position=rightname,
 number font-weight=\normalfont,
 number color=black,
 chapter color=black,
 chapter font-size=Large,
 chapter font-weight=normalfont,
 chapter before=\hfill,
 chapter after=,
 chapter margin left=0pt,
 title font-family=sffamily,
 title font-color=black,
 title font-weight=normalfont,
 title font-size=large,
 title font-shape=upshape,
 chapter title align=centering,
 chapter title width=0.8\textwidth,
 chapter afterindent=true,
 title before=\centerline{\rule{2cm}{0.4pt}}\offinterlineskip%
                    \addvspace{7pt},
 title after=\par\offinterlineskip\addvspace{10pt}\centerline{\rule{3cm}{0.4pt}}
                 \par,
 title margin top=1sp,
 title margin bottom=30pt,
 header style=empty}}

\cxset{style59}

\chapter[Combinations of Colours for producing various Hues]{COMBINATIONS OF COLOURS FOR\\ PRODUCING VARIOUS HUES.}

Artists used to go at great lengths to produce their own colours, although presently the artist material industry is so huge that this has become unecessary. In \emph{The Use and Abuse of Colours and Mediums in Oil Painting}, H.C. Savage provided a study of how pigments were historically produced as well as recipes of how colours could be produced. The book was published in London in 1892 by Reeves \& Sons, Ltd. 

The book is in small format and is produced in the form of a ``report’’ as the headings are named |sections|. So the first thing we need to do in our template is to set the |chapter name| accordingly.

\begin{figure}[ht]
\centering
\includegraphics[width=0.8\textwidth]{pigments}
\caption{Style 53 Page spread.}
\end{figure}

Another interesting feature is the hyphenation of the title in some of the headings, which breaks all current typographical practices. I left the settings for the hyphenation untouched but if you want to emulate it, you
can change the setting to:

\begin{verbatim}
\cxset{style61,
          chapter name=SECTION,
          chapter title hyphenation=on}
\end{verbatim}










