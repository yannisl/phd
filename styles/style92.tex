%\makeatletter\@runinheadtrue\makeatother

\cxset{style92/.style={
chapter opening=any,
chapter border-bottom-width=1pt,
chapter padding=0pt,
chapter display=block,
chapter float=center,
 name=,
 numbering=WORDS,
 number font-size=LARGE,
 number font-weight=bfseries,
 number font-family= sffamily,
 number font-shape=scshape,
 number before=,
 number after=,
 number dot=,
 number position=rightname,
 number color=black!80,
 number border-top-width=0pt,
 number border-left-width=0pt,
 number border-right-width=0pt,
 number border-bottom-width=1pt,
 number border-style=solid,
 number padding=1.5pt,
 number padding-left=0pt,
 number padding-right=0pt,
 number padding-bottom=3pt,
 number float=center,
 number display=inline,
 %chapter name
 chapter float=center,
 chapter display=block,
 chapter color=black!80,
 chapter font-size= Large,
 chapter font-weight=bfseries,
 chapter font-family=rmfamily,
 chapter before=,
 chapter spaceout=soul,
 chapter after=,
 chapter margin left=0cm,
 chapter margin top=0sp,
 chapter padding=0pt,
 chapter padding-bottom=2pt,
 chapter border-bottom-color=black,
  %chapter title
 title font-family=sffamily,
 title font-color=black!80,
 title font-weight=mdseries,
 title font-shape=scshape,
 title font-size=HUGE,
 chapter title align=none,
 title margin-left=0cm,
 title margin bottom=60pt,
 title margin top=60pt,
 title border-left-width=0pt,
 chapter title align=centering,
 chapter title text-align=center,
 chapter title width=\textwidth,
 title before=,
 title after=,
 title beforeskip=,
 title afterskip=,
 title display=block,
 title margin-top=20pt,
 author  names=Dr Yiannis Lazarides,
 author block=false,
 section font-size=Large,
 section font-weight=bfseries,
 section indent=0pt,
 section beforeskip=20pt,
 section afterskip=20pt,
 epigraph width=\dimexpr(\textwidth-2cm)\relax,
 epigraph align=center,
 epigraph text align=center,
 section font-weight=mdseries,
 section align=center,
 epigraph rule width=0pt,
 header style=plain}}

\cxset{style92}
\parindent2em

\chapter[Template 92]{Kazanzakis And God}
\thispagestyle{plain}
\pagestyle{headings}

\dropcap{N}{}ikos Kazanzakis struggled with the idea of God throughout his life. The traditional view of God in the Abrahamic religions (Judaism, Christianity and Islam) has been in retreat for many years. Despite the turbulent relations between the three religions, believers in these traditions have kept a bsic common belief. God is an immutable, omniscient, omnipotent and omnibenevolent. \emph{Kazantzakis and God} by Daniel A. Dombrowski
and published by State University of New York Press (1997). 

\begin{figure}[ht]
\centering
\includegraphics[width=0.45\textwidth]{kazanzakis-01}
\includegraphics[width=0.455\textwidth]{kazanzakis-02}
\caption{Style 92 spread.}
\end{figure}

You may need to set the fonts and spacing to your liking here. The main point was to introduce the property
|border-style=double|. This can be applied either selectively or globally to all elements. The border styles can be extended by the use of the property |number border-bottom-style-custom|, where the value is 
any valid CSS name. 

\cxset{chapter opening=anywhere}

\example
\begin{verbatim}
\cxset{style92,
   chapter opening=anywhere,
   number border-bottom-style=solid
  }
\end{verbatim}


% reset settings for other chapters
\cxset{chapter opening=anywhere,
          chapter toc=none,
          section font-weight=bold,
          section afterskip=1pt plus0.5em minus0.5em,
          number border-style=double,
          number border-bottom-width=3pt}
          
 \chapter[Template 92]{Eating and\\[10pt] Spiritual\\[10pt] Exercises}         
          
 \dropcap{T}{}he ingenious work of Greek author Nikos Kazantzakis who also wrote "Zorba the Greek". Even though "The Last Temptation of Christ" has been regarded as heretical and blasphemous, and it had even been included in the Vatican's Index of forbidden books, it carries a profound message that will touch all open-minded people. It was written with the deepest love and utmost respect for Jesus Christ and as the author himself has said, he was weeping so profusely while writing it, that his tears drenched the page so that he would have to stop and wait for the paper to dry in order to continue. The main premise of N.Kazantzakis is that Jesus had been tempted by all human temptations and had conqured them all; he was tempted by the devil to eat and drink while fasting but stood strong. However, food and water are only a mere fragment of earthly temptations. One of the strongest human desires is to be able to live comfortably and without pain or deprivation, to raise a nice family, to have your cellar full of wheat and oil and to die peacefully in your bed at a very old age. That was the last temptation of Christ (and not copulating with Mary Magdalene as many claim to be the hign point of the book thus undervaluing this spiritual masterpiece). This last temptation came to him in a vision by the devil disguised as an angel while Jesus was on the cross. For one split second he experienced a whole lifetime of what could have been if he weren't the Messiah, if the burden of saving the world were not on his shoulders. That life was sweet and comfortable, with wives, children and plentiful possessions, and above all, with no excruciating pain from iron nails in his arms and feet. Anyone could have easily succumbed to that temptation but not Kazantzakis's Christ: He fought the last temptation and implored to go back to the cross and die there for humanity, acknowledging that this was where he truly belonged. Kazantzakis's Christ was the Messiah and the world was saved. 

It was successfully adapted into film by Martin Scorsese with Willem Dafoe as Jesus. 

The title of the movie "The Last Temptation" fleetingly appears in the movie "Donnie Darko" when Donnie is seen coming out of a movie theateom.    

\section{Technical}  

I decided to draw most of the rules with \tikzname\ as it gives more granular control. There are keys galore
in the pgfmanual describing the various lines that are available and ultimate control can be obtained by defining
dashing patterns.

\begin{texexample}{}{}
\drawrule[dashed, draw=blue,double]{5cm}{1pt}
\chapter[dashed rule example]{This has a dashed rule}
\end{texexample}        
      
%from http://www.urbandictionary.com/define.php?term=The+Last+Temptation+of+Christ
 
%https://books.google.ae/books?id=80socP_JycwC&pg=PA2&dq=kazantzakis+freedom+from+religion&hl=en&sa=X&ei=2_wIVegqh9tR6NaCgAc&ved=0CCcQ6AEwAg#v=onepage&q=kazantzakis%20freedom%20from%20religion&f=false