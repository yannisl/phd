\DeclareRobustCommand\rafrule{%
  \leavevmode \hbox to 14.2cm{\leaders \hbox{\vrule height 20pt width10pt\hskip10pt}\hfill}}
\cxset{style98/.style={
chapter opening=any,
chapter border-bottom-width=1pt,
chapter padding=0pt,
chapter display=block,
chapter float=center,
 name=,
 numbering=arabic,
 number font-size=huge,
 number font-weight=mdseries,
 number font-family= rmfamily,
 number font-shape=scshape,
 number before=\rafrule\relax,
 number after content=,
 number dot=,
 number position=rightname,
 number color=black,
 number border-top-width=0pt,
 number border-left-width=0pt,
 number border-right-width=0pt,
 number border-bottom-width=0pt,
 number border-style=solid,
 number padding=1.5pt,
 number padding-top=0pt,
 number padding-left=0pt,
 number padding-right=0pt,
 number padding-bottom=0pt,
 chapter background-color=black,
 number color=black,
 number float=right,
 number display=block,
 %chapter name
 chapter float=none,
 chapter display=block,
 chapter color=thelightgray,
 chapter font-size= huge,
 chapter font-weight=mdseries,
 chapter font-family=rmfamily,
 chapter before content=,
 chapter spaceout=none,
 chapter after=,
 chapter margin left=0cm,
 chapter margin top=0sp,
 chapter padding=0pt,
 chapter padding-bottom=12pt,
 chapter border-bottom-color=black,
 chapter border-top-width=0pt,
 chapter border-right-width=0pt,
 chapter border-bottom-width=0pt,
  %chapter title
 title font-family=rmfamily,
 title font-color=spot!50,
 title font-weight=mdseries,
 title font-shape=upshape,
 title font-size=Huge,
 chapter title align=none,
 title margin-left=0cm,
 title margin bottom=60pt,
 title border-left-width=0pt,
 title border-top-width=0pt,
 title border-bottom-width=0pt,
 title border-top-color=thelightgray,
 chapter title align=right,
 chapter title text-align=right,
 chapter title width=\textwidth,
 title before=,
 title after=,
 title beforeskip=,
 title afterskip=,
 title padding-top=0pt,
 title display=block,
 title margin-top=0pt,
 author  names=Dr Yiannis Lazarides,
 author block=false,
 section font-size=\Large,
 section font-weight=bfseries,
 section indent=0pt,
 section beforeskip=20pt,
 section afterskip=20pt,
 epigraph width=\dimexpr(\textwidth-2cm)\relax,
 epigraph align=center,
 epigraph text align=center,
 section font-weight=mdseries,
 section align=center,
 epigraph rule width=0pt,
 header style=plain}}
 
  

\cxset{style98}
\parindent2em

\chapter[Template 98]{The Red Army Faction: A Documentary History. Dancing with imperialism}
\label{style98}
\thispagestyle{plain}
\pagestyle{headings}


\dropcap{T}{} ey were known as the Baader-Meinhof Gang, but it is now customary in academic circles to refer to them as the Red Army Faction. The RAF was founded in 1970 by Andreas Baader, Gudrun Ensslin, Horst Mahler, and Ulrike Meinhof.

\begin{figure}[ht]
\centering
\includegraphics[width=0.90\textwidth]{raf-01}
\caption{Style 98 spread.}
\end{figure}

The Red Army Faction existed from 1970 to 1998, committing numerous terrorist acts, especially in late 1977, which led to a national crisis that became known as "German Autumn". It was held responsible for thirty-four deaths, including many secondary targets, such as chauffeurs and bodyguards, and many injuries in its almost thirty years of activity. Although better-known, the RAF conducted fewer attacks than the Revolutionary Cells (German: Revolutionäre Zellen, RZ), which is held responsible for 296 bomb attacks, arson and other attacks between 1973 and 1995.[2]

Although Meinhof was not considered to be a leader of the RAF at any time, her involvement in Baader's escape from jail in 1970 and her well-known status as a German journalist led to her name becoming attached to it.[3]
There were three successive incarnations of the organization:

the “first generation” which consisted of Baader and his associates,
the "second generation" RAF, which operated in the mid to late 1970s after several former members of the Socialist Patients' Collective joined, and
the “third generation” RAF, which existed in the 1980s and 1990s.
On 20 April 1998, an eight-page typewritten letter in German was faxed to the Reuters news agency, signed “RAF" with the submachine-gun red star, declaring that the group had dissolved. 
 
\section{Technical}
 
One deviation from a CSS-like approach is the switching of the chapter number from the left to the right of the title
and also the variation of the font combinations. They combinations are striking, but I am not too sure if I like them that much.

%https://books.google.ae/books?id=MbnP1xD577kC&printsec=frontcover&dq=hunger+striker&hl=en&sa=X&ei=MY8eVYWXDdWXuATBvIEQ&ved=0CEUQ6AEwCDh4#v=twopage&q&f=false
\makeatletter\@specialfalse\makeatother    