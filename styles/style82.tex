\cxset{style82/.style={
 chapter opening=any,
 name=CHAPTER,
 numbering=arabic,
 number font-size=Large,
 number font-weight=bfseries,
 number font-family=sffamily,
 number before=,
 number after=,
 number dot=,
 number position=rightname,
 number color=black!80,
 number display=inline,
 number float=none,
 %chapter 
 chapter color=black!80,
 chapter font-size= Large,
 chapter font-weight=bfseries,
 chapter font-family=sffamily,
 chapter before=\centerline{\vrule width 1cm height 8pt},
 chapter letter-spacing=none,
 chapter after=,
 chapter margin-left=0cm,
 chapter margin top=1sp,
 chapter display=block,
 chapter float=center,
  %chapter title
 title font-family=sffamily,
 title font-color=black!80,
 title font-weight=bold,
 title font-size=Large,
 title spaceout=none,
 chapter title align=none,
 title margin top=15pt,
 title margin-left=0cm,
 title margin bottom=30pt,
 title border-top-width=2pt,
 title border-width=0pt,
 chapter title align=none,
 chapter title text-align=center,
 chapter title width=\textwidth,
 title display=block,
 title before=,
 title after=,
 title beforeskip=,
 title afterskip=,
 author block=true,
 author block format=\centering,
 author names=yiannis lazarides,
 section font-size=large,
 section numbering=none,
 section font-family=sffamily,
 section font-weight=bfseries,
 section spaceout=none,
 section indent=0pt,
 section align=center,
 section color=black,
 epigraph width=\dimexpr(\textwidth-1cm)\relax,
 epigraph align=left,
 header style=empty}}

\cxset{style82}

\chapter[template 82]{ZEALOTS AND ASSASINS}


\section{THE ZEALOTS}
The search for durable peace in lands torn by ethno-national conflict is among the most urgent issues of international politics. Looking closely at five flashpoints of regional crisis, Sumantra Bose asks the question upon which our global future may depend: how can peace be made, and kept, between warring groups with seemingly 


\begin{figure}[ht]
\centering
\includegraphics[width=0.9\textwidth]{zealots}
\caption{Style 82.}
\end{figure}

Israelis and Palestinians, Turkish and Greek Cypriots, Bosnia’s Muslims, Serbs, and Croats, Sinhalese and Tamil Sri Lankans, and pro-independence, pro-Pakistan, and pro-India Kashmiris share homelands scarred by clashing aspirations and war. Bose explains why these lands became zones of zero-sum conflict and boldly tackles the question of how durable peace can be achieved. The cases yield important general insights about the benefits of territorial self-rule, cross-border linkages, regional cooperation, and third-party involvement, and the risks of a deliberately gradual (“incremental”) strategy of peace-building.

Rich in narrative and incisive in analysis, this book takes us deep into the heartlands of conflict---Jerusalem, Kashmir’s Line of Control, the divided cities of Mostar in Bosnia and Nicosia in Cyprus, Sri Lanka’s Jaffna peninsula. Contested Lands illuminates how chronic confrontation can yield to compromise and coexistence in the world’s most troubled regions—and what the United States can do to help.


%https://books.google.ae/books?id=lFVVyJr_xbwC&printsec=frontcover&dq=eoka+:pdf&hl=en&sa=X&ei=UFIEVbShLcvsUt3lgLAO&ved=0CE0Q6AEwCTgU#v=onepage&q&f=false