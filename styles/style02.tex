\clearpage
\makeatletter\@debugtrue\makeatother
\cxset{
 chapter toc=true,
 name=CHAPTER,
 chapter numbering=ORDINALS,
 number font-size=Large,
 number font-family=rmfamily,
 number font-weight=bfseries,
 number before=\kern0.5em,
 number dot=,
 number after=\hfill\hfill\par,
 number position=rightname,
 chapter font-family=rmfamily,
 chapter font-weight=bold,
 chapter font-size=Large,
 chapter before={\vspace*{20pt}\par\hfill},
 chapter after=,
 chapter color=black,
 number color=black,
 %
 title margin top=10pt,
 title before=\par\nointerlineskip\hfill,
 title after=\hfill\hfill\par\nointerlineskip,
 title font-family=rmfamily,
 title font-color= black,
 title font-weight=bfseries,
 title font-size=LARGE,
 chapter title width=0.8\textwidth,
 chapter title align=centering,
 title margin-left=0pt,
 author block=false}

\debugtitle
\debugchapter
\chapter[Template 2]{Mondino, the Restorer of Anatomy}

The archive.org is an extraordinary hunting ground  for typographical surprises. On a recent excursion to find some books on Versalius I stubled on a book titled \emph{Andreas Vesalius, the reformer of anatomy} by  Ball, James Moores. It is an old book published in 1910 and has a couple of unusual features. Check the figure below and see if you can identify the challenging feature.

\begin{figure}[ht]
\centering
\includegraphics[width=0.8\textwidth]{versalius}
\caption{J.B. Moore’s \emph{Andreas Versalius, the Reformer of Anatomy} has many unusual features, including chapter numbers using ordinals. }
\end{figure}

\cxset{chapter toc=true,
          chapter opening=anywhere}
          
\chapter{The Template}          
The template is called \emph{Versalius} and is stored under style02. It can be loaded in the normal way using:
\begin{verbatim}
\usepackage[style02]{phd}
\end{verbatim}

I have not reproduced the full extend of the book’s requirements, as some details are quite cumbersome to be automated through \tex. These though can easily be incorporated in a manual way. More about this later.


\section{The Table of Contents}
Another interesting aspect of this book, which is common with many books of its period is the ToC. The ToC shows the full range of the chapter pages, i.e., it is marked as Page 1-16 rather than the common practice nowdays that indicates only the starting page of the chapter. It also has “TABLE OF CONTENTS”  as a heading and not just contents as you would expect from today’s books.

\begin{figure}[ht]
\centering
\includegraphics[width=0.8\textwidth]{versalius-01}
\caption{J.B. Moore’s \emph{Andreas Versalius, the Reformer of Anatomy} has many unusual features, including chapter numbers using ordinals. }
\end{figure}

\section{List of Illustrations}

\begin{figure}[ht]
\centering
\includegraphics[width=0.8\textwidth]{versalius-02}
\caption{J.B. Moore’s \emph{Andreas Versalius, the Reformer of Anatomy} has many unusual features, including chapter numbers using ordinals. }
\end{figure}

\section{The Frontmatter}
As a foreward there is an unumbered chapter called ``Introduction’’. The chapter heading also has a head band.
\begin{figure}[ht]
\centering
\includegraphics[width=0.8\textwidth]{versalius-03}
\caption{J.B. Moore’s \emph{Andreas Versalius, the Reformer of Anatomy} has many unusual features, including chapter numbers using ordinals. }
\label{lettrine}
\end{figure}

\bgroup
\centering
\includegraphics[width=0.7\textwidth]{versalius-headband}

\LARGE\bfseries INTRODUCTION\par
\egroup
\def\dropcapversalius{%
\vbox to 0pt{\vskip6pt\leavevmode\noindent\includegraphics[width=2.39cm]{versalius-dropcap}%
}%
}
\parindent0pt

\hangindent2.6cm \hangafter0
\dropcapversalius \textsc{he dropcap will have to be inserted}, either using the lettrine package or do be achieved via a parshape command and manual entry. You can also write your own macro command using the details we provide under the Paragraphs chapter. On this page I have manually inserted it, as I used an image from the book for the dropcap. If you were to use the template for a full book, it will be then preferable to use

the lettrine package to set the dropcaps. If you observe Figure~\ref{lettrine} carefully, you will notice the first line of theopening paragraph is in small caps. As \tex typesets the full paragraph this is almost an impossible task to achieve through normal \tex commands and in order not to overcomplicate the discussion it can be achieved manually via trial and error. 

\section{Figures}

Most of the figures are wrapped illustrations. A couple are full page figures and bear no caption numbering. One such illustration is shown on page~\pageref{fig:vesalius}. Do note that the List of Illustrations does have the illustrations listed with additional information to that shown in the captions. 

\begin{figure}[p]
\centering
\includegraphics[width=\textwidth]{vesalius}
\centering
ANDREAS VESALIUS\par
(From an old copperplate engraving)\par
\label{fig:vesalius}
\end{figure}






