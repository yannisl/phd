\makeatletter
\@runinheadtrue
\makeatother
\parskip1pt
\cxset{quotation font-size=\large}
\cxset{style80/.style={
 name=,
 numbering=arabic,
 number font-size=\LARGE,
 number font-weight=\bfseries,
 number before={},
 number after=,
 number position=rightname,
 number dot=.,
 chapter color=black!80,
 chapter font-size=,
 chapter before=,
 number after=,
 chapter after=,
 number color=black!80,
 title font-family=sffamily,
 title font-color=black!80,
 title before=,
 title after=\par,
 title font-weight=\bfseries,
 title font-size=\LARGE,
 title beforeskip=\space,
 title margin bottom=40pt,
 chapter title width=0.6\textwidth,
 chapter afterindent=true,
 chapter title align=left,
  header style=empty,
 author block=false,
 author names=\textsc{\aegean James A. Russel and\\[-1.5pt] Jos\'e Miguel Fernandez-Dols },
 author block format=\normalfont\large\vskip20pt,
 epigraph width=0.8\textwidth,
 epigraph align=center,
 epigraph text align=left,
 section indent=0pt,
 section font-weight=bold,
 section align=left,
 section font-size=Large,
 subsection font-size=large,
 section number after=,}}

\cxset{style80}
\chapter[Template 80]{Sorting, separating, sealing}
\label{ch:style80}

Those who had worked so diligently to destroy civility and the intermingling of cultures in Cyprus achieved their goal of total separation of the two communities in 1974. It was a separation about as complete as well-organized
ethnic cleansing (a little killing, a little terror, plnty of pressure and persuasion) can make it.

\begin{figure}[ht]
\centering
\includegraphics[width=.9\textwidth]{line}
\caption{Style 80.}
\label{fig:style80}
\end{figure} 

It is a young man’s misfortune to be drafted to the army the year the enemy invaded his country, but that is what happened to me. This was an event that marked my generation and as our grandfathers before us, when we get together we always divide memories to those before the war and after the war. For us it was our War, even if for the rest of the world we were but a small footnote in the eight o’clock news. 

To this day the Island is still divided. Evil Empires came down and the Berlin War is now a distant memory. Cockburn tells the story of how the “Green Line” came about. 
On the Greek Cypriot side there were no angels either. 


\begin{quotation}
I had a meeting with four widows who now live in the village of Ta\c{s}kent, on the southern foothills of the Pentadactylos, where they were rehoused by the authorities, as a group arriving in the north. \ldots On 15 August, the date of the second Turkish advance in the north, a gang of \textsc{EOKA-B} men, led by a certain `Adrikos’, rounded up sixty-nine of their menfolk, the youngest thirteen, the oldest seventy-four. 
\end{quotation}

Cockburn writes well. The book is a gender study but for me was a jog to my memories. 

\section{Loading the Template style 80}
The Line: Women, Partition and the Gender Order in Cyprus
 By Cynthia Cockburn
\begin{scriptexample}{Loading the Template}
\begin{verbatim}
\usepackage{phd}
\cxset{style50}
\end{verbatim}
\end{scriptexample}

\section{The Chapter head}
\begin{scriptexample}{Loading the Template}{}
\begin{verbatim}
\cxset{chapter format=runin}
\end{verbatim}
\end{scriptexample}


\makeatletter\@runinheadfalse\makeatother

%https://books.google.ae/books?id=itMmWnc6oK0C&printsec=frontcover&dq=eoka+:pdf&hl=en&sa=X&ei=Iz8EVdr3DoX-UOPwgfAE&ved=0CFcQ6AEwCTgK#v=twopage&q&f=false

