\makeatletter
\@runinheadtrue
\makeatother
\cxset{style57/.style={
 title margin top=50pt,
 name=,
 numbering=arabic,
 number font-size=\LARGE,
 number font-weight=\bfseries,
 number before={},
 number after=\quad,
 number position=rightname,
 number dot=,
 chapter color=black!80,
 chapter font-size=,
 chapter before=,
 number after=,
 chapter after=,
 number color=black!80,
 title font-family=\rmfamily,
 title font-color=black!80,
 title before=,
 title after=\par,
 title font-weight=\bfseries,
 title font-size=\LARGE,
 title beforeskip=\space,
 title afterskip=\vspace*{20pt},
 chapter title width=0.6\textwidth,
 chapter title align=raggedright,
 header style=empty,
 author block=true,
 author names=\textsc{\aegean James A. Russel and\\[-1.5pt] Jos\'e Miguel Fernandez-Dols },
 author block format=\normalfont\large\vskip20pt,
 epigraph width=0.8\textwidth,
 epigraph align=center,
 epigraph text align=left,
 section indent=0pt,
 section font-weight=bold,
 section align=left,
 section font-size=Large,
 subsection font-size=large,
 }}

\cxset{style57}
\chapter{Diversity and Evolution}
\label{ch:style50}


\section{The Chapter head}

This book has some very very unusual chapter headings that make it difficult to typeset using \tex. The chapter title
has a background image of the title, typeset in larger letters and in two or more lines. Here is an example from
our typesetting routine.  My first attempt to typeset was to use \tikzname, however I finally opted for basic boxing macros from \latexe in order to not overcomplicate the typesetting and with the advantage that typesetting can be carried out with one run. With \tikzname\ and its |remember picture, overlay| key settings two or more runs would be necessary.
\begin{figure}[htb]
\centering
\includegraphics[width=0.8\textwidth]{insects-01}
\caption{Spread from the book \emph{The Evolution of the Insects.}}
\end{figure}


What we have to do is first to define what problems we are trying to solve and then to typeset it first outside the \pkgname{phd} package macros. 
\bigskip


\newsavebox\backtitle
\makeatletter
\let\oldHUGE\HUGE
\def\HUGE{\@setfontsize\Huge{38}{38}}

\newcommand\bgtitle[2]{%
\savebox\backtitle{%
   \lineskip0pt
   \parbox{0.8\linewidth}{%
      \rmfamily\color{spot!10}%
      \HUGE\language-1%
      #1
    }%  
}%
{\sffamily\HHUGE 1\kern0.3cm\raisebox{10pt}{\hbox to 0pt{\usebox\backtitle}}\kern0.5cm
\parbox{0.7\textwidth}{\Huge #2}}%
}

\makeatother

\bgtitle{Diversity and\newline Evolution}{ Diversity and Evolution}

Examining the title a bit more carefully reveals that the back title is typeset using two different fonts one in a roman family and the front in a sans serif family. In order for us to provide the hook at a position that the phd routines can use it, we need to put it in the titlebox. This is automatically carried out by the package. Ideally what we would like to do is allow the user to type |\chapter{Diversity and Evolution}| and let the \latexe code take care of the line breaking and the typesetting of the background title. However, examining some more chapters we see that we may get titles like those shown in the next figure (Figure~\ref{fig:insects2}).

\begin{figure}[htb]
\centering
\fbox{\includegraphics[width=0.8\textwidth]{insects-02}}
\caption{Spread from the book \emph{The Evolution of the Insects, showing positioning of epigraph. Another unususal feature of this book is the repeating of the title as a background.}}
\label{fig:insects2}
\end{figure}

There is no logical rule we can apply to the breaking of the back title. In the first case the line break is after a short word ``The’’ whereas in the second case it is much longer. The only reasonable thing we can do is to let the user break it manually. 


\cxset{
 chapter opening=anywhere,
 title font-size=Huge,
 title before=\raisebox{0pt}{\hbox to 0pt{\kern20pt\usebox{\backtitle}}}%
}

\bgtitle{The\\ Holometabola}{The Holometabola}

\section{Epigraphs}

The book has sixteen chapters, but epigraphs only appear only at three chapters. In Figure~\ref{fig:insects2} the epigraph extends to the full height of the chapter head block, whereas in Figure~ it appears on top of the heading.

\begin{figure}[htb]
\centering
\fbox{\includegraphics[width=0.8\textwidth]{insects-03}}
\caption{Spread from the book \emph{The Evolution of the Insects, showing positioning of epigraph. Another unususal feature of this book is the repeating of the title as a background.}}
\label{fig:insects3}
\end{figure}

\begin{figure}[htb]
\centering
\fbox{\includegraphics[width=0.8\textwidth]{insects-04}}
\caption{Spread from the book \emph{The Evolution of the Insects, showing positioning of epigraph. Another unususal feature of this book is the repeating of the title as a background.}}
\label{fig:insects4}
\end{figure}

\section{The Final Template and Commands}

It would be much easier if we could make the template as a special? 
\savebox{\backtitle}{test}

%\usebox{\backtitle}

\bgtitle{Creating\\ Book Designs}{Creating Book Designs}

\clearpage
\lipsum
\makeatletter\@runinheadfalse\makeatother
\let\HUGE\oldHUGE
\cxset{title before={},
          chapter opening=any}%empty the title until we get it right for the rest