\cxset{style71/.style={
 name=,
 numbering=arabic,
 number font-size=huge,
 number font-weight=mdweight,
 number font-family=sffamily,
 number before=\vskip-30pt \hfill \huge\bfseries[\thinspace,
 number after=\bfseries\huge\thinspace]\hfil\hfill \hfill\vskip2pt,
 number dot=,
 number position=rightname,
 number color=black!80,
 %chapter name
 chapter color=black!80,
 chapter font-size= Large,
 chapter font-weight=bfseries,
 chapter font-family=rmfamily,
 chapter before=,
 chapter spaceout=none,
 chapter after=\par,
 chapter margin left=0cm,
 chapter margin top=1sp,
  %chapter title
 title font-family=sffamily,
 title font-color=black!80,
 title font-weight=bold,
 title font-size=huge,
 chapter title align=none,
 title margin-left=0cm,
 title margin bottom=2cm,
 title margin top=1sp,
 chapter title align=centering,
 chapter title width=11cm,
 title before=,
 title after=,
 title beforeskip=,
 title afterskip=,
 author block=false,
 section font-size=Large,
 section font-weight=\bfseries,
 section indent=0pt,
 section align=centering,
 section numbering=none,
 epigraph width=\dimexpr(\textwidth-2cm)\relax,
 epigraph align=center,
 epigraph text align=center,
% section font-weight=mdseries,
 epigraph rule width=0pt,
  header style=empty}}

\cxset{style71}

\chapter{The American Railroad}


\epigraph{%
Un train de chemin der fer est dans ce est dans pays-l`a consider\'e comme une
voiture ordinaire. On est habitue a s’en garder conne nous nous gardons d’un cabriolet qui passe das la rue.}{}
\cxset{epigraph width=\dimexpr(\textwidth-1.5cm)\relax,}
\epigraph{%
[In that country, a railroad train is just another vehicle. People are accustomed to watch out for it in the way we watch out so as not to be knocked over by a buggy, in the street.]
}{---French travel account, 1848}


Evolution is happening before our eyes, but we fail to see it. Authors make a compelling argument that the last
10,000 years of civilization has shaped humans in unprecendent ways.

\begin{figure}[ht]
\centering
\includegraphics[width=0.9\textwidth]{american-railroad}
\caption{Style 71 spread.}
\end{figure}

The shapng

\section{Template 71}

The unusual feature of this template, but a regular feature of University of California Press is the square brackets framing the chapter number. The template is not too difficult to set with the keys options provided by the \pkgname{phd} package; we need to take care of alignment of the brackets and to set the fonts right.

\section{Footnotes}



\cxset{chapter opening=anywhere}

\example
\begin{verbatim}
\cxset{style71,
   chapter opening=anywhere,
  }
\end{verbatim}


% reset settings for other chapters
\cxset{chapter opening=any,}

%https://books.google.ae/books?id=890nCC_kZeIC&printsec=frontcover&dq=university+of+california+press&hl=en&sa=X&ei=BQn_VIjsA4XWU7rSgbAK&ved=0CC4Q6AEwAjjMAw#v=twopage&q&f=false
