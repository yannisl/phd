% \iffalse meta-comment
%
%% File: l3cctab.dtx Copyright (C) 2018 The LaTeX3 Project
%
% It may be distributed and/or modified under the conditions of the
% LaTeX Project Public License (LPPL), either version 1.3c of this
% license or (at your option) any later version.  The latest version
% of this license is in the file
%
%    http://www.latex-project.org/lppl.txt
%
% This file is part of the "l3trial bundle" (The Work in LPPL)
% and all files in that bundle must be distributed together.
%
% -----------------------------------------------------------------------
%
% The development version of the bundle can be found at
%
%    https://github.com/latex3/latex3
%
% for those people who are interested.
%
%<*driver|package>
\RequirePackage{expl3}
\def\IniTeX{\sffamily IniTeX}
%</driver|package>
%<*driver>
\documentclass[full,book,colorize,oneside,cm-default]{phddoc}
\usepackage{phd-lowersections}
%%
%% This is file `phd-documentation-defaults.def',
%% generated with the docstrip utility.
%%
%% The original source files were:
%%
%% phd-fontmanager.dtx  (with options: `DFLT')
%% phd-colorpalette.dtx  (with options: `DFLT')
%% phd-lowersections.dtx  (with options: `DFLT')
%% phd-toc.dtx  (with options: `DFLT')
%% phd-documentation.dtx  (with options: `DFLT')
%% ----------------------------------------------------------------
%% phd --- A package to beautify documents.
%% E-mail: yannislaz@gmail.com
%% Released under the LaTeX Project Public License v1.3c or later
%% See http://www.latex-project.org/lppl.txt
%% ----------------------------------------------------------------
\cxset{
   % settings for document fonts.
    main font-size                 = 10pt,
    main font-face                 = Georgia,
    main sans font-face            = Georgia, %Arial,
    main mono font-face            = B-612,%Source Code Pro, %Consolas, %Apl385, %Consolas,%Source Code Pro,
    chapter label font-face        = Georgia,
    chapter number font-face       = Arial,
    chapter title font-face        = Times New Roman,
    section label font-face        = Arial,
    section number font-face       = Arial,
    section title font-face        = Arial,
    subsection label font-face     = Arial,
    subsection number font-face    = Arial,
    subsection title font-face     = Arial,
    subsubsection label font-face  = Arial,
    subsubsection number font-face = Arial,
    subsubsection title font-face  = Arial,
    paragraph label font-face      = Arial,
    paragraph number font-face     = Arial,
    paragraph title font-face      = Arial,
    subparagraph label font-face   = Arial,
    subparagraph number font-face  = Arial,
    subparagraph title font-face   = Arial,
    % default palette
    palette orange sakura,
    part format                       = traditional,
    chapter title margin-top-width    =  0cm,
    chapter title margin-right-width  =  1cm,
    chapter title margin-bottom-width = 10pt,
    chapter title margin-left-width   = 0pt,
    chapter align                     = left,
    chapter title align               = left, %checked
    chapter name                      = chapter,
    chapter format                    = fashion,
    chapter font-size                 = Huge,
    chapter font-weight               = bold,
    chapter font-family               = sffamily,
    chapter font-shape                = upshape,
    chapter color                     = black,
    chapter number prefix             = ,
    chapter number suffix             = ,
    chapter numbering                 = arabic,
    chapter indent                    = 0pt,
    chapter beforeskip                = -3cm,
    chapter afterskip                 = 30pt,
    chapter afterindent               = off,
    chapter number after              = ,
    chapter arc                       = 0mm,
    chapter background-color          = white,
    chapter afterindent               = off,
    chapter grow left                 = 0mm,
    chapter grow right                = 0mm,
    chapter rounded corners           = northeast,
    chapter shadow                    = fuzzy halo,
    chapter border-left-width         = 0pt,
    chapter border-right-width        = 0pt,
    chapter border-top-width          = 0pt,
    chapter border-bottom-width       = 0pt,
    chapter padding-left-width        = 0pt,
    chapter padding-right-width       = 10pt,
    chapter padding-top-width         = 10pt,
    chapter padding-bottom-width      = 10pt,
    chapter number color              = white,
    chapter label color               = black,
    chapter number font-size        = huge,
    chapter number font-weight      = bfseries,
    chapter number font-family      = sffamily,
    chapter number font-shape       = upshape,
    chapter number align            = Centering,
    chapter title font-size        = Huge,
    chapter title font-weight      = bold,
    chapter title font-family      = sffamily,
    chapter title font-shape       = upshape,
    chapter title color            = black,
    section name                   = Section,
    section format                 = traditional,
    section align                  = Centering,
    section title align            = Centering, %checked
    section font-size              = Large,
    section font-weight            = bfseries,
    section font-family            = serif,
    section font-shape             = upshape,
    section number font-size       = Large,
    section number font-weight     = bfseries,
    section number font-family     = serif,
    section number font-shape      = upshape,
    section title font-size        = Large,
    section title font-weight      = bfseries,
    section title font-family      = serif,
    section title font-shape       = upshape,
    section color                  = black,
    section number prefix          = \thechapter.,
    section number suffix          =,
    section numbering              = arabic,
    section indent                 = 0pt,
    section beforeskip             = 3ex,
    section afterskip              = 1.5ex plus .1ex,
    section afterindent            = on,
    section number after           = \quad,
    section arc                            = 3pt,
    section background-color       = white,
    section afterindent                = on,
    section grow left                   = 0mm,
    section grow right                 = 0mm,
    section rounded corners        = northeast,
    section border-left-width      = 0pt,
    section border-right-width     = 0pt,
    section border-top-width       = 2pt,
    section border-bottom-width    = 2pt,
    section padding-left-width     = 0pt,
    section padding-right-width    = 10pt,
    section padding-top-width      = 2pt,
    section padding-bottom-width   = 2pt,
    section title margin-top-width = 2pt,
    section title color            = thesectiontitlecolor,
    section shadow                 = no shadow,
%% sybsection
    subsection name                   = Subsection,
    subsection format                 = hang,
    subsection font-size              = large,
    subsection font-weight            = bfseries,
    subsection font-family            = rmfamily,
    subsection font-shape             = upshape,
    subsection number font-size       = large,
    subsection number font-weight     = bfseries,
    subsection number font-family     = rmfamily,
    subsection number font-shape      = upshape,
    subsection title font-size        = Large,
    subsection title font-weight      = bfseries,
    subsection title font-family      = sffamily,
    subsection title font-shape       = upshape,
    subsection title color            = bgsexy,
    subsection color                  = bgsexy,
    subsection numbering              = arabic,
    subsection align                  = Centering, %checked
    subsection title align            = Centering, %checked
    subsection beforeskip             = -3.25explus -1ex minus -.2ex,
    subsection afterskip              = 1.5ex plus .2ex,
    subsection number prefix          = \thesection.,
    subsection indent                 = 0pt,
    subsection number after           = 0pt,
    subsection background-color       = white,
    subsection border-left-width      = 0pt,
    subsection border-right-width     = 0pt,
    subsection border-top-width       = 5pt,
    subsection border-bottom-width    = 5pt,
    subsection padding-left-width     = 0pt,
    subsection padding-right-width    = 0pt,
    subsection padding-top-width      = 20pt,
    subsection padding-bottom-width   = 20pt,
    subsection shadow                 = drop shadow,
    subsubsection name                    = Subsubsection,
    subsubsection format                  = hang,
    subsubsection background-color        = white, %checked
    subsubsection afterindent             = off,
    subsubsection font-family             = rmfamily,
    subsubsection font-size               = large,
    subsubsection font-weight             = bfseries,
    subsubsection font-family             = tiresias,
    subsubsection font-shape              = upshape,
    subsubsection font-family             = sffamily,
    subsubsection font-size               = large,
    subsubsection font-weight             = bfseries,
    subsubsection font-family             = tiresias,
    subsubsection font-shape              = upshape,
    subsubsection color                   = black,
    subsubsection number prefix           = \thesubsection,
    subsubsection number suffix           = ,
    subsubsection numbering               = arabic,
    subsubsection indent                  = 0pt,
    subsubsection beforeskip              = -3.25explus -1ex minus -.2ex,
    subsubsection afterskip               = 1.5ex plus .2ex,
    subsubsection align                   = center,
    subsubsection title align             = center,
    subsubsection number after     =,
    subsubsection border-left-width       = 0pt,
    subsubsection border-right-width      = 0pt,
    subsubsection border-top-width        = 2pt,
    subsubsection border-bottom-width     = 0pt,
    subsubsection padding-left-width      = 0pt,
    subsubsection padding-right-width     = 0pt,
    subsubsection padding-top-width       = 20pt,
    subsubsection padding-bottom-width    = 20pt,
    subsubsection shadow                  = no shadow,
    subsubsection title font-size         = large,
    subsubsection title font-weight       = bfseries,
    subsubsection title font-family       = serif,
    subsubsection title font-shape        = itshape,
    subsubsection title color             = thesubsectiontitlecolor,
    paragraph name                = paragraph,
    paragraph format              = inline,
    paragraph name                = paragraph,
    paragraph font-size           = large,
    paragraph font-weight         = bfseries,
    paragraph font-family         = rmfamily,
    paragraph font-shape          = upshape,
    paragraph numbering           = alpha,
    paragraph number prefix       = \thesubsubsection,
    paragraph align               = flushleft,
    paragraph beforeskip          = 3.25ex plus1ex minus.2ex,
    paragraph afterskip           = -1em,
    paragraph indent              = 0pt,
    paragraph number after        = \quad,
    paragraph color               = bgsexy,
    paragraph background-color    = white,
    paragraph shadow              = no shadow,
    paragraph afterindent         = off
    subparagraph name             = subparagraph,
    subparagraph format           = inline,
    subparagraph name             = subparagraph,
    subparagraph font-size        = large,
    subparagraph font-weight      = bfseries,
    subparagraph font-family      = rmfamily,
    subparagraph font-shape       = upshape,
    subparagraph color            = bgsexy,
    subparagraph background-color = bgsexy,
    subparagraph numbering        = none,
    subparagraph align            = flushleft,
    subparagraph beforeskip       = 3.25ex plus1ex minus .2ex,
    subparagraph afterskip        = -1em,
    subparagraph indent           = 0pt,
    subparagraph number after     = ,
    %subparagraph shadow           = off,
%% toc contents element settings
    toc name               = Table of Contents,
    toc  before            =,
    toc  after             =,
    toc  numwidth          = 0pt,
    toc  color             = thetocname,
    toc  background-color  = bgsexy!20,
    toc  frame-color       = red,
    toc  shadow            = none,
    toc  font-weight       = normal,
    toc  font-family       = rmfamily,
    toc  font-shape        = upshape,
    toc  font-size         = Huge,
    toc  afterskip         = 30pt,
    toc  after             = ,
    toc  align             = left,
    toc  indent            = 0pt,
    toc case               = none,
    toc  page after        = A,
    toc  pagestyle         = headings,
    toc  rmarg             = 4em,
%% TOC part keys
    toc part font-size    = LARGE,
    toc part color        =  black,
    toc part beforeskip   =  1em,
    toc part page before  =,
    toc part indent       =  0pt,
    toc part numwidth     = 1.5em,
    % table of contents defaults
    % toc chapter keys
    toc chapter font-size   = Large,
    toc chapter font-family = rmfamily,
    toc chapter font-weight = normal,
    % the toc chapter color thetocchapter
    % is fetched from the palette define
    % your own color in the palette rather than
    % change this here
    toc chapter color       = thetocchapter,
    toc chapter beforeskip  =1em,
    toc chapter afterskip   = 12pt plus0.2pt minus .2pt,
    toc chapter case        = upper,
    toc chapter numwidth    = 1.5em,
    %  TOC chapter page formatting
    toc chapter page font-size        = Large,
    toc chapter page font-shape       = upshape,
    toc chapter page font-weight      = normal,
    toc chapter page font-family      = rmfamily,
    toc chapter page color            = black,
    toc chapter page background-color = white,
    toc chapter page before           =,
    toc chapter page after            =,
      %TOC section
      % indentation
       toc section indent=1.5em,
       toc section numwidth= 2.3em,
       toc section beforeskip=0pt,
       toc section afterskip=0pt,
      % page number fonts
       toc  section page font-size          = large,
       toc  section page font-shape         = upshape,
       toc  section page font-weight        = normal,
       toc  section page font-family        = rmfamily,
       % page number colors
       toc  section page color              = bgsexy,
       toc  section page background-color   = white, %theblue!10,
       % page number before after elements
       toc  section page before             =,
       toc  section page after              =,
       toc section page after = ,
       toc section page before =,
%%
%% subsection defaults
    toc subsection indent        = 3.8em,
    toc subsection numwidth      = 3.2em,
    toc subsection page before   = {},
    toc subsection page after    = {},
%%
    % List of Figures
    lof name              = List of Figures,
    lof before            =,
    lof after             =,
    lof numwidth          = 0pt,
    lof color             = thelofname,
    lof background-color  = white,
    lof frame-color       = white,
    lof shadow            = none,
    lof font-weight       = normal,
    lof font-family       = rmfamily,
    lof font-shape        = upshape,
    lof font-size         = Huge,
    lof afterskip         = 40pt,
    lof after             = ,
    lof align             = left,
    lof indent            = 0pt,
    lof case              = none,
    lof page after        = ,
   color command     = themacrocolor,
   color environment = theenvironment,
   color key         = thekey,
   color value       = thevalue,
   color color       = black,%leaks to index
   color option      = theoption,
   color meta        = themeta,
   color frame       = theframe,
   % indexing
   index actual  = {@},
   index quote   = {!},
   index level   = {>},
   index doc settings,
  docexample/.style={colframe=ExampleFrame,colback=ExampleBack,fontlower=\footnotesize},
  documentation minted style=,
  documentation minted options={tabsize=2,fontsize=\small},
  english language/.code={\cxset{doclang/.cd,
    color=color,colors=Colors,
    environment content=environment content,
    environment=environment,environments=Environments,
    key=key,keys=Keys,
    index=Index,
    pageshort={P.},
    value=value,values=Values}},
 }
\endinput
%%
%% End of file `phd-documentation-defaults.def'.

\cxset{palette black tulip,
       section format=hang,
       subsection afterindent=on,
       section title color=bgsexy}
\cxset{section number prefix =}

\begin{document}
\AddPrefix{|cctab|intarray|sys}
%\AddPrefix{|sys}
\DEBUGON
  \DocInput{\jobname.dtx}
\end{document}
%</driver>
% \fi
%
% \title{^^A
%   The \pkg{l3cctab} package\\ Experimental category code tables^^A
% }
%
% \author{^^A
%  The \LaTeX3 Project\thanks
%    {^^A
%      E-mail:
%        \href{mailto:latex-team@latex-project.org}
%          {latex-team@latex-project.org}^^A
%    }^^A
% }
%
% \date{Released 2018-09-24}
%
% \maketitle
%
% \begin{documentation}
%
% \section{\pkg{l3cctab} documentation}
%
% A category code table enables rapid switching of all category codes in
% one operation. For \LuaTeX{}, this is possible over the entire Unicode
% range. For other engines, only the $8$-bit range ($0$-$255$) is covered by
% such tables.
%
% \begin{function}{\cctab_new:N}
%   \begin{syntax}
%     \cs{cctab_new:N} \meta{category code table}
%   \end{syntax}
%   Creates a new category code table, initially with the codes as
%   used by \IniTeX{}.
% \end{function}
%
% \begin{function}{\cctab_const:Nn}
%   \begin{syntax}
%     \cs{cctab_const:Nn} \meta{category code table} \Arg{category code set up}
%   \end{syntax}
%   Creates a new category code table and applies the
%   \meta{category code set up} on top of prevailing settings, then saves
%   as a constant table.
% \end{function}
%
% \begin{function}{\cctab_gset:Nn}
%   \begin{syntax}
%     \cs{cctab_gset:Nn} \meta{category code table} \Arg{category code set up}
%   \end{syntax}
%   Sets the \meta{category code table} to apply the category codes
%   which apply when the prevailing régime is modified by the
%   \meta{category code set up}. Thus within a standard code block
%   the starting point will be the code applied by \cs{c_code_cctab}.
%   The assignment of the table is global: the underlying primitive does
%   not respect grouping.
% \end{function}
%
% \begin{function}{\cctab_begin:N}
%   \begin{syntax}
%     \cs{cctab_begin:N} \meta{category code table}
%   \end{syntax}
%   Switches the category codes in force to those stored in the
%   \meta{category code table}.  The prevailing codes before the
%   function is called are added to a stack, for use with
%   \cs{cctab_end:}.
% \end{function}
%
% \begin{function}{\cctab_end:}
%   \begin{syntax}
%     \cs{cctab_end:}
%   \end{syntax}
%   Ends the scope of a \meta{category code table} started using
%   \cs{cctab_begin:N}, retuning the codes to those in force before the
%   matching \cs{cctab_begin:N} was used.
% \end{function}
%
% \begin{variable}{\c_code_cctab}
%   Category code table for the code environment. This does not include
%   setting the behaviour of the line-end character, which is only
%   altered by \cs{ExplSyntaxOn}.
% \end{variable}
%
% \begin{variable}{\c_document_cctab}
%   Category code table for a standard \LaTeX{} document. This does not
%   include setting the behaviour of the line-end character, which is
%   only altered by \cs{ExplSyntaxOff}.
% \end{variable}
%
% \begin{variable}{\c_initex_cctab}
%   Category code table as set up by \IniTeX{}.
% \end{variable}
%
% \begin{variable}{\c_other_cctab}
%   Category code table where all characters have category code $12$
%   (other).
% \end{variable}
%
% \begin{variable}{\c_str_cctab}
%   Category code table where all characters have category code $12$
%   (other) with the exception of spaces, which have category code
%   $10$ (space).
% \end{variable}
%
% \end{documentation}
%
% \begin{implementation}
%
% \section{\pkg{l3cctab} implementation}
%
%    \begin{macrocode}
%<*initex|package>
%    \end{macrocode}
%
%    \begin{macrocode}
%<@@=cctab>
%    \end{macrocode}
%
%    \begin{macrocode}
%<*package>
\ProvidesExplPackage{l3cctab}{2018-09-24}{}
  {L3 Experimental category code tables}
%</package>
%    \end{macrocode}
%
% \begin{variable}{\g_@@_allocate_int}
% \begin{variable}{\g_@@_stack_int}
% \begin{variable}{\g_@@_stack_seq}
%   To allocate category code tables, both the read-only and stack
%   tables need to be followed. There is also a sequence stack for the
%   dynamic tables themselves.
%    \begin{macrocode}
%<*initex>
\int_new:N  \g_@@_allocate_int
\int_gset:Nn \g_@@_allocate_int { -1 }
%</initex>
\int_new:N \g_@@_stack_int
\seq_new:N \g_@@_stack_seq
%    \end{macrocode}
% \end{variable}
% \end{variable}
% \end{variable}
%
% \begin{variable}{\l_@@_tmp_tl}
%   Scratch space.
%    \begin{macrocode}
\tl_new:N \l_@@_tmp_tl
%    \end{macrocode}
% \end{variable}
%
% \begin{macro}{\cctab_new:N}
% \begin{macro}{\cctab_begin:N}
% \begin{macro}{\cctab_end:}
% \begin{macro}{\cctab_gset:Nn}
%   As \LuaTeX{} offers engine support for category code tables, and this is
%   entirely lacking from the other engines, we need two complementary
%   approaches here. Rather than intermix them, we split the set up based on
%   engine. (Some future \XeTeX{} may add support, at which point the
%   conditional here would be subtly different.)
%
%   First, the \LuaTeX{} case.
%    \begin{macrocode}
\sys_if_engine_luatex:TF
  {
%    \end{macrocode}
%   Creating a new category code table is done slightly differently
%   from other registers. Low-numbered tables are more efficiently-stored
%   than high-numbered ones. There is also a need to have a stack of
%   flexible tables as well as the set of read-only ones. To satisfy both
%   of these requirements, odd numbered tables are used for read-only
%   tables, and even ones for the stack. Here, therefore, the odd numbers
%   are allocated.
%    \begin{macrocode}
    \cs_new_protected:Npn \cctab_new:N #1
      {
        \__kernel_chk_if_free_cs:N #1
%<*initex>
        \int_gadd:Nn \g_@@_allocate_int { 2 }
        \int_compare:nNnTF
          \g_@@_allocate_int < { \c_max_register_int + 1 }
           {
             \tex_global:D \tex_chardef:D #1 \g_@@_allocate_int
             \tex_initcatcodetable:D #1
           }
           {
             \__kernel_msg_fatal:nnx
               { kernel } { out-of-registers } { cctab }
           }
%</initex>
%<*package>
        \newcatcodetable #1
        \tex_initcatcodetable:D #1
%</package>
      }
%    \end{macrocode}
%   The aim here is to ensure that the saved tables are read-only. This is
%   done by using a stack of tables which are not read only, and actually
%   having them as \enquote{in use} copies.
%    \begin{macrocode}
    \cs_new_protected:Npn \cctab_begin:N #1
      {
        \seq_gpush:Nx \g_@@_stack_seq { \tex_the:D \tex_catcodetable:D }
        \tex_catcodetable:D #1
        \int_gadd:Nn \g_@@_stack_int { 2 }
        \int_compare:nNnT \g_@@_stack_int > \c_max_register_int
          { \__kernel_msg_fatal:nn { kernel } { cctab-stack-full } }
        \tex_savecatcodetable:D \g_@@_stack_int
        \tex_catcodetable:D \g_@@_stack_int
      }
    \cs_new_protected:Npn \cctab_end:
      {
        \int_gsub:Nn \g_@@_stack_int { 2 }
        \seq_if_empty:NTF \g_@@_stack_seq
          { \tl_set:Nn \l_@@_tmp_tl { 0 } }
          { \seq_gpop:NN \g_@@_stack_seq \l_@@_tmp_tl }
        \tex_catcodetable:D \l_@@_tmp_tl \scan_stop:
      }
%    \end{macrocode}
%   Category code tables are always global, so only one version is needed.
%   The set up here is simple, and means that at the point of use there is
%   no need to worry about escaping category codes.
%    \begin{macrocode}
    \cs_new_protected:Npn \cctab_gset:Nn #1#2
      {
        \group_begin:
          #2
          \tex_savecatcodetable:D #1
        \group_end:
      }
  }
%    \end{macrocode}
%   Now the case for other engines. Here, we use an integer array for each
%   table. The index base is out-by-one, so we have an internal function to
%   handle that. The rest of the approach here is pretty simple: use a stack
%   of tables, and save to them at each |begin|. Unlike the \LuaTeX{} case,
%   we can't accidentally alter a saved table, which makes life a little
%   easier.
%    \begin{macrocode}
  {
    \cs_new_protected:Npn \@@_gstore:Nnn #1#2#3
      { \intarray_gset:Nnn #1 { #2 + 1 } {#3} }
%    \end{macrocode}
%   Following the \LuaTeX{} pattern, a new table starts with \IniTeX{} codes.
%    \begin{macrocode}
    \cs_new_protected:Npn \cctab_new:N #1
      {
        \intarray_new:Nn #1 { 256 }
        \int_step_inline:nn { 256 }
          { \intarray_gset:Nnn #1 {##1} { 12 } }
        \@@_gstore:Nnn #1 { 0 } { 9 }
        \@@_gstore:Nnn #1 { 13 } { 5 }
        \@@_gstore:Nnn #1 { 32 } { 10 }
        \@@_gstore:Nnn #1 { 37 } { 14 }
        \int_step_inline:nnn { 65 } { 90 }
          { \intarray_gset:Nnn #1 {##1} { 11 } }
        \@@_gstore:Nnn #1 { 92 } { 0 }
        \int_step_inline:nnn { 97 } { 122 }
          { \@@_gstore:Nnn #1 {##1} { 11 } }
        \@@_gstore:Nnn #1 { 127 } { 15 }
      }
    \cs_new_protected:Npn \cctab_begin:N #1
      {
        \int_gadd:Nn \g_@@_stack_int { 1 }
        \int_compare:nNnT \g_@@_stack_int > \c_max_register_int
          { \__kernel_msg_fatal:nn { kernel } { cctab-stack-full } }
        \cs_if_exist:cF { g_@@_ \int_use:N \g_@@_stack_int _cctab }
          {
            \intarray_new:cn
              { g_@@_ \int_use:N \g_@@_stack_int _cctab }
              { 256 }
          }
        \int_step_inline:nn { 256 }
          {
            \intarray_gset:cnn
              { g_@@_ \int_use:N \g_@@_stack_int _cctab }
              {##1}
              { \char_value_catcode:n { ##1 - 1 } }
          }
        \int_step_inline:nn { 256 }
          {
            \char_set_catcode:nn { ##1 - 1 }
              { \intarray_item:Nn #1 {##1} }
          }
      }
    \cs_generate_variant:Nn \intarray_new:Nn { c }
    \cs_generate_variant:Nn \intarray_gset:Nnn { c }
    \cs_new_protected:Npn \cctab_end:
      {
        \int_step_inline:nn { 256 }
          {
            \char_set_catcode:nn { ##1 - 1 }
              {
                \intarray_item:cn
                  { g_@@_ \int_use:N \g_@@_stack_int _cctab }
                  {##1}
              }
          }
        \int_gsub:Nn \g_@@_stack_int { 1 }
      }
    \cs_generate_variant:Nn \intarray_item:Nn { c }
    \cs_new_protected:Npn \cctab_gset:Nn #1#2
      {
        \group_begin:
          #2
          \int_step_inline:nn { 256 }
            {
              \intarray_gset:Nnn #1 {##1}
                { \char_value_catcode:n { ##1 - 1 } }
            }
        \group_end:
      }
  }
%    \end{macrocode}
% \end{macro}
% \end{macro}
% \end{macro}
% \end{macro}
%
% \begin{variable}{\g_@@_tmp_cctab}
%   Scratch space.
%    \begin{macrocode}
\cctab_new:N \g_@@_tmp_cctab
%    \end{macrocode}
% \end{variable}
%
% \begin{macro}{\cctab_const:Nn}
%   Creating constant tables is a bit tricky: we do it in a two part
%   fashion via a temporary one.
%    \begin{macrocode}
\cs_new_protected:Npn \cctab_const:Nn #1#2
  {
    \cctab_gset:Nn \g_@@_tmp_cctab {#2}
    \cs_new_eq:NN #1 \g_@@_tmp_cctab
  }
%    \end{macrocode}
% \end{macro}
%
% \begin{variable}
%   {
%     \c_code_cctab     ,
%     \c_document_cctab ,
%     \c_initex_cctab   ,
%     \c_other_cctab    ,
%     \c_str_cctab
%   }  
%   Creating category code tables is easy using the function above.
%   The \texttt{other} and \texttt{string} ones are done by completely
%   ignoring the existing codes as this makes life a lot less complex.
%    \begin{macrocode}
\cctab_const:Nn \c_code_cctab { }
\cctab_const:Nn \c_document_cctab
  {
    \char_set_catcode_space:n          { 9 }
    \char_set_catcode_space:n          { 32 }
    \char_set_catcode_other:n          { 58 }
    \char_set_catcode_math_subscript:n { 95 }
    \char_set_catcode_active:n         { 126 }
  }
\cctab_const:Nn \c_other_cctab
  {
    \int_step_inline:nnn { 0 } { 127 }
      { \char_set_catcode_other:n {#1} }
  }
\cctab_const:Nn \c_str_cctab
  {
    \int_step_inline:nnn { 0 } { 127 }
      { \char_set_catcode_other:n {#1} }
    \char_set_catcode_space:n { 32 }
  }
%    \end{macrocode}
% \end{variable}
%
% \subsection{Messages}
%
%    \begin{macrocode}
\__kernel_msg_new:nnnn { kernel } { cctab-stack-full }
  { The~category~code~table~stack~is~exhausted. }
  {
    LaTeX~has~been~asked~to~switch~to~a~new~category~code~table,~
    but~there~is~no~more~space~to~do~this!
  }
%    \end{macrocode}
%
%    \begin{macrocode}
%</initex|package>
%    \end{macrocode}
%
%\end{implementation}
%
%\PrintIndex