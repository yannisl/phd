% https://tex.stackexchange.com/questions/7037/what-does-marks-do
% 
\documentclass{book}

\newmarks\mymark
\newmarks\mysecondmark
\usepackage{fancyhdr}
\pagestyle{fancy}
\fancyhf{}
\lhead{\topmarks\mymark \botmarks\mymark  \firstmarks\mysecondmark  }
\begin{document}
text
\marks\mymark{A first mark}
\marks\mysecondmark{TITLE OF PAGE}
\marks\mymark{A second mark}

\newpage

text \marks\mysecondmark{SUBTITLE}

\mysecondmark
\end{document}
http://tex.loria.fr/moteurs/etex_ref.html#marks
\marks:
this is one of Knuth's `possibly good ideas', listed at the end of TeX82.Bug; whereas TeX has only one \mark, which has to be over-loaded if more than one class of information is to be saved (e.g. over-loading is necessary if separate information for recto and verso pages is to be maintained), e-TeX has a whole class of \marks (256, in the first release); thus rather than writing \mark <general text> as in TeX, in e-TeX one writes \marks 8-bit number <general text>. For example, \marks 0 could be used to retain information for the verso page, whilst \marks 1 could retain information for the recto. There are equivalent classes for the five \marks variables \botmarks, \firstmarks, \topmarks, \splitfirstmarks and \splitbotmarks. It should be noted that \marks 0 and \mark are in fact identical, as are \topmarks 0 and \topmark, \botmarks 0 and \botmark and so on.

For those of you who are interested, after having toiled over this problem
with the kind help of Dr. Michael Barr at McGill University, here is a
summary of using the \mark command in LaTeX and some trouble spots/fixes.
------------------------------------------------------------------------------
Ok, here's the scoop.  TeX has a command, namely \mark, which allows you to
mark certain locations on a page.  You can mark anything, and that will be
remembered by TeX when the page is output.  It will not be expanded in the
text, so you will have to put it in twice, e.g.:
marked text\mark{marked text}

In order to refer to the marks, then, you need to use the commands \firstmark,
\botmark, and \topmark.  \firstmark is the first mark on the page to be
output, and \botmark is the last mark on the page to be output.  (There is
also \topmark, but I cannot recall what this one does.  I think it is
assigned the last value of \botmark before the page is \shipout'd.  Refer to
the TeXbook for the facts.)

However, there's a catch.  In using article.sty of LaTeX, the first
\firstmark of the first page will be set to nothing.  There's a line in
article.sty, "\mark{{}{}}", which does that.  If you need to you can make a
copy of article.sty and comment this line out.  This is what I had to do.

Also, there's another catch.  If you are using twocolumn format, you will
have to redefine \@outputdblcol to save the \firstmark of the first column.
The problem lies in the fact that each column, in LaTeX, is treated as a
separate page, thus the \firstmark command won't work, since it will
actually be the \firstmark in the second column, not the first.

Anyway, include this in your code (from M.Barr):

\makeatletter
\def\@outputdblcol{\if@firstcolumn
\edef\firstcolmark{\firstmark}%%%%%%%%%% Added line %%%%%%%%%%
\global\@firstcolumnfalse
    \global\setbox\@leftcolumn\box\@outputbox
  \else \global\@firstcolumntrue
    \setbox\@outputbox\vbox{\hbox to\textwidth{\hbox to\columnwidth
      {\box\@leftcolumn \hss}\hfil \vrule width\columnseprule\hfil
       \hbox to\columnwidth{\box\@outputbox \hss}}}\@combinedblfloats
       \@outputpage \begingroup \@dblfloatplacement \@startdblcolumn
       \@whilesw\if@fcolmade \fi{\@outputpage\@startdblcolumn}\endgroup
    \fi}
\makeatother

Then, when you want to refer to the first mark on the page, use
\firstcolmark instead.

Thus, in summary, for one column format, \firstmark and \lastmark are what
you need, but remember to edit out the \mark{{}{}} command in article.sty if
you want the first mark on the first page.  For two column format, use the
redefinition code above, and use \firstcolmark and \lastmark.  But, as well,
remember to edit out the \mark{{}{}} command in article.sty.

I must give credit to Michael Barr at McGill University for all of his help.
In fact, most of the credit is due to him and I am very grateful.

Lastly, if anyone has anything else to add regarding this, please do.  I'm
not a TeX expert, so I probably have overlooked some finer points.  For my
purposes, however, this summary will suffice.  I should also note that I am
using this material in conjunction with the fancyheadings style, though this
is not necessary for those who would prefer to use either \markboth,
\markright, etc., or their own set of macros.

--
Andrew Biewer
Language Resource Center
University of Wisconsin--Milwaukee
----------------------------------
``Using TeX/LaTeX is sometimes like living in Alice's Wonderland.''
        Alan D. Corr\'{e} (paraphrase)